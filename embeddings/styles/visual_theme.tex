% Visual Theme and Consistency Framework for Embeddings Presentation
% Provides unified visual language across all slides

% ====================================================================
% COLOR CODING SYSTEM
% ====================================================================

% Concept Categories - Each major concept has a signature color
\definecolor{conceptEmbedding}{RGB}{70,130,180}    % Steel Blue - for embedding concepts
\definecolor{conceptTraining}{RGB}{255,140,0}       % Dark Orange - for training/learning
\definecolor{conceptMath}{RGB}{148,0,211}          % Dark Violet - for mathematical concepts
\definecolor{conceptDimensional}{RGB}{220,20,60}   % Crimson - for dimensionality issues
\definecolor{conceptContext}{RGB}{46,139,87}       % Sea Green - for context/attention
\definecolor{conceptApplication}{RGB}{255,215,0}    % Gold - for applications

% Difficulty Levels - Visual indicators for complexity
\definecolor{levelBasic}{RGB}{144,238,144}         % Light Green
\definecolor{levelIntermediate}{RGB}{255,218,185}  % Peach
\definecolor{levelAdvanced}{RGB}{255,182,193}      % Light Pink
\definecolor{levelExpert}{RGB}{230,230,250}        % Lavender

% Alert/Status Colors
\definecolor{alertSuccess}{RGB}{34,139,34}         % Forest Green
\definecolor{alertWarning}{RGB}{255,140,0}         % Dark Orange
\definecolor{alertError}{RGB}{178,34,34}           % Fire Brick
\definecolor{alertInfo}{RGB}{70,130,180}           % Steel Blue

% ====================================================================
% VISUAL ELEMENTS AND BOXES
% ====================================================================

% Concept Box - For introducing new concepts
\newtcolorbox{conceptbox}[2][]{
  colback=conceptEmbedding!5!white,
  colframe=conceptEmbedding!75!black,
  fonttitle=\bfseries,
  title=#2,
  arc=3mm,
  boxrule=1pt,
  #1
}

% Math Box - For mathematical definitions
\newtcolorbox{mathbox}[1][]{
  colback=conceptMath!5!white,
  colframe=conceptMath!50!black,
  arc=2mm,
  boxrule=0.5pt,
  #1
}

% Try This Box - For exercises and activities
\newtcolorbox{trybox}{
  colback=conceptApplication!10!white,
  colframe=conceptApplication!60!black,
  title={\faLightbulbO\ Try This!},
  fonttitle=\bfseries,
  arc=3mm,
  boxrule=1pt
}

% Key Insight Box - For important takeaways
\newtcolorbox{insightbox}{
  colback=alertInfo!10!white,
  colframe=alertInfo!60!black,
  title={\faKey\ Key Insight},
  fonttitle=\bfseries,
  arc=3mm,
  boxrule=1.5pt
}

% Warning Box - For common pitfalls
\newtcolorbox{warningbox}{
  colback=alertWarning!10!white,
  colframe=alertWarning!60!black,
  title={\faWarning\ Common Pitfall},
  fonttitle=\bfseries,
  arc=3mm,
  boxrule=1pt
}

% ====================================================================
% COMPLEXITY INDICATORS
% ====================================================================

% Difficulty stars command
\newcommand{\difficulty}[1]{%
  \ifcase#1\relax
  \or \textcolor{levelBasic}{\faStar\faStarO\faStarO\faStarO\faStarO} % 1 star
  \or \textcolor{levelBasic}{\faStar\faStar\faStarO\faStarO\faStarO} % 2 stars
  \or \textcolor{levelIntermediate}{\faStar\faStar\faStar\faStarO\faStarO} % 3 stars
  \or \textcolor{levelAdvanced}{\faStar\faStar\faStar\faStar\faStarO} % 4 stars
  \or \textcolor{levelExpert}{\faStar\faStar\faStar\faStar\faStar} % 5 stars
  \fi
}

% Progress bar for multi-part concepts
\newcommand{\conceptprogress}[2]{%
  \begin{tikzpicture}[baseline=-0.5ex]
    \draw[thick,gray!30] (0,0) rectangle (3,0.3);
    \fill[conceptEmbedding!70] (0,0) rectangle ({3*#1/#2},0.3);
    \node[right] at (3.1,0.15) {\small #1/#2};
  \end{tikzpicture}
}

% ====================================================================
% ICON SYSTEM FOR OPERATIONS
% ====================================================================

% Define consistent icons for common operations
\newcommand{\iconAdd}{\textcolor{alertSuccess}{\faPlus}}
\newcommand{\iconSubtract}{\textcolor{alertError}{\faMinus}}
\newcommand{\iconMultiply}{\textcolor{conceptMath}{\faTimes}}
\newcommand{\iconTransform}{\textcolor{conceptTraining}{\faRandom}}
\newcommand{\iconProject}{\textcolor{conceptDimensional}{\faArrowsAlt}}
\newcommand{\iconAttention}{\textcolor{conceptContext}{\faEye}}
\newcommand{\iconEmbedding}{\textcolor{conceptEmbedding}{\faCube}}
\newcommand{\iconNetwork}{\textcolor{conceptTraining}{\faShareAlt}}
\newcommand{\iconGradient}{\textcolor{conceptMath}{\faLineChart}}
\newcommand{\iconOptimize}{\textcolor{alertSuccess}{\faCogs}}

% ====================================================================
% VISUAL CONSISTENCY COMMANDS
% ====================================================================

% Highlighted concept introduction
\newcommand{\concept}[1]{\textcolor{conceptEmbedding}{\textbf{#1}}}
\newcommand{\mathconcept}[1]{\textcolor{conceptMath}{\textbf{#1}}}
\newcommand{\trainingconcept}[1]{\textcolor{conceptTraining}{\textbf{#1}}}

% Dimension notation with visual hint
\newcommand{\dimension}[1]{\textcolor{conceptDimensional}{$d=#1$}}

% Vector notation with consistent style
\renewcommand{\vec}[1]{\boldsymbol{#1}}
\newcommand{\embedvec}[1]{\textcolor{conceptEmbedding}{\vec{#1}}}

% Matrix notation
\newcommand{\embedmatrix}[1]{\textcolor{conceptEmbedding}{\mathbf{#1}}}

% ====================================================================
% SLIDE TEMPLATES
% ====================================================================

% Section intro slide with visual theme
\newcommand{\sectionintro}[3]{% {section name}{icon}{description}
  \begin{frame}[plain]
    \begin{center}
      \vspace{2cm}
      {\Huge #2}\\[0.5cm]
      {\Large \textcolor{conceptEmbedding}{#1}}\\[0.3cm]
      {\normalsize #3}
    \end{center}
  \end{frame}
}

% Concept introduction slide
\newenvironment{conceptslide}[2]{% {title}{difficulty}
  \begin{frame}{#1 \hfill \difficulty{#2}}
}{
  \end{frame}
}

% ====================================================================
% PROGRESS TRACKING
% ====================================================================

% Visual progress indicator for sections
\newcommand{\sectionprogress}{%
  \begin{tikzpicture}[remember picture,overlay]
    \node[anchor=south east,xshift=-0.5cm,yshift=0.5cm] at (current page.south east) {
      \conceptprogress{\thesection}{\totalsections}
    };
  \end{tikzpicture}
}

% Time indicator for presentations
\newcommand{\timeindicator}[1]{%
  \begin{tikzpicture}[remember picture,overlay]
    \node[anchor=south west,xshift=0.5cm,yshift=0.5cm] at (current page.south west) {
      \textcolor{gray}{\faClock\ #1 min}
    };
  \end{tikzpicture}
}

% ====================================================================
% ANIMATION HELPERS
% ====================================================================

% Step-by-step reveal with visual cues
\newcommand{\step}[2]{\only<#1->{\textcolor<#1>{conceptEmbedding}{#2}}}
\newcommand{\highlight}[2]{\textcolor<#1>{alertWarning}{#2}}
\newcommand{\fade}[2]{\textcolor<#1>{gray!30}{#2}}

% Progressive complexity reveal
\newenvironment{progressive}[1]{% {number of steps}
  \setcounter{beamerpauses}{#1}
}{
}

% ====================================================================
% FORMULA WITH CHART COMMAND
% ====================================================================

% Command to display formula with accompanying chart
\newcommand{\formulaWithChart}[3]{%
% #1: formula content
% #2: chart filename (without path/extension)
% #3: chart caption
\begin{columns}[T]
\column{0.48\textwidth}
\vspace{0.5cm}
\begin{block}{Mathematical Formula}
#1
\end{block}
\column{0.48\textwidth}
\begin{figure}
\centering
\includegraphics[width=\textwidth]{../figures/embeddings/formulas/#2.pdf}
\caption{\small #3}
\end{figure}
\end{columns}
}

% Alternative layout for larger formulas
\newcommand{\formulaWithChartWide}[3]{%
% #1: formula content  
% #2: chart filename (without path/extension)
% #3: chart caption
\begin{block}{Mathematical Formula}
#1
\end{block}
\vspace{0.3cm}
\begin{center}
\includegraphics[width=0.8\textwidth]{../figures/embeddings/formulas/#2.pdf}
\end{center}
\vspace{-0.2cm}
\textit{\small #3}
}

% ====================================================================
% VISUAL CONSISTENCY RULES
% ====================================================================

% Set default itemize/enumerate colors
\setbeamercolor{itemize item}{fg=conceptEmbedding}
\setbeamercolor{itemize subitem}{fg=conceptEmbedding!70}
\setbeamercolor{itemize subsubitem}{fg=conceptEmbedding!50}

\setbeamercolor{enumerate item}{fg=conceptMath}
\setbeamercolor{enumerate subitem}{fg=conceptMath!70}
\setbeamercolor{enumerate subsubitem}{fg=conceptMath!50}

% Consistent alert colors
\setbeamercolor{alerted text}{fg=alertWarning}

% Block colors
\setbeamercolor{block title}{bg=conceptEmbedding,fg=white}
\setbeamercolor{block body}{bg=conceptEmbedding!10}

\setbeamercolor{block title alerted}{bg=alertWarning,fg=white}
\setbeamercolor{block body alerted}{bg=alertWarning!10}

\setbeamercolor{block title example}{bg=conceptApplication,fg=black}
\setbeamercolor{block body example}{bg=conceptApplication!10}