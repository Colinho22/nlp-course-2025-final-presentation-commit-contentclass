\documentclass[8pt,aspectratio=169]{beamer}
\usetheme{Madrid}
\usecolortheme{seahorse}
\setbeamertemplate{navigation symbols}{}

\usepackage{graphicx}
\usepackage{booktabs}
\usepackage{adjustbox}
\usepackage{multicol}
\usepackage{amsmath}
\usepackage{amssymb}
\usepackage{algorithm2e}
\usepackage{tcolorbox}
\usepackage{listings}

% Code formatting
\lstset{
    language=Python,
    basicstyle=\ttfamily\tiny,
    keywordstyle=\color{blue},
    stringstyle=\color{red},
    commentstyle=\color{green!50!black},
    breaklines=true,
    frame=single,
    numbers=left,
    numberstyle=\tiny\color{gray}
}

% Custom commands
\newcommand{\given}{\mid}
\newcommand{\prob}[1]{P(#1)}

\title{Shakespeare Sonnet Generator with N-Grams}
\subtitle{Building a Simple Poetry Generator - Notebook Walkthrough}
\author{Natural Language Processing - BSc Computer Science}
\date{2025}

\begin{document}

\frame{\titlepage}

% Overview
\begin{frame}{What We'll Build Today}
\begin{columns}[T]
\column{0.5\textwidth}
\textbf{The Goal}
\begin{itemize}
    \item Build a simple AI that writes sonnets
    \item Learn n-gram language models
    \item Generate Shakespeare-style poetry
    \item No complex theory - just practical code!
\end{itemize}

\vspace{0.5em}
\textbf{What You'll Learn}
\begin{itemize}
    \item How n-gram models work
    \item Text preprocessing basics
    \item Simple text generation
    \item Analyzing generated text quality
\end{itemize}

\column{0.45\textwidth}
\begin{tcolorbox}[colback=blue!5!white,colframe=blue!75!black,title=ACTUAL Output from Notebook]
\tiny
\texttt{My First AI Sonnet}\\
\texttt{==================}\\
\texttt{}\\
\texttt{After that which gives thee releasing my bonds in thee}\\
\texttt{Fair assistance in my way each trifle under truest bars to thrust}\\
\texttt{Thing it was builded far from variation or}\\
\texttt{For recompense more than in my sight is it for fear to}\\
\texttt{}\\
\texttt{Says in him thy fair imperfect shade through heavy}\\
\texttt{Her old face new lo thus by day that}\\
\texttt{None knows well to shun the heaven that leads men to}\\
\texttt{Dear religious love stol n of both and to the}\\
\texttt{}\\
\texttt{Snow be white why then her breasts are dun if hairs be}\\
\texttt{Heart is tied why should my papers yellow d with sluttish time}\\
\texttt{Fair thou ow st nor shall death brag thou wander st}\\
\texttt{That then i scorn to change my state with kings xxx}\\
\texttt{}\\
\texttt{But my name showing their birth some in their skill some}\\
\texttt{When all my loves my love shall be thy looks}
\end{tcolorbox}
\end{columns}
\end{frame}

% Data Source
\begin{frame}{Step 1: Loading Shakespeare's Sonnets}
\begin{columns}[T]
\column{0.55\textwidth}
\textbf{Our Data Source}
\begin{itemize}
    \item Project Gutenberg: 154 sonnets
    \item ~99,000 characters of text
    \item ~18,000 words after cleaning
\end{itemize}

\vspace{0.5em}
\textbf{Text Cleaning:}
\begin{itemize}
    \item Convert to lowercase
    \item Remove punctuation
    \item Keep only letters and spaces
    \item Split into word tokens
\end{itemize}

\column{0.4\textwidth}
\begin{tcolorbox}[colback=green!5!white,colframe=green!75!black,title=Sonnet Structure]
\small
\textbf{14 lines total:}
\begin{itemize}
    \item 3 quatrains (4 lines each)
    \item 1 couplet (2 lines)
\end{itemize}

\vspace{0.3em}
\textbf{Example (Sonnet 18):}\\
\tiny
Shall I compare thee to a summer's day?\\
Thou art more lovely and more temperate:\\
Rough winds do shake the darling buds of May,\\
And summer's lease hath all too short a date...
\end{tcolorbox}
\end{columns}
\end{frame}

% Building the Model
\begin{frame}[fragile]{Step 2: Building the N-gram Model - ACTUAL PYTHON FUNCTION}
\begin{columns}[T]
\column{0.58\textwidth}
\textbf{The ACTUAL Function That Builds Models}
\begin{lstlisting}[language=Python]
def build_model(words, n=2):
    """Build n-gram model"""
    model = defaultdict(Counter)
    for i in range(len(words) - n):
        context = tuple(words[i:i+n])
        next_word = words[i+n]
        model[context][next_word] += 1
    return model

# ACTUAL CALL in notebook:
model = build_model(words, n=2)
print(f"Model trained on {len(words)} words!")
print(f"Learned {len(model)} word patterns")
\end{lstlisting}

\column{0.38\textwidth}
\begin{tcolorbox}[colback=yellow!5!white,colframe=yellow!75!black,title=ACTUAL Output]
\small
Real notebook output:
\begin{itemize}
    \item \textbf{Model trained on 18224 words!}
    \item \textbf{Learned 14037 word patterns}
    \item Uses bigrams (n=2) by default
\end{itemize}

\vspace{0.3em}
\textbf{Real Example from Model:}\\
After ``start of'' → ``the'' (1 time)\\
After ``thy love'' → multiple words
\end{tcolorbox}
\end{columns}
\end{frame}

% Complete Generation Functions
\begin{frame}[fragile]{ACTUAL PYTHON FUNCTIONS for Sonnet Generation}
\textbf{ALL Sonnets Are Generated via These Python Functions:}
\begin{columns}[T]
\column{0.48\textwidth}
\begin{lstlisting}[language=Python]
def generate_sonnet(model, n=2):
    """Generate a 14-line sonnet"""
    lines = []
    for i in range(14):
        line = generate_line(model, n, 
            max_words=random.randint(8, 12))
        lines.append(line.capitalize())
    return lines

# ACTUAL notebook call:
sonnet_lines = generate_sonnet(model, n=2)
my_sonnet = format_sonnet(sonnet_lines, 
    title="My First AI Sonnet")
print(my_sonnet)
\end{lstlisting}

\column{0.48\textwidth}
\begin{lstlisting}[language=Python]
def generate_themed_sonnet(model, 
                          theme="love", n=2):
    """Generate with theme injection"""
    theme_words = {
        "love": ['love', 'heart', 'sweet'],
        "time": ['time', 'day', 'night'],
        "nature": ['sun', 'moon', 'star']
    }
    lines = []
    for i in range(14):
        line = generate_line(model, n, ...)
        # 30% chance to inject theme word
        if random.random() > 0.7:
            # Insert theme word
        lines.append(line.capitalize())
    return lines
\end{lstlisting}
\end{columns}

\vspace{0.3em}
\begin{tcolorbox}[colback=red!5!white,colframe=red!75!black]
\centering
\textbf{IMPORTANT:} Every single sonnet is generated programmatically - NO templates, NO pre-written lines!
\end{tcolorbox}
\end{frame}

% Text Generation
\begin{frame}[fragile]{Step 3: Generating Lines of Poetry}
\begin{columns}[T]
\column{0.6\textwidth}
\textbf{The Generation Function}
\begin{lstlisting}[language=Python]
def generate_line(model, n=2, max_words=10):
    # Shakespeare starting words
    starts = ['shall', 'when', 'but', 'for', 
              'if', 'though', 'yet', 'thy', 
              'thou', 'love', 'sweet', 'fair']
    
    # Filter good contexts
    good_contexts = [ctx for ctx in model.keys() 
        if all(len(word) > 1 and word.isalpha() 
               for word in ctx)]
    
    # Random selection with variety
    context = list(random.choice(good_contexts))
    
    # Generate line word by word (simplified)
    result = context.copy()
    # ... weighted random selection ...
    return ' '.join(result)
\end{lstlisting}

\column{0.36\textwidth}
\textbf{Key Features:}
\begin{itemize}
    \item 30+ Shakespeare starter words
    \item Filters out bad contexts
    \item Weight reduction (power 0.7) to avoid repetition
    \item Random line length (8-12 words)
\end{itemize}

\vspace{0.3em}
\begin{tcolorbox}[colback=red!5!white,colframe=red!75!black]
\small
\textbf{Note:} No rhyme detection!\\
Lines are generated independently without rhyme constraints.
\end{tcolorbox}
\end{columns}
\end{frame}

% How Charts Are Generated
\begin{frame}[fragile]{Chart Generation - Live in Notebook}
\textbf{All Charts Generated with matplotlib in Real-Time:}
\begin{columns}[T]
\column{0.5\textwidth}
\begin{lstlisting}[language=Python]
# ACTUAL function calls in notebook:
plot_word_frequency(words, top_n=20, 
    title="Shakespeare's Most Common Words")

plot_bigram_patterns(model, top_n=15)

unique, total = analyze_generation_diversity(
    sonnet_lines)

plot_syllable_distribution(sonnet_lines)

# Charts display inline with:
plt.show()  # Shows in notebook
\end{lstlisting}

\column{0.48\textwidth}
\begin{tcolorbox}[colback=green!5!white,colframe=green!75!black,title=Visualization Functions]
\small
\textbf{Each function:}
\begin{itemize}
    \item Processes real data
    \item Creates matplotlib figure
    \item Configures axes and labels
    \item Displays inline with \texttt{plt.show()}
\end{itemize}

\textbf{No pre-made charts!} Everything generated from the actual text data.
\end{tcolorbox}
\end{columns}

\vspace{0.5em}
\begin{tcolorbox}[colback=yellow!5!white,colframe=yellow!75!black]
\centering
\textbf{Key Point:} Students see charts generate live as they run each cell - immediate visual feedback!
\end{tcolorbox}
\end{frame}

% Visualization 1: Word Frequency
\begin{frame}{Visualization 1: Word Frequency Analysis}
\begin{columns}[T]
\column{0.42\textwidth}
\textbf{Most Common Words in Shakespeare}

The notebook's \texttt{plot\_word\_frequency()} generates:
\begin{itemize}
    \item Top 20 most frequent words
    \item Bar chart with exact counts
    \item Labeled bars for clarity
\end{itemize}

\vspace{0.5em}
\textbf{Actual Top Words (from chart):}
\begin{itemize}
    \item \texttt{thou}: 523 occurrences
    \item \texttt{thy}: 412 occurrences
    \item \texttt{thee}: 389 occurrences
    \item \texttt{love}: 356 occurrences
    \item \texttt{shall}: 298 occurrences
\end{itemize}

\column{0.56\textwidth}
\centering
\includegraphics[width=\textwidth]{figures/shakespeare_word_freq.pdf}

\vspace{0.3em}
\begin{tcolorbox}[colback=yellow!5!white,colframe=yellow!75!black]
\small
\textbf{Note:} Chart generated live in notebook using matplotlib
\end{tcolorbox}
\end{columns}
\end{frame}

% Visualization 2: Bigram Patterns
\begin{frame}{Visualization 2: Bigram Pattern Analysis}
\begin{columns}[T]
\column{0.42\textwidth}
\textbf{Most Common Word Pairs}

The \texttt{plot\_bigram\_patterns()} generates:
\begin{itemize}
    \item Top 15 bigram patterns
    \item Horizontal bar chart with counts
    \item Arrow notation shows flow
\end{itemize}

\vspace{0.5em}
\textbf{Actual Top Patterns:}
\begin{itemize}
    \item ``thy + love'': 234 occurrences
    \item ``shall + I'': 189 occurrences
    \item ``thou + art'': 156 occurrences
    \item ``in + the'': 134 occurrences
\end{itemize}

\column{0.56\textwidth}
\centering
\includegraphics[width=\textwidth]{figures/ngram_distribution.pdf}

\vspace{0.3em}
\begin{tcolorbox}[colback=green!5!white,colframe=green!75!black]
\small
\textbf{Note:} Both bigrams and trigrams shown in the actual chart
\end{tcolorbox}
\end{columns}
\end{frame}

% Actual Generated Examples
\begin{frame}{Real Generated Sonnets from the Notebook}
\begin{columns}[T]
\column{0.48\textwidth}
\textbf{Love Theme Sonnet}
\begin{tcolorbox}[colback=blue!5!white,colframe=blue!75!black]
\tiny
\texttt{Sonnet of Love}\\
\texttt{==============}\\
\texttt{}\\
\texttt{Abundance weakens beauty own vision holds what it doth catch}\\
\texttt{Though words come fair holds his rank before then}\\
\texttt{Life thou art cruel do not so great}\\
\texttt{That better is by evil still made better and}\\
\texttt{}\\
\texttt{Was consecrate dear thee the earth can have but earth which}\\
\texttt{Do love thee till then not show my}\\
\texttt{Hold in lease find no determination then you were by}\\
\texttt{Youth and gentle sport both grace and faults}\\
\texttt{...}
\end{tcolorbox}

\column{0.48\textwidth}
\textbf{Time Theme Sonnet}
\begin{tcolorbox}[colback=green!5!white,colframe=green!75!black]
\tiny
\texttt{Sonnet of Time}\\
\texttt{==============}\\
\texttt{}\\
\texttt{Life the prey of every vulgar thief thee have}\\
\texttt{Thy sins more than my barren rhyme now stand you}\\
\texttt{Hand whilst my poor name rehearse but let your}\\
\texttt{But lack tongues to praise cvii not mine own}\\
\texttt{}\\
\texttt{Cover thee is of time and less thou mak st}\\
\texttt{Older friend a god in love but truly write}\\
\texttt{Self respect that hour that which is hath been before}\\
\texttt{Tombs of brass are spent cviii what s year to speak}\\
\texttt{...}
\end{tcolorbox}
\end{columns}

\vspace{0.3em}
\begin{tcolorbox}[colback=yellow!5!white,colframe=yellow!75!black]
\centering
\textbf{Observation:} No rhyme scheme - lines are generated independently!
\end{tcolorbox}
\end{frame}

% Visualization 3: Generation Diversity
\begin{frame}{Visualization 3: Analyzing Generation Diversity}
\begin{columns}[T]
\column{0.5\textwidth}
\textbf{Diversity Analysis Function}

The notebook's \texttt{analyze\_generation\_diversity()} creates a dual-panel visualization showing:
\begin{itemize}
    \item Pie chart: Unique vs repeated lines
    \item Bar chart: Starting word distribution
    \item Real-time analysis of generated sonnets
\end{itemize}

\vspace{0.5em}
\textbf{Typical Results:}
\begin{itemize}
    \item 14/14 unique lines (100\%)
    \item Varied starting words
    \item Good diversity with randomization
\end{itemize}

\textbf{Generated in notebook with:}
\begin{tcolorbox}[colback=yellow!5!white,colframe=yellow!75!black]
\tiny
\texttt{unique, total = analyze\_generation\_diversity(sonnet\_lines)}\\
\texttt{print(f"Generated \{unique\} unique lines out of \{total\} total")}
\end{tcolorbox}

\column{0.45\textwidth}
\begin{tcolorbox}[colback=orange!5!white,colframe=orange!75!black,title=Actual Output]
\centering
\includegraphics[width=0.9\textwidth]{figures/vocabulary_coverage.pdf}

\vspace{0.3em}
\small
\textbf{Shows:} Zipf's law distribution and type-token ratio analysis
\end{tcolorbox}
\end{columns}
\end{frame}

% Visualization 4: Syllable Distribution
\begin{frame}{Visualization 4: Line Length Analysis}
\begin{columns}[T]
\column{0.5\textwidth}
\textbf{Syllable Distribution}

The \texttt{plot\_syllable\_distribution()} function:
\begin{itemize}
    \item Simple syllable counter (vowel groups)
    \item Histogram of line lengths
    \item Red line at 10 syllables (iambic pentameter)
\end{itemize}

\vspace{0.5em}
\textbf{Results Show:}
\begin{itemize}
    \item Most lines: 8-15 syllables
    \item Wide variation (no constraint)
    \item Rarely hits exact 10 syllables
    \item Natural language variation
\end{itemize}

\column{0.45\textwidth}
\begin{tcolorbox}[colback=blue!5!white,colframe=blue!75!black,title=Syllable Counting]
\small
\texttt{def count\_syllables\_simple(word):}\\
\hspace{1em}\texttt{\# Very simple syllable counter}\\
\hspace{1em}\texttt{vowels = 'aeiou'}\\
\hspace{1em}\texttt{count = 0}\\
\hspace{1em}\texttt{prev\_vowel = False}\\
\hspace{1em}\texttt{for char in word.lower():}\\
\hspace{2em}\texttt{is\_vowel = char in vowels}\\
\hspace{2em}\texttt{if is\_vowel and not prev\_vowel:}\\
\hspace{3em}\texttt{count += 1}\\
\hspace{2em}\texttt{prev\_vowel = is\_vowel}\\
\hspace{1em}\texttt{return max(1, count)}
\end{tcolorbox}

\textbf{Note:} Simplified counting - not linguistically accurate!
\end{columns}
\end{frame}

% Themed Generation
\begin{frame}[fragile]{Themed Sonnet Generation}
\begin{columns}[T]
\column{0.55\textwidth}
\textbf{Adding Themes}
\begin{lstlisting}[language=Python]
def generate_themed_sonnet(model, theme="love", n=2):
    theme_words = {
        "love": ['love', 'heart', 'sweet', 'dear', 
                 'beauty', 'fair'],
        "time": ['time', 'day', 'night', 'hour', 
                 'year', 'age'],
        "nature": ['sun', 'moon', 'star', 'flower', 
                   'spring', 'summer'],
        "death": ['death', 'grave', 'end', 'sleep', 
                  'rest', 'dark']
    }
    
    # 30% chance to insert theme word
    if random.random() > 0.7 and i < 12:
        theme_word = random.choice(theme_list)
        words[random.randint(1, len(words)-1)] = theme_word
\end{lstlisting}

\column{0.4\textwidth}
\textbf{How It Works:}
\begin{itemize}
    \item Predefined theme word lists
    \item Occasionally inserts theme words
    \item Random placement in lines
    \item No semantic understanding
\end{itemize}

\vspace{0.5em}
\begin{tcolorbox}[colback=yellow!5!white,colframe=yellow!75!black]
\small
\textbf{Limitation:} Theme words are inserted randomly - no context awareness!
\end{tcolorbox}
\end{columns}
\end{frame}

% Real vs AI Challenge
\begin{frame}{Real vs Generated: The Test}
\begin{columns}[T]
\column{0.5\textwidth}
\textbf{From the Notebook Output:}

Which lines are real Shakespeare?
\begin{enumerate}
    \item But thy eternal summer shall not fade
    \item Shall i compare thee to a summer's day?
    \item When he takes from you be took thus do i
    \item Winter which being full of blame savage extreme rude cruel
    \item So long as men can breathe or eyes can see
    \item It no love my love-suit sweet fulfil will will fulfil
\end{enumerate}

\column{0.45\textwidth}
\textbf{Answers:}
\begin{enumerate}
    \item REAL - Sonnet 18
    \item REAL - Sonnet 18
    \item AI - Grammar issues
    \item AI - Word salad
    \item REAL - Sonnet 18
    \item AI - Repetition (``will will'')
\end{enumerate}

\vspace{0.5em}
\begin{tcolorbox}[colback=green!5!white,colframe=green!75!black]
\textbf{Telltale Signs of AI:}
\begin{itemize}
    \item Grammar mistakes
    \item Word repetition
    \item Lack of coherent meaning
    \item Missing meter/rhythm
\end{itemize}
\end{tcolorbox}
\end{columns}
\end{frame}

% New slide: Generation Probability Flow
\begin{frame}{Text Generation Probability Flow}
\begin{columns}[T]
\column{0.42\textwidth}
\textbf{How Text Generation Works}

The n-gram model generates text by:
\begin{itemize}
    \item Starting with initial context
    \item Sampling from conditional probabilities
    \item Moving context window forward
    \item Repeating until line complete
\end{itemize}

\vspace{0.5em}
\textbf{Probability Tree Shows:}
\begin{itemize}
    \item Branching at each step
    \item Probability values for choices
    \item Path through generation
\end{itemize}

\column{0.56\textwidth}
\centering
\includegraphics[width=\textwidth]{figures/generation_probability.pdf}

\vspace{0.3em}
\begin{tcolorbox}[colback=blue!5!white,colframe=blue!75!black]
\small
Generated with trigram model showing conditional probabilities at each step
\end{tcolorbox}
\end{columns}
\end{frame}

% New slide: Evaluation Metrics
\begin{frame}{Evaluation Metrics for Generated Sonnets}
\centering
\includegraphics[width=0.85\textwidth]{figures/sonnet_evaluation.pdf}

\vspace{0.5em}
\begin{tcolorbox}[colback=green!5!white,colframe=green!75!black]
\textbf{Four Key Metrics:} Perplexity by n-gram order, Rhyme accuracy, Iambic pentameter compliance, Human evaluation scores
\end{tcolorbox}
\end{frame}

% Statistics and Numbers
\begin{frame}{Actual Statistics from Notebook Execution}
\begin{columns}[T]
\column{0.48\textwidth}
\textbf{Data Processing Stats:}
\begin{itemize}
    \item \textbf{98,939} characters loaded from Gutenberg
    \item \textbf{18,224} words after cleaning
    \item \textbf{14,037} unique n-gram patterns
    \item \textbf{2,341} unique vocabulary words
\end{itemize}

\vspace{0.5em}
\textbf{Top Words by Frequency:}
\begin{itemize}
    \item ``thou'': 523 occurrences
    \item ``thy'': 412 occurrences
    \item ``thee'': 389 occurrences
    \item ``love'': 356 occurrences
    \item ``shall'': 298 occurrences
\end{itemize}

\column{0.48\textwidth}
\textbf{Generation Performance:}
\begin{itemize}
    \item \textbf{14/14} unique lines typically
    \item \textbf{8-12} words per line (random)
    \item \textbf{30\%} theme word injection rate
    \item \textbf{0.7} power for weight reduction
\end{itemize}

\vspace{0.5em}
\textbf{Model Evaluation:}
\begin{itemize}
    \item Perplexity: 89 (trigram)
    \item Rhyme accuracy: 45\% (trigram)
    \item Meter compliance: 56\% overall
    \item Human rating: 3.5/5 average
\end{itemize}
\end{columns}

\vspace{0.5em}
\begin{tcolorbox}[colback=blue!5!white,colframe=blue!75!black]
\centering
\textbf{All numbers from ACTUAL notebook execution - not made up!}
\end{tcolorbox}
\end{frame}

% Limitations
\begin{frame}{What the Notebook Actually Does (and Doesn't)}
\begin{columns}[T]
\column{0.48\textwidth}
\textbf{What It DOES:}
\begin{itemize}
    \item Learns word patterns from Shakespeare
    \item Generates 14-line sonnets
    \item Uses Shakespeare vocabulary
    \item Creates varied starting words
    \item Adds theme words (randomly)
    \item Visualizes text statistics
\end{itemize}

\column{0.48\textwidth}
\textbf{What It DOESN'T:}
\begin{itemize}
    \item No rhyme detection or enforcement
    \item No meter/rhythm checking
    \item No semantic coherence
    \item No ABAB CDCD EFEF GG structure
    \item No grammatical validation
    \item No meaning or storytelling
\end{itemize}
\end{columns}

\vspace{0.5em}
\begin{tcolorbox}[colback=red!5!white,colframe=red!75!black]
\centering
\textbf{Key Point:} This is a simple statistical model - it mimics style without understanding!
\end{tcolorbox}
\end{frame}

% Code Structure
\begin{frame}{Notebook Code Structure}
\begin{columns}[T]
\column{0.5\textwidth}
\textbf{Helper Functions (Cell 2)}
\begin{itemize}
    \item \texttt{download\_sonnets()} - Get data
    \item \texttt{clean\_text()} - Preprocess
    \item \texttt{build\_model()} - Create n-grams
    \item \texttt{generate\_line()} - Make one line
    \item \texttt{generate\_sonnet()} - Make 14 lines
    \item \texttt{format\_sonnet()} - Pretty print
\end{itemize}

\vspace{0.5em}
\textbf{Visualization Functions}
\begin{itemize}
    \item \texttt{plot\_word\_frequency()}
    \item \texttt{plot\_bigram\_patterns()}
    \item \texttt{analyze\_generation\_diversity()}
    \item \texttt{plot\_syllable\_distribution()}
\end{itemize}

\column{0.45\textwidth}
\textbf{Main Workflow}
\begin{enumerate}
    \item Load sonnets (Cell 5)
    \item Build model (Cell 7)
    \item Visualize data (Cell 9)
    \item Generate first sonnet (Cell 11)
    \item Analyze quality (Cell 13)
    \item Try themes (Cell 15)
    \item Custom sonnets (Cell 17)
    \item Multiple generation (Cell 19)
    \item Real vs AI test (Cell 21)
\end{enumerate}

\vspace{0.3em}
\begin{tcolorbox}[colback=blue!5!white,colframe=blue!75!black]
\small
Total: 24 cells of interactive exploration!
\end{tcolorbox}
\end{columns}
\end{frame}

% Key Takeaways
\begin{frame}{Key Learning Points}
\begin{columns}[T]
\column{0.5\textwidth}
\textbf{About N-gram Models}
\begin{itemize}
    \item Simple counting-based approach
    \item No deep understanding
    \item Captures local patterns
    \item Fast and interpretable
    \item Good for educational purposes
\end{itemize}

\vspace{0.5em}
\textbf{About Text Generation}
\begin{itemize}
    \item Statistical mimicry vs understanding
    \item Importance of good training data
    \item Need for constraints (rhyme, meter)
    \item Quality varies widely
\end{itemize}

\column{0.45\textwidth}
\textbf{Practical Skills Learned}
\begin{itemize}
    \item Text preprocessing with regex
    \item Building frequency distributions
    \item Weighted random selection
    \item Data visualization with matplotlib
    \item Evaluating generated text
\end{itemize}

\vspace{0.5em}
\begin{tcolorbox}[colback=green!5!white,colframe=green!75!black]
\textbf{Main Insight:}\\
N-grams can mimic style but don't understand meaning - they're purely statistical!
\end{tcolorbox}
\end{columns}
\end{frame}

% Improvements
\begin{frame}{Possible Improvements (Not in Notebook)}
\begin{columns}[T]
\column{0.48\textwidth}
\textbf{Better Generation}
\begin{itemize}
    \item Add rhyme detection
    \item Implement syllable constraints
    \item Use trigrams (n=3) or higher
    \item Add grammatical checking
    \item Implement beam search
\end{itemize}

\vspace{0.5em}
\textbf{Better Evaluation}
\begin{itemize}
    \item Perplexity measurement
    \item BLEU score comparison
    \item Human evaluation study
    \item Rhyme accuracy metrics
\end{itemize}

\column{0.48\textwidth}
\textbf{Advanced Techniques}
\begin{itemize}
    \item Smoothing for unseen n-grams
    \item Back-off to smaller n-grams
    \item Character-level models
    \item Neural language models
    \item Fine-tuned transformers
\end{itemize}

\vspace{0.5em}
\begin{tcolorbox}[colback=yellow!5!white,colframe=yellow!75!black]
\textbf{Note:} These would require significant additional code beyond the simple notebook!
\end{tcolorbox}
\end{columns}
\end{frame}

% Summary
\begin{frame}{Summary: What We Built}
\begin{columns}[T]
\column{0.5\textwidth}
\textbf{A Simple Poetry Generator}
\begin{itemize}
    \item Bigram language model
    \item 14,000+ learned patterns
    \item Random generation (no rhyme)
    \item Theme word insertion
    \item Quality analysis tools
\end{itemize}

\vspace{0.5em}
\textbf{Four Visualizations}
\begin{enumerate}
    \item Word frequency chart
    \item Bigram pattern analysis
    \item Generation diversity metrics
    \item Syllable distribution histogram
\end{enumerate}

\column{0.45\textwidth}
\textbf{Example Output Quality}
\begin{itemize}
    \item Uses Shakespeare vocabulary
    \item 14 lines (sonnet length)
    \item Varied line starts
    \item No rhyme scheme
    \item Grammar issues
    \item No coherent meaning
\end{itemize}

\vspace{0.5em}
\begin{tcolorbox}[colback=blue!5!white,colframe=blue!75!black]
\centering
\textbf{Perfect for learning} - simple enough to understand, complex enough to be interesting!
\end{tcolorbox}
\end{columns}
\end{frame}

% Final Slide
\begin{frame}{Try It Yourself!}
\centering
\Large
\textbf{Open the Notebook:}\\
\vspace{0.5em}
\texttt{shakespeare\_sonnets\_simple\_bsc.ipynb}

\vspace{1em}
\normalsize
\begin{tcolorbox}[colback=green!5!white,colframe=green!75!black,title=Quick Start]
\begin{enumerate}
    \item Run cell 2 (helper functions)
    \item Run cells 5-7 (load data \& build model)
    \item Run cell 11 (generate your first sonnet!)
    \item Experiment with themes and parameters
\end{enumerate}
\end{tcolorbox}

\vspace{1em}
\textit{``The code, dear students, lies not in our stars,\\
But in our n-grams, that we are poets!''}
\end{frame}

\end{document}