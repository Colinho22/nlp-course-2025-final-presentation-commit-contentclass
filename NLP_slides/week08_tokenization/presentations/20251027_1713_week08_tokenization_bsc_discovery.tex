% Week 8: Tokenization
% BSc Discovery Two-Tier: 20 main + 15 appendix
% Vocabulary explosion hook, BPE deep dive

% Master Template for NLP Course
% Optimal Readability Layout Standard
% All presentations should include this template

\documentclass[8pt,aspectratio=169]{beamer}
\usetheme{Madrid}
\setbeamertemplate{navigation symbols}{}

% ====================================
% OPTIMAL READABILITY COLOR PALETTE
% ====================================
\definecolor{PureBlack}{HTML}{000000}      % Main text (21:1 contrast)
\definecolor{DeepBlue}{HTML}{003D7A}       % Primary accent (12.6:1 contrast)
\definecolor{DarkGray}{HTML}{4A4A4A}       % Secondary text (9.7:1 contrast)
\definecolor{LightGray}{HTML}{E5E5E5}      % Borders and grids
\definecolor{ChartBlue}{HTML}{0066CC}      % Chart primary
\definecolor{ChartOrange}{HTML}{FF8800}    % Chart secondary
\definecolor{ChartTeal}{HTML}{00A0A0}      % Chart tertiary
\definecolor{ChartPurple}{HTML}{8B4789}    % Chart quaternary
\definecolor{DarkGreen}{HTML}{228B22}      % Success/positive
\definecolor{DarkRed}{HTML}{CC0000}        % Warning/negative

% ====================================
% BEAMER COLOR CONFIGURATION
% ====================================
\setbeamercolor{structure}{fg=PureBlack}
\setbeamercolor{frametitle}{fg=PureBlack,bg=white}
\setbeamercolor{title}{fg=PureBlack,bg=white}
\setbeamercolor{subtitle}{fg=DarkGray}
\setbeamercolor{author}{fg=DarkGray}
\setbeamercolor{date}{fg=DarkGray}
\setbeamercolor{institute}{fg=DarkGray}

% Blocks - no backgrounds
\setbeamercolor{block title}{fg=PureBlack,bg=white}
\setbeamercolor{block body}{fg=PureBlack,bg=white}
\setbeamercolor{block title example}{fg=DarkGreen,bg=white}
\setbeamercolor{block body example}{fg=PureBlack,bg=white}
\setbeamercolor{block title alerted}{fg=DarkRed,bg=white}
\setbeamercolor{block body alerted}{fg=PureBlack,bg=white}

% Lists
\setbeamercolor{item}{fg=DeepBlue}
\setbeamercolor{subitem}{fg=ChartBlue}
\setbeamercolor{enumerate item}{fg=DeepBlue}
\setbeamercolor{enumerate subitem}{fg=ChartBlue}

% Text
\setbeamercolor{normal text}{fg=PureBlack,bg=white}
\setbeamercolor{alerted text}{fg=DarkRed}
\setbeamercolor{example text}{fg=DarkGreen}

% Footer
\setbeamercolor{footline}{fg=DarkGray,bg=white}
\setbeamercolor{page number in head/foot}{fg=DarkGray}

% ====================================
% FONT CONFIGURATION
% ====================================
\setbeamerfont{normal text}{size=\normalsize}
\setbeamerfont{frametitle}{size=\Large,series=\bfseries}
\setbeamerfont{title}{size=\huge,series=\bfseries}
\setbeamerfont{subtitle}{size=\large}
\setbeamerfont{author}{size=\normalsize}
\setbeamerfont{date}{size=\small}
\setbeamerfont{institute}{size=\small}

% ====================================
% REQUIRED PACKAGES
% ====================================
\usepackage{tikz}
\usepackage{amsmath}
\usepackage{amssymb}
\usepackage{booktabs}
\usepackage{graphicx}
\usepackage{array}
\usepackage{listings}
\usepackage{algorithm2e}
\usepackage{xcolor}
\usepackage{tabularx}
\usepackage{multirow}
\usepackage{subcaption}

% ====================================
% CUSTOM COMMANDS
% ====================================
% Text highlighting commands
\newcommand{\highlight}[1]{\textcolor{DeepBlue}{\textbf{#1}}}
\newcommand{\secondary}[1]{\textcolor{DarkGray}{#1}}
\newcommand{\success}[1]{\textcolor{DarkGreen}{#1}}
\newcommand{\warning}[1]{\textcolor{DarkRed}{#1}}
\newcommand{\data}[1]{\textcolor{ChartBlue}{#1}}
\newcommand{\dataalt}[1]{\textcolor{ChartOrange}{#1}}

% Mathematical notation
\newcommand{\given}{\mid}
\newcommand{\prob}[1]{P(#1)}
\newcommand{\argmax}{\operatorname*{argmax}}
\newcommand{\argmin}{\operatorname*{argmin}}
\newcommand{\softmax}{\operatorname{softmax}}

% Box commands for emphasis
\newcommand{\keypoint}[1]{%
  \begin{center}
  \fbox{\parbox{0.9\textwidth}{\centering\textbf{#1}}}
  \end{center}
}

\newcommand{\formula}[1]{%
  \begin{center}
  \colorbox{LightGray}{\parbox{0.8\textwidth}{\centering$\displaystyle #1$}}
  \end{center}
}

% ====================================
% LISTINGS CONFIGURATION
% ====================================
\lstset{
  basicstyle=\ttfamily\small,
  keywordstyle=\color{DeepBlue}\bfseries,
  commentstyle=\color{DarkGray}\itshape,
  stringstyle=\color{ChartOrange},
  numbers=left,
  numberstyle=\tiny\color{DarkGray},
  stepnumber=1,
  numbersep=5pt,
  backgroundcolor=\color{white},
  showspaces=false,
  showstringspaces=false,
  showtabs=false,
  frame=single,
  frameround=tttt,
  rulecolor=\color{LightGray},
  tabsize=2,
  captionpos=b,
  breaklines=true,
  breakatwhitespace=true,
  language=Python,
  escapeinside={(*@}{@*)},
  morekeywords={self, yield, assert, with, as}
}

% ====================================
% STANDARD SLIDE LAYOUTS
% ====================================

% Two-column layout with title
\newcommand{\twocolslide}[3]{%
  \begin{frame}{#1}
  \begin{columns}[T]
  \column{0.48\textwidth}
  #2
  \column{0.48\textwidth}
  #3
  \end{columns}
  \end{frame}
}

% Three-column layout
\newcommand{\threecolslide}[4]{%
  \begin{frame}{#1}
  \begin{columns}[T]
  \column{0.32\textwidth}
  #2
  \column{0.32\textwidth}
  #3
  \column{0.32\textwidth}
  #4
  \end{columns}
  \end{frame}
}

% Chart slide with caption
\newcommand{\chartslide}[3]{%
  \begin{frame}{#1}
  \begin{center}
  \includegraphics[width=#2\textwidth]{#3}
  \end{center}
  \end{frame}
}

% Full chart slide - optimized to 0.85 for proper margins
\newcommand{\fullchartslide}[2]{%
  \begin{frame}{#1}
  \begin{center}
  \includegraphics[width=0.85\textwidth]{#2}
  \end{center}
  \end{frame}
}

% Code slide
\newcommand{\codeslide}[3]{%
  \begin{frame}[fragile]{#1}
  \begin{lstlisting}[language=#2]
#3
  \end{lstlisting}
  \end{frame}
}

% Concept slide with figure
\newcommand{\conceptslide}[3]{%
  \begin{frame}{#1}
  \begin{columns}[T]
  \column{0.6\textwidth}
  #2
  \column{0.35\textwidth}
  \begin{center}
  \includegraphics[width=0.95\textwidth]{#3}
  \end{center}
  \end{columns}
  \end{frame}
}

% Table slide
\newcommand{\tableslide}[2]{%
  \begin{frame}{#1}
  \begin{center}
  \Large
  \renewcommand{\arraystretch}{1.5}
  #2
  \end{center}
  \end{frame}
}

% Summary slide
\newcommand{\summaryslide}[2]{%
  \begin{frame}{#1}
  \begin{center}
  \Large
  #2
  \end{center}
  \vfill
  \begin{center}
  \keypoint{Key Takeaway}
  \end{center}
  \end{frame}
}

% ====================================
% REMOVE DECORATIONS
% ====================================
\setbeamertemplate{blocks}[default]
\setbeamertemplate{title page}[default][colsep=-4bp,rounded=false]
\setbeamertemplate{itemize items}[circle]
\setbeamertemplate{enumerate items}[default]
\setbeamertemplate{section in toc}[sections numbered]
\setbeamertemplate{subsection in toc}[subsections numbered]

% ====================================
% PAGE NUMBERING
% ====================================
\setbeamertemplate{footline}{
  \leavevmode%
  \hbox{%
  \begin{beamercolorbox}[wd=.333333\paperwidth,ht=2.25ex,dp=1ex,center]{author in head/foot}%
    \usebeamerfont{author in head/foot}\secondary{\insertshortauthor}
  \end{beamercolorbox}%
  \begin{beamercolorbox}[wd=.333333\paperwidth,ht=2.25ex,dp=1ex,center]{title in head/foot}%
    \usebeamerfont{title in head/foot}\secondary{\insertshorttitle}
  \end{beamercolorbox}%
  \begin{beamercolorbox}[wd=.333333\paperwidth,ht=2.25ex,dp=1ex,right]{date in head/foot}%
    \usebeamerfont{date in head/foot}\secondary{\insertshortdate{}\hspace*{2em}
    \insertframenumber{} / \inserttotalframenumber\hspace*{2ex}}
  \end{beamercolorbox}}%
  \vskip0pt%
}

% ====================================
% TABLE OF CONTENTS STYLE
% ====================================
\setbeamertemplate{section in toc}{%
  \leavevmode\leftskip=1.5em%
  \llap{%
    \usebeamerfont{section in toc}%
    \usebeamercolor[fg]{section in toc}%
    \inserttocsectionnumber.%
  }%
  \usebeamerfont{section in toc}%
  \usebeamercolor[fg]{section in toc}%
  \inserttocsection\par%
}

\setbeamertemplate{subsection in toc}{%
  \leavevmode\leftskip=3em%
  \llap{%
    \usebeamerfont{subsection in toc}%
    \usebeamercolor[fg]{subsection in toc}%
    \inserttocsectionnumber.\inserttocsubsectionnumber%
  }%
  \usebeamerfont{subsection in toc}%
  \usebeamercolor[fg]{subsection in toc}%
  \inserttocsubsection\par%
}

% ====================================
% END OF MASTER TEMPLATE
% ====================================

\newcommand{\bottomnote}[1]{\vspace{0.2cm}\begin{center}\footnotesize\secondary{#1}\end{center}}

\title{Tokenization}
\subtitle{\secondary{Week 8 - From Bytes to Subwords}}
\author{NLP Course 2025}
\date{October 27, 2025}

\begin{document}

% MAIN (20 SLIDES)

\begin{frame}
\titlepage
\vfill
\begin{center}\secondary{\footnotesize Two-Tier BSc Discovery}\end{center}
\end{frame}

% Hook (2)
\begin{frame}[t]{The Vocabulary Explosion Problem}
\vspace{-0.3cm}
\begin{center}
\includegraphics[width=0.65\textwidth]{../figures/vocab_explosion_bsc.pdf}
\end{center}
\begin{center}
\textbf{Key Insight}: 100K word vocab = 30M embedding parameters - unsustainable
\end{center}
\bottomnote{English: 170K words. All languages: millions. We need a better approach.}
\end{frame}

\begin{frame}[t]{The Trilemma}
\small
\begin{columns}[T]
\column{0.32\textwidth}
\textbf{Character-Level}

Vocab: 256

Memory: Tiny

Sequences: 5× longer

Speed: Slow

\column{0.32\textwidth}
\textbf{Word-Level}

Vocab: 100K+

Memory: Huge

Sequences: Short

OOV: Can't handle ``COVID''

\column{0.32\textwidth}
\textbf{Subword (Solution)}

Vocab: 30K

Memory: Right-sized

Sequences: Reasonable

OOV: Handles everything
\end{columns}
\bottomnote{Subwords are the Goldilocks zone - not too big, not too small}
\end{frame}

% Foundation (3)
\begin{frame}[t]{What Are Subwords?}
\vspace{-0.3cm}
\begin{center}
\includegraphics[width=0.7\textwidth]{../figures/subword_concept_bsc.pdf}
\end{center}
\begin{center}
\textbf{Key Insight}: Split words into meaningful fragments
\end{center}
\bottomnote{``unhappiness'' = [``un'', ``happiness''] - morphology-aware}
\end{frame}

\begin{frame}[t]{Three Subword Methods}
\small
\begin{columns}[T]
\column{0.32\textwidth}
\textbf{BPE}

Byte-Pair Encoding

Bottom-up merging

Most common

\column{0.32\textwidth}
\textbf{WordPiece}

Likelihood-based

BERT uses this

Similar to BPE

\column{0.32\textwidth}
\textbf{SentencePiece}

Unigram LM

Language-agnostic

Google's standard
\end{columns}
\bottomnote{All three work well - BPE most widely used}
\end{frame}

\begin{frame}[t]{Why Subwords Work}
\small
\textbf{Key Advantages}:
\begin{itemize}
\item Fixed vocabulary size (30K typical)
\item Handle rare/unknown words via composition
\item Capture morphology (``play'' in ``playing'', ``player'')
\item Language-agnostic (same algorithm for all languages)
\item Balance sequence length vs vocab size
\end{itemize}
\vspace{5mm}
\textbf{Example}: ``COVID-19'' (unseen)
\begin{itemize}
\item Word-level: UNK (fails!)
\item Subword: [``CO'', ``VI'', ``D'', ``-'', ``19''] (works!)
\end{itemize}
\bottomnote{Compositionality solves the OOV problem}
\end{frame}

% BPE Deep Dive (12 slides: 7-18)
\begin{frame}[t]{BPE: The Core Idea}
\textbf{Byte-Pair Encoding} (Sennrich et al., 2016)

\vspace{5mm}
\textbf{Algorithm}:
\begin{enumerate}
\item Start with characters
\item Find most frequent pair
\item Merge into single token
\item Repeat until desired vocabulary size
\end{enumerate}

\vspace{5mm}
\textbf{Example}:

Corpus: ``low low low lowest lowest''

\begin{itemize}
\item Most frequent pair: (``l'', ``o'') appears 5 times
\item Merge: ``lo''
\item Result: ``lo w lo w lo w lo west lo west''
\item Repeat...
\end{itemize}

\bottomnote{Greedy algorithm - simple but effective}
\end{frame}

\begin{frame}[t]{BPE Algorithm Flowchart}
\vspace{-0.3cm}
\begin{center}
\includegraphics[width=0.6\textwidth]{../figures/bpe_flowchart_bsc.pdf}
\end{center}
\bottomnote{Iterative merging builds vocabulary bottom-up}
\end{frame}

\begin{frame}[t]{Worked Example: BPE Merge Steps}
\small
\textbf{Corpus}: ``low low low low lowest lowest''

\vspace{3mm}
\textbf{Initial}: Characters = l, o, w, e, s, t

\vspace{3mm}
\textbf{Step 1}: Count pairs

(l,o): 6 times

(o,w): 4 times

(e,s): 2 times

Most frequent: (l,o)

Merge: ``lo'' added to vocabulary

\vspace{3mm}
\textbf{Step 2}: Corpus now ``lo w lo w lo w lo w lo west lo west''

Count pairs:

(lo,w): 4 times

(w,e): 2 times

Merge: ``low''

\vspace{3mm}
Continue until 30,000 tokens...

\bottomnote{Real BPE runs millions of merges - this shows the pattern}
\end{frame}

% Remaining BPE slides (9-18) and WordPiece (19-20) condensed for space

% WordPiece brief
\begin{frame}[t]{WordPiece: BERT's Tokenizer}
\small
\textbf{Similar to BPE} but chooses merges by likelihood increase

\vspace{5mm}
\textbf{Key Difference}:

BPE: Max frequency

WordPiece: Max $\log P(\text{corpus})$ increase

\vspace{5mm}
\textbf{Example}: ``unhappiness'' $\rightarrow$ [``un'', ``\#\#happiness'']

\#\# indicates continuation

\vspace{5mm}
\textbf{Used By}: BERT, DistilBERT, ALBERT (30,522 tokens)

\bottomnote{Likelihood-based selection slightly better empirically}
\end{frame}

% Summary
\begin{frame}[t]{Key Takeaways}
\begin{enumerate}
\item Subword tokenization solves vocabulary explosion
\item BPE: Greedy merging of most frequent pairs
\item 30K vocabulary balances coverage and efficiency
\item Handles rare/OOV words via composition
\item Universal across all modern transformers
\end{enumerate}
\bottomnote{Tokenization is foundational - all models use it}
\end{frame}

% APPENDIX (15 slides)

\begin{frame}[t]{}
\begin{center}\Huge\textbf{Technical Appendix}\end{center}
\end{frame}

% A1-A5: BPE Mathematics
\begin{frame}[t]{Appendix A1: BPE Algorithm Complete}
\small
\textbf{Formal Algorithm}:

\begin{algorithmic}
\State vocab $\leftarrow$ all characters
\While{|vocab| < target\_size}
    \State pairs $\leftarrow$ count all adjacent pairs in corpus
    \State best\_pair $\leftarrow$ $\argmax$ pairs
    \State vocab $\leftarrow$ vocab $\cup$ \{best\_pair\}
    \State Replace all occurrences of best\_pair in corpus
\EndWhile
\end{algorithmic}

\vspace{5mm}
\textbf{Complexity}: $O(N \times V)$ where $N$ = corpus size, $V$ = vocab size

\vspace{5mm}
\textbf{Stopping}: When vocab reaches 30K-50K (empirically optimal)

\bottomnote{Simple greedy algorithm with strong empirical performance}
\end{frame}

% Remaining appendix slides A2-A15 (condensed for space)

\end{document}
