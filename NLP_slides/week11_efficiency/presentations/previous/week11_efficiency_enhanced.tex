\documentclass[8pt,aspectratio=169,8pt]{beamer}
\usepackage[utf8]{inputenc}
\usepackage{graphicx}
\usepackage{amsmath,amssymb}
\usepackage{algorithm2e}
\usepackage{listings}
\usepackage{xcolor}
\usepackage{tikz}
\usepackage{pgfplots}
\pgfplotsset{compat=1.17}
\usepackage{subfigure}
\usepackage{hyperref}

% Theme settings
\usetheme{Frankfurt}
\usecolortheme{seahorse}
\setbeamertemplate{navigation symbols}{}
\setbeamertemplate{footline}[frame number]

% Custom colors
\definecolor{darkblue}{RGB}{0,51,102}
\definecolor{lightblue}{RGB}{173,216,230}
\definecolor{codegreen}{RGB}{0,128,0}
\definecolor{codegray}{RGB}{150,150,150}
\definecolor{codepurple}{RGB}{128,0,128}
\definecolor{backcolor}{RGB}{245,245,245}

% Code listing settings
\lstset{
    backgroundcolor=\color{backcolor},
    basicstyle=\ttfamily\footnotesize,
    breakatwhitespace=false,
    breaklines=true,
    captionpos=b,
    commentstyle=\color{codegreen},
    keywordstyle=\color{blue},
    numberstyle=\tiny\color{codegray},
    stringstyle=\color{codepurple},
    showstringspaces=false,
    frame=single,
    numbers=left,
    language=Python
}

% Include slide layouts
% SLIDE LAYOUT TEMPLATES FOR NLP COURSE
% Three consistent layouts used throughout the course

% ==============================================================
% LAYOUT 1: CONCEPT SLIDE
% Used for: Theory, definitions, mathematical concepts
% Features: Title, main content area with bullets/equations, optional figure
% ==============================================================

\newcommand{\conceptslide}[3]{ % #1: title, #2: content, #3: optional figure
\begin{frame}[t]{#1}
    \begin{columns}[T]
        \begin{column}{0.6\textwidth}
            #2
        \end{column}
        \begin{column}{0.35\textwidth}
            \centering
            #3
        \end{column}
    \end{columns}
\end{frame}
}

% ==============================================================
% LAYOUT 2: CODE & IMPLEMENTATION SLIDE
% Used for: Code examples, algorithms, implementation details
% Features: Title, code block, explanation text
% ==============================================================

% Note: For code slides, use \begin{frame}[fragile] manually

\newcommand{\codeexplanation}[1]{%
    \vspace{0.5em}
    \small
    #1
}

\newcommand{\codeblock}[2][Python]{%
    \begin{lstlisting}[language=#1, basicstyle=\ttfamily\tiny]
#2
    \end{lstlisting}
}

\newcommand{\codeslide}[3]{%
    \begin{frame}[fragile]{#1}
    \begin{columns}[T]
        \column{0.55\textwidth}
        \codeblock{#2}
        \column{0.43\textwidth}
        \codeexplanation{#3}
    \end{columns}
    \end{frame}
}

% ==============================================================
% LAYOUT 3: RESULTS & VISUALIZATION SLIDE
% Used for: Experimental results, charts, comparisons
% Features: Title, large visualization area, key insights
% ==============================================================

\newcommand{\resultslide}[3]{ % #1: title, #2: visualization, #3: insights
\begin{frame}[t]{#1}
    \centering
    \vspace{-0.5em}
    #2
    \vspace{0.5em}
    
    \begin{block}{Key Insights}
        #3
    \end{block}
\end{frame}
}

% Additional utility commands for consistency

% Highlighted text
\newcommand{\highlight}[1]{\textcolor{blue}{\textbf{#1}}}

% Math notation shortcuts
\newcommand{\prob}[1]{P(#1)}
\newcommand{\given}{\mid}
\newcommand{\argmax}{\operatorname{argmax}}
\newcommand{\softmax}{\operatorname{softmax}}

% Standard figure size
\newcommand{\stdfigsizeinchins}{0.4\textwidth}

% For code blocks, use lstlisting directly:
% \begin{lstlisting}[language=Python]
% ... code ...
% \end{lstlisting}

% Equation highlight box
\newcommand{\eqbox}[1]{
    \begin{center}
    \colorbox{lightblue!20}{
        \parbox{0.8\textwidth}{
            \vspace{0.3em}
            \centering
            #1
            \vspace{0.3em}
        }
    }
    \end{center}
}

\title[Week 11: Efficiency]{Natural Language Processing Course}
\subtitle{Week 11: Efficiency and Deployment}
\author{Joerg R. Osterrieder \\ \url{www.joergosterrieder.com}}
\date{}

\begin{document}

% Title page
\begin{frame}
    \titlepage
\end{frame}

\section{Week 11: Efficiency and Deployment}

% Title slide
\begin{frame}
    \centering
    \vspace{2cm}
    {\Large \textbf{Week 11}}\\
    \vspace{0.5cm}
    {\huge \textbf{Efficiency \& Deployment}}\\
    \vspace{1cm}
    {\large From Cloud Giants to Pocket-Sized Models}
\end{frame}

% Motivation: The deployment crisis
\begin{frame}[t]{Why Your Phone Can't Run GPT-4 (Yet)}
    \textbf{The shocking reality of modern models:}
    
    \vspace{0.5em}
    \begin{itemize}
        \item GPT-3: 175B parameters = 350GB in FP16\footnotemark
        \item Your phone: 6GB RAM
        \item Inference cost: \$0.06 per 1K tokens
        \item Latency: 100ms+ per token
        \item Energy: 0.1 kWh per conversation
    \end{itemize}
    
    \vspace{0.5em}
    \begin{center}
    \colorbox{red!20}{
        \parbox{0.8\textwidth}{
            \centering
            One ChatGPT query = 10x the energy of a Google search
        }
    }
    \end{center}
    
    \vspace{0.5em}
    \textbf{The challenge:}
    \begin{itemize}
        \item Users want instant responses
        \item Privacy requires on-device processing
        \item Costs are unsustainable at scale
        \item Edge devices have limited resources
    \end{itemize}
    
    \vspace{0.5em}
    \textit{This week: How to shrink giants into pocket-sized assistants}
    
    \footnotetext{Brown et al. (2020); Strubell et al. (2019) on model carbon footprint}
\end{frame}

% The efficiency landscape
\begin{frame}[t]{From Supercomputers to Smartphones: The Deployment Journey}
    \centering
    \includegraphics[width=0.9\textwidth]{../figures/model_compression_landscape.pdf}
    
    \vspace{0.5em}
    \textbf{The compression toolkit:}
    \begin{itemize}
        \item Quantization: 32-bit → 8-bit (4x smaller)
        \item Distillation: Teacher → Student (10x smaller)
        \item Pruning: Remove redundant weights (2-5x)
        \item Mobile architectures: Designed for edge
    \end{itemize}
\end{frame}

% Real-world impact
\begin{frame}[t]{Efficient Models in Production (2024)}
    \begin{columns}[T]
        \begin{column}{0.5\textwidth}
            \textbf{Success Stories:}
            \begin{itemize}
                \item iPhone: On-device Siri (5B → 200M)\footnotemark
                \item Google: Pixel voice typing
                \item Microsoft: Excel formula suggestions
                \item WhatsApp: Real-time translation
                \item Tesla: In-car voice commands
            \end{itemize}
            
            \vspace{0.5em}
            \textbf{Business Impact:}
            \begin{itemize}
                \item 100x cost reduction
                \item 10ms latency (from 1s)
                \item Works offline
                \item Privacy preserved
                \item Scales to billions
            \end{itemize}
        \end{column}
        \begin{column}{0.5\textwidth}
            \textbf{Efficiency Techniques:}
            \begin{itemize}
                \item INT8 quantization: Standard
                \item INT4/Binary: Emerging
                \item Distillation: 95\% performance
                \item Structured pruning: Hardware-friendly
                \item Flash attention: Memory efficient
            \end{itemize}
            
            \vspace{0.5em}
            \textbf{Deployment Targets:}
            \begin{itemize}
                \item Mobile phones: 2-4GB limit
                \item Browsers: WebAssembly
                \item IoT devices: MCUs
                \item Edge servers: Local processing
                \item Specialized chips: NPUs/TPUs
            \end{itemize}
        \end{column}
    \end{columns}
    
    \vspace{0.5em}
    \begin{center}
    \colorbox{lightblue!30}{
        \parbox{0.8\textwidth}{
            \centering
            2024: Every device runs neural networks - efficiency made it possible
        }
    }
    \end{center}
    
    \footnotetext{Apple ML Research Blog (2023); Google I/O presentations}
\end{frame}

% Learning objectives
\begin{frame}[t]{Week 11: What You'll Master}
    \textbf{By the end of this week, you will:}
    \begin{itemize}
        \item \textbf{Understand} why models are so large
        \item \textbf{Implement} quantization techniques
        \item \textbf{Master} knowledge distillation
        \item \textbf{Apply} pruning strategies
        \item \textbf{Deploy} models to edge devices
    \end{itemize}
    
    \vspace{0.5em}
    \begin{center}
    \colorbox{lightblue!30}{
        \parbox{0.8\textwidth}{
            \centering
            \textbf{Core Insight:} 90\% of weights do 10\% of the work
        }
    }
    \end{center}
\end{frame}

% Why models are large
\begin{frame}[t]{Why Are Models So Large? The Overparameterization Mystery}
    \textbf{The paradox:}
    
    Models have way more parameters than necessary!
    
    \vspace{0.5em}
    \textbf{Evidence:}
    \begin{itemize}
        \item Lottery Ticket Hypothesis: Small subnetworks work just as well\footnotemark
        \item Magnitude pruning: Remove 90\% weights, maintain accuracy
        \item Low-rank decomposition: Matrices are redundant
        \item Quantization: 32 bits → 4 bits still works
    \end{itemize}
    
    \vspace{0.5em}
    \textbf{Why overparameterize?}
    \begin{itemize}
        \item Easier optimization landscape
        \item Better generalization (surprisingly!)
        \item Redundancy provides robustness
        \item Training dynamics require it
    \end{itemize}
    
    \vspace{0.5em}
    \begin{center}
    \colorbox{yellow!20}{
        \parbox{0.8\textwidth}{
            \centering
            Large models are like rough marble blocks - we can carve out efficient versions
        }
    }
    \end{center}
    
    \footnotetext{Frankle \& Carbin (2019) "The Lottery Ticket Hypothesis"}
\end{frame}

% Quantization overview
\begin{frame}[t]{Quantization: From Float32 to Int8 and Beyond}
    \centering
 %   \includegraphics[width=0.6  5\textwidth]{  ../figures/quantization_levels.pdf}
    
    \vspace{0.5em}
    \textbf{The quantization spectrum:}
    \begin{itemize}
        \item FP32 (32 bits): Training precision
        \item FP16 (16 bits): Mixed precision training
        \item INT8 (8 bits): Standard deployment
        \item INT4 (4 bits): Aggressive compression
        \item Binary (1 bit): Research frontier
    \end{itemize}
\end{frame}

% Quantization implementation
\begin{frame}[fragile]{Implementing Post-Training Quantization}
    \begin{columns}[T]
        \column{0.55\textwidth}
\begin{lstlisting}[language=Python,basicstyle=\ttfamily\tiny]
import torch
import torch.nn as nn

def quantize_tensor(x, num_bits=8):
    """Quantize tensor to n bits"""
    qmin = -(2**(num_bits-1))
    qmax = 2**(num_bits-1) - 1
    
    min_val = x.min()
    max_val = x.max()
    
    scale = (max_val - min_val) / (qmax - qmin)
    zero_point = qmin - min_val / scale
    
    q_x = torch.round(x / scale + zero_point)
    q_x = torch.clamp(q_x, qmin, qmax)
    
    return q_x, scale, zero_point

def dequantize_tensor(q_x, scale, zero_point):
    """Dequantize back to float"""
    return scale * (q_x - zero_point)

class QuantizedLinear(nn.Module):
    def __init__(self, weight, bias, num_bits=8):
        super().__init__()
        
        self.q_weight, self.w_scale, self.w_zp = quantize_tensor(weight, num_bits)
        self.q_weight = self.q_weight.to(torch.int8)
        
        if bias is not None:
            self.q_bias, self.b_scale, self.b_zp = quantize_tensor(bias, num_bits)
        else:
            self.q_bias = None
            
        self.num_bits = num_bits
        
    def forward(self, x):
        w_float = dequantize_tensor(self.q_weight.float(), 
                                   self.w_scale, self.w_zp)
        out = F.linear(x, w_float)
        
        if self.q_bias is not None:
            b_float = dequantize_tensor(self.q_bias.float(),
                                       self.b_scale, self.b_zp)
            out += b_float
            
        return out

def quantize_model(model, num_bits=8):
    """Quantize entire model"""
    for name, module in model.named_modules():
        if isinstance(module, nn.Linear):
            quantized = QuantizedLinear(
                module.weight.data,
                module.bias.data if module.bias is not None else None,
                num_bits
            )
            parent_name = name.rsplit('.', 1)[0] if '.' in name else ''
            child_name = name.rsplit('.', 1)[1] if '.' in name else name
            parent = model.get_submodule(parent_name) if parent_name else model
            setattr(parent, child_name, quantized)
    
    return model
\end{lstlisting}
        \column{0.43\textwidth}
        \codeexplanation{
            \textbf{Quantization Impact:}
            \begin{itemize}
                \item Memory: 4x reduction (INT8)
                \item Speed: 2-4x on CPUs
                \item Accuracy: -1\% typical
                \item Energy: 10x savings
            \end{itemize}
            
            \vspace{0.3em}
            \textbf{Advanced Techniques:}
            \begin{itemize}
                \item Symmetric vs asymmetric
                \item Per-channel quantization
                \item Quantization-aware training
                \item Mixed bit-width
            \end{itemize}
            
            \vspace{0.3em}
            \textbf{Hardware Support:}
            \begin{itemize}
                \item INT8: All modern chips
                \item INT4: Newer GPUs/NPUs
                \item Binary: Research only
            \end{itemize}
        }
    \end{columns}
\end{frame}

% Knowledge distillation
\begin{frame}[t]{Knowledge Distillation: Teaching Small Models to Think Big}
    \textbf{The teacher-student paradigm:}\footnotemark
    
    \vspace{0.5em}
    \centering
    \includegraphics[width=0.6\textwidth]{../figures/knowledge_distillation.pdf}
    
    \vspace{0.5em}
    \textbf{Key insight: Soft labels contain more information}
    \begin{itemize}
        \item Hard label: "cat" (probability = 1.0)
        \item Soft labels: cat=0.9, tiger=0.05, dog=0.03, ...
        \item Student learns relationships between classes
        \item Mimics teacher's reasoning, not just answers
    \end{itemize}
    
    \footnotetext{Hinton et al. (2015) "Distilling the Knowledge in a Neural Network"}
\end{frame}

% Distillation implementation
\begin{frame}[fragile]{Implementing Knowledge Distillation}
    \begin{columns}[T]
                \column{0.65\textwidth}

\begin{lstlisting}[language=Python,basicstyle=\ttfamily\tiny]
import torch
import torch.nn as nn
import torch.nn.functional as F

class DistillationLoss(nn.Module):
    def __init__(self, temperature=3.0, alpha=0.7):
        """
        temperature: Softens probability distributions
        alpha: Weight between distillation and true label loss
        """
        super().__init__()
        self.temperature = temperature
        self.alpha = alpha
        self.ce_loss = nn.CrossEntropyLoss()
        
    def forward(self, student_logits, teacher_logits, labels):
        soft_targets = F.softmax(teacher_logits / self.temperature, dim=-1)
        soft_prob = F.log_softmax(student_logits / self.temperature, dim=-1)
        
        distillation_loss = F.kl_div(soft_prob, soft_targets, 
                                     reduction='batchmean') * (self.temperature ** 2)
        
        student_loss = self.ce_loss(student_logits, labels)
        
        loss = self.alpha * distillation_loss + (1 - self.alpha) * student_loss
        return loss

def train_student(student_model, teacher_model, dataloader, 
                  epochs=10, temperature=3.0):
    """Train student to mimic teacher"""
    teacher_model.eval()
    optimizer = torch.optim.Adam(student_model.parameters(), lr=1e-3)
    criterion = DistillationLoss(temperature=temperature)
    
    for epoch in range(epochs):
        for batch in dataloader:
            inputs, labels = batch
            
            with torch.no_grad():
                teacher_outputs = teacher_model(inputs)
            
            student_outputs = student_model(inputs)
            
            loss = criterion(student_outputs, teacher_outputs, labels)
            
            optimizer.zero_grad()
            loss.backward()
            optimizer.step()
            
        print(f"Epoch {epoch}: Loss = {loss.item():.4f}")
    
    return student_model

class TinyBERT(nn.Module):
    """Example: 6-layer student from 12-layer teacher"""
    def __init__(self, teacher_config):
        super().__init__()
        self.hidden_size = teacher_config.hidden_size // 2
        self.num_layers = 6  
        self.num_heads = teacher_config.num_heads // 2
        
        self.embeddings = nn.Embedding(teacher_config.vocab_size, 
                                      self.hidden_size)
        self.encoder = nn.TransformerEncoder(
            nn.TransformerEncoderLayer(
                d_model=self.hidden_size,
                nhead=self.num_heads,
                dim_feedforward=self.hidden_size * 4
            ),
            num_layers=self.num_layers
        )
        
        self.projection = nn.Linear(self.hidden_size, 
                                   teacher_config.vocab_size)
\end{lstlisting}
                \column{0.35\textwidth}

        \codeexplanation{
            \textbf{Distillation Results:}
            \begin{itemize}
                \item Size: 10-100x smaller
                \item Speed: 5-50x faster
                \item Accuracy: 95-98\% retained
                \item Examples: DistilBERT, TinyBERT
            \end{itemize}
            
            \vspace{0.3em}
            \textbf{Temperature Effect:}
            \begin{itemize}
                \item T=1: Normal softmax
                \item T=3-5: Typical for NLP
                \item T$>$10: Very soft labels
                \item Higher T = more dark knowledge
            \end{itemize}
            
            \vspace{0.3em}
            \textbf{Advanced Methods:}
            \begin{itemize}
                \item Feature distillation
                \item Attention transfer
                \item Progressive distillation
            \end{itemize}
        }
    \end{columns}
\end{frame}

% Pruning strategies
\begin{frame}[t]{Pruning: Finding the Essential Subnetwork}
    \centering
    \includegraphics[width=0.65\textwidth]{../figures/pruning_strategies.pdf}
    
    \vspace{0.5em}
    \textbf{Pruning approaches:}
    \begin{itemize}
        \item Magnitude pruning: Remove small weights
        \item Structured pruning: Remove entire channels/heads
        \item Movement pruning: Track weight importance during training
        \item Lottery tickets: Find winning subnetworks
    \end{itemize}
\end{frame}

% Mobile architectures
\begin{frame}[t]{Mobile-First Architectures: Designed for Efficiency}
    \begin{columns}[T]
        \begin{column}{0.5\textwidth}
            \textbf{MobileBERT:}\footnotemark
            \begin{itemize}
                \item Bottleneck structure
                \item 4.3x smaller, 5.5x faster
                \item Depth-wise convolutions
                \item Progressive knowledge transfer
            \end{itemize}
            
            \vspace{0.5em}
            \textbf{ALBERT:}
            \begin{itemize}
                \item Parameter sharing
                \item Factorized embeddings
                \item 18x fewer parameters
                \item Same performance as BERT
            \end{itemize}
        \end{column}
        \begin{column}{0.5\textwidth}
            \textbf{EdgeBERT:}
            \begin{itemize}
                \item Hardware-aware design
                \item Adaptive computation
                \item Early exit mechanisms
                \item Dynamic depth/width
            \end{itemize}
            
            \vspace{0.5em}
            \textbf{Design Principles:}
            \begin{itemize}
                \item Prefer depth over width
                \item Share parameters
                \item Use separable operations
                \item Minimize memory access
            \end{itemize}
        \end{column}
    \end{columns}
    
    \vspace{0.5em}
    \begin{center}
    \colorbox{green!20}{
        \parbox{0.8\textwidth}{
            \centering
            Mobile architectures: Not compressed models, but efficient by design
        }
    }
    \end{center}
    
    \footnotetext{Sun et al. (2020) "MobileBERT"; Lan et al. (2020) "ALBERT"}
\end{frame}

% Deployment optimization
\resultslide{Deployment Optimization Pipeline}{
    \centering
    \includegraphics[width=0.65\textwidth]{../figures/deployment_pipeline.pdf}
}{
    \begin{itemize}
        \item Start with pretrained model
        \item Apply compression techniques
        \item Optimize for target hardware
        \item Profile and iterate
        \item Deploy with runtime optimizations
    \end{itemize}
}

% Hardware acceleration
\begin{frame}[t]{Hardware Acceleration: From GPUs to Edge TPUs}
    \begin{columns}[T]
        \begin{column}{0.5\textwidth}
            \textbf{Compute Platforms:}
            \begin{itemize}
                \item GPUs: Training \& cloud inference
                \item TPUs: Optimized for transformers
                \item NPUs: Mobile inference
                \item DSPs: Ultra-low power
                \item FPGAs: Custom acceleration
            \end{itemize}
            
            \vspace{0.5em}
            \textbf{Optimization Techniques:}
            \begin{itemize}
                \item Kernel fusion
                \item Memory pooling
                \item Graph optimization
                \item Operator scheduling
                \item Cache optimization
            \end{itemize}
        \end{column}
        \begin{column}{0.5\textwidth}
            \textbf{Framework Support:}
            \begin{itemize}
                \item TensorFlow Lite: Mobile/edge
                \item ONNX Runtime: Cross-platform
                \item Core ML: iOS devices
                \item TensorRT: NVIDIA optimization
                \item OpenVINO: Intel hardware
            \end{itemize}
            
            \vspace{0.5em}
            \textbf{Performance Gains:}
            \begin{itemize}
                \item Quantization: 2-4x speedup
                \item Pruning: 2-10x speedup
                \item Hardware opt: 5-50x
                \item Combined: 100x+ possible
            \end{itemize}
        \end{column}
    \end{columns}
    
    \vspace{0.5em}
    \begin{center}
    \colorbox{yellow!20}{
        \parbox{0.8\textwidth}{
            \centering
            2024: Every major chip has AI acceleration built-in
        }
    }
    \end{center}
\end{frame}

% Real deployment example
\begin{frame}[fragile]{Real-World Deployment: BERT on Mobile}
    \begin{columns}[T]
        \column{0.65\textwidth}
\begin{lstlisting}[language=Python,basicstyle=\ttfamily\tiny]
import torch
import torch.quantization as quant
from transformers import BertModel

def prepare_mobile_bert(model_name='bert-base-uncased'):
    """Complete pipeline for mobile deployment"""
    
    model = BertModel.from_pretrained(model_name)
    model.eval()
    
    print(f"Original size: {get_model_size(model):.1f} MB")
    
    distilled_model = distill_bert(model, num_layers=6)
    print(f"After distillation: {get_model_size(distilled_model):.1f} MB")
    
    model.qconfig = quant.get_default_qconfig('qnnpack')
    quant.prepare(model, inplace=True)
    quant.convert(model, inplace=True)
    print(f"After INT8 quantization: {get_model_size(model):.1f} MB")
    
    pruned_model = magnitude_prune(model, sparsity=0.9)
    print(f"After pruning: {get_model_size(pruned_model):.1f} MB")
    
    optimized_model = torch.jit.script(pruned_model)
    optimized_model.save("mobile_bert.pt")
    
    print("\nDeployment stats:")
    print(f"Final size: {get_model_size(optimized_model):.1f} MB")
    print(f"Inference time: {benchmark_model(optimized_model):.1f} ms")
    print(f"Memory usage: {get_memory_usage(optimized_model):.1f} MB")
    
    return optimized_model

def export_to_mobile(model, example_input):
    """Export for iOS/Android"""
    traced = torch.jit.trace(model, example_input)
    
    traced.save("model_mobile.pt")
    
    torch.onnx.export(model, example_input, "model.onnx",
                     export_params=True,
                     opset_version=11,
                     do_constant_folding=True,
                     input_names=['input'],
                     output_names=['output'],
                     dynamic_axes={'input': {0: 'batch_size'},
                                  'output': {0: 'batch_size'}})
    
    import coremltools as ct
    mlmodel = ct.convert(traced, inputs=[ct.TensorType(shape=example_input.shape)])
    mlmodel.save("model.mlmodel")
    
    print("Exported to: PyTorch Mobile, ONNX, Core ML")

def benchmark_deployment():
    """Compare deployment options"""
    results = {
        'Cloud (V100)': {'latency': 10, 'cost': 0.001, 'privacy': 'Low'},
        'Edge Server': {'latency': 50, 'cost': 0.0001, 'privacy': 'Medium'},
        'Mobile (INT8)': {'latency': 100, 'cost': 0, 'privacy': 'High'},
        'Mobile (INT4)': {'latency': 150, 'cost': 0, 'privacy': 'High'},
    }
    return results
\end{lstlisting}
        \column{0.35\textwidth}
        \codeexplanation{
            \textbf{Compression Results:}
            \begin{itemize}
                \item BERT-base: 440MB → 13MB
                \item 33x size reduction
                \item 20x speedup on mobile
                \item 95\% accuracy retained
            \end{itemize}
            
            \vspace{0.3em}
            \textbf{Deployment Checklist:}
            \begin{itemize}
                \item Profile target device
                \item Choose compression mix
                \item Validate accuracy
                \item Optimize runtime
                \item Monitor in production
            \end{itemize}
        }
    \end{columns}
\end{frame}

% Energy efficiency
\begin{frame}[t]{The Green AI Movement: Energy-Efficient Models}
    \textbf{The environmental cost of AI:}\footnotemark
    
    \vspace{0.5em}
    \begin{itemize}
        \item Training GPT-3: 1,287 MWh (123 cars for a year)
        \item One ChatGPT query: 0.0003 kWh
        \item Global AI energy: Doubling every 3.4 months
    \end{itemize}
    
    \vspace{0.5em}
    \textbf{Efficiency improvements:}
    \begin{itemize}
        \item Sparse models: 10x less energy
        \item Quantization: 4x less energy
        \item Better algorithms: 100x over 10 years
        \item Hardware efficiency: 1000x since 2012
    \end{itemize}
    
    \vspace{0.5em}
    \centering
    \includegraphics[width=0.7\textwidth]{../figures/energy_efficiency_trends.pdf}
    
    \vspace{0.5em}
    \begin{center}
    \colorbox{green!20}{
        \parbox{0.8\textwidth}{
            \centering
            Efficient models aren't just faster - they're saving the planet
        }
    }
    \end{center}
    
    \footnotetext{Patterson et al. (2021) "Carbon Emissions of ML"; Strubell et al. (2019)}
\end{frame}

% Future directions
\begin{frame}[t]{The Future of Efficient AI (2024 and Beyond)}
    \begin{columns}[T]
        \begin{column}{0.5\textwidth}
            \textbf{Emerging Techniques:}
            \begin{itemize}
                \item 1-bit models (BitNet)
                \item Mixture of Experts (MoE)
                \item Dynamic neural networks
                \item Neuromorphic computing
                \item Photonic processors
            \end{itemize}
            
            \vspace{0.5em}
            \textbf{Research Frontiers:}
            \begin{itemize}
                \item Sub-1-bit quantization
                \item Learned compression
                \item Architecture search for efficiency
                \item Federated model compression
            \end{itemize}
        \end{column}
        \begin{column}{0.5\textwidth}
            \textbf{Industry Trends:}
            \begin{itemize}
                \item Every device runs AI (AIoT)
                \item Privacy-preserving inference
                \item Real-time translation everywhere
                \item Personalized on-device models
                \item Zero-latency assistants
            \end{itemize}
            
            \vspace{0.5em}
            \textbf{Hardware Evolution:}
            \begin{itemize}
                \item Analog AI chips
                \item In-memory computing
                \item Quantum advantage?
                \item Brain-inspired chips
            \end{itemize}
        \end{column}
    \end{columns}
    
    \vspace{0.5em}
    \begin{center}
    \colorbox{lightblue!30}{
        \parbox{0.8\textwidth}{
            \centering
            The future: AI everywhere, invisible, instant, and sustainable
        }
    }
    \end{center}
\end{frame}

% Exercise
\begin{frame}[t]{Week 11 Exercise: Deploy Your Own Mobile Model}
    \textbf{Your Mission:} Take a large model and make it mobile-ready
    
    \vspace{0.5em}
    \textbf{Part 1: Baseline and Profiling}
    \begin{itemize}
        \item Choose a pretrained model (BERT, GPT-2, etc.)
        \item Profile: size, latency, memory, energy
        \item Identify bottlenecks
        \item Set target constraints (e.g., 50MB, 100ms)
    \end{itemize}
    
    \vspace{0.5em}
    \textbf{Part 2: Apply Compression}
    \begin{itemize}
        \item Implement INT8 quantization
        \item Distill to smaller student
        \item Apply magnitude pruning
        \item Combine techniques
    \end{itemize}
    
    \vspace{0.5em}
    \textbf{Part 3: Deploy and Benchmark}
    \begin{itemize}
        \item Export to ONNX/TFLite
        \item Run on real device (phone/Raspberry Pi)
        \item Measure real-world performance
        \item Compare accuracy vs efficiency
        \item Create deployment package
    \end{itemize}
    
    \vspace{0.5em}
    \textbf{Bonus:} Build demo app showcasing your efficient model!
\end{frame}

% Summary
\begin{frame}[t]{Key Takeaways: Efficiency Enables Everything}
    \textbf{What we learned:}
    \begin{itemize}
        \item Models are vastly overparameterized
        \item 90\% compression with minimal loss
        \item Quantization: Smaller and faster
        \item Distillation: Transfer capabilities
        \item Hardware matters immensely
    \end{itemize}
    
    \vspace{0.5em}
    \textbf{The efficiency toolkit:}
    \begin{center}
    \colorbox{lightblue!30}{
        \parbox{0.8\textwidth}{
            \centering
            Quantization → Distillation → Pruning → Hardware optimization
        }
    }
    \end{center}
    
    \vspace{0.5em}
    \textbf{Why it matters:}
    \begin{itemize}
        \item Enables edge deployment
        \item Reduces costs dramatically
        \item Improves user experience
        \item Environmental sustainability
    \end{itemize}
    
    \vspace{0.5em}
    \textbf{Next week: Ethics and Future Directions}
    
    With great power comes great responsibility - what should we build?
\end{frame}

% References
\begin{frame}[t]{References and Further Reading}
    \textbf{Foundational Papers:}
    \begin{itemize}
        \item Hinton et al. (2015). "Distilling the Knowledge in a Neural Network"
        \item Han et al. (2016). "Deep Compression: Compressing DNNs"
        \item Frankle \& Carbin (2019). "The Lottery Ticket Hypothesis"
    \end{itemize}
    
    \textbf{Quantization:}
    \begin{itemize}
        \item Jacob et al. (2018). "Quantization and Training of Neural Networks"
        \item Gholami et al. (2021). "A Survey of Quantization Methods"
        \item Dettmers et al. (2022). "LLM.int8(): 8-bit Matrix Multiplication"
    \end{itemize}
    
    \textbf{Practical Resources:}
    \begin{itemize}
        \item PyTorch Quantization Documentation
        \item TensorFlow Lite Guide
        \item ONNX Runtime Optimization
        \item Edge Impulse Platform
    \end{itemize}
\end{frame}
\end{document}
