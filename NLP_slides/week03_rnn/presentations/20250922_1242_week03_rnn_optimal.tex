% Week 3: Recurrent Neural Networks (RNNs)
% Using the Master Optimal Readability Template

% Master Template for NLP Course
% Optimal Readability Layout Standard
% All presentations should include this template

\documentclass[8pt,aspectratio=169]{beamer}
\usetheme{Madrid}
\setbeamertemplate{navigation symbols}{}

% ====================================
% OPTIMAL READABILITY COLOR PALETTE
% ====================================
\definecolor{PureBlack}{HTML}{000000}      % Main text (21:1 contrast)
\definecolor{DeepBlue}{HTML}{003D7A}       % Primary accent (12.6:1 contrast)
\definecolor{DarkGray}{HTML}{4A4A4A}       % Secondary text (9.7:1 contrast)
\definecolor{LightGray}{HTML}{E5E5E5}      % Borders and grids
\definecolor{ChartBlue}{HTML}{0066CC}      % Chart primary
\definecolor{ChartOrange}{HTML}{FF8800}    % Chart secondary
\definecolor{ChartTeal}{HTML}{00A0A0}      % Chart tertiary
\definecolor{ChartPurple}{HTML}{8B4789}    % Chart quaternary
\definecolor{DarkGreen}{HTML}{228B22}      % Success/positive
\definecolor{DarkRed}{HTML}{CC0000}        % Warning/negative

% ====================================
% BEAMER COLOR CONFIGURATION
% ====================================
\setbeamercolor{structure}{fg=PureBlack}
\setbeamercolor{frametitle}{fg=PureBlack,bg=white}
\setbeamercolor{title}{fg=PureBlack,bg=white}
\setbeamercolor{subtitle}{fg=DarkGray}
\setbeamercolor{author}{fg=DarkGray}
\setbeamercolor{date}{fg=DarkGray}
\setbeamercolor{institute}{fg=DarkGray}

% Blocks - no backgrounds
\setbeamercolor{block title}{fg=PureBlack,bg=white}
\setbeamercolor{block body}{fg=PureBlack,bg=white}
\setbeamercolor{block title example}{fg=DarkGreen,bg=white}
\setbeamercolor{block body example}{fg=PureBlack,bg=white}
\setbeamercolor{block title alerted}{fg=DarkRed,bg=white}
\setbeamercolor{block body alerted}{fg=PureBlack,bg=white}

% Lists
\setbeamercolor{item}{fg=DeepBlue}
\setbeamercolor{subitem}{fg=ChartBlue}
\setbeamercolor{enumerate item}{fg=DeepBlue}
\setbeamercolor{enumerate subitem}{fg=ChartBlue}

% Text
\setbeamercolor{normal text}{fg=PureBlack,bg=white}
\setbeamercolor{alerted text}{fg=DarkRed}
\setbeamercolor{example text}{fg=DarkGreen}

% Footer
\setbeamercolor{footline}{fg=DarkGray,bg=white}
\setbeamercolor{page number in head/foot}{fg=DarkGray}

% ====================================
% FONT CONFIGURATION
% ====================================
\setbeamerfont{normal text}{size=\normalsize}
\setbeamerfont{frametitle}{size=\Large,series=\bfseries}
\setbeamerfont{title}{size=\huge,series=\bfseries}
\setbeamerfont{subtitle}{size=\large}
\setbeamerfont{author}{size=\normalsize}
\setbeamerfont{date}{size=\small}
\setbeamerfont{institute}{size=\small}

% ====================================
% REQUIRED PACKAGES
% ====================================
\usepackage{tikz}
\usepackage{amsmath}
\usepackage{amssymb}
\usepackage{booktabs}
\usepackage{graphicx}
\usepackage{array}
\usepackage{listings}
\usepackage{algorithm2e}
\usepackage{xcolor}
\usepackage{tabularx}
\usepackage{multirow}
\usepackage{subcaption}

% ====================================
% CUSTOM COMMANDS
% ====================================
% Text highlighting commands
\newcommand{\highlight}[1]{\textcolor{DeepBlue}{\textbf{#1}}}
\newcommand{\secondary}[1]{\textcolor{DarkGray}{#1}}
\newcommand{\success}[1]{\textcolor{DarkGreen}{#1}}
\newcommand{\warning}[1]{\textcolor{DarkRed}{#1}}
\newcommand{\data}[1]{\textcolor{ChartBlue}{#1}}
\newcommand{\dataalt}[1]{\textcolor{ChartOrange}{#1}}

% Mathematical notation
\newcommand{\given}{\mid}
\newcommand{\prob}[1]{P(#1)}
\newcommand{\argmax}{\operatorname*{argmax}}
\newcommand{\argmin}{\operatorname*{argmin}}
\newcommand{\softmax}{\operatorname{softmax}}

% Box commands for emphasis
\newcommand{\keypoint}[1]{%
  \begin{center}
  \fbox{\parbox{0.9\textwidth}{\centering\textbf{#1}}}
  \end{center}
}

\newcommand{\formula}[1]{%
  \begin{center}
  \colorbox{LightGray}{\parbox{0.8\textwidth}{\centering$\displaystyle #1$}}
  \end{center}
}

% ====================================
% LISTINGS CONFIGURATION
% ====================================
\lstset{
  basicstyle=\ttfamily\small,
  keywordstyle=\color{DeepBlue}\bfseries,
  commentstyle=\color{DarkGray}\itshape,
  stringstyle=\color{ChartOrange},
  numbers=left,
  numberstyle=\tiny\color{DarkGray},
  stepnumber=1,
  numbersep=5pt,
  backgroundcolor=\color{white},
  showspaces=false,
  showstringspaces=false,
  showtabs=false,
  frame=single,
  frameround=tttt,
  rulecolor=\color{LightGray},
  tabsize=2,
  captionpos=b,
  breaklines=true,
  breakatwhitespace=true,
  language=Python,
  escapeinside={(*@}{@*)},
  morekeywords={self, yield, assert, with, as}
}

% ====================================
% STANDARD SLIDE LAYOUTS
% ====================================

% Two-column layout with title
\newcommand{\twocolslide}[3]{%
  \begin{frame}{#1}
  \begin{columns}[T]
  \column{0.48\textwidth}
  #2
  \column{0.48\textwidth}
  #3
  \end{columns}
  \end{frame}
}

% Three-column layout
\newcommand{\threecolslide}[4]{%
  \begin{frame}{#1}
  \begin{columns}[T]
  \column{0.32\textwidth}
  #2
  \column{0.32\textwidth}
  #3
  \column{0.32\textwidth}
  #4
  \end{columns}
  \end{frame}
}

% Chart slide with caption
\newcommand{\chartslide}[3]{%
  \begin{frame}{#1}
  \begin{center}
  \includegraphics[width=#2\textwidth]{#3}
  \end{center}
  \end{frame}
}

% Full chart slide - optimized to 0.85 for proper margins
\newcommand{\fullchartslide}[2]{%
  \begin{frame}{#1}
  \begin{center}
  \includegraphics[width=0.85\textwidth]{#2}
  \end{center}
  \end{frame}
}

% Code slide
\newcommand{\codeslide}[3]{%
  \begin{frame}[fragile]{#1}
  \begin{lstlisting}[language=#2]
#3
  \end{lstlisting}
  \end{frame}
}

% Concept slide with figure
\newcommand{\conceptslide}[3]{%
  \begin{frame}{#1}
  \begin{columns}[T]
  \column{0.6\textwidth}
  #2
  \column{0.35\textwidth}
  \begin{center}
  \includegraphics[width=0.95\textwidth]{#3}
  \end{center}
  \end{columns}
  \end{frame}
}

% Table slide
\newcommand{\tableslide}[2]{%
  \begin{frame}{#1}
  \begin{center}
  \Large
  \renewcommand{\arraystretch}{1.5}
  #2
  \end{center}
  \end{frame}
}

% Summary slide
\newcommand{\summaryslide}[2]{%
  \begin{frame}{#1}
  \begin{center}
  \Large
  #2
  \end{center}
  \vfill
  \begin{center}
  \keypoint{Key Takeaway}
  \end{center}
  \end{frame}
}

% ====================================
% REMOVE DECORATIONS
% ====================================
\setbeamertemplate{blocks}[default]
\setbeamertemplate{title page}[default][colsep=-4bp,rounded=false]
\setbeamertemplate{itemize items}[circle]
\setbeamertemplate{enumerate items}[default]
\setbeamertemplate{section in toc}[sections numbered]
\setbeamertemplate{subsection in toc}[subsections numbered]

% ====================================
% PAGE NUMBERING
% ====================================
\setbeamertemplate{footline}{
  \leavevmode%
  \hbox{%
  \begin{beamercolorbox}[wd=.333333\paperwidth,ht=2.25ex,dp=1ex,center]{author in head/foot}%
    \usebeamerfont{author in head/foot}\secondary{\insertshortauthor}
  \end{beamercolorbox}%
  \begin{beamercolorbox}[wd=.333333\paperwidth,ht=2.25ex,dp=1ex,center]{title in head/foot}%
    \usebeamerfont{title in head/foot}\secondary{\insertshorttitle}
  \end{beamercolorbox}%
  \begin{beamercolorbox}[wd=.333333\paperwidth,ht=2.25ex,dp=1ex,right]{date in head/foot}%
    \usebeamerfont{date in head/foot}\secondary{\insertshortdate{}\hspace*{2em}
    \insertframenumber{} / \inserttotalframenumber\hspace*{2ex}}
  \end{beamercolorbox}}%
  \vskip0pt%
}

% ====================================
% TABLE OF CONTENTS STYLE
% ====================================
\setbeamertemplate{section in toc}{%
  \leavevmode\leftskip=1.5em%
  \llap{%
    \usebeamerfont{section in toc}%
    \usebeamercolor[fg]{section in toc}%
    \inserttocsectionnumber.%
  }%
  \usebeamerfont{section in toc}%
  \usebeamercolor[fg]{section in toc}%
  \inserttocsection\par%
}

\setbeamertemplate{subsection in toc}{%
  \leavevmode\leftskip=3em%
  \llap{%
    \usebeamerfont{subsection in toc}%
    \usebeamercolor[fg]{subsection in toc}%
    \inserttocsectionnumber.\inserttocsubsectionnumber%
  }%
  \usebeamerfont{subsection in toc}%
  \usebeamercolor[fg]{subsection in toc}%
  \inserttocsubsection\par%
}

% ====================================
% END OF MASTER TEMPLATE
% ====================================

\title{Recurrent Neural Networks}
\subtitle{\secondary{Week 3 - Teaching Networks to Remember}}
\author{NLP Course 2025}
\date{\today}

\begin{document}

% Title slide
\begin{frame}
\titlepage
\vfill
\begin{center}
\secondary{\footnotesize From Feedforward to Recurrent: Adding Memory to Neural Networks}
\end{center}
\end{frame}

% Overview
\begin{frame}{Week 3: The Memory Revolution}
\begin{center}
{\Large \textbf{Processing Sequences with State}}
\end{center}

\vspace{10mm}

\begin{columns}[T]
\column{0.32\textwidth}
\textbf{The Challenge}
\begin{itemize}
\item Sequential data everywhere
\item \warning{Order matters}
\item Variable length inputs
\item Long-term dependencies
\end{itemize}

\column{0.32\textwidth}
\textbf{The Solution}
\begin{itemize}
\item \highlight{Recurrent connections}
\item Hidden state memory
\item Parameter sharing
\item Backprop through time
\end{itemize}

\column{0.32\textwidth}
\textbf{The Evolution}
\begin{itemize}
\item Vanilla RNN → \success{LSTM}
\item Solving gradients
\item Gating mechanisms
\item Modern variants (GRU)
\end{itemize}
\end{columns}

\vfill
\secondary{\footnotesize The foundation of sequence modeling before transformers}
\end{frame}

% Course Journey
\begin{frame}{Course Journey: Where We Are}
\begin{center}
\includegraphics[width=0.9\textwidth]{../figures/course_timeline.pdf}
\end{center}

\vspace{5mm}

\begin{columns}[T]
\column{0.48\textwidth}
\textbf{Journey So Far:}
\begin{itemize}
\item Week 1: Statistical language models (n-grams)
\item Week 2: Word embeddings (Word2Vec, dense vectors)
\item \highlight{Week 3: Sequential processing with memory}
\end{itemize}

\column{0.48\textwidth}
\textbf{Coming Next:}
\begin{itemize}
\item Week 4: Encoder-decoder architectures
\item Week 5: Attention mechanisms
\item Week 6+: Transformers and beyond
\end{itemize}
\end{columns}

\vfill
\keypoint{RNNs bridge the gap between static embeddings and modern attention}
\end{frame}

% Part 1: Why Sequential Processing Matters
\begin{frame}{\Large Part 1: Why Sequential Processing Matters}
\begin{center}
{\huge \textbf{The Importance of Order}}
\end{center}

\vfill

\begin{columns}[T]
\column{0.48\textwidth}
\textbf{Word Order Changes Meaning}

\begin{itemize}
\item "Dog bites man" $\neq$ "Man bites dog"
\item "Not bad" $\neq$ "Bad, not!"
\item Context flows through sequence
\end{itemize}

\vspace{5mm}

\textbf{Feedforward Limitations}
\begin{itemize}
\item Fixed input size
\item No memory between inputs
\item Can't model sequences naturally
\item Position information lost
\end{itemize}

\column{0.48\textwidth}
\textbf{Sequential Tasks in NLP}

\begin{tabular}{ll}
\toprule
\textbf{Task} & \textbf{Type} \\
\midrule
Language Model & Many-to-many \\
Translation & Seq-to-seq \\
Sentiment & Many-to-one \\
Named Entity & Many-to-many \\
Speech Rec. & Seq-to-seq \\
\bottomrule
\end{tabular}

\vspace{5mm}

\keypoint{Most NLP is inherently sequential}
\end{columns}

\vfill
\secondary{\footnotesize Sequential processing is fundamental to understanding language}
\end{frame}

% The Core RNN Idea
\begin{frame}{The Core Idea: Recurrence}
\begin{columns}[T]
\column{0.48\textwidth}
\textbf{Mathematical Definition}

\formula{h_t = \tanh(W_{hh} h_{t-1} + W_{xh} x_t + b_h)}
\formula{y_t = W_{hy} h_t + b_y}

Where:
\begin{itemize}
\item $h_t$ = hidden state at time $t$
\item $x_t$ = input at time $t$
\item $y_t$ = output at time $t$
\item $W_{*}$ = weight matrices (shared!)
\end{itemize}

\vspace{5mm}

\textbf{Key Insight}: Same weights at every timestep

\column{0.48\textwidth}
\includegraphics[width=\textwidth]{../figures/rnn_unrolled.pdf}

\vspace{5mm}

\textbf{Unrolled View Shows:}
\begin{itemize}
\item Information flows left-to-right
\item Hidden state carries memory
\item Parameters shared across time
\item Can handle any sequence length
\end{itemize}
\end{columns}

\vfill
\secondary{\footnotesize One network, applied repeatedly through time}
\end{frame}

% Forward Pass Visualization
\begin{frame}{Forward Pass: Step by Step}
\begin{center}
\includegraphics[width=0.85\textwidth]{../figures/rnn_forward_pass.pdf}
\end{center}

\vspace{5mm}

\begin{columns}[T]
\column{0.32\textwidth}
\textbf{Step 1: Initialize}
\begin{itemize}
\item $h_0 = \vec{0}$ (zeros)
\item Load first input $x_1$
\item Apply RNN cell
\end{itemize}

\column{0.32\textwidth}
\textbf{Step 2: Process}
\begin{itemize}
\item Compute $h_1$ from $h_0, x_1$
\item Generate output $y_1$
\item Pass $h_1$ forward
\end{itemize}

\column{0.32\textwidth}
\textbf{Step 3: Continue}
\begin{itemize}
\item Repeat for all timesteps
\item Each $h_t$ depends on history
\item Outputs can be at each step
\end{itemize}
\end{columns}

\vfill
\secondary{\footnotesize Hidden state accumulates information over time}
\end{frame}

% Code Example
\begin{frame}[fragile]{Implementation: Simple RNN Cell}
\begin{columns}[T]
\column{0.55\textwidth}
\begin{lstlisting}[language=Python]
import numpy as np

class RNNCell:
    def __init__(self, input_size, hidden_size):
        # Initialize weights
        self.Wxh = np.random.randn(input_size, hidden_size) * 0.01
        self.Whh = np.random.randn(hidden_size, hidden_size) * 0.01
        self.Why = np.random.randn(hidden_size, output_size) * 0.01
        self.bh = np.zeros((1, hidden_size))
        self.by = np.zeros((1, output_size))

    def step(self, x, h_prev):
        # Single timestep forward
        h = np.tanh(np.dot(x, self.Wxh) +
                   np.dot(h_prev, self.Whh) + self.bh)
        y = np.dot(h, self.Why) + self.by
        return y, h

    def forward(self, inputs):
        h = np.zeros((1, self.hidden_size))
        outputs = []

        for x in inputs:
            y, h = self.step(x, h)
            outputs.append(y)

        return outputs
\end{lstlisting}

\column{0.43\textwidth}
\textbf{PyTorch Equivalent:}
\begin{lstlisting}[language=Python]
import torch.nn as nn

# Built-in RNN
rnn = nn.RNN(
    input_size=100,
    hidden_size=256,
    num_layers=1,
    batch_first=True
)

# Or use LSTM/GRU
lstm = nn.LSTM(
    input_size=100,
    hidden_size=256,
    num_layers=2,
    dropout=0.2,
    bidirectional=True
)

# Forward pass
output, (hn, cn) = lstm(input_seq)
\end{lstlisting}

\vspace{3mm}
\keypoint{Modern frameworks handle the complexity}
\end{columns}
\end{frame}

% Part 2: The Vanishing Gradient Problem
\begin{frame}{\Large Part 2: The Vanishing Gradient Problem}
\begin{center}
{\huge \textbf{Why Simple RNNs Fail}}
\end{center}

\vfill

\begin{columns}[T]
\column{0.48\textwidth}
\textbf{The Problem}

Gradient through time:
\formula{\frac{\partial L}{\partial h_0} = \frac{\partial L}{\partial h_T} \prod_{t=1}^{T} \frac{\partial h_t}{\partial h_{t-1}}}

Each term: $\frac{\partial h_t}{\partial h_{t-1}} = W_h^T \cdot \text{diag}(f'(h_{t-1}))$

\vspace{5mm}

For tanh: $|f'(x)| \leq 1$

If $||W_h|| < 1$: gradients → 0 \warning{(vanish)}

If $||W_h|| > 1$: gradients → ∞ \warning{(explode)}

\column{0.48\textwidth}
\includegraphics[width=\textwidth]{../figures/vanishing_gradient.pdf}

\vspace{5mm}

\textbf{Consequences:}
\begin{itemize}
\item Can't learn long dependencies
\item Gradient ≈ 0 after 10-20 steps
\item Network "forgets" early inputs
\item Training becomes ineffective
\end{itemize}
\end{columns}

\vfill
\secondary{\footnotesize The fundamental limitation that led to LSTM/GRU}
\end{frame}

% Gradient Flow Visualization
\begin{frame}{Visualizing Gradient Flow}
\begin{center}
\includegraphics[width=0.85\textwidth]{../figures/gradient_flow_comparison.pdf}
\end{center}

\begin{columns}[T]
\column{0.48\textwidth}
\textbf{Vanilla RNN}
\begin{itemize}
\item Exponential decay/growth
\item Gradient magnitude: $O(\lambda^T)$
\item Effective memory: 5-10 steps
\item \warning{Cannot learn long patterns}
\end{itemize}

\column{0.48\textwidth}
\textbf{LSTM (Next Section)}
\begin{itemize}
\item Constant error flow
\item Gradient highways
\item Effective memory: 100+ steps
\item \success{Learns long dependencies}
\end{itemize}
\end{columns}

\vfill
\keypoint{Gradient flow determines what the network can learn}
\end{frame}

% Part 3: LSTM - The Solution
\begin{frame}{\Large Part 3: Long Short-Term Memory (LSTM)}
\begin{center}
{\huge \textbf{Engineering Memory}}
\end{center}

\vfill

\begin{columns}[T]
\column{0.48\textwidth}
\textbf{The Innovation (1997)}

Hochreiter & Schmidhuber's insight:
\begin{itemize}
\item Add a \highlight{memory cell} $C_t$
\item Control flow with \highlight{gates}
\item Create gradient highways
\item Selective reading/writing
\end{itemize}

\vspace{5mm}

\textbf{Three Gates:}
\begin{enumerate}
\item \textbf{Forget}: What to discard
\item \textbf{Input}: What to store
\item \textbf{Output}: What to expose
\end{enumerate}

\column{0.48\textwidth}
\textbf{LSTM Equations}

\begin{align*}
f_t &= \sigma(W_f \cdot [h_{t-1}, x_t] + b_f) \\
i_t &= \sigma(W_i \cdot [h_{t-1}, x_t] + b_i) \\
\tilde{C}_t &= \tanh(W_C \cdot [h_{t-1}, x_t] + b_C) \\
C_t &= f_t * C_{t-1} + i_t * \tilde{C}_t \\
o_t &= \sigma(W_o \cdot [h_{t-1}, x_t] + b_o) \\
h_t &= o_t * \tanh(C_t)
\end{align*}

\vspace{3mm}

\keypoint{Gates use sigmoid (0-1) for control}
\end{columns}

\vfill
\secondary{\footnotesize The architecture that made deep sequence modeling possible}
\end{frame}

% LSTM Architecture Visualization
\begin{frame}{LSTM Architecture: Gate Mechanisms}
\begin{center}
\includegraphics[width=0.9\textwidth]{../figures/lstm_architecture.pdf}
\end{center}

\vspace{5mm}

\begin{columns}[T]
\column{0.32\textwidth}
\textbf{Forget Gate}
\begin{itemize}
\item Decides what to forget
\item $f_t \in [0,1]^d$
\item 0 = forget completely
\item 1 = remember fully
\end{itemize}

\column{0.32\textwidth}
\textbf{Input Gate}
\begin{itemize}
\item Controls new information
\item $i_t$ = how much to write
\item $\tilde{C}_t$ = what to write
\item Selective memory update
\end{itemize}

\column{0.32\textwidth}
\textbf{Output Gate}
\begin{itemize}
\item Filters cell state
\item $o_t$ = what to output
\item Hidden state from cell
\item Task-specific exposure
\end{itemize}
\end{columns}

\vfill
\secondary{\footnotesize Information flow is carefully controlled at each step}
\end{frame}

% LSTM vs RNN Comparison
\begin{frame}{RNN vs LSTM: Key Differences}
\begin{center}
\begin{tabular}{lcc}
\toprule
\textbf{Aspect} & \textbf{Vanilla RNN} & \textbf{LSTM} \\
\midrule
Parameters & $O(h^2)$ & $O(4h^2)$ \\
Memory & Short (5-10 steps) & Long (100+ steps) \\
Gradient flow & Multiplicative & Additive \\
Training speed & Fast & Slower (4x params) \\
Gradient problem & Severe & Largely solved \\
Use cases & Short sequences & Most applications \\
\bottomrule
\end{tabular}
\end{center}

\vspace{8mm}

\begin{columns}[T]
\column{0.48\textwidth}
\textbf{When to Use RNN:}
\begin{itemize}
\item Very short sequences
\item Real-time constraints
\item Limited compute
\item Simple patterns
\end{itemize}

\column{0.48\textwidth}
\textbf{When to Use LSTM:}
\begin{itemize}
\item Long dependencies
\item Complex patterns
\item Production systems
\item Default choice (pre-2017)
\end{itemize}
\end{columns}

\vfill
\keypoint{LSTM's complexity is justified by superior performance}
\end{frame}

% GRU - Simplified Alternative
\begin{frame}{GRU: Gated Recurrent Unit}
\begin{columns}[T]
\column{0.48\textwidth}
\textbf{Simplification of LSTM (2014)}

Cho et al. merged gates:
\begin{itemize}
\item Only \highlight{2 gates} instead of 3
\item No separate cell state
\item Fewer parameters (3x vs 4x)
\item Similar performance
\end{itemize}

\vspace{5mm}

\textbf{GRU Equations:}
\begin{align*}
z_t &= \sigma(W_z \cdot [h_{t-1}, x_t]) \\
r_t &= \sigma(W_r \cdot [h_{t-1}, x_t]) \\
\tilde{h}_t &= \tanh(W \cdot [r_t * h_{t-1}, x_t]) \\
h_t &= (1-z_t) * h_{t-1} + z_t * \tilde{h}_t
\end{align*}

\column{0.48\textwidth}
\includegraphics[width=\textwidth]{../figures/gru_architecture.pdf}

\vspace{5mm}

\textbf{Gates:}
\begin{itemize}
\item \textbf{Update gate} ($z_t$): How much to update
\item \textbf{Reset gate} ($r_t$): How much past matters
\end{itemize}

\vspace{5mm}

\keypoint{GRU ≈ LSTM with fewer parameters}
\end{columns}

\vfill
\secondary{\footnotesize Often preferred for smaller datasets}
\end{frame}

% Bidirectional RNNs
\begin{frame}[fragile]{Bidirectional RNNs: Using Future Context}
\begin{center}
\includegraphics[width=0.85\textwidth]{../figures/bidirectional_rnn.pdf}
\end{center}

\vspace{5mm}

\begin{columns}[T]
\column{0.48\textwidth}
\textbf{Motivation}

"The bank by the river" vs "The bank denied the loan"
\begin{itemize}
\item "bank" ambiguous without future context
\item Forward RNN only sees past
\item Solution: process both directions
\end{itemize}

\vspace{5mm}

\textbf{Architecture:}
\begin{itemize}
\item Two separate RNNs
\item Forward: left → right
\item Backward: right → left
\item Concatenate hidden states
\end{itemize}

\column{0.48\textwidth}
\textbf{Implementation:}
\begin{lstlisting}[language=Python]
# PyTorch BiLSTM
bilstm = nn.LSTM(
    input_size=100,
    hidden_size=128,
    bidirectional=True
)

# Output size = 2 * hidden_size
# output: [batch, seq, 256]

# Separate directions:
output = output.view(
    batch, seq, 2, 128
)
forward = output[:,:,0,:]
backward = output[:,:,1,:]
\end{lstlisting}

\vspace{3mm}
\success{Standard for non-causal tasks}
\end{columns}

\vfill
\secondary{\footnotesize Double the parameters, significant performance gain}
\end{frame}

% Practical Training Tips
\begin{frame}[fragile]{Training RNNs: Practical Tips}
\begin{columns}[T]
\column{0.48\textwidth}
\textbf{Common Issues \& Solutions}

\begin{enumerate}
\item \textbf{Gradient Explosion}
   \begin{itemize}
   \item Solution: Gradient clipping
   \item `torch.nn.utils.clip\_grad\_norm\_`
   \item Typical value: 1.0 - 5.0
   \end{itemize}

\item \textbf{Initialization}
   \begin{itemize}
   \item Xavier/He initialization
   \item Forget gate bias = 1.0 (LSTM)
   \item Helps gradient flow
   \end{itemize}

\item \textbf{Overfitting}
   \begin{itemize}
   \item Dropout (between layers)
   \item Recurrent dropout (careful!)
   \item Weight decay
   \end{itemize}
\end{enumerate}

\column{0.48\textwidth}
\textbf{Hyperparameters}

\begin{tabular}{ll}
\toprule
\textbf{Parameter} & \textbf{Typical Range} \\
\midrule
Hidden size & 128 - 512 \\
Num layers & 1 - 3 \\
Learning rate & 1e-3 - 1e-2 \\
Batch size & 32 - 128 \\
Sequence length & 20 - 200 \\
Gradient clip & 1.0 - 5.0 \\
Dropout & 0.2 - 0.5 \\
\bottomrule
\end{tabular}

\vspace{5mm}

\textbf{Training Strategy:}
\begin{itemize}
\item Start with small sequences
\item Gradually increase length
\item Monitor gradient norms
\item Use teacher forcing wisely
\end{itemize}
\end{columns}

\vfill
\keypoint{RNN training requires careful tuning and monitoring}
\end{frame}

% Applications
\begin{frame}{Real-World Applications}
\begin{columns}[T]
\column{0.48\textwidth}
\textbf{Natural Language Processing}
\begin{itemize}
\item Language modeling (pre-2018)
\item Machine translation (pre-2017)
\item Speech recognition (still used)
\item Named entity recognition
\item Sentiment analysis
\end{itemize}

\vspace{5mm}

\textbf{Time Series}
\begin{itemize}
\item Stock price prediction
\item Weather forecasting
\item Anomaly detection
\item Signal processing
\end{itemize}

\column{0.48\textwidth}
\textbf{Modern Context (2024)}

\includegraphics[width=\textwidth]{../figures/rnn_vs_transformer_timeline.pdf}

\vspace{5mm}

\textbf{Where RNNs Still Win:}
\begin{itemize}
\item Streaming/online processing
\item Edge devices (memory constraints)
\item Variable-length sequences
\item Time series with clear temporal patterns
\end{itemize}
\end{columns}

\vfill
\secondary{\footnotesize RNNs remain relevant for specific use cases}
\end{frame}

% Connection to Modern Methods
\begin{frame}{From RNNs to Transformers: Evolution}
\begin{center}
\includegraphics[width=0.9\textwidth]{../figures/evolution_to_transformers.pdf}
\end{center}

\vspace{5mm}

\begin{columns}[T]
\column{0.32\textwidth}
\textbf{RNN Era (2013-2017)}
\begin{itemize}
\item Sequential processing
\item Hidden state memory
\item Cannot parallelize
\item Long-range struggles
\end{itemize}

\column{0.32\textwidth}
\textbf{Attention Addition (2015)}
\begin{itemize}
\item RNN + attention
\item Direct connections
\item Better gradients
\item Still sequential
\end{itemize}

\column{0.32\textwidth}
\textbf{Transformers (2017+)}
\begin{itemize}
\item Pure attention
\item Fully parallel
\item Global context
\item Dominates today
\end{itemize}
\end{columns}

\vfill
\keypoint{RNNs taught us the importance of modeling sequences}
\end{frame}

% Key Takeaways
\begin{frame}{Key Takeaways}
\begin{center}
{\Large \textbf{What We Learned About RNNs}}
\end{center}

\vspace{10mm}

\begin{columns}[T]
\column{0.32\textwidth}
\textbf{Core Concepts}
\begin{itemize}
\item \highlight{Recurrence} for sequences
\item Hidden state as memory
\item Parameter sharing
\item Backprop through time
\end{itemize}

\column{0.32\textwidth}
\textbf{Challenges}
\begin{itemize}
\item Vanishing gradients
\item Sequential bottleneck
\item Training difficulty
\item Limited context window
\end{itemize}

\column{0.32\textwidth}
\textbf{Solutions}
\begin{itemize}
\item LSTM/GRU gates
\item Gradient clipping
\item Bidirectional processing
\item Attention (next week!)
\end{itemize}
\end{columns}

\vspace{10mm}

\keypoint{RNNs introduced memory to neural networks - a crucial innovation}

\vspace{8mm}

\begin{center}
\secondary{\Large Next Week: Sequence-to-Sequence Models}\\
\secondary{How to translate, summarize, and generate with encoder-decoder architectures}
\end{center}
\end{frame}

% References
\begin{frame}{References \& Further Reading}
\textbf{Foundational Papers:}
\begin{itemize}
\item Hochreiter \& Schmidhuber (1997). "Long Short-Term Memory"
\item Cho et al. (2014). "Learning Phrase Representations using RNN Encoder-Decoder" (GRU)
\item Graves (2013). "Generating Sequences With RNNs"
\item Karpathy (2015). "The Unreasonable Effectiveness of RNNs" (blog)
\end{itemize}

\vspace{5mm}

\textbf{Practical Resources:}
\begin{itemize}
\item PyTorch RNN Tutorial: \url{pytorch.org/tutorials/intermediate/char_rnn}
\item Understanding LSTMs: \url{colah.github.io/posts/2015-08-Understanding-LSTMs/}
\item Stanford CS224N Lecture 6: RNNs and Language Models
\end{itemize}

\vspace{5mm}

\textbf{Code Examples:}
\begin{itemize}
\item Week 3 Lab: `week03\_rnn\_lab.ipynb`
\item GitHub: Various char-RNN implementations
\item Hugging Face: Modern RNN models
\end{itemize}
\end{frame}

\end{document}