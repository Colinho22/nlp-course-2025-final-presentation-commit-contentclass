\documentclass[8pt,aspectratio=169]{beamer}
\usetheme{Madrid}
\usecolortheme{seahorse}
\setbeamertemplate{navigation symbols}{}

% Packages
\usepackage{graphicx}
\usepackage{amsmath}
\usepackage{amssymb}
\usepackage{tikz}
\usepackage{listings}
\usepackage{xcolor}
\usepackage{colortbl}
\usepackage{tcolorbox}

% Math commands
\newcommand{\given}{\mid}

% BSc Pedagogical boxes
\newtcolorbox{checkpoint}[1][]{
    colback=yellow!10!white,
    colframe=yellow!75!black,
    title=\textbf{Checkpoint: #1},
    fonttitle=\bfseries
}

\newtcolorbox{intuition}[1][]{
    colback=purple!5!white,
    colframe=purple!75!black,
    title=\textbf{Intuition: #1},
    fonttitle=\bfseries
}

\newtcolorbox{realworld}[1][]{
    colback=orange!5!white,
    colframe=orange!75!black,
    title=\textbf{Real World: #1},
    fonttitle=\bfseries
}

% Title information
\title{LSTM - Long Short-Term Memory}
\subtitle{Understanding Through a Complete Example}
\author{}
\date{}

\begin{document}

% Slide 1: Title
\begin{frame}
    \titlepage
\end{frame}

% Slide 2: THE CORE - Show the Magic First!
\begin{frame}{Watch LSTM Process a Sentence}
    \scriptsize
    \textbf{Sentence:} ``The cat was hungry. The dog was sleeping.''

    \vspace{-0.3em}
    \begin{center}
    \renewcommand{\arraystretch}{0.9}
    \begin{tabular}{|c|l|l|l|l|}
    \hline
    \textbf{Word} & \textbf{Forget} & \textbf{Input} & \textbf{Output} & \textbf{Memory} \\
    \hline
    ``The'' & 0.9 & 0.3 & 0.2 & [article] \\
    \hline
    ``cat'' & 0.8 & 0.9 & 0.8 & [cat] \\
    \hline
    ``was'' & 0.9 & 0.7 & 0.9 & [cat, was] \\
    \hline
    ``hungry'' & 0.8 & 0.8 & 0.7 & [cat, hungry] \\
    \hline
    \rowcolor{yellow!20}
    ``.'' & \textbf{0.1} & 0.4 & 0.3 & [end] \\
    \hline
    ``The'' & 0.1 & 0.8 & 0.2 & [article] \\
    \hline
    \rowcolor{green!20}
    ``dog'' & 0.7 & \textbf{0.9} & 0.9 & [dog] \\
    \hline
    ``was'' & 0.9 & 0.8 & \textbf{0.9} & [dog, was] \\
    \hline
    \end{tabular}
    \end{center}

    \vspace{0.1em}
    \begin{columns}
        \begin{column}{0.48\textwidth}
            \textbf{Notice:}
            \begin{itemize}\setlength\itemsep{-0.1em}
                \item Three mysterious numbers
                \item Memory changes
                \item ``cat'' disappears
                \item ``dog'' appears
            \end{itemize}
        \end{column}

        \begin{column}{0.48\textwidth}
            \begin{intuition}[The Magic]
            How does LSTM know to:
            \begin{itemize}\setlength\itemsep{-0.1em}
                \item Forget ``cat''?
                \item Remember ``dog''?
                \item Keep right info?
            \end{itemize}
            \textbf{Let's find out...}
            \end{intuition}
        \end{column}
    \end{columns}
\end{frame}

% Slide 3: The Problem This Solves
\begin{frame}{Why Do We Need This?}
    \begin{center}
        \includegraphics[width=0.7\textwidth]{../figures/vanishing_gradient_problem_bsc.pdf}
    \end{center}

    \vspace{-0.6em}
    \scriptsize
    \begin{columns}[T]
        \begin{column}{0.48\textwidth}
            \textbf{RNN Problem:}
            \begin{itemize}\setlength\itemsep{-0.1em}
                \item Gradients vanish
                \item Forgets early info
                \item Can't handle long deps
                \item Loses ``cat''
            \end{itemize}

            \vspace{0.2em}
            RNN forgets ``cat'' by ``dog''
        \end{column}

        \begin{column}{0.48\textwidth}
            \textbf{LSTM Solution:}
            \begin{itemize}\setlength\itemsep{-0.1em}
                \item Cell state highway
                \item Addition not multiply
                \item Three gates
                \item Preserves then clears
            \end{itemize}

            \vspace{0.2em}
            \begin{checkpoint}[Key]
            Three columns = \textbf{gates}!
            \end{checkpoint}
        \end{column}
    \end{columns}
\end{frame}

% Slide 4: Gate 1 - Forget (Explain Row 5)
\begin{frame}{Gate 1: Forget - What to Erase?}
    \begin{center}
        \includegraphics[width=0.7\textwidth]{../figures/forget_gate_detail_bsc.pdf}
    \end{center}

    \vspace{-0.5em}
    \scriptsize
    \begin{columns}[T]
        \begin{column}{0.48\textwidth}
            \textbf{Remember Our Table?}

            Row 5: ``.'' had \textbf{Forget = 0.1}

            \vspace{0.3em}
            \textbf{What This Means:}
            \begin{itemize}\setlength\itemsep{0em}
                \item 0.0 = forget everything
                \item 1.0 = keep everything
                \item 0.1 = forget most things
            \end{itemize}

            \vspace{0.3em}
            \textbf{Why at period?}
            \begin{itemize}\setlength\itemsep{0em}
                \item New sentence starting
                \item Old subject no longer relevant
                \item Clear ``cat'' from memory
            \end{itemize}
        \end{column}

        \begin{column}{0.48\textwidth}
            \textbf{Formula:}
            \[
            f_t = \sigma(W_f \cdot [h_{t-1}, x_t] + b_f)
            \]

            \vspace{0.2em}
            \textbf{How It Decides:}
            \begin{enumerate}\setlength\itemsep{0em}
                \item Look at current word (``.'')
                \item Look at previous output
                \item Compute: Should we forget?
                \item Output value 0 to 1
            \end{enumerate}

            \vspace{0.2em}
            \textbf{Cell state update:}
            \[
            C_t = f_t \odot C_{t-1} + \ldots
            \]
            Multiply old memory by 0.1 = mostly erase!
        \end{column}
    \end{columns}
\end{frame}

% Slide 5: Gate 2 - Input (Explain Row 7)
\begin{frame}{Gate 2: Input - What to Add?}
    \begin{center}
        \includegraphics[width=0.7\textwidth]{../figures/input_gate_detail_bsc.pdf}
    \end{center}

    \vspace{-0.5em}
    \scriptsize
    \begin{columns}[T]
        \begin{column}{0.48\textwidth}
            \textbf{Remember Our Table?}

            Row 7: ``dog'' had \textbf{Input = 0.9}

            \vspace{0.3em}
            \textbf{What This Means:}
            \begin{itemize}\setlength\itemsep{0em}
                \item 0.0 = add nothing
                \item 1.0 = add everything
                \item 0.9 = add most of it
            \end{itemize}

            \vspace{0.3em}
            \textbf{Why at ``dog''?}
            \begin{itemize}\setlength\itemsep{0em}
                \item New subject appearing
                \item Important to remember
                \item Will need it for ``was''
            \end{itemize}
        \end{column}

        \begin{column}{0.48\textwidth}
            \textbf{Formulas (Two Parts):}
            \[
            i_t = \sigma(W_i \cdot [h_{t-1}, x_t] + b_i)
            \]
            \[
            \tilde{C}_t = \tanh(W_C \cdot [h_{t-1}, x_t] + b_C)
            \]

            \vspace{0.2em}
            \textbf{How It Works:}
            \begin{enumerate}\setlength\itemsep{0em}
                \item Create candidate info ($\tilde{C}_t$)
                \item Decide how much to use ($i_t$)
                \item Multiply them together
                \item Add to cell state
            \end{enumerate}

            \vspace{0.2em}
            \textbf{Cell state update:}
            \[
            C_t = \ldots + i_t \odot \tilde{C}_t
            \]
        \end{column}
    \end{columns}
\end{frame}

% Slide 6: Gate 3 - Output (Explain Row 8)
\begin{frame}{Gate 3: Output - What to Share?}
    \begin{center}
        \includegraphics[width=0.7\textwidth]{../figures/output_gate_detail_bsc.pdf}
    \end{center}

    \vspace{-0.5em}
    \scriptsize
    \begin{columns}[T]
        \begin{column}{0.48\textwidth}
            \textbf{Remember Our Table?}

            Row 8: ``was'' had \textbf{Output = 0.9}

            \vspace{0.3em}
            \textbf{What This Means:}
            \begin{itemize}\setlength\itemsep{0em}
                \item 0.0 = output nothing
                \item 1.0 = output everything
                \item 0.9 = output most of memory
            \end{itemize}

            \vspace{0.3em}
            \textbf{Why at ``was''?}
            \begin{itemize}\setlength\itemsep{0em}
                \item Need to predict next word
                \item Subject info is relevant
                \item ``dog'' determines verb form
            \end{itemize}
        \end{column}

        \begin{column}{0.48\textwidth}
            \textbf{Formulas:}
            \[
            o_t = \sigma(W_o \cdot [h_{t-1}, x_t] + b_o)
            \]
            \[
            h_t = o_t \odot \tanh(C_t)
            \]

            \vspace{0.2em}
            \textbf{How It Works:}
            \begin{enumerate}\setlength\itemsep{0em}
                \item Look at cell state (has ``dog'')
                \item Decide what's relevant now
                \item Filter memory through gate
                \item Send $h_t$ to next layer
            \end{enumerate}

            \vspace{0.2em}
            \textbf{Key:} Cell state stays protected, but we share filtered version
        \end{column}
    \end{columns}
\end{frame}

% Slide 7: Architecture Overview - How Gates Work Together
\begin{frame}{The Big Picture: Three Gates + Cell State}
    \begin{center}
        \includegraphics[width=0.6\textwidth]{../figures/lstm_architecture_overview_bsc.pdf}
    \end{center}

    \vspace{-0.6em}
    \scriptsize
    \begin{columns}[T]
        \begin{column}{0.48\textwidth}
            \textbf{Cell State Highway:}
            \begin{itemize}\setlength\itemsep{-0.1em}
                \item Protected memory
                \item Info flows easily
                \item Gates control entry/exit
                \item Gradients don't vanish!
            \end{itemize}

            \vspace{0.1em}
            \begin{intuition}[Analogy]
            \begin{itemize}\setlength\itemsep{-0.1em}
                \item \textcolor{red}{Forget} = Eraser
                \item \textcolor{blue}{Input} = Pen
                \item \textcolor{green}{Output} = Highlighter
            \end{itemize}
            \end{intuition}
        \end{column}

        \begin{column}{0.48\textwidth}
            \textbf{All Three Work Together:}

            At each word:
            \begin{enumerate}\setlength\itemsep{-0.1em}
                \item \textbf{Forget:} Remove old (0.1 at ``.'' = erase ``cat'')
                \item \textbf{Input:} Add new (0.9 at ``dog'' = add subject)
                \item \textbf{Output:} Share (0.9 at ``was'' = use ``dog'')
            \end{enumerate}

            \vspace{0.1em}
            \textbf{Result:} Memory evolves!
        \end{column}
    \end{columns}
\end{frame}

% Slide 8: CORE REVISITED - Now You Understand!
\begin{frame}{Now Let's Look Again - You Understand It!}
    \textbf{Sentence:} ``The cat was hungry. The dog was sleeping.''

    \vspace{0.3em}
    \small
    \begin{center}
    \renewcommand{\arraystretch}{1.2}
    \begin{tabular}{|c|l|l|l|l|}
    \hline
    \textbf{Word} & \textbf{Forget} & \textbf{Input} & \textbf{Output} & \textbf{Memory} \\
    \hline
    ``The'' & 0.9 (keep) & 0.3 (small add) & 0.2 (hide) & [article] \\
    \hline
    ``cat'' & 0.8 (keep) & 0.9 (add subject!) & 0.8 (show) & [cat] \\
    \hline
    ``was'' & 0.9 (keep) & 0.7 (add verb) & 0.9 (need it!) & [cat, was] \\
    \hline
    ``hungry'' & 0.8 (keep) & 0.8 (add state) & 0.7 (show) & [cat, hungry] \\
    \hline
    \rowcolor{yellow!20}
    ``.'' & \textbf{0.1 (ERASE!)} & 0.4 (ending) & 0.3 (hide) & [end] \\
    \hline
    ``The'' & 0.1 (clear old) & 0.8 (new start) & 0.2 (hide) & [article] \\
    \hline
    \rowcolor{green!20}
    ``dog'' & 0.7 (keep) & \textbf{0.9 (NEW subject!)} & 0.9 (show) & [dog] \\
    \hline
    \rowcolor{blue!20}
    ``was'' & 0.9 (keep) & 0.8 (add verb) & \textbf{0.9 (USE dog!)} & [dog, was] \\
    \hline
    \end{tabular}
    \end{center}

    \vspace{0.5em}
    \begin{checkpoint}[The Critical Transition]
    Watch rows 4-7: ``hungry'' $\rightarrow$ ``.'' $\rightarrow$ ``The'' $\rightarrow$ ``dog''

    \textbf{Memory evolves:} [cat, hungry] $\rightarrow$ \textcolor{red}{FORGET} $\rightarrow$ [end] $\rightarrow$ \textcolor{green}{ADD} $\rightarrow$ [dog]

    This is what RNNs cannot do! LSTM uses gates to control memory intelligently.
    \end{checkpoint}
\end{frame}

% Slide 9: Mathematical Detail
\begin{frame}{The Complete Mathematics}
    \begin{columns}[T]
        \begin{column}{0.48\textwidth}
            \textbf{All Three Gates:}

            \vspace{0.5em}
            \textbf{Forget Gate:}
            \[
            f_t = \sigma(W_f \cdot [h_{t-1}, x_t] + b_f)
            \]

            \textbf{Input Gate:}
            \[
            i_t = \sigma(W_i \cdot [h_{t-1}, x_t] + b_i)
            \]
            \[
            \tilde{C}_t = \tanh(W_C \cdot [h_{t-1}, x_t] + b_C)
            \]

            \textbf{Output Gate:}
            \[
            o_t = \sigma(W_o \cdot [h_{t-1}, x_t] + b_o)
            \]
        \end{column}

        \begin{column}{0.48\textwidth}
            \textbf{Cell State Update:}
            \[
            C_t = f_t \odot C_{t-1} + i_t \odot \tilde{C}_t
            \]

            \textbf{Output:}
            \[
            h_t = o_t \odot \tanh(C_t)
            \]

            \vspace{1em}
            \textbf{In Our Example:}
            \begin{itemize}
                \item $x_t$ = current word embedding
                \item $h_{t-1}$ = previous output
                \item $C_t$ = cell state (memory)
                \item $\sigma$ = sigmoid (0 to 1)
                \item $\tanh$ = tanh (-1 to 1)
                \item $\odot$ = element-wise multiply
            \end{itemize}
        \end{column}
    \end{columns}
\end{frame}

% Slide 10: Summary
\begin{frame}{Summary: LSTM in Practice}
    \begin{columns}[T]
        \begin{column}{0.48\textwidth}
            \textbf{What We Learned:}
            \begin{enumerate}
                \item LSTM uses three gates to control memory
                \item Forget gate: what to erase (0.1 at ``.'' = erase ``cat'')
                \item Input gate: what to add (0.9 at ``dog'' = add subject)
                \item Output gate: what to use (0.9 at ``was'' = use ``dog'')
                \item Cell state flows information forward
            \end{enumerate}

            \vspace{1em}
            \textbf{When to Use LSTM:}
            \begin{itemize}
                \item Long sequences (100+ words)
                \item Long-term dependencies
                \item Context matters
                \item Grammar and structure
            \end{itemize}
        \end{column}

        \begin{column}{0.48\textwidth}
            \begin{realworld}[Applications]
            \textbf{Where LSTMs Excel:}
            \begin{itemize}
                \item Machine Translation (Google Translate)
                \item Speech Recognition (Siri, Alexa)
                \item Text Generation (ChatGPT foundations)
                \item Video Analysis
                \item Music Generation
                \item Handwriting Recognition
            \end{itemize}
            \end{realworld}

            \vspace{1em}
            \textbf{Key Takeaway:}

            The sentence example showed you \textit{exactly} how LSTM gates work in practice. That's the real magic!
        \end{column}
    \end{columns}

    \vspace{1em}
    \begin{center}
        \Large \textbf{Questions?}
    \end{center}
\end{frame}

\end{document}
