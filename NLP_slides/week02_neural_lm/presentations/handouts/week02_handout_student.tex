\documentclass[11pt,a4paper]{article}
\usepackage[margin=1in]{geometry}
\usepackage{amsmath,amssymb}
\usepackage{graphicx}
\usepackage{enumitem}
\usepackage{tcolorbox}
\usepackage{array}
\usepackage{multirow}
\usepackage{tikz}
\usetikzlibrary{positioning,arrows.meta,shapes}

% Custom commands
\newcommand{\highlight}[1]{\textbf{#1}}

% Box for exercises
\newtcolorbox{exercise}[1][]{
    colback=blue!5!white,
    colframe=blue!75!black,
    title=#1,
    fonttitle=\bfseries
}

\newtcolorbox{hint}[1][]{
    colback=yellow!10!white,
    colframe=orange!75!black,
    title=Hint,
    fonttitle=\bfseries
}

\title{\textbf{Week 2: Neural Language Models}\\
\large Discovering Word Meanings Through Context\\
\large Pre-Lab Exercise (No Programming Required)}
\author{NLP Course 2025 - Student Version}
\date{}

\begin{document}
\maketitle

\noindent\textbf{Time:} 30-40 minutes\\
\textbf{Objective:} Understand the core concepts behind word embeddings and neural language models through hands-on discovery.

\section*{Part 1: Context Discovery (10 minutes)}

\begin{exercise}[The Mystery Word ``Glork'']
Read the following sentences carefully. The word ``glork'' is not a real English word, but you should be able to figure out what it means from context.

\begin{enumerate}[label=\alph*)]
    \item The glork meowed loudly at night, keeping everyone awake.
    \item I need to feed my glork before leaving for work.
    \item The glork chased the mouse across the kitchen floor.
    \item My neighbor has three glorks, all different colors.
    \item The veterinarian said my glork is perfectly healthy.
    \item The glork purred contentedly while sitting on my lap.
\end{enumerate}

\textbf{Questions:}
\begin{enumerate}
    \item What is a ``glork''? \rule{3cm}{0.4pt}
    \item List three words that helped you figure this out: 
    \begin{itemize}
        \item \rule{3cm}{0.4pt}
        \item \rule{3cm}{0.4pt}
        \item \rule{3cm}{0.4pt}
    \end{itemize}
    \item This demonstrates the \textbf{distributional hypothesis}: ``You shall know a word by the company it keeps.'' 
    
    In your own words, what does this mean?
    
    \vspace{2cm}
\end{enumerate}
\end{exercise}

\section*{Part 2: Word Similarity Matrix (15 minutes)}

\begin{exercise}[Building a Similarity Space]
Rate the similarity between each pair of words on a scale from 0 (completely unrelated) to 10 (nearly identical). Fill in the matrix below:

\begin{center}
\begin{tabular}{|c||c|c|c|c|c|c|c|c|}
\hline
 & king & queen & man & woman & Paris & France & Berlin & Germany \\
\hline\hline
king & 10 & & & & & & & \\
\hline
queen & & 10 & & & & & & \\
\hline
man & & & 10 & & & & & \\
\hline
woman & & & & 10 & & & & \\
\hline
Paris & & & & & 10 & & & \\
\hline
France & & & & & & 10 & & \\
\hline
Berlin & & & & & & & 10 & \\
\hline
Germany & & & & & & & & 10 \\
\hline
\end{tabular}
\end{center}

\textbf{Analysis Questions:}
\begin{enumerate}
    \item Which word pair has the highest similarity (excluding identical words)? \rule{4cm}{0.4pt}
    \item Do you notice any patterns? For example, are certain groups of words more similar to each other?
    
    \vspace{2cm}
    
    \item If you subtract your ``man'' ratings from your ``king'' ratings, what pattern emerges?
    
    \vspace{1.5cm}
\end{enumerate}
\end{exercise}

\newpage

\section*{Part 3: The Dimension Problem (10 minutes)}

\begin{exercise}[Describing Words with Numbers]
Imagine you need to describe animals using only numbers (like coordinates in space). Each property becomes a dimension.

\textbf{Task:} Rate these animals on each dimension (0-10):

\begin{center}
\begin{tabular}{|l||c|c|c|c|c|}
\hline
Animal & Size & Friendliness & Domesticated & Dangerous & Flying ability \\
\hline\hline
Cat & & & & & \\
\hline
Dog & & & & & \\
\hline
Lion & & & & & \\
\hline
Eagle & & & & & \\
\hline
Goldfish & & & & & \\
\hline
\end{tabular}
\end{center}

\textbf{Questions:}
\begin{enumerate}
    \item Which two animals are most similar based on your numbers? Calculate by finding the pair with the smallest total difference across all dimensions.
    
    \vspace{2cm}
    
    \item How many dimensions would you need to perfectly distinguish between all animals in the world? Circle one:
    \begin{itemize}
        \item[$\square$] 5-10
        \item[$\square$] 50-100
        \item[$\square$] 100-500
        \item[$\square$] 1000+
    \end{itemize}
    
    \item Word2Vec typically uses 100-300 dimensions. Why might this be enough even though there are hundreds of thousands of words?
    
    \vspace{2cm}
\end{enumerate}
\end{exercise}

\section*{Part 4: Word Arithmetic Discovery (10 minutes)}

\begin{exercise}[Vector Arithmetic with Words]
If words are points in space (with many dimensions), we can do arithmetic with them!

\begin{hint}
Think of relationships as directions in space. The direction from ``man'' to ``king'' might be similar to the direction from ``woman'' to...?
\end{hint}

\textbf{Analogies as Arithmetic:}
\begin{enumerate}
    \item If we compute: \textbf{king} - \textbf{man} + \textbf{woman} = ?
    
    Your answer: \rule{3cm}{0.4pt}
    
    \item If we compute: \textbf{Paris} - \textbf{France} + \textbf{Germany} = ?
    
    Your answer: \rule{3cm}{0.4pt}
    
    \item Create your own word arithmetic problem:
    
    \rule{2cm}{0.4pt} - \rule{2cm}{0.4pt} + \rule{2cm}{0.4pt} = \rule{2cm}{0.4pt}
    
    \item Why does this work? What does the subtraction capture?
    
    \vspace{2cm}
\end{enumerate}
\end{exercise}

\section*{Part 5: Reflection Questions (5 minutes)}

\begin{enumerate}
    \item \textbf{Why Numbers?} Why would representing words as numbers (vectors) help computers understand language?
    
    \vspace{2cm}
    
    \item \textbf{Ambiguity Problem:} The word ``bank'' can mean a financial institution or the side of a river. How does this complicate our vector representation? What might be a solution?
    
    \vspace{2cm}
    
    \item \textbf{Teaching Relationships:} If you had to teach a computer that ``puppy'' and ``dog'' are related (without explicitly programming rules), how would our context-based approach help?
    
    \vspace{2cm}
\end{enumerate}

\vspace{1cm}
\noindent\rule{\textwidth}{0.4pt}
\begin{center}
\textit{End of Student Exercise}
\end{center}

\end{document}