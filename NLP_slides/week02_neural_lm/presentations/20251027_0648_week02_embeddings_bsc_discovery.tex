% Week 2: Word Embeddings - Neural Language Models
% BSc Discovery-Based Presentation - Two-Tier Structure
% 20 concise main slides + 15 deep appendix slides

% Master Template for NLP Course
% Optimal Readability Layout Standard
% All presentations should include this template

\documentclass[8pt,aspectratio=169]{beamer}
\usetheme{Madrid}
\setbeamertemplate{navigation symbols}{}

% ====================================
% OPTIMAL READABILITY COLOR PALETTE
% ====================================
\definecolor{PureBlack}{HTML}{000000}      % Main text (21:1 contrast)
\definecolor{DeepBlue}{HTML}{003D7A}       % Primary accent (12.6:1 contrast)
\definecolor{DarkGray}{HTML}{4A4A4A}       % Secondary text (9.7:1 contrast)
\definecolor{LightGray}{HTML}{E5E5E5}      % Borders and grids
\definecolor{ChartBlue}{HTML}{0066CC}      % Chart primary
\definecolor{ChartOrange}{HTML}{FF8800}    % Chart secondary
\definecolor{ChartTeal}{HTML}{00A0A0}      % Chart tertiary
\definecolor{ChartPurple}{HTML}{8B4789}    % Chart quaternary
\definecolor{DarkGreen}{HTML}{228B22}      % Success/positive
\definecolor{DarkRed}{HTML}{CC0000}        % Warning/negative

% ====================================
% BEAMER COLOR CONFIGURATION
% ====================================
\setbeamercolor{structure}{fg=PureBlack}
\setbeamercolor{frametitle}{fg=PureBlack,bg=white}
\setbeamercolor{title}{fg=PureBlack,bg=white}
\setbeamercolor{subtitle}{fg=DarkGray}
\setbeamercolor{author}{fg=DarkGray}
\setbeamercolor{date}{fg=DarkGray}
\setbeamercolor{institute}{fg=DarkGray}

% Blocks - no backgrounds
\setbeamercolor{block title}{fg=PureBlack,bg=white}
\setbeamercolor{block body}{fg=PureBlack,bg=white}
\setbeamercolor{block title example}{fg=DarkGreen,bg=white}
\setbeamercolor{block body example}{fg=PureBlack,bg=white}
\setbeamercolor{block title alerted}{fg=DarkRed,bg=white}
\setbeamercolor{block body alerted}{fg=PureBlack,bg=white}

% Lists
\setbeamercolor{item}{fg=DeepBlue}
\setbeamercolor{subitem}{fg=ChartBlue}
\setbeamercolor{enumerate item}{fg=DeepBlue}
\setbeamercolor{enumerate subitem}{fg=ChartBlue}

% Text
\setbeamercolor{normal text}{fg=PureBlack,bg=white}
\setbeamercolor{alerted text}{fg=DarkRed}
\setbeamercolor{example text}{fg=DarkGreen}

% Footer
\setbeamercolor{footline}{fg=DarkGray,bg=white}
\setbeamercolor{page number in head/foot}{fg=DarkGray}

% ====================================
% FONT CONFIGURATION
% ====================================
\setbeamerfont{normal text}{size=\normalsize}
\setbeamerfont{frametitle}{size=\Large,series=\bfseries}
\setbeamerfont{title}{size=\huge,series=\bfseries}
\setbeamerfont{subtitle}{size=\large}
\setbeamerfont{author}{size=\normalsize}
\setbeamerfont{date}{size=\small}
\setbeamerfont{institute}{size=\small}

% ====================================
% REQUIRED PACKAGES
% ====================================
\usepackage{tikz}
\usepackage{amsmath}
\usepackage{amssymb}
\usepackage{booktabs}
\usepackage{graphicx}
\usepackage{array}
\usepackage{listings}
\usepackage{algorithm2e}
\usepackage{xcolor}
\usepackage{tabularx}
\usepackage{multirow}
\usepackage{subcaption}

% ====================================
% CUSTOM COMMANDS
% ====================================
% Text highlighting commands
\newcommand{\highlight}[1]{\textcolor{DeepBlue}{\textbf{#1}}}
\newcommand{\secondary}[1]{\textcolor{DarkGray}{#1}}
\newcommand{\success}[1]{\textcolor{DarkGreen}{#1}}
\newcommand{\warning}[1]{\textcolor{DarkRed}{#1}}
\newcommand{\data}[1]{\textcolor{ChartBlue}{#1}}
\newcommand{\dataalt}[1]{\textcolor{ChartOrange}{#1}}

% Mathematical notation
\newcommand{\given}{\mid}
\newcommand{\prob}[1]{P(#1)}
\newcommand{\argmax}{\operatorname*{argmax}}
\newcommand{\argmin}{\operatorname*{argmin}}
\newcommand{\softmax}{\operatorname{softmax}}

% Box commands for emphasis
\newcommand{\keypoint}[1]{%
  \begin{center}
  \fbox{\parbox{0.9\textwidth}{\centering\textbf{#1}}}
  \end{center}
}

\newcommand{\formula}[1]{%
  \begin{center}
  \colorbox{LightGray}{\parbox{0.8\textwidth}{\centering$\displaystyle #1$}}
  \end{center}
}

% ====================================
% LISTINGS CONFIGURATION
% ====================================
\lstset{
  basicstyle=\ttfamily\small,
  keywordstyle=\color{DeepBlue}\bfseries,
  commentstyle=\color{DarkGray}\itshape,
  stringstyle=\color{ChartOrange},
  numbers=left,
  numberstyle=\tiny\color{DarkGray},
  stepnumber=1,
  numbersep=5pt,
  backgroundcolor=\color{white},
  showspaces=false,
  showstringspaces=false,
  showtabs=false,
  frame=single,
  frameround=tttt,
  rulecolor=\color{LightGray},
  tabsize=2,
  captionpos=b,
  breaklines=true,
  breakatwhitespace=true,
  language=Python,
  escapeinside={(*@}{@*)},
  morekeywords={self, yield, assert, with, as}
}

% ====================================
% STANDARD SLIDE LAYOUTS
% ====================================

% Two-column layout with title
\newcommand{\twocolslide}[3]{%
  \begin{frame}{#1}
  \begin{columns}[T]
  \column{0.48\textwidth}
  #2
  \column{0.48\textwidth}
  #3
  \end{columns}
  \end{frame}
}

% Three-column layout
\newcommand{\threecolslide}[4]{%
  \begin{frame}{#1}
  \begin{columns}[T]
  \column{0.32\textwidth}
  #2
  \column{0.32\textwidth}
  #3
  \column{0.32\textwidth}
  #4
  \end{columns}
  \end{frame}
}

% Chart slide with caption
\newcommand{\chartslide}[3]{%
  \begin{frame}{#1}
  \begin{center}
  \includegraphics[width=#2\textwidth]{#3}
  \end{center}
  \end{frame}
}

% Full chart slide - optimized to 0.85 for proper margins
\newcommand{\fullchartslide}[2]{%
  \begin{frame}{#1}
  \begin{center}
  \includegraphics[width=0.85\textwidth]{#2}
  \end{center}
  \end{frame}
}

% Code slide
\newcommand{\codeslide}[3]{%
  \begin{frame}[fragile]{#1}
  \begin{lstlisting}[language=#2]
#3
  \end{lstlisting}
  \end{frame}
}

% Concept slide with figure
\newcommand{\conceptslide}[3]{%
  \begin{frame}{#1}
  \begin{columns}[T]
  \column{0.6\textwidth}
  #2
  \column{0.35\textwidth}
  \begin{center}
  \includegraphics[width=0.95\textwidth]{#3}
  \end{center}
  \end{columns}
  \end{frame}
}

% Table slide
\newcommand{\tableslide}[2]{%
  \begin{frame}{#1}
  \begin{center}
  \Large
  \renewcommand{\arraystretch}{1.5}
  #2
  \end{center}
  \end{frame}
}

% Summary slide
\newcommand{\summaryslide}[2]{%
  \begin{frame}{#1}
  \begin{center}
  \Large
  #2
  \end{center}
  \vfill
  \begin{center}
  \keypoint{Key Takeaway}
  \end{center}
  \end{frame}
}

% ====================================
% REMOVE DECORATIONS
% ====================================
\setbeamertemplate{blocks}[default]
\setbeamertemplate{title page}[default][colsep=-4bp,rounded=false]
\setbeamertemplate{itemize items}[circle]
\setbeamertemplate{enumerate items}[default]
\setbeamertemplate{section in toc}[sections numbered]
\setbeamertemplate{subsection in toc}[subsections numbered]

% ====================================
% PAGE NUMBERING
% ====================================
\setbeamertemplate{footline}{
  \leavevmode%
  \hbox{%
  \begin{beamercolorbox}[wd=.333333\paperwidth,ht=2.25ex,dp=1ex,center]{author in head/foot}%
    \usebeamerfont{author in head/foot}\secondary{\insertshortauthor}
  \end{beamercolorbox}%
  \begin{beamercolorbox}[wd=.333333\paperwidth,ht=2.25ex,dp=1ex,center]{title in head/foot}%
    \usebeamerfont{title in head/foot}\secondary{\insertshorttitle}
  \end{beamercolorbox}%
  \begin{beamercolorbox}[wd=.333333\paperwidth,ht=2.25ex,dp=1ex,right]{date in head/foot}%
    \usebeamerfont{date in head/foot}\secondary{\insertshortdate{}\hspace*{2em}
    \insertframenumber{} / \inserttotalframenumber\hspace*{2ex}}
  \end{beamercolorbox}}%
  \vskip0pt%
}

% ====================================
% TABLE OF CONTENTS STYLE
% ====================================
\setbeamertemplate{section in toc}{%
  \leavevmode\leftskip=1.5em%
  \llap{%
    \usebeamerfont{section in toc}%
    \usebeamercolor[fg]{section in toc}%
    \inserttocsectionnumber.%
  }%
  \usebeamerfont{section in toc}%
  \usebeamercolor[fg]{section in toc}%
  \inserttocsection\par%
}

\setbeamertemplate{subsection in toc}{%
  \leavevmode\leftskip=3em%
  \llap{%
    \usebeamerfont{subsection in toc}%
    \usebeamercolor[fg]{subsection in toc}%
    \inserttocsectionnumber.\inserttocsubsectionnumber%
  }%
  \usebeamerfont{subsection in toc}%
  \usebeamercolor[fg]{subsection in toc}%
  \inserttocsubsection\par%
}

% ====================================
% END OF MASTER TEMPLATE
% ====================================

% Define bottomnote command
\newcommand{\bottomnote}[1]{%
    \vspace{0.2cm}
    \begin{center}
    \footnotesize\secondary{#1}
    \end{center}
}

\title{Word Embeddings}
\subtitle{\secondary{Week 2 - When Words Became Vectors}}
\author{NLP Course 2025}
\date{October 27, 2025}

\begin{document}

% ===== MAIN PRESENTATION (20 SLIDES) =====

% Slide 1: Title
\begin{frame}
\titlepage
\vfill
\begin{center}
\secondary{\footnotesize Two-Tier BSc Discovery Presentation}
\end{center}
\end{frame}

% ===== OPENING: DISCOVERY HOOK CASCADE (Slides 2-5) =====

% Slide 2: Hook #1 - Word Arithmetic
\begin{frame}[t]{Hook \#1: Words That Do Algebra}
\vspace{-0.3cm}
\begin{center}
\includegraphics[width=0.65\textwidth]{../figures/word_arithmetic_3d_bsc.pdf}
\end{center}

\begin{center}
\Large
\textbf{king - man + woman = queen}
\end{center}

\vspace{3mm}
\begin{center}
\textbf{Key Insight}: Embeddings capture semantic relationships geometrically
\end{center}

\vspace{0.2cm}
\bottomnote{How can words do algebra? The answer changes everything in NLP}
\end{frame}

% Slide 3: Hook #2 - Semantic Similarity
\begin{frame}[t]{Hook \#2: Similarity That N-grams Miss}
\vspace{-0.3cm}
\begin{center}
\includegraphics[width=0.7\textwidth]{../figures/similarity_clustering_2d_bsc.pdf}
\end{center}

\begin{center}
\textbf{Key Insight}: Similar words cluster together in embedding space
\end{center}

\vspace{0.2cm}
\bottomnote{N-grams treat all words equally distant - embeddings capture meaning}
\end{frame}

% Slide 4: Hook #3 - Dimensionality Paradox
\begin{frame}[t]{Hook \#3: Compression That Improves Quality}
\vspace{-0.3cm}
\begin{center}
\includegraphics[width=0.7\textwidth]{../figures/dimensionality_comparison_bsc.pdf}
\end{center}

\begin{center}
\textbf{Key Insight}: 50,000 sparse $\rightarrow$ 300 dense with MORE information
\end{center}

\vspace{0.2cm}
\bottomnote{Dense representations are more powerful than sparse ones}
\end{frame}

% Slide 5: Hook #4 - Context Dependence
\begin{frame}[t]{Hook \#4: Words With Multiple Meanings}
\vspace{-0.3cm}
\begin{center}
\includegraphics[width=0.65\textwidth]{../figures/context_dependent_bank_bsc.pdf}
\end{center}

\begin{center}
\textbf{Key Insight}: Static embeddings average meanings (limitation to address later)
\end{center}

\vspace{0.2cm}
\bottomnote{``bank'' has one vector - foreshadows contextualized embeddings (Week 6)}
\end{frame}

% ===== FOUNDATION (Slides 6-9) =====

% Slide 6: What Are Word Embeddings?
\begin{frame}[t]{What Are Word Embeddings?}
\vspace{-0.3cm}
\begin{center}
\includegraphics[width=0.65\textwidth]{../figures/word_as_vector_concept_bsc.pdf}
\end{center}

\begin{center}
\textbf{Definition}: Dense, low-dimensional, continuous vector representations of words
\end{center}

\vspace{0.2cm}
\bottomnote{Words become points in semantic space}
\end{frame}

% Slide 7: From One-Hot to Dense
\begin{frame}[t]{From Sparse One-Hot to Dense Embeddings}
\small
\begin{columns}[T]
\column{0.48\textwidth}
\raggedright
\textbf{One-Hot Encoding (Old Way)}:

Each word = vector of size $|V|$

\vspace{3mm}
Example ($V$ = 5):
\begin{itemize}
\item ``cat'' = [1, 0, 0, 0, 0]
\item ``dog'' = [0, 1, 0, 0, 0]
\item ``mat'' = [0, 0, 1, 0, 0]
\end{itemize}

\vspace{3mm}
\textbf{Problems}:
\begin{itemize}
\item Huge dimensionality (50K typical)
\item All words equally distant
\item No similarity information
\item Sparse (99.998\% zeros)
\end{itemize}

\column{0.48\textwidth}
\raggedright
\textbf{Dense Embeddings (New Way)}:

Each word = vector of size $d$ (300 typical)

\vspace{3mm}
Example ($d$ = 3):
\begin{itemize}
\item ``cat'' = [0.2, 0.8, -0.3]
\item ``dog'' = [0.1, 0.7, -0.2]
\item ``mat'' = [-0.5, 0.1, 0.6]
\end{itemize}

\vspace{3mm}
\textbf{Advantages}:
\begin{itemize}
\item Low dimensionality (100-300)
\item Similarity encoded (cat $\approx$ dog)
\item Continuous values
\item Information-dense
\end{itemize}

\end{columns}

\vspace{0.2cm}
\bottomnote{Dense $<$ Sparse but contains MORE information - this is the magic}
\end{frame}

% Slide 8: Vector Space Model
\begin{frame}[t]{The Vector Space Model}
\vspace{-0.3cm}
\begin{center}
\includegraphics[width=0.7\textwidth]{../figures/semantic_space_2d_bsc.pdf}
\end{center}

\begin{center}
\textbf{Key Insight}: Semantic similarity = geometric proximity
\end{center}

\vspace{0.2cm}
\bottomnote{Meaning becomes geometry - we can compute with meaning}
\end{frame}

% Slide 9: Why Distributed Representations Work
\begin{frame}[t]{Why Distributed Representations Work}
\vspace{-0.3cm}
\begin{center}
\includegraphics[width=0.65\textwidth]{../figures/distributed_features_bsc.pdf}
\end{center}

\begin{center}
\textbf{Key Insight}: Each dimension captures some semantic feature
\end{center}

\vspace{0.2cm}
\bottomnote{300 dimensions $=$ 300 semantic features combining to represent meaning}
\end{frame}

% ===== WORD2VEC (Slides 10-16) - 7 Slides =====

% Slide 10: The Core Idea
\begin{frame}[t]{Word2Vec: The Core Idea}
\begin{center}
\Large
\textit{``You shall know a word by the company it keeps''}

\vspace{5mm}
\secondary{- J.R. Firth (1957)}
\end{center}

\vspace{10mm}

\textbf{Distributional Hypothesis}:

Words appearing in similar contexts have similar meanings

\vspace{5mm}
\textbf{Word2Vec Approach}:

Learn word vectors by predicting context

\vspace{0.2cm}
\bottomnote{Context prediction forces similar words to have similar vectors}
\end{frame}

% Slides 11-12: Skip-gram Architecture (Dual)
\begin{frame}[t]{Skip-gram: Predict Context from Word}
\vspace{-0.3cm}
\begin{center}
\includegraphics[width=0.7\textwidth]{../figures/skipgram_architecture_bsc.pdf}
\end{center}

\begin{center}
\textbf{Key Insight}: Given center word, predict surrounding words
\end{center}

\vspace{0.2cm}
\bottomnote{This is the most common Word2Vec variant - simple and effective}
\end{frame}

\begin{frame}[t]{Skip-gram: The Architecture}
\small
\begin{columns}[T]
\column{0.48\textwidth}
\raggedright
\textbf{Input}: Center word (one-hot)

$x \in \mathbb{R}^{|V|}$

\vspace{3mm}
\textbf{Hidden Layer}: Embedding lookup

$h = W^T x \in \mathbb{R}^{d}$

This IS the word embedding!

\vspace{3mm}
\textbf{Output Layer}: Context predictions

$y = W' h \in \mathbb{R}^{|V|}$

Softmax for probabilities

\vspace{3mm}
\textbf{Parameters}:
\begin{itemize}
\item $W$: $|V| \times d$ (input embeddings)
\item $W'$: $d \times |V|$ (output weights)
\end{itemize}

\column{0.48\textwidth}
\raggedright
\textbf{Training Objective}:

Maximize probability of context words

$$\max \sum_{t=1}^{T} \sum_{-c \leq j \leq c, j \neq 0} \log P(w_{t+j} | w_t)$$

\vspace{3mm}
\textbf{Example}:

Sentence: ``The cat sat on the mat''

Center: ``cat''

Context window $c=2$:

Predict: ``the'', ``sat''

\vspace{3mm}
\textbf{Key Trick}:

Share embeddings! $W$ is what we keep after training

\end{columns}

\vspace{0.2cm}
\bottomnote{Simple 3-layer network - embeddings are the weights}
\end{frame}

% Slide 13: Worked Example - Skip-gram
\begin{frame}[t]{Worked Example: Skip-gram Forward Pass}
\small
\textbf{Given}: ``The cat sat'', center = ``cat'', predict ``sat''

\vspace{3mm}
\textbf{Step 1}: One-hot encode center word

``cat'' = word ID 3797

$x = [0, 0, ..., 1_{3797}, ..., 0] \in \mathbb{R}^{50000}$

\vspace{3mm}
\textbf{Step 2}: Embedding lookup (hidden layer)

$$h = W^T x = W_{3797} \in \mathbb{R}^{300}$$

This is just row 3797 of $W$! (Lookup, no multiplication needed)

Example: $h = [0.23, -0.41, 0.15, ..., 0.08]$

\vspace{3mm}
\textbf{Step 3}: Compute output scores

$$score(``sat'') = W'_{sat} \cdot h = 0.85$$

\vspace{3mm}
\textbf{Step 4}: Softmax over vocabulary

$$P(``sat'' | ``cat'') = \frac{\exp(0.85)}{\sum_{w \in V} \exp(score(w))} = 0.0023$$

\vspace{3mm}
\textbf{Step 5}: Compute loss

$$Loss = -\log(0.0023) = 6.07$$

\vspace{0.2cm}
\bottomnote{Backprop updates $W$ to increase P(sat|cat) - embeddings improve}
\end{frame}

% Slide 14: The Softmax Problem
\begin{frame}[t]{The Computational Bottleneck}
\vspace{-0.3cm}
\begin{center}
\includegraphics[width=0.65\textwidth]{../figures/softmax_problem_bsc.pdf}
\end{center}

\begin{center}
\textbf{Key Insight}: Computing softmax over 50K words is prohibitively expensive
\end{center}

\vspace{0.2cm}
\bottomnote{Softmax denominator requires summing over entire vocabulary}
\end{frame}

% Slide 15: Negative Sampling Solution
\begin{frame}[t]{Negative Sampling: The Trick That Makes It Practical}
\small
\textbf{The Problem}: Softmax over 50K words per training example

$$P(context | word) = \frac{\exp(score)}{\sum_{w=1}^{50000} \exp(score_w)}$$

Requires 50K exponentials per example!

\vspace{5mm}
\textbf{The Solution}: Negative Sampling

\begin{itemize}
\item 1 positive pair: (``cat'', ``sat'') - actual context
\item $k$ negative pairs: (``cat'', ``xylophone''), (``cat'', ``democracy''), ... - random words
\item Typical: $k=5$ for small datasets, $k=2-5$ for large
\end{itemize}

\vspace{5mm}
\textbf{New Objective}:

$$\log \sigma(v'_{sat} \cdot v_{cat}) + \sum_{i=1}^{k} \log \sigma(-v'_{w_i} \cdot v_{cat})$$

Only $k+1$ computations instead of 50,000!

\vspace{5mm}
\textbf{Example}: Positive pair + 5 negatives = 6 computations vs 50,000

Speedup: \textcolor{green}{\textbf{8,333x faster!}}

\vspace{0.2cm}
\bottomnote{Negative sampling approximates softmax - critical for practical training}
\end{frame}

% Slide 16: CBOW Architecture
\begin{frame}[t]{CBOW: The Reverse Approach}
\vspace{-0.3cm}
\begin{center}
\includegraphics[width=0.7\textwidth]{../figures/cbow_architecture_bsc.pdf}
\end{center}

\begin{center}
\textbf{Key Insight}: Predict word from context (reverse of Skip-gram)
\end{center}

\vspace{3mm}
\small
\begin{columns}[T]
\column{0.48\textwidth}
\raggedright
\textbf{CBOW}: Continuous Bag of Words

Given: ``the'', ``sat''

Predict: ``cat''

\textbf{When to use}: Faster, good for frequent words

\column{0.48\textwidth}
\raggedright
\textbf{Skip-gram}:

Given: ``cat''

Predict: ``the'', ``sat''

\textbf{When to use}: Better for rare words, more common
\end{columns}

\vspace{0.2cm}
\bottomnote{Both work - Skip-gram slightly better empirically}
\end{frame}

% ===== INTEGRATION (Slides 17-20) =====

% Slide 17: Evaluation - Analogy Tasks
\begin{frame}[t]{Evaluation: Word Analogies}
\vspace{-0.3cm}
\begin{center}
\includegraphics[width=0.65\textwidth]{../figures/analogy_results_bsc.pdf}
\end{center}

\begin{center}
\textbf{Key Insight}: Good embeddings solve analogies via vector arithmetic
\end{center}

\vspace{0.2cm}
\bottomnote{king:queen :: man:woman achieved 72\% accuracy (Word2Vec, 2013)}
\end{frame}

% Slide 18: Pre-trained Embeddings
\begin{frame}[t]{Pre-trained Embeddings}
\vspace{-0.3cm}
\begin{center}
\includegraphics[width=0.7\textwidth]{../figures/pretrained_comparison_bsc.pdf}
\end{center}

\begin{center}
\textbf{Key Insight}: Use pre-trained embeddings as starting point
\end{center}

\vspace{0.2cm}
\bottomnote{Word2Vec, GloVe, FastText - all available pre-trained}
\end{frame}

% Slide 19: Key Takeaways
\begin{frame}[t]{Key Takeaways}
\begin{enumerate}
\item \textbf{Embeddings as dense vectors}

  Words $\rightarrow$ continuous vectors in $\mathbb{R}^d$ (typically $d=300$)

\item \textbf{Skip-gram predicts context from word}

  Train by predicting surrounding words in large corpus

\item \textbf{Negative sampling enables efficient training}

  Approximate softmax with $k$ negative samples (8000x speedup)

\item \textbf{Geometric semantics}

  Similarity = cosine, analogies = vector arithmetic

\item \textbf{Foundation for neural NLP}

  All neural models start with embedding layer
\end{enumerate}

\vspace{0.2cm}
\bottomnote{Embeddings revolutionized NLP in 2013 - still used everywhere today}
\end{frame}

% Slide 20: Lab Preview
\begin{frame}[t]{Next: Visual Exploration}
\begin{columns}[T]
\column{0.48\textwidth}
\raggedright
\textbf{Lab Activities}:

\begin{itemize}
\item Load pre-trained Word2Vec
\item Visualize in 2D and 3D
\item Perform word arithmetic
\item Find most similar words
\item Explore semantic clusters
\item Compare Word2Vec vs GloVe
\item Visualize training evolution
\end{itemize}

\column{0.48\textwidth}
\raggedright
\textbf{Visualizations You'll Create}:

\begin{itemize}
\item 2D PCA projections
\item Interactive 3D scatter plots
\item Analogy arrows in 2D
\item Similarity heatmaps
\item Semantic cluster plots
\item Training convergence
\end{itemize}

\vspace{3mm}
\textbf{Goal}: Build intuition through visualization

\end{columns}

\vspace{3mm}
\begin{center}
\textbf{See embeddings come alive!}
\end{center}

\vspace{0.2cm}
\bottomnote{Understanding embeddings requires seeing them - lab is essential}
\end{frame}

% ===== TECHNICAL APPENDIX (15 SLIDES) =====

\begin{frame}[t]{}
\begin{center}
\Huge\textbf{Technical Appendix}

\vspace{5mm}
\Large\secondary{Complete Mathematical Treatment}
\end{center}
\end{frame}

% ===== WORD2VEC MATHEMATICS (A1-A5) =====

% Slide A1: Skip-gram Objective Function
\begin{frame}[t]{Appendix A1: Skip-gram Objective Function}
\small
\textbf{Goal}: Maximize probability of context words given center word

\vspace{3mm}
\textbf{Objective}:

$$\mathcal{L} = \frac{1}{T} \sum_{t=1}^{T} \sum_{\substack{-c \leq j \leq c \\ j \neq 0}} \log P(w_{t+j} | w_t)$$

where:
\begin{itemize}
\item $T$: Total words in corpus
\item $c$: Context window size (typically 5)
\item $w_t$: Center word at position $t$
\item $w_{t+j}$: Context word at offset $j$
\end{itemize}

\vspace{3mm}
\textbf{Conditional Probability (Naive Softmax)}:

$$P(w_O | w_I) = \frac{\exp(v'_{w_O} \cdot v_{w_I})}{\sum_{w=1}^{|V|} \exp(v'_w \cdot v_{w_I})}$$

where $v_{w_I}$ is input embedding, $v'_{w_O}$ is output embedding

\vspace{3mm}
\textbf{Problem}: Denominator sums over entire vocabulary - $O(|V|)$ per example

For $|V|=50K$, $T=1B$ words, $c=5$: 500 trillion softmax computations!

\vspace{0.2cm}
\bottomnote{This objective is correct but computationally infeasible}
\end{frame}

% Slide A2: Negative Sampling Mathematics
\begin{frame}[t]{Appendix A2: Negative Sampling Mathematics}
\small
\textbf{Key Idea}: Binary classification instead of multi-class

\vspace{3mm}
\textbf{Reformulation}:

Instead of predicting which word from vocabulary,

Predict: Is this word in context (yes/no)?

\vspace{5mm}
\textbf{Negative Sampling Objective}:

$$\log \sigma(v'_{w_O} \cdot v_{w_I}) + \sum_{i=1}^{k} \mathbb{E}_{w_i \sim P_n(w)} [\log \sigma(-v'_{w_i} \cdot v_{w_I})]$$

where:
\begin{itemize}
\item $\sigma(x) = \frac{1}{1+\exp(-x)}$ (sigmoid)
\item $w_O$: Actual context word (positive example)
\item $w_i$: Sampled negative words
\item $P_n(w)$: Noise distribution (typically $P(w)^{0.75}$)
\item $k$: Number of negative samples (5-20)
\end{itemize}

\vspace{5mm}
\textbf{Why $P(w)^{0.75}$}?

Raises probability of rare words, lowers frequent words

More balanced negative sampling

\vspace{0.2cm}
\bottomnote{This approximates softmax with $k$ samples instead of $|V|$ computations}
\end{frame}

% Slide A3: Hierarchical Softmax
\begin{frame}[t]{Appendix A3: Hierarchical Softmax Alternative}
\small
\textbf{Different Approach}: Binary tree instead of flat softmax

\vspace{3mm}
\textbf{Key Idea}:

\begin{itemize}
\item Arrange vocabulary in binary tree
\item Prediction = path through tree
\item Each node: Binary decision (left vs right)
\item Depth: $\log_2 |V|$ decisions instead of $|V|$ computations
\end{itemize}

\vspace{5mm}
\textbf{Complexity}:

\begin{itemize}
\item Naive softmax: $O(|V|)$ per example
\item Hierarchical softmax: $O(\log |V|)$ per example
\item For $|V|=50K$: $O(50000)$ vs $O(16)$ - 3000x speedup!
\end{itemize}

\vspace{5mm}
\textbf{Trade-offs}:

\begin{itemize}
\item \textcolor{green}{Faster than naive softmax}
\item \textcolor{red}{Slower than negative sampling}
\item \textcolor{green}{Exact (no approximation)}
\item \textcolor{red}{Tree construction matters}
\end{itemize}

\vspace{3mm}
\textbf{Usage}: Less common than negative sampling

\vspace{0.2cm}
\bottomnote{Hierarchical softmax elegant but negative sampling more practical}
\end{frame}

% Slide A4: Gradient Descent and Backpropagation
\begin{frame}[t]{Appendix A4: Training via Gradient Descent}
\small
\textbf{Optimization}: Stochastic Gradient Descent (SGD)

\vspace{3mm}
\textbf{Gradients for Skip-gram with Negative Sampling}:

For positive pair $(w_I, w_O)$:

$$\frac{\partial}{\partial v_{w_I}} = (1 - \sigma(v'_{w_O} \cdot v_{w_I})) v'_{w_O}$$

For negative pairs $(w_I, w_i)$:

$$\frac{\partial}{\partial v_{w_I}} = -\sigma(-v'_{w_i} \cdot v_{w_I}) v'_{w_i}$$

\vspace{5mm}
\textbf{Update Rule}:

$$v_{w_I}^{new} = v_{w_I}^{old} - \eta \cdot \frac{\partial L}{\partial v_{w_I}}$$

where $\eta$ is learning rate (typically 0.025, decays to 0.0001)

\vspace{5mm}
\textbf{Training Details}:

\begin{itemize}
\item Mini-batch size: 100-1000 word pairs
\item Epochs: 5-15 over corpus
\item Learning rate decay: Linear
\item Convergence: 1-3 days on CPU for 1B words
\end{itemize}

\vspace{0.2cm}
\bottomnote{Simple SGD - no fancy optimizers needed}
\end{frame}

% Slide A5: Computational Complexity
\begin{frame}[t]{Appendix A5: Computational Complexity Analysis}
\small
\begin{center}
\begin{tabular}{lccc}
\toprule
\textbf{Method} & \textbf{Per Example} & \textbf{Training Time} & \textbf{Quality} \\
\midrule
Naive Softmax & $O(|V| \cdot d)$ & Weeks & Best \\
Hierarchical Softmax & $O(\log |V| \cdot d)$ & Days & Good \\
Negative Sampling & $O(k \cdot d)$ & Hours & Good \\
\bottomrule
\multicolumn{4}{l}{\footnotesize Typical: $|V|=50K$, $d=300$, $k=5$, corpus=1B words} \\
\end{tabular}
\end{center}

\vspace{5mm}
\textbf{Memory Requirements}:

\begin{itemize}
\item Embeddings: $|V| \times d \times 4$ bytes = 50K $\times$ 300 $\times$ 4 = 60MB
\item Context matrix: Another 60MB
\item Total: ~120MB (fits in RAM easily)
\end{itemize}

\vspace{5mm}
\textbf{Parallelization}:

\begin{itemize}
\item Word2Vec easily parallelizable (independent windows)
\item Multi-threading: Near-linear speedup
\item GPU: 10-50x faster than CPU
\end{itemize}

\vspace{0.2cm}
\bottomnote{Efficient algorithm + modern hardware = practical at scale}
\end{frame}

% ===== GLOVE (A6-A10) =====

% Slide A6: GloVe - Global Vectors
\begin{frame}[t]{Appendix A6: GloVe - Global Vectors for Word Representation}
\small
\textbf{Different Philosophy}: Explicit matrix factorization

\vspace{3mm}
\textbf{Key Idea}:

\begin{itemize}
\item Word2Vec: Local context window (implicit matrix factorization)
\item GloVe: Global co-occurrence statistics (explicit matrix factorization)
\end{itemize}

\vspace{5mm}
\textbf{Co-occurrence Matrix} $X$:

$$X_{ij} = \text{number of times word } j \text{ appears in context of word } i$$

\vspace{3mm}
Example snippet: Count how often ``cat'' and ``dog'' appear near each other across entire corpus

\vspace{5mm}
\textbf{GloVe Objective}:

$$\mathcal{L} = \sum_{i,j=1}^{|V|} f(X_{ij}) (v_i^T v_j + b_i + b_j - \log X_{ij})^2$$

where $f(x)$ is weighting function (down-weight rare co-occurrences)

\vspace{0.2cm}
\bottomnote{GloVe combines global statistics with local context}
\end{frame}

% Slide A7: GloVe Objective Explained
\begin{frame}[t]{Appendix A7: GloVe Objective Function Breakdown}
\small
\textbf{Goal}: Dot product of vectors should match log co-occurrence

$$v_i^T v_j \approx \log X_{ij}$$

\vspace{5mm}
\textbf{Weighted Least Squares}:

$$\min \sum_{i,j=1}^{|V|} f(X_{ij}) (v_i^T v_j + b_i + b_j - \log X_{ij})^2$$

\vspace{3mm}
\textbf{Weighting Function}:

$$f(x) = \begin{cases}
(x/x_{max})^\alpha & \text{if } x < x_{max} \\
1 & \text{otherwise}
\end{cases}$$

Typical: $x_{max}=100$, $\alpha=0.75$

\vspace{5mm}
\textbf{Why Weighting Matters}:

\begin{itemize}
\item Very rare co-occurrences: Noisy, unreliable
\item Very frequent: Dominate loss (``the the'', ``of the'')
\item Middle ground: Most informative
\end{itemize}

\vspace{0.2cm}
\bottomnote{Weighting function is critical for GloVe performance}
\end{frame}

% Slide A8: Matrix Factorization Perspective
\begin{frame}[t]{Appendix A8: Matrix Factorization Connection}
\small
\textbf{Insight}: Both Word2Vec and GloVe factorize co-occurrence matrix

\vspace{5mm}
\textbf{Pointwise Mutual Information (PMI)}:

$$PMI(i,j) = \log \frac{P(i,j)}{P(i)P(j)} = \log \frac{X_{ij} \cdot |X|}{\sum_k X_{ik} \cdot \sum_k X_{kj}}$$

Measures how much more likely words co-occur than by chance

\vspace{5mm}
\textbf{Connection}:

Word2Vec (Skip-gram with negative sampling) implicitly factorizes shifted PMI matrix:

$$v_i^T v_j \approx PMI(i,j) - \log k$$

GloVe explicitly factorizes log co-occurrence matrix

\vspace{5mm}
\textbf{Unifying View}:

Both methods learn low-rank approximation of word-context statistics

Different loss functions, similar result

\vspace{0.2cm}
\bottomnote{This explains why Word2Vec and GloVe produce similar embeddings}
\end{frame}

% Slide A9: GloVe Training
\begin{frame}[t]{Appendix A9: GloVe Training Algorithm}
\small
\textbf{Steps}:

\begin{enumerate}
\item Build co-occurrence matrix $X$ from corpus (count pairs within window)
\item Initialize word vectors $v_i$ and biases $b_i$ randomly
\item Optimize via AdaGrad:

  For each $(i,j)$ pair with $X_{ij} > 0$:

  $$v_i^{new} = v_i - \eta \frac{\partial L}{\partial v_i}$$

  where gradient includes $f(X_{ij})$ weighting

\item Iterate until convergence (50-100 epochs typical)
\item Final embeddings: $v_i$ (can optionally average with $v_j$)
\end{enumerate}

\vspace{5mm}
\textbf{Hyperparameters}:

\begin{itemize}
\item Embedding dimension $d$: 100-300
\item Context window: 10-15 (larger than Word2Vec's 5)
\item $x_{max}$: 100
\item $\alpha$: 0.75
\item Learning rate: 0.05 with AdaGrad
\end{itemize}

\vspace{3mm}
\textbf{Training Time}: Similar to Word2Vec (hours to days)

\vspace{0.2cm}
\bottomnote{GloVe implementation simpler than Word2Vec (no neural network)}
\end{frame}

% Slide A10: Word2Vec vs GloVe Comparison
\begin{frame}[t]{Appendix A10: Word2Vec vs GloVe - When to Use Each}
\small
\begin{center}
\begin{tabular}{lcc}
\toprule
\textbf{Aspect} & \textbf{Word2Vec} & \textbf{GloVe} \\
\midrule
Method & Local context window & Global co-occurrence \\
Objective & Predict context & Factorize matrix \\
Complexity & $O(k \cdot d)$ per pair & $O(nnz)$ total \\
Training & Online (streaming) & Batch (requires X) \\
Memory & Low (embeddings only) & High (needs matrix) \\
Quality & Excellent & Excellent \\
Speed & Fast & Moderate \\
Rare words & Better (Skip-gram) & Moderate \\
Analogies & 72\% & 75\% \\
Best for & Large corpora, streaming & Small/medium corpora \\
\bottomrule
\end{tabular}
\end{center}

\vspace{5mm}
\textbf{Empirical Results} (on same corpus):

\begin{itemize}
\item Performance: Nearly identical (GloVe 3-5\% better on some tasks)
\item Training time: Word2Vec faster (no matrix construction)
\item Implementation: Word2Vec simpler (fewer hyperparameters)
\end{itemize}

\vspace{3mm}
\textbf{Recommendation}: Start with Word2Vec, try GloVe if you have time

\vspace{0.2cm}
\bottomnote{Both are excellent - choice matters less than proper training}
\end{frame}

% ===== ADVANCED TOPICS (A11-A15) =====

% Slide A11: Implementation Details
\begin{frame}[t]{Appendix A11: Implementation Details and Best Practices}
\small
\textbf{Corpus Preparation}:

\begin{itemize}
\item Lowercase everything (or preserve case)
\item Remove rare words (< 5 occurrences)
\item Subsampling frequent words: $P(w_i) = 1 - \sqrt{t/f(w_i)}$ where $t=10^{-5}$
\item Helps balance frequent/rare words
\end{itemize}

\vspace{5mm}
\textbf{Hyperparameter Choices}:

\begin{center}
\small
\begin{tabular}{lll}
\toprule
\textbf{Parameter} & \textbf{Small Corpus} & \textbf{Large Corpus} \\
\midrule
Embedding dim $d$ & 100-200 & 300-500 \\
Window size $c$ & 3-5 & 5-10 \\
Negative samples $k$ & 5-10 & 2-5 \\
Min word count & 5 & 10 \\
Learning rate $\eta$ & 0.025 & 0.025 \\
Epochs & 10-20 & 5-10 \\
\bottomrule
\end{tabular}
\end{center}

\vspace{5mm}
\textbf{Debugging Tips}:

\begin{itemize}
\item Check: Loss should decrease steadily
\item Test: Run analogies after each epoch
\item Validate: Hold out 10\% for validation
\end{itemize}

\vspace{0.2cm}
\bottomnote{These details matter - bad hyperparameters give poor embeddings}
\end{frame}

% Slide A12: FastText - Subword Embeddings
\begin{frame}[t]{Appendix A12: FastText - Character N-grams}
\small
\textbf{Motivation}: Word2Vec/GloVe ignore word morphology

\vspace{3mm}
\textbf{FastText Innovation} (Facebook AI, 2017):

Represent words as bags of character n-grams

\vspace{3mm}
\textbf{Example}: ``playing'' = <pl, pla, lay, ayi, yin, ing, ng>

\vspace{3mm}
\textbf{Embedding}:

$$v_{playing} = \sum_{g \in ngrams(``playing'')} v_g$$

Sum of character n-gram embeddings

\vspace{5mm}
\textbf{Advantages}:

\begin{itemize}
\item Handle OOV words (unseen in training)
\item Capture morphology (``play'' in ``playing'', ``player'')
\item Better for morphologically rich languages
\item Small vocabulary (can't memorize all words)
\end{itemize}

\vspace{5mm}
\textbf{Trade-offs}:

\begin{itemize}
\item \textcolor{green}{Handles rare/unseen words}
\item \textcolor{red}{More parameters (n-grams)}
\item \textcolor{green}{Morphological awareness}
\item \textcolor{red}{Slower training}
\end{itemize}

\vspace{0.2cm}
\bottomnote{FastText extends Word2Vec - useful for morphology-heavy tasks}
\end{frame}

% Slide A13: ELMo - Contextualized Embeddings Preview
\begin{frame}[t]{Appendix A13: ELMo - Deep Contextualized Embeddings}
\small
\textbf{Limitation of Word2Vec/GloVe}: One vector per word (no context)

``bank'' always has same embedding (averages river and money meanings)

\vspace{5mm}
\textbf{ELMo Solution} (2018):

\begin{itemize}
\item Embeddings from Language Model (ELMo)
\item BiLSTM reads sentence
\item Each word gets different embedding depending on context
\end{itemize}

\vspace{5mm}
\textbf{Example}:

\begin{itemize}
\item ``The bank of the river'' $\rightarrow$ $v_{bank}^{river}$
\item ``Money in the bank'' $\rightarrow$ $v_{bank}^{money}$
\item $v_{bank}^{river} \neq v_{bank}^{money}$
\end{itemize}

\vspace{5mm}
\textbf{Connection to Week 6}:

ELMo $\rightarrow$ BERT $\rightarrow$ GPT progression

All use context to modify embeddings

\vspace{3mm}
\textbf{Note}: Static embeddings (Word2Vec/GloVe) still useful for many tasks

\vspace{0.2cm}
\bottomnote{ELMo bridged static embeddings to contextual (BERT) - important milestone}
\end{frame}

% Slide A14: Evaluation Metrics
\begin{frame}[t]{Appendix A14: Evaluation Metrics}
\small
\textbf{Intrinsic Evaluation} (embedding quality directly):

\begin{itemize}
\item \textbf{Word Similarity}: Correlation with human judgments (WordSim-353, SimLex-999)
\item \textbf{Word Analogies}: Accuracy on a:b :: c:d tasks (Google analogy dataset)
\item \textbf{Clustering}: Do semantic categories cluster?
\end{itemize}

\vspace{5mm}
\textbf{Extrinsic Evaluation} (downstream task performance):

\begin{itemize}
\item Use embeddings in actual NLP task
\item Sentiment classification accuracy
\item Named entity recognition F1
\item Question answering performance
\end{itemize}

\vspace{5mm}
\textbf{Trade-offs}:

\begin{itemize}
\item Intrinsic: Fast, but doesn't guarantee downstream success
\item Extrinsic: Slow, but measures real usefulness
\end{itemize}

\vspace{5mm}
\textbf{Best Practice}: Use both - intrinsic for development, extrinsic for final validation

\vspace{0.2cm}
\bottomnote{Good intrinsic scores usually (but not always) lead to good extrinsic performance}
\end{frame}

% Slide A15: Connection to Modern Transformers
\begin{frame}[t]{Appendix A15: From Word2Vec to Transformers}
\small
\textbf{The Evolution} (2013-2024):

\begin{center}
\begin{tabular}{lll}
\toprule
\textbf{Year} & \textbf{Model} & \textbf{Innovation} \\
\midrule
2013 & Word2Vec & Static distributed representations \\
2014 & GloVe & Matrix factorization perspective \\
2017 & FastText & Subword embeddings \\
2018 & ELMo & Contextualized (BiLSTM) \\
2018 & BERT & Transformer encoder (Week 6) \\
2018 & GPT & Transformer decoder (Week 6) \\
2024 & GPT-4/Claude & 1T+ parameters, multimodal \\
\bottomrule
\end{tabular}
\end{center}

\vspace{5mm}
\textbf{What Stayed from Word2Vec}:

\begin{itemize}
\item Embedding layer (first layer of all neural models)
\item Distributional hypothesis
\item Pre-training on large corpora
\item Vector arithmetic intuition
\end{itemize}

\vspace{5mm}
\textbf{What Changed}:

\begin{itemize}
\item Static $\rightarrow$ Contextualized
\item Single vector $\rightarrow$ Different vectors per context
\item Window $\rightarrow$ Full sentence attention
\end{itemize}

\vspace{0.2cm}
\bottomnote{Word2Vec started the revolution - transformers completed it}
\end{frame}

\end{document}
