\documentclass[8pt,aspectratio=169,8pt]{beamer}
\usepackage[utf8]{inputenc}
\usepackage{graphicx}
\usepackage{amsmath,amssymb}
\usepackage{algorithm2e}
\usepackage{listings}
\usepackage{xcolor}
\usepackage{tikz}
\usepackage{pgfplots}
\pgfplotsset{compat=1.17}
\usepackage{subfigure}
\usepackage{hyperref}

% Theme settings
\usetheme{Frankfurt}
\usecolortheme{seahorse}
\setbeamertemplate{navigation symbols}{}
\setbeamertemplate{footline}[frame number]

% Custom colors
\definecolor{darkblue}{RGB}{0,51,102}
\definecolor{lightblue}{RGB}{173,216,230}
\definecolor{codegreen}{RGB}{0,128,0}
\definecolor{codegray}{RGB}{150,150,150}
\definecolor{codepurple}{RGB}{128,0,128}
\definecolor{backcolor}{RGB}{245,245,245}

% Code listing settings
\lstset{
    backgroundcolor=\color{backcolor},
    basicstyle=\ttfamily\footnotesize,
    breakatwhitespace=false,
    breaklines=true,
    captionpos=b,
    commentstyle=\color{codegreen},
    keywordstyle=\color{blue},
    numberstyle=\tiny\color{codegray},
    stringstyle=\color{codepurple},
    showstringspaces=false,
    frame=single,
    numbers=left,
    language=Python
}

% Include slide layouts
% SLIDE LAYOUT TEMPLATES FOR NLP COURSE
% Three consistent layouts used throughout the course

% ==============================================================
% LAYOUT 1: CONCEPT SLIDE
% Used for: Theory, definitions, mathematical concepts
% Features: Title, main content area with bullets/equations, optional figure
% ==============================================================

\newcommand{\conceptslide}[3]{ % #1: title, #2: content, #3: optional figure
\begin{frame}[t]{#1}
    \begin{columns}[T]
        \begin{column}{0.6\textwidth}
            #2
        \end{column}
        \begin{column}{0.35\textwidth}
            \centering
            #3
        \end{column}
    \end{columns}
\end{frame}
}

% ==============================================================
% LAYOUT 2: CODE & IMPLEMENTATION SLIDE
% Used for: Code examples, algorithms, implementation details
% Features: Title, code block, explanation text
% ==============================================================

% Note: For code slides, use \begin{frame}[fragile] manually

\newcommand{\codeexplanation}[1]{%
    \vspace{0.5em}
    \small
    #1
}

\newcommand{\codeblock}[2][Python]{%
    \begin{lstlisting}[language=#1, basicstyle=\ttfamily\tiny]
#2
    \end{lstlisting}
}

\newcommand{\codeslide}[3]{%
    \begin{frame}[fragile]{#1}
    \begin{columns}[T]
        \column{0.55\textwidth}
        \codeblock{#2}
        \column{0.43\textwidth}
        \codeexplanation{#3}
    \end{columns}
    \end{frame}
}

% ==============================================================
% LAYOUT 3: RESULTS & VISUALIZATION SLIDE
% Used for: Experimental results, charts, comparisons
% Features: Title, large visualization area, key insights
% ==============================================================

\newcommand{\resultslide}[3]{ % #1: title, #2: visualization, #3: insights
\begin{frame}[t]{#1}
    \centering
    \vspace{-0.5em}
    #2
    \vspace{0.5em}
    
    \begin{block}{Key Insights}
        #3
    \end{block}
\end{frame}
}

% Additional utility commands for consistency

% Highlighted text
\newcommand{\highlight}[1]{\textcolor{blue}{\textbf{#1}}}

% Math notation shortcuts
\newcommand{\prob}[1]{P(#1)}
\newcommand{\given}{\mid}
\newcommand{\argmax}{\operatorname{argmax}}
\newcommand{\softmax}{\operatorname{softmax}}

% Standard figure size
\newcommand{\stdfigsizeinchins}{0.4\textwidth}

% For code blocks, use lstlisting directly:
% \begin{lstlisting}[language=Python]
% ... code ...
% \end{lstlisting}

% Equation highlight box
\newcommand{\eqbox}[1]{
    \begin{center}
    \colorbox{lightblue!20}{
        \parbox{0.8\textwidth}{
            \vspace{0.3em}
            \centering
            #1
            \vspace{0.3em}
        }
    }
    \end{center}
}

\title[Week 4: Seq2Seq]{Natural Language Processing Course}
\subtitle{Week 4: Sequence-to-Sequence Models}
\author{Joerg R. Osterrieder \\ \url{www.joergosterrieder.com}}
\date{}

\begin{document}

% Title page
\begin{frame}
    \titlepage
\end{frame}

\section{Week 4: Sequence-to-Sequence Models}

% Title slide
\begin{frame}
    \centering
    \vspace{2cm}
    {\Large \textbf{Week 4}}\\
    \vspace{0.5cm}
    {\huge \textbf{Sequence-to-Sequence Models}}\\
    \vspace{1cm}
    {\large Breaking the Fixed-Length Barrier}
\end{frame}

% Motivation: Translation problem
\begin{frame}[t]{Why Google Translate Struggled for Years}
    \textbf{The Challenge:}
    
    \vspace{0.5em}
    English: "I love you" (3 words)
    
    French: "Je t'aime" (2 words)
    
    German: "Ich liebe dich" (3 words)
    
    Japanese: "aishiteru" (1 word)
    
    \vspace{0.5em}
    \begin{center}
    \colorbox{red!20}{
        \parbox{0.8\textwidth}{
            \centering
            Different languages express the same idea with different lengths!
        }
    }
    \end{center}
    
    \vspace{0.5em}
    \textbf{The Problem:} RNNs produce one output per input
    
    \textbf{The Solution:} Separate encoding from decoding
    
    \vspace{0.5em}
    \textit{This breakthrough improved Google Translate accuracy by 60\% in 2016}\footnotemark
    
    \footnotetext{Wu et al. (2016). "Google's Neural Machine Translation System", arXiv}
\end{frame}

% Real-world impact
\begin{frame}[t]{Seq2Seq Powers Your Daily Interactions}
    \begin{columns}[T]
        \begin{column}{0.5\textwidth}
            \textbf{Translation (2024):}
            \begin{itemize}
                \item Google: 1B+ translations daily\footnotemark
                \item DeepL: Seq2seq + attention
                \item Real-time conversation mode
            \end{itemize}
            
            \vspace{0.5em}
            \textbf{Chatbots:}
            \begin{itemize}
                \item Customer service (80\% first-line)\footnotemark
                \item Variable-length responses
                \item Context-aware replies
            \end{itemize}
        \end{column}
        \begin{column}{0.5\textwidth}
            \textbf{Text Summarization:}
            \begin{itemize}
                \item News article $\rightarrow$ headline
                \item Email $\rightarrow$ one-liner
                \item Document $\rightarrow$ abstract
            \end{itemize}
            
            \vspace{0.5em}
            \textbf{Code Generation:}
            \begin{itemize}
                \item Comment $\rightarrow$ code
                \item Natural language $\rightarrow$ SQL
                \item Bug description $\rightarrow$ fix
            \end{itemize}
        \end{column}
    \end{columns}
    
    \vspace{0.5em}
    \begin{center}
    \colorbox{lightblue!30}{
        \parbox{0.8\textwidth}{
            \centering
            Key Innovation: Input and output can have different lengths!
        }
    }
    \end{center}
    
    \footnotetext[1]{Google Translate statistics 2024}
    \footnotetext[2]{Gartner report on AI customer service}
\end{frame}

% Learning objectives
\begin{frame}[t]{Week 4: What You'll Master}
    \textbf{By the end of this week, you will:}
    \begin{itemize}
        \item \textbf{Understand} why variable-length I/O is crucial
        \item \textbf{Build} intuition for encoder-decoder architecture
        \item \textbf{Implement} a complete seq2seq translator
        \item \textbf{Discover} why attention changes everything
        \item \textbf{Create} a chatbot that handles any conversation length
    \end{itemize}
    
    \vspace{0.5em}
    \begin{center}
    \colorbox{lightblue!30}{
        \parbox{0.8\textwidth}{
            \centering
            \textbf{Core Insight:} Compress entire input to a "thought", then expand to output
        }
    }
    \end{center}
\end{frame}

% The fixed-length problem
\begin{frame}[t]{The Fixed-Length Prison}
    \textbf{What RNNs can do:}
    \begin{itemize}
        \item Input: "The cat sat" $\rightarrow$ Output: "on the mat" (same length)
        \item Each input produces exactly one output
    \end{itemize}
    
    \vspace{0.5em}
    \textbf{What we actually need:}
    \begin{itemize}
        \item "Hello" $\rightarrow$ "Bonjour" (different lengths)
        \item "How are you?" $\rightarrow$ "¿Cómo estás?" (different structure)
        \item Long paragraph $\rightarrow$ Short summary (compression)
    \end{itemize}
    
    \vspace{0.5em}
    \textbf{Failed Approaches:}
    \begin{enumerate}
        \item Pad everything to max length (wastes computation)
        \item Truncate to fixed length (loses information)
        \item Multiple passes (complicated and slow)
    \end{enumerate}
    
    \vspace{0.5em}
    \begin{center}
    \colorbox{yellow!20}{
        \parbox{0.8\textwidth}{
            \centering
            We need to decouple input processing from output generation!
        }
    }
    \end{center}
\end{frame}

% Encoder-decoder intuition
\begin{frame}[t]{The Brilliant Solution: Encoder-Decoder}
    \textbf{How humans translate:}
    \begin{enumerate}
        \item Read entire English sentence
        \item Understand the complete meaning
        \item Express that meaning in French
    \end{enumerate}
    
    \vspace{0.5em}
    \textbf{Seq2seq does exactly this:}\footnotemark
    \begin{enumerate}
        \item \textbf{Encoder:} Process entire input into a "thought vector"
        \item \textbf{Context Vector:} Fixed-size representation of meaning
        \item \textbf{Decoder:} Generate output from the thought vector
    \end{enumerate}
    
    \vspace{0.5em}
    \textbf{The Magic:}
    \begin{itemize}
        \item Encoder handles any input length
        \item Decoder produces any output length
        \item Middle "thought" is always same size!
    \end{itemize}
    
    \vspace{0.5em}
    \begin{center}
    \colorbox{lightblue!30}{
        \parbox{0.8\textwidth}{
            \centering
            Context vector = Compressed understanding of entire input
        }
    }
    \end{center}
    
    \footnotetext{Sutskever, Vinyals \& Le (2014). "Sequence to Sequence Learning with Neural Networks", NeurIPS}
\end{frame}

% Visual architecture
\begin{frame}[t]{Visualizing Seq2Seq Architecture}
    \centering
    \includegraphics[width=0.9\textwidth]{../figures/seq2seq_architecture.pdf}
    
    \vspace{0.5em}
    \textbf{Key insights:}
    \begin{itemize}
        \item Encoder reads left-to-right (or bidirectional)
        \item Final hidden state contains entire input meaning
        \item Decoder generates one word at a time
        \item Special tokens: START begins generation, END stops it
    \end{itemize}
\end{frame}

% Mathematical formulation
\conceptslide{Seq2Seq Mathematics: Two RNNs Working Together}{
    \textbf{Encoder (processes input):}
    
    \vspace{0.5em}
    \eqbox{
        $\begin{aligned}
        h_t^{enc} &= \text{LSTM}_{enc}(x_t, h_{t-1}^{enc}) \\
        c &= h_{final}^{enc} \text{ (context vector)}
        \end{aligned}$
    }
    
    \vspace{0.5em}
    \textbf{Decoder (generates output):}
    
    \vspace{0.5em}
    \eqbox{
        $\begin{aligned}
        h_0^{dec} &= c \text{ (initialize with context)} \\
        h_t^{dec} &= \text{LSTM}_{dec}(y_{t-1}, h_{t-1}^{dec}) \\
        y_t &= \text{softmax}(W_y h_t^{dec})
        \end{aligned}$
    }
    
    \vspace{0.5em}
    \textbf{In plain English:}
    \begin{itemize}
        \item Encoder: Compress input into fixed-size thought
        \item Decoder: Expand thought into variable-length output
    \end{itemize}
}{
    % Could add a simple flow diagram
}

% Implementation
\begin{frame}[fragile]{Building Seq2Seq: Complete Implementation}
    \begin{columns}[T]
\column{0.55\textwidth}
\begin{lstlisting}[language=Python,basicstyle=\ttfamily\tiny]
import torch
import torch.nn as nn

class Encoder(nn.Module):
    def __init__(self, input_size, hidden_size):
        """Compress input sequence to fixed-size vector"""
        super().__init__()
        self.hidden_size = hidden_size
        self.embedding = nn.Embedding(input_size, hidden_size)
        self.lstm = nn.LSTM(hidden_size, hidden_size)
        
    def forward(self, input_seq):
        """Process entire input sequence"""
        embedded = self.embedding(input_seq)
        outputs, (hidden, cell) = self.lstm(embedded)
        # Return final hidden state as context
        return hidden, cell

class Decoder(nn.Module):
    def __init__(self, output_size, hidden_size):
        """Generate output sequence from context"""
        super().__init__()
        self.hidden_size = hidden_size
        self.embedding = nn.Embedding(output_size, hidden_size)
        self.lstm = nn.LSTM(hidden_size, hidden_size)
        self.out = nn.Linear(hidden_size, output_size)
        
    def forward(self, input_token, hidden, cell):
        """Generate one output token at a time"""
        embedded = self.embedding(input_token.unsqueeze(0))
        output, (hidden, cell) = self.lstm(embedded, (hidden, cell))
        prediction = self.out(output.squeeze(0))
        return prediction, hidden, cell
\end{lstlisting}
\column{0.43\textwidth}
        \codeexplanation{
            \textbf{Design Choices:}
            \begin{itemize}
                \item Hidden size: 256-512 typical\footnotemark
                \item Bidirectional encoder often better
                \item Teacher forcing during training
            \end{itemize}
            
            \vspace{0.3em}
            \textbf{Usage Pattern:}
            \begin{small}
            \texttt{\# Encode}\\
            \texttt{context = encoder(input\_seq)}\\
            \texttt{\# Decode}\\
            \texttt{output = decoder(context)}
            \end{small}
            
            \footnotetext{Britz et al. (2017) extensive comparison}
        }
    \end{columns}
\end{frame}

% Complete seq2seq model
\begin{frame}[fragile]{Complete Seq2Seq Model}
    \begin{columns}[T]
        \column{0.55\textwidth}
\begin{lstlisting}[language=Python,basicstyle=\ttfamily\tiny]
class Seq2Seq(nn.Module):
    def __init__(self, encoder, decoder, device):
        """Complete translation model"""
        super().__init__()
        self.encoder = encoder
        self.decoder = decoder
        self.device = device
        
    def forward(self, src, trg, teacher_forcing_ratio=0.5):
        """Translate source to target"""
        batch_size = src.shape[1]
        trg_len = trg.shape[0]
        trg_vocab_size = self.decoder.out.out_features
        
        # Store decoder outputs
        outputs = torch.zeros(trg_len, batch_size, trg_vocab_size).to(self.device)
        
        # Encode entire source sequence
        hidden, cell = self.encoder(src)
        
        # First decoder input is <START> token
        decoder_input = trg[0,:]
        
        for t in range(1, trg_len):
            # Decode one token
            output, hidden, cell = self.decoder(decoder_input, hidden, cell)
            outputs[t] = output
            
            # Teacher forcing: use true target or predicted
            teacher_force = random.random() < teacher_forcing_ratio
            top1 = output.argmax(1)
            decoder_input = trg[t] if teacher_force else top1
            
        return outputs
\end{lstlisting}
        \column{0.45\textwidth}
        \codeexplanation{
            \textbf{Training Tricks:}
            \begin{itemize}
                \item Teacher forcing prevents drift\footnotemark
                \item Gradually reduce forcing ratio
                \item Beam search for better decoding
            \end{itemize}
            
            \vspace{0.3em}
            \textbf{Performance (2014):}
            \begin{itemize}
                \item BLEU: 34.8 on EN-FR\footnotemark
                \item Beats phrase-based SMT
                \item 1000x faster than RNN search
            \end{itemize}
            
            \footnotetext[1]{Williams \& Zipser (1989)}
            \footnotetext[2]{Original seq2seq paper results}
        }
    \end{columns}
\end{frame}

% The bottleneck problem
\begin{frame}[t]{The Information Bottleneck Problem}
    \textbf{Imagine compressing a book into one sentence...}
    
    \vspace{0.5em}
    \textbf{What happens with long sequences:}
    \begin{itemize}
        \item 10 words: Context vector remembers well
        \item 20 words: Starting to forget early words
        \item 50+ words: Information bottleneck!\footnotemark
    \end{itemize}
    
    \vspace{0.5em}
    \textbf{The Problem Visualized:}
    
    \centering
    \includegraphics[height=0.5\textheight]{../figures/bottleneck_problem.pdf}
    
    \vspace{0.5em}
    \begin{center}
    \colorbox{red!20}{
        \parbox{0.8\textwidth}{
            \centering
            Fixed-size context vector must capture EVERYTHING!
        }
    }
    \end{center}
    
    \footnotetext{Cho et al. (2014) showed performance degrades after 30 tokens}
\end{frame}

% Attention mechanism introduction
\begin{frame}[t]{The Attention Revolution (2015)}
    \textbf{How humans actually translate:}
    
    "The black cat sat on the mat" $\rightarrow$ "Le chat noir..."
    
    \begin{itemize}
        \item When translating "chat" (cat), we look back at "cat"
        \item When translating "noir" (black), we look back at "black"
        \item We don't compress everything into one thought!
    \end{itemize}
    
    \vspace{0.5em}
    \textbf{Attention mechanism:}\footnotemark
    \begin{enumerate}
        \item Keep ALL encoder hidden states
        \item When decoding, look back at relevant parts
        \item Weight importance of each input word
        \item Focus on what matters right now
    \end{enumerate}
    
    \vspace{0.5em}
    \begin{center}
    \colorbox{green!20}{
        \parbox{0.8\textwidth}{
            \centering
            Attention = Let the decoder peek back at the input whenever needed
        }
    }
    \end{center}
    
    \footnotetext{Bahdanau, Cho \& Bengio (2015). "Neural Machine Translation by Jointly Learning to Align and Translate"}
\end{frame}

% Attention visualization
\begin{frame}[t]{Visualizing Attention: What the Model Looks At}
    \centering
    \includegraphics[width=0.85\textwidth]{../figures/attention_visualization.pdf}
    
    \vspace{0.5em}
    \textbf{Key observations:}
    \begin{itemize}
        \item Diagonal pattern shows alignment
        \item Some words need multiple source words
        \item Reordering handled naturally
        \item Model learns what to attend to!
    \end{itemize}
\end{frame}

% Attention implementation
\begin{frame}[fragile]{Implementing Attention}
    \begin{columns}[T]
       \column{0.55\textwidth}
\begin{lstlisting}[language=Python,basicstyle=\ttfamily\tiny]
class Attention(nn.Module):
    def __init__(self, hidden_size):
        """Calculate attention weights"""
        super().__init__()
        self.attn = nn.Linear(hidden_size * 2, hidden_size)
        self.v = nn.Linear(hidden_size, 1, bias=False)
        
    def forward(self, hidden, encoder_outputs):
        """Compute attention over encoder outputs"""
        batch_size = encoder_outputs.shape[1]
        src_len = encoder_outputs.shape[0]
        
        # Repeat decoder hidden state
        hidden = hidden.unsqueeze(1).repeat(1, src_len, 1)
        encoder_outputs = encoder_outputs.permute(1, 0, 2)
        
        # Calculate attention scores
        energy = torch.tanh(self.attn(
            torch.cat((hidden, encoder_outputs), dim=2)
        ))
        attention = self.v(energy).squeeze(2)
        
        # Convert to probabilities
        return F.softmax(attention, dim=1)

class AttentionDecoder(nn.Module):
    def __init__(self, output_size, hidden_size, attention):
        """Decoder with attention mechanism"""
        super().__init__()
        self.attention = attention
        self.lstm = nn.LSTM(hidden_size * 2, hidden_size)
        self.out = nn.Linear(hidden_size, output_size)
\end{lstlisting}
\column{0.45\textwidth}
        \codeexplanation{
            \textbf{How Attention Works:}
            \begin{itemize}
                \item Score each encoder state
                \item Softmax creates distribution
                \item Weighted sum of encoder states
                \item Concatenate with decoder state
            \end{itemize}
            
            \vspace{0.3em}
            \textbf{Performance Boost:}
            \begin{itemize}
                \item BLEU: 34.8 $\rightarrow$ 41.8\footnotemark
                \item Handles 2x longer sequences
                \item Interpretable alignments
            \end{itemize}
            
            \footnotetext{15\% improvement with attention}
        }
    \end{columns}
\end{frame}

% Results comparison
\resultslide{Seq2Seq Impact: Translation Quality Leap}{
    \centering
    \includegraphics[width=0.4\textwidth]{../figures/seq2seq_performance.pdf}
}{
    \begin{itemize}
        \item Phrase-based SMT dominated for decades
        \item Seq2seq matched SMT in 2014
        \item Attention surpassed all previous methods
        \item Google deployed in production 2016
        \item Now foundation for all translation systems
    \end{itemize}
}

% Modern applications
\begin{frame}[t]{Beyond Translation: Seq2Seq Everywhere (2024)}
    \begin{columns}[T]
        \begin{column}{0.5\textwidth}
            \textbf{Text Generation:}
            \begin{itemize}
                \item Email auto-complete\footnotemark
                \item Code documentation
                \item Story continuation
                \item Dialogue systems
            \end{itemize}
            
            \vspace{0.5em}
            \textbf{Summarization:}
            \begin{itemize}
                \item News $\rightarrow$ headlines
                \item Papers $\rightarrow$ abstracts
                \item Meetings $\rightarrow$ notes
                \item Videos $\rightarrow$ descriptions
            \end{itemize}
        \end{column}
        \begin{column}{0.5\textwidth}
            \textbf{Data Transformation:}
            \begin{itemize}
                \item Text $\rightarrow$ SQL queries\footnotemark
                \item Natural language $\rightarrow$ code
                \item Speech $\rightarrow$ text
                \item Image $\rightarrow$ caption
            \end{itemize}
            
            \vspace{0.5em}
            \textbf{Still Competitive:}
            \begin{itemize}
                \item Smaller than transformers
                \item Streaming applications
                \item Low-latency requirements
                \item Resource-constrained devices
            \end{itemize}
        \end{column}
    \end{columns}
    
    \vspace{0.5em}
    \begin{center}
    \colorbox{lightblue!30}{
        \parbox{0.8\textwidth}{
            \centering
            Seq2seq principle: Any input $\rightarrow$ any output transformation
        }
    }
    \end{center}
    
    \footnotetext[1]{Gmail Smart Compose uses seq2seq variants}
    \footnotetext[2]{Text-to-SQL still uses specialized seq2seq models}
\end{frame}

% Exercise
\begin{frame}[t]{Week 4 Exercise: Build a Smart Chatbot}
    \textbf{Your Mission:} Create a chatbot that gives variable-length responses
    
    \vspace{0.5em}
    \textbf{Example Behavior:}
    \begin{itemize}
        \item "Hi" $\rightarrow$ "Hello! How can I help you today?"
        \item "What's the weather?" $\rightarrow$ "I'd need to check current conditions. What's your location?"
        \item "Tell me a joke" $\rightarrow$ (Generates different length jokes)
    \end{itemize}
    
    \vspace{0.5em}
    \textbf{Implementation Steps:}
    \begin{enumerate}
        \item Build encoder-decoder architecture
        \item Train on conversation pairs
        \item Implement attention mechanism
        \item Compare responses with/without attention
        \item Visualize what words the model attends to
    \end{enumerate}
    
    \vspace{0.5em}
    \textbf{Bonus Challenges:}
    \begin{itemize}
        \item Implement beam search (top-k responses)
        \item Add personality tokens (formal/casual)
        \item Handle multi-turn conversations
        \item Measure response diversity
    \end{itemize}
    
    \vspace{0.5em}
    \textbf{You'll discover:} Why chatbots sometimes give generic responses!
\end{frame}

% Summary
\begin{frame}[t]{Key Takeaways: Breaking Free from Fixed Length}
    \textbf{What we learned:}
    \begin{itemize}
        \item Language tasks need variable-length input/output
        \item Encoder-decoder separates understanding from generation
        \item Context vector creates information bottleneck
        \item Attention lets decoder access all input information
        \item Seq2seq enables any-to-any sequence transformation
    \end{itemize}
    
    \vspace{0.5em}
    \textbf{The evolution:}
    \begin{center}
    \colorbox{lightblue!30}{
        \parbox{0.8\textwidth}{
            \centering
            Fixed I/O (RNN) $\rightarrow$ Variable I/O (Seq2seq) $\rightarrow$ Full visibility (Attention)
        }
    }
    \end{center}
    
    \vspace{0.5em}
    \textbf{Critical Innovation:}
    Attention shows us what the model is "thinking" - interpretability!
    
    \vspace{0.5em}
    \textbf{Next week: The Transformer}
    
    What if we used ONLY attention, no RNNs at all?
\end{frame}

% References
\begin{frame}[t]{References and Further Reading}
    \textbf{Foundational Papers:}
    \begin{itemize}
        \item Sutskever et al. (2014). "Sequence to Sequence Learning with Neural Networks", NeurIPS
        \item Bahdanau et al. (2015). "Neural Machine Translation by Jointly Learning to Align and Translate"
        \item Cho et al. (2014). "Learning Phrase Representations using RNN Encoder-Decoder"
    \end{itemize}
    
    \textbf{Applications:}
    \begin{itemize}
        \item Wu et al. (2016). "Google's Neural Machine Translation System"
        \item See et al. (2017). "Get To The Point: Summarization with Pointer-Generator Networks"
        \item Vinyals \& Le (2015). "A Neural Conversational Model"
    \end{itemize}
    
    \textbf{Recommended Resources:}
    \begin{itemize}
        \item TensorFlow Seq2seq Tutorial (with attention visualization)
        \item Visualizing and Understanding Neural Machine Translation
        \item PyTorch Chatbot Tutorial
    \end{itemize}
\end{frame}
\end{document}
