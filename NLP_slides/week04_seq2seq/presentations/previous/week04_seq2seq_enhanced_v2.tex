\documentclass[8pt,aspectratio=169]{beamer}
\usepackage[utf8]{inputenc}
\usepackage{graphicx}
\usepackage{amsmath,amssymb}
\usepackage{algorithm2e}
\usepackage{listings}
\usepackage{xcolor}
\usepackage{tikz}
\usepackage{pgfplots}
\pgfplotsset{compat=1.17}
\usepackage{subfigure}
\usepackage{hyperref}
\usepackage{tcolorbox}

% Theme settings
\usetheme{Frankfurt}
\usecolortheme{seahorse}
\setbeamertemplate{navigation symbols}{}
\setbeamertemplate{footline}[frame number]

% Custom colors (matching educational color scheme)
\definecolor{darkblue}{RGB}{0,51,102}
\definecolor{lightblue}{RGB}{173,216,230}
\definecolor{codegreen}{RGB}{0,128,0}
\definecolor{codegray}{RGB}{150,150,150}
\definecolor{codepurple}{RGB}{128,0,128}
\definecolor{backcolor}{RGB}{245,245,245}

% Code listing settings
\lstset{
    backgroundcolor=\color{backcolor},
    basicstyle=\ttfamily\tiny,
    breakatwhitespace=false,
    breaklines=true,
    captionpos=b,
    commentstyle=\color{codegreen},
    keywordstyle=\color{blue},
    numberstyle=\tiny\color{codegray},
    stringstyle=\color{codepurple},
    showstringspaces=false,
    frame=single,
    numbers=left,
    language=Python
}

% Include shared slide layouts
% SLIDE LAYOUT TEMPLATES FOR NLP COURSE
% Three consistent layouts used throughout the course

% ==============================================================
% LAYOUT 1: CONCEPT SLIDE
% Used for: Theory, definitions, mathematical concepts
% Features: Title, main content area with bullets/equations, optional figure
% ==============================================================

\newcommand{\conceptslide}[3]{ % #1: title, #2: content, #3: optional figure
\begin{frame}[t]{#1}
    \begin{columns}[T]
        \begin{column}{0.6\textwidth}
            #2
        \end{column}
        \begin{column}{0.35\textwidth}
            \centering
            #3
        \end{column}
    \end{columns}
\end{frame}
}

% ==============================================================
% LAYOUT 2: CODE & IMPLEMENTATION SLIDE
% Used for: Code examples, algorithms, implementation details
% Features: Title, code block, explanation text
% ==============================================================

% Note: For code slides, use \begin{frame}[fragile] manually

\newcommand{\codeexplanation}[1]{%
    \vspace{0.5em}
    \small
    #1
}

\newcommand{\codeblock}[2][Python]{%
    \begin{lstlisting}[language=#1, basicstyle=\ttfamily\tiny]
#2
    \end{lstlisting}
}

\newcommand{\codeslide}[3]{%
    \begin{frame}[fragile]{#1}
    \begin{columns}[T]
        \column{0.55\textwidth}
        \codeblock{#2}
        \column{0.43\textwidth}
        \codeexplanation{#3}
    \end{columns}
    \end{frame}
}

% ==============================================================
% LAYOUT 3: RESULTS & VISUALIZATION SLIDE
% Used for: Experimental results, charts, comparisons
% Features: Title, large visualization area, key insights
% ==============================================================

\newcommand{\resultslide}[3]{ % #1: title, #2: visualization, #3: insights
\begin{frame}[t]{#1}
    \centering
    \vspace{-0.5em}
    #2
    \vspace{0.5em}
    
    \begin{block}{Key Insights}
        #3
    \end{block}
\end{frame}
}

% Additional utility commands for consistency

% Highlighted text
\newcommand{\highlight}[1]{\textcolor{blue}{\textbf{#1}}}

% Math notation shortcuts
\newcommand{\prob}[1]{P(#1)}
\newcommand{\given}{\mid}
\newcommand{\argmax}{\operatorname{argmax}}
\newcommand{\softmax}{\operatorname{softmax}}

% Standard figure size
\newcommand{\stdfigsizeinchins}{0.4\textwidth}

% For code blocks, use lstlisting directly:
% \begin{lstlisting}[language=Python]
% ... code ...
% \end{lstlisting}

% Equation highlight box
\newcommand{\eqbox}[1]{
    \begin{center}
    \colorbox{lightblue!20}{
        \parbox{0.8\textwidth}{
            \vspace{0.3em}
            \centering
            #1
            \vspace{0.3em}
        }
    }
    \end{center}
}

\title[Week 4: Seq2Seq]{Natural Language Processing Course}
\subtitle{Week 4: Sequence-to-Sequence Models}
\author{Breaking the Fixed-Length Barrier}
\institute{NLP Course 2025}
\date{}

\begin{document}

% Title page
\begin{frame}
    \titlepage
\end{frame}

% Learning objectives with modern context
\begin{frame}{Week 4: What You'll Master Today}
    \textbf{By the end of this week, you will:}
    \begin{itemize}
        \item \highlight{Understand} why ChatGPT's predecessors needed variable-length processing
        \item \highlight{Build} intuition for encoder-decoder architecture (used in Google Translate)
        \item \highlight{Discover} the attention mechanism that powers modern AI
        \item \highlight{Implement} a complete seq2seq model from scratch
        \item \highlight{Connect} these concepts to today's transformer-based systems
    \end{itemize}
    
    \vspace{1em}
    \eqbox{
        \textbf{Core Breakthrough:} Input and output sequences can have different lengths!
    }
    
    \vspace{0.5em}
    \textit{This innovation (2014-2016) laid the foundation for everything from Google Translate to GitHub Copilot}
\end{frame}

% Modern motivation
\conceptslide{Why This Matters: Your Daily AI Interactions (2024)}{
    \textbf{Translation \& Communication:}
    \begin{itemize}
        \item Google Translate: 1+ billion translations daily
        \item DeepL: 1 billion people served monthly
        \item Real-time conversation translation
    \end{itemize}
    
    \vspace{0.5em}
    \textbf{Code \& Development:}
    \begin{itemize}
        \item GitHub Copilot: 40\% faster development
        \item Natural language → SQL queries
        \item Comment → complete function generation
    \end{itemize}
    
    \vspace{0.5em}
    \textbf{Content \& Communication:}
    \begin{itemize}
        \item Email summarization (Outlook, Gmail)
        \item Meeting transcript → action items
        \item Customer service automation (80\% first-line)
    \end{itemize}
}{
    \includegraphics[width=\textwidth]{../figures/week4_modern_applications_ecosystem.pdf}
}

% The fundamental problem
\conceptslide{The Variable-Length Challenge}{
    \textbf{The Core Problem:}
    
    \vspace{0.5em}
    Different languages express ideas with different lengths:
    \begin{itemize}
        \item English: "I love you" (3 words)
        \item French: "Je t'aime" (2 words)  
        \item German: "Ich liebe dich" (3 words)
        \item Japanese: "aishiteru" (1 compound word)
    \end{itemize}
    
    \vspace{0.5em}
    \textbf{Traditional RNN Limitation:}
    \begin{itemize}
        \item Each input produces exactly one output
        \item Fixed-length constraint breaks translation
        \item Can't handle summarization (long → short)
        \item Fails at code generation (comment → function)
    \end{itemize}
}{
    \vspace{1em}
    \textbf{Failed Approaches:}
    \begin{enumerate}
        \item Pad to maximum length (wasteful)
        \item Truncate long sequences (loses information)
        \item Force 1:1 word mapping (doesn't work)
    \end{enumerate}
    
    \vspace{1em}
    \colorbox{red!20}{
        \parbox{0.9\textwidth}{
            \centering
            \textbf{We need to decouple input and output lengths!}
        }
    }
}

% The brilliant solution
\resultslide{The Breakthrough: Encoder-Decoder Architecture}{
    \includegraphics[width=\textwidth]{../figures/week4_encoder_decoder_architecture.pdf}
}{
    \begin{itemize}
        \item \textbf{Encoder:} Compresses input sequence into fixed-size "thought" vector
        \item \textbf{Decoder:} Expands "thought" into variable-length output
        \item \textbf{Key Innovation:} Separate encoding from decoding phases
        \item \textbf{Result:} Input and output can have completely different lengths
    \end{itemize}
}

% Mathematical formulation
\conceptslide{Seq2Seq Mathematics: The Core Equations}{
    \textbf{Encoder Phase:}
    \eqbox{
        $h_t = \text{LSTM}(h_{t-1}, x_t)$ \quad \text{where } $h_T = \text{context vector}$
    }
    
    \textbf{Decoder Phase:}
    \eqbox{
        $s_t = \text{LSTM}(s_{t-1}, y_{t-1})$ \quad \text{conditioned on context}
    }
    
    \textbf{Output Probability:}
    \eqbox{
        $P(y_t \given y_{<t}, x) = \softmax(W_s s_t + b)$
    }
    
    \vspace{0.5em}
    \textbf{Training Objective:}
    \eqbox{
        $\max \sum_{t=1}^{T'} \log P(y_t^* \given y_{<t}^*, x)$
    }
}{
    \textbf{Key Insights:}
    \begin{itemize}
        \item Context vector = compressed representation
        \item Decoder generates one word at a time
        \item Training uses teacher forcing
        \item Inference uses beam search
    \end{itemize}
    
    \vspace{1em}
    \textbf{Dimensions:}
    \begin{itemize}
        \item Input length: $T$ (variable)
        \item Output length: $T'$ (variable)
        \item Context size: $d$ (fixed)
    \end{itemize}
}

% Code implementation
\begin{frame}[fragile]{Building Seq2Seq: PyTorch Implementation}
\begin{columns}[T]
\column{0.55\textwidth}
\begin{lstlisting}[language=Python]
class Seq2SeqEncoder(nn.Module):
    def __init__(self, vocab_size, embed_size, hidden_size):
        super().__init__()
        self.embedding = nn.Embedding(vocab_size, embed_size)
        self.lstm = nn.LSTM(embed_size, hidden_size, batch_first=True)
    
    def forward(self, x):
        # x: [batch, seq_len]
        embedded = self.embedding(x)  # [batch, seq_len, embed]
        output, (h_n, c_n) = self.lstm(embedded)
        return h_n, c_n  # Final hidden and cell states

class Seq2SeqDecoder(nn.Module):
    def __init__(self, vocab_size, embed_size, hidden_size):
        super().__init__()
        self.embedding = nn.Embedding(vocab_size, embed_size)
        self.lstm = nn.LSTM(embed_size, hidden_size, batch_first=True)
        self.output_projection = nn.Linear(hidden_size, vocab_size)
    
    def forward(self, x, hidden, cell):
        # x: [batch, 1] (one word at a time)
        embedded = self.embedding(x)
        output, (h_n, c_n) = self.lstm(embedded, (hidden, cell))
        logits = self.output_projection(output)
        return logits, h_n, c_n
\end{lstlisting}
\column{0.43\textwidth}
\textbf{Key Components:}
\begin{itemize}
    \item \textcolor{blue}{\textbf{Encoder}}: Processes entire input sequence
    \item \textcolor{green}{\textbf{Context}}: Hidden state transfer
    \item \textcolor{red}{\textbf{Decoder}}: Generates output step-by-step
\end{itemize}

\vspace{0.5em}
\textbf{Training Process:}
\begin{enumerate}
    \item Encode input sentence
    \item Initialize decoder with context
    \item Teacher forcing during training
    \item Beam search during inference
\end{enumerate}

\vspace{0.5em}
\textbf{Modern Usage:}
\begin{itemize}
    \item Foundation of Transformer encoder-decoder
    \item Still used in specialized applications
    \item Basis for understanding attention
\end{itemize}
\end{columns}
\end{frame}

% The bottleneck problem
\resultslide{The Information Bottleneck Problem}{
    \includegraphics[width=\textwidth]{../figures/week4_information_bottleneck_analysis.pdf}
}{
    \begin{itemize}
        \item \textbf{Problem:} All input information compressed into single vector
        \item \textbf{Effect:} Quality degrades rapidly with sentence length
        \item \textbf{Solution Preview:} Attention mechanism (next slides!)
        \item \textbf{Real Impact:} Google Translate was limited to ~20 words before attention
    \end{itemize}
}

% Attention introduction
\conceptslide{The Attention Revolution (2015)}{
    \textbf{The Key Insight:}
    \begin{itemize}
        \item Don't compress everything into one vector
        \item Keep \textit{all} encoder hidden states
        \item Let decoder \textit{choose} what to focus on
        \item Different output words attend to different input words
    \end{itemize}
    
    \vspace{0.5em}
    \textbf{Attention Mechanism:}
    \eqbox{
        $c_t = \sum_{i=1}^{T} \alpha_{t,i} h_i$ \quad where \quad $\alpha_{t,i} = \frac{\exp(e_{t,i})}{\sum_{j=1}^{T} \exp(e_{t,j})}$
    }
    
    \vspace{0.5em}
    \textbf{Alignment Score:}
    \eqbox{
        $e_{t,i} = \text{align}(s_{t-1}, h_i)$ \quad (similarity function)
    }
}{
    \textbf{Breakthrough Results:}
    \begin{itemize}
        \item Google NMT (2016): 60\% improvement
        \item Handles sentences up to 80+ words
        \item Foundation for Transformers
        \item Powers all modern LLMs
    \end{itemize}
    
    \vspace{1em}
    \colorbox{green!20}{
        \parbox{0.9\textwidth}{
            \centering
            \textbf{"Attention is All You Need"}\\
            (Transformer paper, 2017)
        }
    }
}

% Attention visualization
\resultslide{Attention in Action: What the Model Looks At}{
    \includegraphics[width=\textwidth]{../figures/week4_attention_comprehensive_analysis.pdf}
}{
    \begin{itemize}
        \item \textbf{Heatmap:} Darker = higher attention weight
        \item \textbf{Alignment:} Model learns word-to-word correspondences
        \item \textbf{Flexibility:} One target word can attend to multiple source words
        \item \textbf{Evolution:} Attention patterns improve with model scale
    \end{itemize}
}

% Attention implementation
\begin{frame}[fragile]{Implementing Attention: The Complete Mechanism}
\begin{columns}[T]
\column{0.55\textwidth}
\begin{lstlisting}[language=Python]
class AttentionMechanism(nn.Module):
    def __init__(self, hidden_size):
        super().__init__()
        self.hidden_size = hidden_size
        self.W_a = nn.Linear(hidden_size, hidden_size)
        self.U_a = nn.Linear(hidden_size, hidden_size)
        self.v_a = nn.Linear(hidden_size, 1)
    
    def forward(self, decoder_hidden, encoder_outputs):
        # decoder_hidden: [batch, hidden_size]
        # encoder_outputs: [batch, seq_len, hidden_size]
        
        batch_size, seq_len, _ = encoder_outputs.size()
        
        # Expand decoder hidden for all time steps
        decoder_hidden = decoder_hidden.unsqueeze(1).repeat(1, seq_len, 1)
        
        # Compute alignment scores
        energy = torch.tanh(
            self.W_a(decoder_hidden) + self.U_a(encoder_outputs)
        )
        attention_scores = self.v_a(energy).squeeze(2)
        
        # Apply softmax to get attention weights
        attention_weights = F.softmax(attention_scores, dim=1)
        
        # Compute context vector
        context = torch.bmm(
            attention_weights.unsqueeze(1),
            encoder_outputs
        ).squeeze(1)
        
        return context, attention_weights
\end{lstlisting}
\column{0.43\textwidth}
\textbf{Step-by-Step Process:}
\begin{enumerate}
    \item Compute similarity between decoder state and all encoder states
    \item Apply softmax to get probability distribution
    \item Weight encoder states by attention
    \item Sum to get context vector
\end{enumerate}

\vspace{0.5em}
\textbf{Three Attention Types:}
\begin{itemize}
    \item \textcolor{blue}{\textbf{Dot Product}}: $h_i \cdot s_t$
    \item \textcolor{green}{\textbf{Additive}}: $v^T \tanh(W_h h_i + W_s s_t)$
    \item \textcolor{red}{\textbf{Scaled Dot}}: $\frac{h_i \cdot s_t}{\sqrt{d}}$
\end{itemize}

\vspace{0.5em}
\textbf{Modern Impact:}
\begin{itemize}
    \item Transformer uses scaled dot-product
    \item Multi-head attention in GPT/BERT
    \item Self-attention in modern LLMs
\end{itemize}
\end{columns}
\end{frame}

% Beam search
\resultslide{Beam Search: Finding the Best Translation}{
    \includegraphics[width=\textwidth]{../figures/week4_beam_search_algorithm.pdf}
}{
    \begin{itemize}
        \item \textbf{Problem:} Exponentially many possible outputs
        \item \textbf{Solution:} Keep top-k most likely partial sequences
        \item \textbf{Trade-off:} Beam size vs computational cost
        \item \textbf{Practice:} Beam size 4-8 for translation, 1-2 for creative text
    \end{itemize}
}

% Performance evolution
\resultslide{From 2014 to 2024: The Performance Revolution}{
    \includegraphics[width=\textwidth]{../figures/week4_performance_evolution_timeline.pdf}
}{
    \begin{itemize}
        \item \textbf{2014:} Vanilla Seq2Seq - proof of concept
        \item \textbf{2015:} Attention - breakthrough for long sequences
        \item \textbf{2017:} Transformer - "Attention is All You Need"
        \item \textbf{2024:} Modern LLMs - seq2seq principles scaled massively
    \end{itemize}
}

% Modern connections
\conceptslide{Connecting to Modern AI: GPT, BERT, and Beyond}{
    \textbf{Seq2Seq Lives On in Modern Models:}
    \begin{itemize}
        \item \textbf{ChatGPT:} Uses encoder-decoder principles internally
        \item \textbf{GitHub Copilot:} Comment → code is seq2seq
        \item \textbf{BERT:} Encoder-only architecture
        \item \textbf{T5:} "Text-to-Text Transfer Transformer" - pure seq2seq
    \end{itemize}
    
    \vspace{0.5em}
    \textbf{Key Evolution Points:}
    \begin{itemize}
        \item RNN → Transformer (parallelization)
        \item Single attention → Multi-head attention
        \item Fixed context → Self-attention
        \item Supervised → Pre-training + fine-tuning
    \end{itemize}
    
    \vspace{0.5em}
    \eqbox{
        \textbf{Core Principle Unchanged:} Variable-length input → Variable-length output
    }
}{
    \textbf{2024 Statistics:}
    \begin{itemize}
        \item 127B+ model parameters (vs 10M in 2014)
        \item Trillions of tokens training data
        \item Sub-second response times
        \item Multi-modal capabilities
    \end{itemize}
    
    \vspace{1em}
    \textbf{Industry Impact:}
    \begin{itemize}
        \item \$127B+ market size
        \item 40\% developer productivity gains
        \item 1B+ daily translations
        \item 80\% customer service automation
    \end{itemize}
}

% Hands-on exercise preview
\begin{frame}{Week 4 Lab: Build Your Own Translator}
    \textbf{In today's lab, you will:}
    \begin{enumerate}
        \item Implement a complete seq2seq model from scratch
        \item Train it on English-French translation
        \item Add attention mechanism and see the improvement
        \item Visualize attention weights on real examples
        \item Compare different beam search strategies
        \item Connect your implementation to modern transformers
    \end{enumerate}
    
    \vspace{1em}
    \textbf{Dataset:} 10K English-French sentence pairs
    
    \textbf{Tools:} PyTorch, Jupyter notebook with interactive visualizations
    
    \textbf{Outcome:} Working translator that you can test with your own sentences!
    
    \vspace{1em}
    \begin{center}
    \colorbox{lightblue!30}{
        \parbox{0.8\textwidth}{
            \centering
            \textbf{Ready to build the technology behind Google Translate?}
        }
    }
    \end{center}
\end{frame}

% Key takeaways
\begin{frame}{Key Takeaways: Breaking Free from Fixed Length}
    \textbf{Core Concepts Mastered:}
    \begin{enumerate}
        \item \highlight{Variable-length problem:} Why 1:1 mapping fails
        \item \highlight{Encoder-decoder solution:} Separate encoding from decoding
        \item \highlight{Information bottleneck:} Single context vector limitation
        \item \highlight{Attention mechanism:} Dynamic focus on relevant inputs
        \item \highlight{Beam search:} Efficient search through output space
    \end{enumerate}
    
    \vspace{1em}
    \textbf{Modern Relevance:}
    \begin{itemize}
        \item Foundation of all current language models
        \item Core principle in ChatGPT, Copilot, and translation systems
        \item Attention evolved into self-attention (Transformers)
        \item Encoder-decoder architectures still dominant
    \end{itemize}
    
    \vspace{1em}
    \eqbox{
        \textbf{Next Week:} Transformers - "Attention is All You Need"
    }
    
    \vspace{0.5em}
    \textit{You now understand the foundation that made modern AI possible!}
\end{frame}

% References
\begin{frame}{References and Further Reading}
    \textbf{Foundational Papers:}
    \begin{itemize}
        \item Sutskever et al. (2014). "Sequence to Sequence Learning with Neural Networks"
        \item Bahdanau et al. (2015). "Neural Machine Translation by Jointly Learning to Align and Translate"
        \item Luong et al. (2015). "Effective Approaches to Attention-based Neural Machine Translation"
        \item Vaswani et al. (2017). "Attention Is All You Need"
    \end{itemize}
    
    \vspace{0.5em}
    \textbf{Modern Applications:}
    \begin{itemize}
        \item Wu et al. (2016). "Google's Neural Machine Translation System"
        \item Radford et al. (2019). "Language Models are Unsupervised Multitask Learners" (GPT-2)
        \item Brown et al. (2020). "Language Models are Few-Shot Learners" (GPT-3)
    \end{itemize}
    
    \vspace{0.5em}
    \textbf{Interactive Resources:}
    \begin{itemize}
        \item The Illustrated Transformer (Jay Alammar)
        \item OpenAI's GPT papers and blog posts
        \item Hugging Face Transformers documentation
        \item Today's lab notebook with complete implementations
    \end{itemize}
\end{frame}

\end{document}