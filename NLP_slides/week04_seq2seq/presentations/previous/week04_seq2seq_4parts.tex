\documentclass[8pt,aspectratio=169]{beamer}
\usepackage[utf8]{inputenc}
\usepackage{graphicx}
\usepackage{amsmath,amssymb}
\usepackage{algorithm2e}
\usepackage{listings}
\usepackage{xcolor}
\usepackage{tikz}
\usepackage{pgfplots}
\pgfplotsset{compat=1.17}
\usepackage{subfigure}
\usepackage{hyperref}
\usepackage{tcolorbox}

% Theme settings
\usetheme{Frankfurt}
\usecolortheme{seahorse}
\setbeamertemplate{navigation symbols}{}
\setbeamertemplate{footline}[frame number]

% Custom colors (matching educational color scheme)
\definecolor{darkblue}{RGB}{0,51,102}
\definecolor{lightblue}{RGB}{173,216,230}
\definecolor{codegreen}{RGB}{0,128,0}
\definecolor{codegray}{RGB}{150,150,150}
\definecolor{codepurple}{RGB}{128,0,128}
\definecolor{backcolor}{RGB}{245,245,245}

% Code listing settings
\lstset{
    backgroundcolor=\color{backcolor},
    basicstyle=\ttfamily\tiny,
    breakatwhitespace=false,
    breaklines=true,
    captionpos=b,
    commentstyle=\color{codegreen},
    keywordstyle=\color{blue},
    numberstyle=\tiny\color{codegray},
    stringstyle=\color{codepurple},
    showstringspaces=false,
    frame=single,
    numbers=left,
    language=Python
}

% Include shared slide layouts
% SLIDE LAYOUT TEMPLATES FOR NLP COURSE
% Three consistent layouts used throughout the course

% ==============================================================
% LAYOUT 1: CONCEPT SLIDE
% Used for: Theory, definitions, mathematical concepts
% Features: Title, main content area with bullets/equations, optional figure
% ==============================================================

\newcommand{\conceptslide}[3]{ % #1: title, #2: content, #3: optional figure
\begin{frame}[t]{#1}
    \begin{columns}[T]
        \begin{column}{0.6\textwidth}
            #2
        \end{column}
        \begin{column}{0.35\textwidth}
            \centering
            #3
        \end{column}
    \end{columns}
\end{frame}
}

% ==============================================================
% LAYOUT 2: CODE & IMPLEMENTATION SLIDE
% Used for: Code examples, algorithms, implementation details
% Features: Title, code block, explanation text
% ==============================================================

% Note: For code slides, use \begin{frame}[fragile] manually

\newcommand{\codeexplanation}[1]{%
    \vspace{0.5em}
    \small
    #1
}

\newcommand{\codeblock}[2][Python]{%
    \begin{lstlisting}[language=#1, basicstyle=\ttfamily\tiny]
#2
    \end{lstlisting}
}

\newcommand{\codeslide}[3]{%
    \begin{frame}[fragile]{#1}
    \begin{columns}[T]
        \column{0.55\textwidth}
        \codeblock{#2}
        \column{0.43\textwidth}
        \codeexplanation{#3}
    \end{columns}
    \end{frame}
}

% ==============================================================
% LAYOUT 3: RESULTS & VISUALIZATION SLIDE
% Used for: Experimental results, charts, comparisons
% Features: Title, large visualization area, key insights
% ==============================================================

\newcommand{\resultslide}[3]{ % #1: title, #2: visualization, #3: insights
\begin{frame}[t]{#1}
    \centering
    \vspace{-0.5em}
    #2
    \vspace{0.5em}
    
    \begin{block}{Key Insights}
        #3
    \end{block}
\end{frame}
}

% Additional utility commands for consistency

% Highlighted text
\newcommand{\highlight}[1]{\textcolor{blue}{\textbf{#1}}}

% Math notation shortcuts
\newcommand{\prob}[1]{P(#1)}
\newcommand{\given}{\mid}
\newcommand{\argmax}{\operatorname{argmax}}
\newcommand{\softmax}{\operatorname{softmax}}

% Standard figure size
\newcommand{\stdfigsizeinchins}{0.4\textwidth}

% For code blocks, use lstlisting directly:
% \begin{lstlisting}[language=Python]
% ... code ...
% \end{lstlisting}

% Equation highlight box
\newcommand{\eqbox}[1]{
    \begin{center}
    \colorbox{lightblue!20}{
        \parbox{0.8\textwidth}{
            \vspace{0.3em}
            \centering
            #1
            \vspace{0.3em}
        }
    }
    \end{center}
}

\title[Week 4: Seq2Seq]{Natural Language Processing Course}
\subtitle{Week 4: Sequence-to-Sequence Models}
\author{Breaking the Fixed-Length Barrier}
\institute{NLP Course 2025}
\date{}

\begin{document}

% Title page
\begin{frame}
    \titlepage
\end{frame}

% Course overview
\begin{frame}{Week 4: The Four-Part Journey}
    \textbf{Today's Learning Path:}
    
    \vspace{1em}
    \begin{enumerate}
        \item \highlight{The Problem:} Why variable-length sequences matter
        \item \highlight{The Solution:} Encoder-decoder architecture
        \item \highlight{The Breakthrough:} Attention mechanism revolution
        \item \highlight{Modern Impact:} From research to your daily AI tools
    \end{enumerate}
    
    \vspace{1em}
    \eqbox{
        \textbf{Core Question:} How do we translate ``Hello world'' to ``Bonjour le monde''?
    }
    
    \vspace{1em}
    \textbf{By the end:} You'll understand the technology behind Google Translate, GitHub Copilot, and the foundation of ChatGPT!
\end{frame}

% ============================================================
% PART 1: THE PROBLEM
% ============================================================

\section{Part 1: The Problem - Why Variable-Length Sequences Matter}

\begin{frame}[plain]
    \centering
    \vspace{2cm}
    {\Huge \textbf{Part 1}}\\
    \vspace{0.5cm}
    {\LARGE \textbf{The Problem}}\\
    \vspace{0.5cm}
    {\large Why Variable-Length Sequences Matter}\\
    \vspace{1cm}
    {\normalsize Breaking free from the fixed-length prison}
\end{frame}

\conceptslide{Your Daily AI Interactions (2024)}{
    \textbf{Every day, you use AI systems that solve this problem:}
    \begin{itemize}
        \item \textbf{Google Translate:} 1+ billion daily translations
        \item \textbf{GitHub Copilot:} 1+ million developers, 40\% faster coding
        \item \textbf{Email Summaries:} Long email → key points (Gmail, Outlook)
        \item \textbf{Customer Service:} 80\% first-line automation
        \item \textbf{ChatGPT:} Variable question → variable answer
    \end{itemize}
    
    \vspace{0.5em}
    \textbf{The Challenge They All Solve:}
    \begin{itemize}
        \item Input length does not equal Output length
        \item No fixed relationship between input and output size
        \item Need to handle any length combination
    \end{itemize}
}{
    \includegraphics[width=\textwidth]{../figures/week4_modern_applications_ecosystem.pdf}
}

\conceptslide{The Core Challenge: Length Mismatch}{
    \textbf{Translation Examples:}
    \begin{itemize}
        \item English: "I love you" (3 words) → French: "Je t'aime" (2 words)
        \item English: "Thank you" (2 words) → Japanese: "arigato" (1 word)
        \item English: "Good morning" (2 words) → German: "Guten Morgen" (2 words)
    \end{itemize}
    
    \vspace{0.5em}
    \textbf{Other Variable-Length Tasks:}
    \begin{itemize}
        \item \textbf{Summarization:} Article (500 words) → Summary (50 words)
        \item \textbf{Code Generation:} Comment (1 line) → Function (20 lines)
        \item \textbf{Q\&A:} Question (10 words) → Answer (100 words)
    \end{itemize}
    
    \vspace{0.5em}
    \textbf{Traditional RNN Limitation:}
    \begin{itemize}
        \item Each input produces exactly one output
        \item Fixed 1:1 input-output mapping
        \item Cannot handle length mismatches
    \end{itemize}
}{
    \textbf{Failed Approaches:}
    \begin{enumerate}
        \item Pad to maximum length (wasteful)
        \item Truncate long sequences (loses information)
        \item Force 1:1 word mapping (doesn't work)
    \end{enumerate}
    
    \vspace{1em}
    \colorbox{red!20}{
        \parbox{0.9\textwidth}{
            \centering
            \textbf{We need to decouple input and output lengths!}
        }
    }
}

\begin{frame}{Part 1 Key Insight}
    \begin{center}
    \vspace{1.5cm}
    {\Large \textbf{The Fundamental Problem}}\\
    \vspace{1cm}
    
    \colorbox{lightblue!30}{
        \parbox{0.8\textwidth}{
            \centering
            \large
            Traditional neural networks assume:\\
            \textbf{Input Length = Output Length}\\
            \vspace{0.5em}
            But real-world tasks need:\\
            \textbf{Variable Input → Variable Output}
        }
    }
    
    \vspace{1.5cm}
    {\large Next: How do we solve this architectural limitation?}
    \end{center}
\end{frame}

% ============================================================
% PART 2: THE SOLUTION
% ============================================================

\section{Part 2: The Solution - Encoder-Decoder Architecture}

\begin{frame}[plain]
    \centering
    \vspace{2cm}
    {\Huge \textbf{Part 2}}\\
    \vspace{0.5cm}
    {\LARGE \textbf{The Solution}}\\
    \vspace{0.5cm}
    {\large Encoder-Decoder Architecture}\\
    \vspace{1cm}
    {\normalsize Separating understanding from generation}
\end{frame}

\conceptslide{The Brilliant Insight: Separate Encoding from Decoding}{
    \textbf{Human Translation Process:}
    \begin{enumerate}
        \item \textbf{Read} entire source sentence
        \item \textbf{Understand} the meaning (internal representation)
        \item \textbf{Generate} target sentence in new language
    \end{enumerate}
    
    \vspace{0.5em}
    \textbf{Computer Translation (Seq2Seq):}
    \begin{enumerate}
        \item \textbf{Encoder:} Process entire input → "thought" vector
        \item \textbf{Context:} Compressed understanding
        \item \textbf{Decoder:} Generate output from "thought"
    \end{enumerate}
    
    \vspace{0.5em}
    \eqbox{
        \text{Input} \xrightarrow{\text{Encoder}} \text{Context} \xrightarrow{\text{Decoder}} \text{Output}
    }
}{
    \textbf{Key Benefits:}
    \begin{itemize}
        \item Input and output can have different lengths
        \item Encoder focuses on understanding
        \item Decoder focuses on generation
        \item Separates concerns cleanly
    \end{itemize}
    
    \vspace{1em}
    \textbf{Breakthrough Impact (2014):}
    \begin{itemize}
        \item First neural translation system
        \item Outperformed phrase-based systems
        \item End-to-end learning possible
    \end{itemize}
}

% Architecture visualization
\resultslide{The Complete Encoder-Decoder Architecture}{
    \includegraphics[width=\textwidth]{../figures/week4_encoder_decoder_architecture.pdf}
}{
    \begin{itemize}
        \item \textbf{Encoder RNNs:} Process input sequence sequentially
        \item \textbf{Context Vector:} Fixed-size "thought" representation
        \item \textbf{Decoder RNNs:} Generate output conditioned on context
        \item \textbf{Variable Lengths:} Input $T$ steps, output $T'$ steps
    \end{itemize}
}

\conceptslide{Mathematical Formulation}{
    \textbf{Encoder Phase:}
    \eqbox{
        h_t = \text{LSTM}(h_{t-1}, x_t) \quad \text{for } t = 1, \ldots, T
    }
    \eqbox{
        c = h_T \quad \text{(context vector = final encoder state)}
    }
    
    \textbf{Decoder Phase:}
    \eqbox{
        s_t = \text{LSTM}(s_{t-1}, y_{t-1}, c) \quad \text{for } t = 1, \ldots, T'
    }
    \eqbox{
        P(y_t \given y_{<t}, x) = \softmax(W_s s_t + b)
    }
    
    \textbf{Training Objective:}
    \eqbox{
        \max \sum_{t=1}^{T'} \log P(y_t^* \given y_{<t}^*, x)
    }
}{
    \textbf{Key Variables:}
    \begin{itemize}
        \item $T$: Input sequence length (variable)
        \item $T'$: Output sequence length (variable)
        \item $c$: Context vector (fixed size)
        \item $h_t$: Encoder hidden states
        \item $s_t$: Decoder hidden states
    \end{itemize}
    
    \vspace{0.5em}
    \textbf{Training Strategy:}
    \begin{itemize}
        \item Teacher forcing during training
        \item Cross-entropy loss
        \item Backpropagation through time
    \end{itemize}
}

\begin{frame}{Part 2 Key Insight}
    \begin{center}
    \vspace{1.5cm}
    {\Large \textbf{The Architectural Solution}}\\
    \vspace{1cm}
    
    \colorbox{green!20}{
        \parbox{0.8\textwidth}{
            \centering
            \large
            \textbf{Encoder:} Any length input $\rightarrow$ Fixed-size understanding\\
            \vspace{0.5em}
            \textbf{Decoder:} Fixed-size understanding $\rightarrow$ Any length output\\
            \vspace{0.5em}
            \textit{Separating concerns enables variable-length processing!}
        }
    }
    
    \vspace{1.5cm}
    {\large But wait... there's a problem with this approach...}
    \end{center}
\end{frame}

% ============================================================
% PART 3: THE BREAKTHROUGH
% ============================================================

\section{Part 3: The Breakthrough - Attention Mechanism}

\begin{frame}[plain]
    \centering
    \vspace{2cm}
    {\Huge \textbf{Part 3}}\\
    \vspace{0.5cm}
    {\LARGE \textbf{The Breakthrough}}\\
    \vspace{0.5cm}
    {\large Attention Mechanism Revolution}\\
    \vspace{1cm}
    {\normalsize Solving the information bottleneck}
\end{frame}

\resultslide{The Information Bottleneck Problem}{
    \includegraphics[width=\textwidth]{../figures/week4_information_bottleneck_analysis.pdf}
}{
    \begin{itemize}
        \item \textbf{Problem:} All input information compressed into single context vector
        \item \textbf{Effect:} Quality degrades rapidly with sentence length
        \item \textbf{Bottleneck:} 20-word sentence → 256-dim vector (95\% compression!)
        \item \textbf{Real Impact:} Pre-2016 Google Translate limited to ~20 words
    \end{itemize}
}

\conceptslide{The Attention Revolution (2015)}{
    \textbf{The Key Insight:}
    \begin{itemize}
        \item Don't compress everything into one vector
        \item Keep \textit{all} encoder hidden states
        \item Let decoder \textit{choose} what to focus on
        \item Different output words attend to different input words
    \end{itemize}
    
    \vspace{0.5em}
    \textbf{Attention Mechanism:}
    \eqbox{
        c_t = \sum_{i=1}^{T} \alpha_{t,i} h_i
    }
    
    \textbf{Attention Weights:}
    \eqbox{
        \alpha_{t,i} = \frac{\exp(e_{t,i})}{\sum_{j=1}^{T} \exp(e_{t,j})}
    }
    
    \textbf{Alignment Score:}
    \eqbox{
        e_{t,i} = \text{align}(s_{t-1}, h_i)
    }
}{
    \textbf{Three Main Types:}
    \begin{itemize}
        \item \textbf{Dot Product:} $e_{t,i} = s_t \cdot h_i$
        \item \textbf{Additive:} $e_{t,i} = v^T \tanh(W_s s_t + W_h h_i)$
        \item \textbf{Scaled Dot:} $e_{t,i} = \frac{s_t \cdot h_i}{\sqrt{d}}$
    \end{itemize}
    
    \vspace{0.5em}
    \textbf{Revolutionary Results:}
    \begin{itemize}
        \item Google NMT (2016): 60\% improvement
        \item Handles 80+ word sentences
        \item Foundation for all modern LLMs
    \end{itemize}
}

\resultslide{Attention in Action: What the Model Looks At}{
    \includegraphics[width=\textwidth]{../figures/week4_attention_comprehensive_analysis.pdf}
}{
    \begin{itemize}
        \item \textbf{Heatmap Interpretation:} Darker cells = higher attention
        \item \textbf{Alignment Discovery:} Model learns word correspondences automatically
        \item \textbf{Flexibility:} One target word can attend to multiple source words
        \item \textbf{Interpretability:} We can see what the model is "thinking"
    \end{itemize}
}

\resultslide{Beam Search: Finding the Best Translation}{
    \includegraphics[width=\textwidth]{../figures/week4_beam_search_algorithm.pdf}
}{
    \begin{itemize}
        \item \textbf{Problem:} Exponentially many possible output sequences
        \item \textbf{Greedy:} Pick best word at each step (gets stuck)
        \item \textbf{Beam Search:} Keep top-k partial sequences
        \item \textbf{Trade-off:} Beam size vs computational cost
    \end{itemize}
}

\begin{frame}{Part 3 Key Insight}
    \begin{center}
    \vspace{1.5cm}
    {\Large \textbf{The Attention Breakthrough}}\\
    \vspace{1cm}
    
    \colorbox{yellow!20}{
        \parbox{0.8\textwidth}{
            \centering
            \large
            \textbf{Before:} All information → 1 context vector\\
            \vspace{0.5em}
            \textbf{After:} Decoder chooses what to focus on\\
            \vspace{0.5em}
            \textit{"Don't compress everything - let the model decide!"}
        }
    }
    
    \vspace{1.5cm}
    {\large This insight enabled modern AI as we know it...}
    \end{center}
\end{frame}

% ============================================================
% PART 4: MODERN IMPACT
% ============================================================

\section{Part 4: Modern Impact - From Research to Production}

\begin{frame}[plain]
    \centering
    \vspace{2cm}
    {\Huge \textbf{Part 4}}\\
    \vspace{0.5cm}
    {\LARGE \textbf{Modern Impact}}\\
    \vspace{0.5cm}
    {\large From Research to Production}\\
    \vspace{1cm}
    {\normalsize How seq2seq principles power today's AI}
\end{frame}

\resultslide{From 2014 to 2024: The Evolution}{
    \includegraphics[width=\textwidth]{../figures/week4_performance_evolution_timeline.pdf}
}{
    \begin{itemize}
        \item \textbf{2014:} Seq2Seq proof of concept (25 BLEU)
        \item \textbf{2015:} Attention breakthrough (35 BLEU)
        \item \textbf{2016:} Google NMT production deployment (45 BLEU)
        \item \textbf{2017:} Transformer - "Attention is All You Need" (52 BLEU)
        \item \textbf{2024:} Modern LLMs with seq2seq foundations (68+ BLEU)
    \end{itemize}
}

\conceptslide{Connecting to Modern AI Systems}{
    \textbf{How Seq2Seq Principles Live On:}
    \begin{itemize}
        \item \textbf{ChatGPT:} Uses encoder-decoder principles with self-attention
        \item \textbf{GitHub Copilot:} Comment → code is pure seq2seq
        \item \textbf{BERT:} Encoder-only with attention mechanisms
        \item \textbf{T5:} "Text-to-Text Transfer" - explicit seq2seq
        \item \textbf{Google Translate:} Modern transformer encoder-decoder
    \end{itemize}
    
    \vspace{0.5em}
    \textbf{What Changed:}
    \begin{itemize}
        \item RNNs → Transformers (parallelization)
        \item Single attention → Multi-head attention
        \item Fixed vocab → Subword tokenization
        \item Task-specific → Pre-training + fine-tuning
    \end{itemize}
    
    \vspace{0.5em}
    \eqbox{
        \textbf{Core Principle Unchanged:} Variable input → Variable output
    }
}{
    \textbf{2024 Market Impact:}
    \begin{itemize}
        \item Translation: \$15.7B market
        \item Code assistance: \$8.5B market  
        \item Text summarization: \$12.3B market
        \item Conversational AI: \$45.2B market
    \end{itemize}
    
    \vspace{1em}
    \textbf{Scale Evolution:}
    \begin{itemize}
        \item 2014: 10M parameters
        \item 2024: 1.7T parameters
        \item 5,000× scale increase!
    \end{itemize}
}

\conceptslide{Industry Applications You Use Daily}{
    \textbf{Translation \& Localization:}
    \begin{itemize}
        \item Google Translate: Real-time conversation mode
        \item DeepL: 1 billion people served monthly
        \item Microsoft Translator: Built into Office suite
    \end{itemize}
    
    \textbf{Development \& Productivity:}
    \begin{itemize}
        \item GitHub Copilot: Autocomplete for code
        \item Tabnine: AI-powered code suggestions
        \item CodeT5: Natural language → SQL queries
    \end{itemize}
    
    \textbf{Communication \& Content:}
    \begin{itemize}
        \item Email summarization (Gmail Smart Compose)
        \item Meeting notes → action items
        \item Document translation with formatting
    \end{itemize}
}{
    \textbf{Performance Benchmarks (2024):}
    \begin{itemize}
        \item Translation accuracy: 95\%+ for common languages
        \item Response time: <100ms for most queries
        \item Availability: 99.9\%+ uptime
        \item Languages supported: 100+ pairs
    \end{itemize}
    
    \vspace{1em}
    \textbf{Why It Matters to You:}
    \begin{itemize}
        \item Foundational knowledge for AI careers
        \item Basis for understanding modern LLMs
        \item Essential for ML engineering roles
    \end{itemize}
}

\begin{frame}{The Bridge to Week 5: Transformers}
    \textbf{What Transformers Keep from Seq2Seq:}
    \begin{itemize}
        \item Encoder-decoder architecture
        \item Attention mechanism (enhanced to multi-head)
        \item Variable-length input/output
        \item Teacher forcing training
    \end{itemize}
    
    \vspace{0.5em}
    \textbf{What Transformers Add:}
    \begin{itemize}
        \item \highlight{Self-attention:} Attention within sequences
        \item \highlight{Parallelization:} No more sequential RNN processing
        \item \highlight{Multi-head:} Multiple attention patterns simultaneously
        \item \highlight{Position encoding:} Handle word order without RNNs
    \end{itemize}
    
    \vspace{1em}
    \eqbox{
        \textbf{Next Week:} "Attention Is All You Need" - The Transformer Revolution
    }
    
    \vspace{0.5em}
    \textit{You now have the foundation to understand the architecture behind ChatGPT!}
\end{frame}

% Summary slide
\begin{frame}{Week 4 Complete: From Problem to Modern AI}
    \textbf{The Four-Part Journey Completed:}
    
    \vspace{0.5em}
    \begin{enumerate}
        \item \textcolor{red}{\textbf{Problem:}} Variable-length sequences broke traditional RNNs
        \item \textcolor{blue}{\textbf{Solution:}} Encoder-decoder architecture enabled variable I/O
        \item \textcolor{green}{\textbf{Breakthrough:}} Attention solved the information bottleneck
        \item \textcolor{orange}{\textbf{Impact:}} These principles power billion-dollar AI systems
    \end{enumerate}
    
    \vspace{1em}
    \textbf{What You Can Now Do:}
    \begin{itemize}
        \item Explain why ChatGPT's architecture needed these foundations
        \item Design seq2seq systems for real applications
        \item Understand the attention mechanism powering modern AI
        \item Connect 2014 research breakthroughs to 2024 production systems
    \end{itemize}
    
    \vspace{1em}
    \eqbox{
        \textbf{Next:} Transformers - "Attention is All You Need"
    }
\end{frame}

% ============================================================
% APPENDIX: IMPLEMENTATION DETAILS
% ============================================================

\appendix

\section{Appendix: Implementation Details}

\begin{frame}[fragile]{Appendix A: PyTorch Encoder Implementation}
\begin{lstlisting}[language=Python]
class Seq2SeqEncoder(nn.Module):
    def __init__(self, vocab_size, embed_size, hidden_size, num_layers=1):
        super().__init__()
        self.embedding = nn.Embedding(vocab_size, embed_size)
        self.lstm = nn.LSTM(embed_size, hidden_size, num_layers, 
                           batch_first=True, dropout=0.2)
        self.hidden_size = hidden_size
        self.num_layers = num_layers
    
    def forward(self, x, lengths=None):
        # x: [batch_size, seq_len]
        embedded = self.embedding(x)  # [batch, seq_len, embed_size]
        
        # Optional: Pack sequences for efficiency
        if lengths is not None:
            embedded = nn.utils.rnn.pack_padded_sequence(
                embedded, lengths, batch_first=True, enforce_sorted=False)
        
        output, (h_n, c_n) = self.lstm(embedded)
        
        # Unpack if we packed
        if lengths is not None:
            output, _ = nn.utils.rnn.pad_packed_sequence(
                output, batch_first=True)
        
        return output, h_n, c_n
\end{lstlisting}
\end{frame}

\begin{frame}[fragile]{Appendix B: Attention Module Implementation}
\begin{lstlisting}[language=Python]
class AttentionMechanism(nn.Module):
    def __init__(self, hidden_size, attention_type='additive'):
        super().__init__()
        self.hidden_size = hidden_size
        self.attention_type = attention_type
        
        if attention_type == 'additive':
            self.W_a = nn.Linear(hidden_size, hidden_size, bias=False)
            self.U_a = nn.Linear(hidden_size, hidden_size, bias=False)
            self.v_a = nn.Linear(hidden_size, 1, bias=False)
        elif attention_type == 'multiplicative':
            self.W_a = nn.Linear(hidden_size, hidden_size, bias=False)
    
    def forward(self, decoder_hidden, encoder_outputs, mask=None):
        # decoder_hidden: [batch, hidden_size]
        # encoder_outputs: [batch, seq_len, hidden_size]
        
        batch_size, seq_len, hidden_size = encoder_outputs.size()
        
        if self.attention_type == 'dot':
            # Simple dot product attention
            scores = torch.bmm(encoder_outputs, 
                              decoder_hidden.unsqueeze(2)).squeeze(2)
        
        elif self.attention_type == 'additive':
            # Bahdanau attention
            decoder_hidden = decoder_hidden.unsqueeze(1).repeat(1, seq_len, 1)
            energy = torch.tanh(self.W_a(decoder_hidden) + self.U_a(encoder_outputs))
            scores = self.v_a(energy).squeeze(2)
        
        # Apply mask if provided (for padded sequences)
        if mask is not None:
            scores.masked_fill_(mask == 0, -float('inf'))
        
        # Compute attention weights
        attention_weights = F.softmax(scores, dim=1)
        
        # Compute context vector
        context = torch.bmm(attention_weights.unsqueeze(1), encoder_outputs).squeeze(1)
        
        return context, attention_weights
\end{lstlisting}
\end{frame}

\begin{frame}[fragile]{Appendix C: Complete Seq2Seq Model}
\begin{lstlisting}[language=Python]
class Seq2SeqWithAttention(nn.Module):
    def __init__(self, src_vocab_size, tgt_vocab_size, embed_size, hidden_size):
        super().__init__()
        self.encoder = Seq2SeqEncoder(src_vocab_size, embed_size, hidden_size)
        self.attention = AttentionMechanism(hidden_size, 'additive')
        
        # Decoder components
        self.tgt_embedding = nn.Embedding(tgt_vocab_size, embed_size)
        self.decoder_lstm = nn.LSTM(embed_size + hidden_size, hidden_size, batch_first=True)
        self.output_projection = nn.Linear(hidden_size, tgt_vocab_size)
        
    def forward(self, src, tgt, src_lengths=None):
        # Encode
        encoder_outputs, h_n, c_n = self.encoder(src, src_lengths)
        
        # Decode with attention
        batch_size, tgt_len = tgt.size()
        outputs = []
        
        decoder_hidden = h_n[-1]  # Use last layer
        decoder_cell = c_n[-1]
        
        for t in range(tgt_len):
            # Current target token
            tgt_token = tgt[:, t:t+1]  # [batch, 1]
            tgt_embedded = self.tgt_embedding(tgt_token)  # [batch, 1, embed]
            
            # Compute attention
            context, attention_weights = self.attention(decoder_hidden, encoder_outputs)
            
            # Concatenate target embedding with context
            decoder_input = torch.cat([tgt_embedded.squeeze(1), context], dim=1)
            decoder_input = decoder_input.unsqueeze(1)  # [batch, 1, embed+hidden]
            
            # Decoder step
            output, (decoder_hidden, decoder_cell) = self.decoder_lstm(
                decoder_input, (decoder_hidden.unsqueeze(0), decoder_cell.unsqueeze(0)))
            
            decoder_hidden = decoder_hidden.squeeze(0)
            decoder_cell = decoder_cell.squeeze(0)
            
            # Project to vocabulary
            logits = self.output_projection(output.squeeze(1))
            outputs.append(logits)
        
        return torch.stack(outputs, dim=1)  # [batch, tgt_len, vocab_size]
\end{lstlisting}
\end{frame}

\begin{frame}[fragile]{Appendix D: Beam Search Implementation}
\begin{lstlisting}[language=Python]
def beam_search(model, src_tensor, src_lengths, beam_size=4, max_length=50):
    model.eval()
    device = src_tensor.device
    
    # Encode source
    with torch.no_grad():
        encoder_outputs, h_n, c_n = model.encoder(src_tensor, src_lengths)
    
    # Initialize beams
    beams = [{
        'sequence': [START_TOKEN],
        'score': 0.0,
        'hidden': h_n[-1],
        'cell': c_n[-1]
    }]
    
    completed_beams = []
    
    for step in range(max_length):
        candidates = []
        
        for beam in beams:
            if beam['sequence'][-1] == END_TOKEN:
                completed_beams.append(beam)
                continue
            
            # Prepare input
            last_token = torch.tensor([[beam['sequence'][-1]]], device=device)
            
            # Decoder step with attention
            with torch.no_grad():
                context, _ = model.attention(beam['hidden'], encoder_outputs)
                tgt_embedded = model.tgt_embedding(last_token)
                
                decoder_input = torch.cat([tgt_embedded.squeeze(1), context], dim=1)
                decoder_input = decoder_input.unsqueeze(1)
                
                output, (new_hidden, new_cell) = model.decoder_lstm(
                    decoder_input, (beam['hidden'].unsqueeze(0), beam['cell'].unsqueeze(0)))
                
                logits = model.output_projection(output.squeeze(1))
                probs = F.softmax(logits, dim=1)
            
            # Get top-k candidates
            top_probs, top_indices = probs.topk(beam_size)
            
            for prob, idx in zip(top_probs[0], top_indices[0]):
                candidates.append({
                    'sequence': beam['sequence'] + [idx.item()],
                    'score': beam['score'] + torch.log(prob).item(),
                    'hidden': new_hidden.squeeze(0),
                    'cell': new_cell.squeeze(0)
                })
        
        # Keep top beam_size candidates
        beams = sorted(candidates, key=lambda x: x['score'])[-beam_size:]
    
    # Return best completed sequence
    if completed_beams:
        return max(completed_beams, key=lambda x: x['score'])['sequence']
    else:
        return beams[-1]['sequence']
\end{lstlisting}
\end{frame}

\begin{frame}{Appendix E: Hyperparameters and Training Tips}
    \begin{columns}[T]
        \begin{column}{0.5\textwidth}
            \textbf{Typical Hyperparameters:}
            \begin{itemize}
                \item \textbf{Hidden size:} 256-512
                \item \textbf{Embedding size:} 128-300
                \item \textbf{Num layers:} 1-4
                \item \textbf{Dropout:} 0.1-0.3
                \item \textbf{Learning rate:} 1e-4 to 1e-3
                \item \textbf{Beam size:} 4-8
                \item \textbf{Max length:} 50-200
            \end{itemize}
            
            \textbf{Training Best Practices:}
            \begin{itemize}
                \item Gradient clipping (important!)
                \item Learning rate scheduling
                \item Teacher forcing ratio annealing
                \item Early stopping on validation BLEU
            \end{itemize}
        \end{column}
        \begin{column}{0.5\textwidth}
            \textbf{Common Issues:}
            \begin{itemize}
                \item \textbf{Repetitive output:} Add coverage/repetition penalty
                \item \textbf{Poor long sequences:} Increase hidden size or add attention
                \item \textbf{Slow training:} Use packed sequences for variable lengths
                \item \textbf{Exposure bias:} Scheduled sampling or professor forcing
            \end{itemize}
            
            \textbf{Evaluation Metrics:}
            \begin{itemize}
                \item \textbf{BLEU:} Standard for translation (0-100)
                \item \textbf{ROUGE:} For summarization tasks
                \item \textbf{Perplexity:} Language modeling quality
                \item \textbf{Human evaluation:} Ultimate quality measure
            \end{itemize}
        \end{column}
    \end{columns}
\end{frame}

\begin{frame}{Appendix F: References and Further Reading}
    \textbf{Foundational Papers:}
    \begin{itemize}
        \item Sutskever et al. (2014). "Sequence to Sequence Learning with Neural Networks"
        \item Bahdanau et al. (2015). "Neural Machine Translation by Jointly Learning to Align and Translate"
        \item Luong et al. (2015). "Effective Approaches to Attention-based Neural Machine Translation"
        \item Cho et al. (2014). "Learning Phrase Representations using RNN Encoder-Decoder"
    \end{itemize}
    
    \textbf{Modern Context:}
    \begin{itemize}
        \item Wu et al. (2016). "Google's Neural Machine Translation System"
        \item Vaswani et al. (2017). "Attention Is All You Need" (Transformer)
        \item Devlin et al. (2018). "BERT: Pre-training of Deep Bidirectional Transformers"
    \end{itemize}
    
    \textbf{Interactive Resources:}
    \begin{itemize}
        \item Jay Alammar: "Visualizing A Neural Machine Translation Model"
        \item The Illustrated Transformer
        \item Hugging Face Transformers documentation
        \item OpenAI GPT papers and technical reports
    \end{itemize}
    
    \textbf{Implementations:}
    \begin{itemize}
        \item PyTorch seq2seq tutorial
        \item Fairseq toolkit (Facebook's seq2seq library)
        \item OpenNMT (open source neural machine translation)
    \end{itemize}
\end{frame}

\end{document}