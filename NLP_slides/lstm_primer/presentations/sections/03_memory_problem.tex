% Section 3: The Memory Problem

\begin{frame}[t]{The Memory Problem: What Should We Remember?}
\begin{center}
\textbf{Insight from Human Reading}
\end{center}
\vspace{3mm}

\begin{columns}[t]
\column{0.48\textwidth}
\textbf{Imagine Reading a Novel:}

\vspace{3mm}
\textit{Chapter 1: ``Alice was born in London in 1985. She had a happy childhood.''}

\textit{Chapter 3: ``After graduating from university, Alice moved to New York.''}

\textit{Chapter 7: ``Now 38 years old, Alice reflected on her life in \_\_\_''}

\vspace{3mm}
\textbf{What You Remember:}
\begin{itemize}
\item Alice (main character)
\item Born in London
\item Moved to New York
\item Currently 38 years old
\end{itemize}

\vspace{3mm}
\textbf{What You Forgot:}
\begin{itemize}
\item ``had a happy childhood''
\item ``graduating from university''
\item Exact wording
\item Minor details
\end{itemize}

\column{0.48\textwidth}
\textbf{Key Insight:} Memory is \textbf{selective}

\vspace{3mm}
\textbf{Human Memory Strategy:}
\begin{enumerate}
\item \textbf{Decide} what's important
\item \textbf{Keep} relevant information
\item \textbf{Forget} irrelevant details
\item \textbf{Update} as story progresses
\end{enumerate}

\vspace{3mm}
\textbf{What We Need in AI:}

\vspace{3mm}
\textbf{Forget Mechanism:}
\begin{itemize}
\item Remove outdated information
\item Clear memory when topic changes
\item Example: Forget ``chocolate'' after period
\end{itemize}

\vspace{3mm}
\textbf{Input Mechanism:}
\begin{itemize}
\item Decide what to store
\item Remember important words
\item Example: Store ``Paris'' strongly
\end{itemize}

\vspace{3mm}
\textbf{Output Mechanism:}
\begin{itemize}
\item Control what to reveal
\item Use memory when needed
\item Example: Recall ``Paris'' when predicting language
\end{itemize}

\vspace{3mm}
{\footnotesize\color{annotGray}This is exactly what LSTMs do!}

\end{columns}
\end{frame}