% LSTM Primer: BSc 10-Page Version
% Zero Prerequisites - All Notation Explained
\documentclass[8pt,aspectratio=169]{beamer}
\usetheme{Madrid}
\usecolortheme{seahorse}
\setbeamertemplate{navigation symbols}{}

\usepackage{amsmath}
\usepackage{graphicx}
\usepackage{xcolor}
\usepackage{tikz}

% Color definitions
\definecolor{mainGray}{RGB}{64,64,64}
\definecolor{annotGray}{RGB}{180,180,180}
\definecolor{backGray}{RGB}{240,240,240}
\definecolor{forgetRed}{RGB}{231,76,60}
\definecolor{inputGreen}{RGB}{46,204,113}
\definecolor{outputBlue}{RGB}{52,152,219}
\definecolor{cellYellow}{RGB}{241,196,15}
\definecolor{checkpointBlue}{RGB}{70,130,180}
\definecolor{glossaryPurple}{RGB}{155,89,182}

% BSc pedagogical boxes
\newcommand{\termbox}[2]{\colorbox{glossaryPurple!15}{\parbox{0.45\textwidth}{\footnotesize\textbf{#1:} #2}}}
\newcommand{\plainenglish}[1]{\colorbox{backGray}{\parbox{0.9\textwidth}{\footnotesize\textit{In plain English: #1}}}}
\newcommand{\checkpoint}[1]{\colorbox{checkpointBlue!15}{\parbox{0.9\textwidth}{\small\textbf{Checkpoint:} #1}}}

\newcommand{\given}{\mid}

\title{LSTM Networks: Teaching Machines Long-Term Memory}
\subtitle{A 10-Page BSc Introduction - Zero Prerequisites}
\author{NLP Course 2025}
\date{\today}

\begin{document}

\frame{\titlepage}

% PAGE 1: The Autocomplete Challenge
\begin{frame}[t]{Page 1: The Autocomplete Challenge}
\begin{center}
\textbf{The Problem: Predicting What Comes Next}
\end{center}
\vspace{2mm}

\begin{columns}[t]
\column{0.48\textwidth}
\includegraphics[width=\textwidth]{../figures/autocomplete_screenshot.pdf}

\vspace{2mm}
\textbf{What You Type:}

\textit{``I grew up in Paris. I went to school there for 12 years. I learned to speak fluent \_\_\_''}

\vspace{2mm}
\textbf{The Challenge:}
\begin{itemize}
\item Need to remember ``Paris'' (18 words back!)
\item Ignore recent irrelevant words
\item Connect city to language
\item Answer: \textbf{French}
\end{itemize}

\column{0.48\textwidth}
\textbf{Technical Terms Explained:}

\vspace{2mm}
\colorbox{glossaryPurple!15}{\parbox{0.45\textwidth}{\footnotesize\textbf{Sequence Modeling:} Predicting the next element based on all previous elements in a sequence (like words in a sentence)}}

\vspace{2mm}
\colorbox{glossaryPurple!15}{\parbox{0.45\textwidth}{\footnotesize\textbf{Context:} The previous words that give meaning to what comes next}}

\vspace{2mm}
\colorbox{glossaryPurple!15}{\parbox{0.45\textwidth}{\footnotesize\textbf{Long-term Dependency:} When information from far back (like ``Paris'') is needed to make a prediction now}}

\vspace{3mm}
\textbf{Why It's Hard:}
\begin{itemize}
\item Context can be 50-100+ words long
\item Only some words matter (``Paris'')
\item Others are noise (``there'', ``for'', ``12'')
\item Need selective memory
\end{itemize}

\vspace{3mm}
{\footnotesize\color{annotGray}This is the problem LSTMs were invented to solve}

\end{columns}
\end{frame}

% PAGE 2: Why Simple Approaches Fail
\begin{frame}[t]{Page 2: Why Simple Approaches Fail}
\begin{center}
\textbf{N-gram Models: The Baseline (And Why They Don't Work)}
\end{center}
\vspace{2mm}

\begin{columns}[t]
\column{0.55\textwidth}
\includegraphics[width=\textwidth]{../figures/context_window_comparison.pdf}

\vspace{2mm}
\textbf{What N-grams Do:}
\begin{itemize}
\item Count word sequences in training data
\item Look at last 1-2 words only
\item Pick most common next word
\end{itemize}

\vspace{2mm}
\plainenglish{If you've seen ``speak fluent'' 100 times in training, and ``English'' came next 60 times, predict ``English'' (even though ``French'' is correct here)}

\column{0.42\textwidth}
\textbf{Notation Explained:}

\vspace{2mm}
\colorbox{glossaryPurple!15}{\parbox{0.4\textwidth}{\footnotesize\textbf{N-gram:} A sequence of N words. Bigram = 2 words (``I love''), Trigram = 3 words (``I love chocolate'')}}

\vspace{2mm}
\colorbox{glossaryPurple!15}{\parbox{0.4\textwidth}{\footnotesize\textbf{$P(w_t \given w_{t-1})$:} Probability of word $w_t$ given previous word $w_{t-1}$. The vertical bar means ``given''}}

\vspace{3mm}
\textbf{Fatal Limitations:}

\vspace{2mm}
\textbf{1. Fixed Window:}
\begin{itemize}
\item Can only see 1-2 words back
\item ``Paris'' is 18 words away
\item Impossible to capture!
\end{itemize}

\vspace{2mm}
\textbf{2. No Understanding:}
\begin{itemize}
\item Pure counting, no meaning
\item Can't generalize patterns
\item Needs to see exact sequence
\end{itemize}

\vspace{3mm}
{\footnotesize\color{annotGray}We need a model with variable-length memory}

\end{columns}
\end{frame}

% PAGE 3: The Memory Problem
\begin{frame}[t]{Page 3: The Memory Problem - What We Need}
\begin{center}
\textbf{Insight from Human Reading}
\end{center}
\vspace{2mm}

\begin{columns}[t]
\column{0.48\textwidth}
\textbf{Human Memory Example:}

\vspace{2mm}
\textit{Chapter 1: ``Alice was born in London in 1985. She had a happy childhood.''}

\textit{Chapter 3: ``After graduating from university, Alice moved to New York.''}

\textit{Chapter 7: ``Now 38 years old, Alice reflected on her life in \_\_\_''}

\vspace{2mm}
\textbf{What You Remember:}
\begin{itemize}
\item Alice (main character) [YES]
\item Born in London [YES]
\item Moved to New York [YES]
\item Currently 38 [YES]
\end{itemize}

\vspace{2mm}
\textbf{What You Forgot:}
\begin{itemize}
\item ``had a happy childhood'' [NO]
\item ``graduating from university'' [NO]
\item Exact wording [NO]
\end{itemize}

\column{0.48\textwidth}
\textbf{Three Mechanisms We Need:}

\vspace{3mm}
\colorbox{forgetRed!15}{\parbox{0.45\textwidth}{\small\textbf{\color{forgetRed}1. Forget Gate:} Decide what to remove from memory\\
\textit{Example: Forget ``chocolate'' after period}}}

\vspace{2mm}
\colorbox{inputGreen!15}{\parbox{0.45\textwidth}{\small\textbf{\color{inputGreen}2. Input Gate:} Decide what to store\\
\textit{Example: Store ``Paris'' strongly}}}

\vspace{2mm}
\colorbox{outputBlue!15}{\parbox{0.45\textwidth}{\small\textbf{\color{outputBlue}3. Output Gate:} Decide what to use now\\
\textit{Example: Recall ``Paris'' when predicting language}}}

\vspace{3mm}
\textbf{Technical Terms:}

\vspace{2mm}
\colorbox{glossaryPurple!15}{\parbox{0.45\textwidth}{\footnotesize\textbf{Gate:} A learned decision mechanism that outputs values between 0 (block) and 1 (allow). Acts like a controllable valve}}

\vspace{2mm}
\colorbox{glossaryPurple!15}{\parbox{0.45\textwidth}{\footnotesize\textbf{Memory Cell:} A storage unit that holds information across time. Can be updated or erased by gates}}

\vspace{3mm}
{\footnotesize\color{annotGray}This is exactly what LSTMs implement with math!}

\end{columns}
\end{frame}

% PAGE 4: LSTM Architecture Overview
\begin{frame}[t]{Page 4: LSTM Architecture - The Complete Solution}
\begin{center}
\textbf{Long Short-Term Memory: Gated Memory Cells}
\end{center}
\vspace{2mm}

\begin{columns}[t]
\column{0.55\textwidth}
\includegraphics[width=\textwidth]{../figures/lstm_architecture.pdf}

\vspace{2mm}
\textbf{Key Components:}

\vspace{2mm}
\colorbox{cellYellow!15}{\parbox{0.5\textwidth}{\small\textbf{\color{cellYellow}Cell State $C_t$:} Long-term memory highway. Information flows through with minimal transformation}}

\vspace{2mm}
\colorbox{mainGray!15}{\parbox{0.5\textwidth}{\small\textbf{Hidden State $h_t$:} Short-term working memory. Used for predictions}}

\column{0.42\textwidth}
\textbf{Notation Glossary:}

\vspace{2mm}
\begin{itemize}
\item \textbf{$t$:} Time step (current word position)
\item \textbf{$t-1$:} Previous time step
\item \textbf{$x_t$:} Input at time $t$ (current word)
\item \textbf{$h_t$:} Hidden state at time $t$
\item \textbf{$C_t$:} Cell state at time $t$
\item \textbf{$\sigma$:} Sigmoid function (outputs 0 to 1)
\item \textbf{$\tanh$:} Tanh function (outputs -1 to 1)
\item \textbf{$\odot$:} Element-wise multiplication
\item \textbf{$W$:} Weight matrix (learned parameters)
\item \textbf{$b$:} Bias vector (learned parameters)
\item \textbf{$[a, b]$:} Concatenation (stick vectors together)
\end{itemize}

\vspace{2mm}
\textbf{Why Two States?}
\begin{itemize}
\item $C_t$: Long-term storage (like hard drive)
\item $h_t$: Current focus (like RAM)
\item Gates control information flow
\end{itemize}

\vspace{2mm}
{\footnotesize\color{annotGray}This dual-state design is the key innovation}

\end{columns}
\end{frame}

% PAGE 5: The Three Gates Explained
\begin{frame}[t]{Page 5: The Three Gates - How They Work}
\begin{center}
\textbf{Gate Mechanisms with Concrete Examples}
\end{center}
\vspace{2mm}

\begin{columns}[t]
\column{0.55\textwidth}
\includegraphics[width=\textwidth]{../figures/gate_activation_heatmap.pdf}

\vspace{2mm}
\textbf{What This Chart Shows:}
\begin{itemize}
\item Real gate values over sentence
\item Darker = higher (close to 1)
\item Lighter = lower (close to 0)
\end{itemize}

\vspace{2mm}
\textbf{Key Patterns:}
\begin{itemize}
\item \textbf{Forget gate:} High throughout, drops at period
\item \textbf{Input gate:} Peaks on content words
\item \textbf{Output gate:} High when predicting
\end{itemize}

\column{0.42\textwidth}
\textbf{{\color{forgetRed}Forget Gate} $f_t$:}

$$f_t = \sigma(W_f [h_{t-1}, x_t] + b_f)$$

\plainenglish{Look at previous state and current word. Decide how much of old memory to keep (0 = forget all, 1 = keep all)}

\vspace{2mm}
\textbf{Example:} At period: $f_t = 0.1$ (forget 90\%)

\vspace{3mm}
\textbf{{\color{inputGreen}Input Gate} $i_t$:}

$$i_t = \sigma(W_i [h_{t-1}, x_t] + b_i)$$
$$\tilde{C}_t = \tanh(W_C [h_{t-1}, x_t] + b_C)$$

\plainenglish{Create new candidate memory $\tilde{C}_t$. Gate $i_t$ decides how much to store (0 = ignore, 1 = fully store)}

\vspace{2mm}
\textbf{Example:} At ``Paris'': $i_t = 0.95$ (store 95\%)

\vspace{3mm}
\textbf{{\color{outputBlue}Output Gate} $o_t$:}

$$o_t = \sigma(W_o [h_{t-1}, x_t] + b_o)$$

\plainenglish{Decide how much memory to reveal for current prediction (0 = hide, 1 = expose all)}

\vspace{2mm}
\textbf{Example:} At ``fluent'': $o_t = 0.9$ (use 90\%)

\end{columns}
\end{frame}

% PAGE 6: How Cell State Works
\begin{frame}[t]{Page 6: Cell State - The Memory Highway}
\begin{center}
\textbf{Why LSTMs Can Remember 50-100+ Steps}
\end{center}
\vspace{2mm}

\begin{columns}[t]
\column{0.55\textwidth}
\includegraphics[width=\textwidth]{../figures/gradient_flow_comparison.pdf}

\vspace{2mm}
\textbf{The Update Equation:}

$$C_t = f_t \odot C_{t-1} + i_t \odot \tilde{C}_t$$

\vspace{2mm}
\plainenglish{New memory = (keep some old) + (add some new). The $\odot$ means multiply each number separately}

\vspace{2mm}
\textbf{Why Addition is Magic:}
\begin{itemize}
\item Old $C_{t-1}$ directly adds to new $C_t$
\item No matrix multiplication!
\item Information preserved unchanged
\item Gradients flow backward easily
\end{itemize}

\column{0.42\textwidth}
\textbf{Comparison - RNN vs LSTM:}

\vspace{2mm}
\textbf{RNN (Multiplicative):}
$$h_t = \tanh(W_h h_{t-1} + ...)$$

After 50 steps:
\begin{itemize}
\item Signal: $0.5^{50} \approx 10^{-15}$
\item Information lost!
\end{itemize}

\vspace{2mm}
\textbf{LSTM (Additive):}
$$C_t = f_t \odot C_{t-1} + ...$$

After 50 steps (with $f_t = 0.95$):
\begin{itemize}
\item Signal: $0.95^{50} \approx 0.08$
\item Still usable!
\end{itemize}

\vspace{2mm}
\colorbox{backGray}{\parbox{0.4\textwidth}{\footnotesize\textbf{10 billion times better!} This is why LSTMs work where RNNs fail}}

\vspace{2mm}
\textbf{Technical Terms:}

\vspace{2mm}
\colorbox{glossaryPurple!15}{\parbox{0.4\textwidth}{\footnotesize\textbf{Gradient:} How much to update weights during training. Vanishes in RNNs, preserved in LSTMs}}

\vspace{2mm}
{\footnotesize\color{annotGray}Addition creates a gradient highway}

\end{columns}
\end{frame}

% PAGE 7: Complete LSTM Equations
\begin{frame}[t]{Page 7: Complete LSTM Equations - All Together}
\begin{center}
\textbf{The Full Forward Pass (One Time Step)}
\end{center}
\vspace{2mm}

\begin{columns}[t]
\column{0.48\textwidth}
\textbf{All Six Equations:}

\vspace{2mm}
\begin{align*}
f_t &= \sigma(W_f [h_{t-1}, x_t] + b_f) && \text{{\color{forgetRed}Forget gate}}\\
i_t &= \sigma(W_i [h_{t-1}, x_t] + b_i) && \text{{\color{inputGreen}Input gate}}\\
\tilde{C}_t &= \tanh(W_C [h_{t-1}, x_t] + b_C) && \text{{\color{inputGreen}Candidate}}\\
C_t &= f_t \odot C_{t-1} + i_t \odot \tilde{C}_t && \text{{\color{cellYellow}Cell update}}\\
o_t &= \sigma(W_o [h_{t-1}, x_t] + b_o) && \text{{\color{outputBlue}Output gate}}\\
h_t &= o_t \odot \tanh(C_t) && \text{Hidden state}
\end{align*}

\vspace{2mm}
\textbf{Activation Functions:}

\vspace{2mm}
\textbf{Sigmoid:} $\sigma(z) = \frac{1}{1 + e^{-z}}$
\begin{itemize}
\item Range: (0, 1)
\item Used for gates (0 = close, 1 = open)
\end{itemize}

\vspace{2mm}
\textbf{Tanh:} $\tanh(z) = \frac{e^z - e^{-z}}{e^z + e^{-z}}$
\begin{itemize}
\item Range: (-1, 1)
\item Used for cell state (can be positive or negative)
\end{itemize}

\column{0.48\textwidth}
\textbf{Concrete Numerical Example:}

\vspace{2mm}
Input: word ``love'' (after ``I'')

\vspace{2mm}
\textbf{Step 1:} Compute gates
\begin{itemize}
\item $f_t = [0.62, 0.45, 0.69, ...]$ (keep some)
\item $i_t = [0.77, 0.69, 0.38, ...]$ (add much)
\item $o_t = [0.71, 0.75, 0.45, ...]$ (reveal most)
\end{itemize}

\vspace{2mm}
\textbf{Step 2:} Create candidate
\begin{itemize}
\item $\tilde{C}_t = [0.54, -0.29, 0.72, ...]$
\end{itemize}

\vspace{2mm}
\textbf{Step 3:} Update cell state
\begin{itemize}
\item Old: $C_{t-1} = [0.5, 0.3, 0.2, ...]$
\item Keep: $f_t \odot C_{t-1} = [0.31, 0.14, 0.14, ...]$
\item Add: $i_t \odot \tilde{C}_t = [0.42, -0.20, 0.27, ...]$
\item New: $C_t = [0.73, -0.06, 0.41, ...]$
\end{itemize}

\vspace{2mm}
\textbf{Step 4:} Compute output
\begin{itemize}
\item $h_t = o_t \odot \tanh(C_t)$
\item $h_t = [0.44, -0.04, 0.17, ...]$
\end{itemize}

\vspace{2mm}
{\footnotesize\color{annotGray}This happens at every word in the sequence!}

\end{columns}
\end{frame}

% PAGE 8: Training LSTMs
\begin{frame}[t]{Page 8: Training LSTMs}
\begin{center}
\textbf{Backpropagation Through Time (BPTT)}
\end{center}
\vspace{2mm}

\begin{columns}[t]
\column{0.55\textwidth}
\includegraphics[width=\textwidth]{../figures/training_progression.pdf}

\vspace{2mm}
\textbf{What This Shows:}
\begin{itemize}
\item Training loss decreasing over time
\item Validation accuracy improving
\item Model learning long-range patterns
\end{itemize}

\vspace{2mm}
\textbf{Training Process:}
\begin{enumerate}
\item Forward pass through entire sequence
\item Compute loss (prediction error)
\item Backward pass to compute gradients
\item Update all weights ($W_f, W_i, W_C, W_o$)
\item Repeat for thousands of sequences
\end{enumerate}

\column{0.42\textwidth}
\textbf{Technical Terms:}

\vspace{2mm}
\colorbox{glossaryPurple!15}{\parbox{0.4\textwidth}{\footnotesize\textbf{BPTT:} Backpropagation Through Time. Compute gradients by unrolling the sequence and flowing errors backward from end to start}}

\vspace{2mm}
\colorbox{glossaryPurple!15}{\parbox{0.4\textwidth}{\footnotesize\textbf{Loss Function:} Measures prediction error. Usually cross-entropy for word prediction}}

\vspace{2mm}
\colorbox{glossaryPurple!15}{\parbox{0.4\textwidth}{\footnotesize\textbf{Gradient Clipping:} Cap gradients at threshold (e.g., 5.0) to prevent explosion}}

\vspace{3mm}
\textbf{Key Hyperparameters:}
\begin{itemize}
\item Hidden size: 256-512 (memory capacity)
\item Learning rate: 0.001 (update step size)
\item Batch size: 32-128 (sequences per update)
\item Sequence length: 50-100 (truncation)
\item Dropout: 0.2-0.5 (prevent overfitting)
\end{itemize}

\vspace{2mm}
{\footnotesize\color{annotGray}Training can take hours to days depending on data size}

\end{columns}
\end{frame}

% PAGE 9: Applications, Summary and Reference
\begin{frame}[t]{Page 9: Applications, Summary \& Quick Reference}
\begin{center}
\textbf{Where LSTMs Are Used + Summary}
\end{center}
\vspace{1mm}

\begin{columns}[t]
\column{0.32\textwidth}
\textbf{Applications:}

\vspace{1mm}
\textbf{NLP:} Translation, generation, sentiment

\textbf{Speech:} Recognition (Siri), music generation

\textbf{Time Series:} Stock, weather, energy, healthcare

\vspace{2mm}
\includegraphics[width=\textwidth]{../figures/model_comparison_table.pdf}

\vspace{1mm}
\textbf{Key Insight:} LSTMs still relevant in 2025! Complementary to Transformers.

\column{0.34\textwidth}
\textbf{Equations Reference:}

{\tiny
\begin{align*}
f_t &= \sigma(W_f [h_{t-1}, x_t] + b_f)\\
i_t &= \sigma(W_i [h_{t-1}, x_t] + b_i)\\
\tilde{C}_t &= \tanh(W_C [h_{t-1}, x_t] + b_C)\\
C_t &= f_t \odot C_{t-1} + i_t \odot \tilde{C}_t\\
o_t &= \sigma(W_o [h_{t-1}, x_t] + b_o)\\
h_t &= o_t \odot \tanh(C_t)
\end{align*}
}

\vspace{1mm}
\textbf{Summary:}
\begin{itemize}
\item \textbf{Problem:} 50-100 step memory needed
\item \textbf{Solution:} 3 gates + additive cell state
\item \textbf{Impact:} Google Translate, foundation for Transformers
\end{itemize}

\column{0.32\textwidth}
\textbf{When to Use:}

\vspace{1mm}
\textbf{LSTMs:}
\begin{itemize}
\item Time series (SOTA)
\item Real-time
\item Mobile/edge
\item Limited data
\end{itemize}

\vspace{1mm}
\textbf{Transformers:}
\begin{itemize}
\item Large datasets
\item Parallel training
\item Bidirectional
\item SOTA NLP
\end{itemize}

\vspace{1mm}
\textbf{Notation:}
\begin{itemize}
\item $t$: time, $x_t$: input
\item $h_t$: hidden, $C_t$: cell
\item $\sigma$: sigmoid (0-1)
\item $\tanh$: (-1-1)
\item $\odot$: element-wise ×
\end{itemize}

\vspace{1mm}
\textbf{Code:}
\begin{itemize}
\item PyTorch: \texttt{nn.LSTM()}
\item TensorFlow: \texttt{tf.keras.layers.LSTM()}
\end{itemize}

\vspace{1mm}
{\footnotesize Hochreiter \& Schmidhuber (1997), Colah's blog, PyTorch tutorials}

\end{columns}
\end{frame}

\end{document}