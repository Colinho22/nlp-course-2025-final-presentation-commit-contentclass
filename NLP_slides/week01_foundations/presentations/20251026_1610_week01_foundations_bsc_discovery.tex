% Week 1: Foundations and Statistical Language Modeling
% BSc Discovery-Based Presentation using Educational Framework
% Complete rewrite with problem-driven narrative

% Master Template for NLP Course
% Optimal Readability Layout Standard
% All presentations should include this template

\documentclass[8pt,aspectratio=169]{beamer}
\usetheme{Madrid}
\setbeamertemplate{navigation symbols}{}

% ====================================
% OPTIMAL READABILITY COLOR PALETTE
% ====================================
\definecolor{PureBlack}{HTML}{000000}      % Main text (21:1 contrast)
\definecolor{DeepBlue}{HTML}{003D7A}       % Primary accent (12.6:1 contrast)
\definecolor{DarkGray}{HTML}{4A4A4A}       % Secondary text (9.7:1 contrast)
\definecolor{LightGray}{HTML}{E5E5E5}      % Borders and grids
\definecolor{ChartBlue}{HTML}{0066CC}      % Chart primary
\definecolor{ChartOrange}{HTML}{FF8800}    % Chart secondary
\definecolor{ChartTeal}{HTML}{00A0A0}      % Chart tertiary
\definecolor{ChartPurple}{HTML}{8B4789}    % Chart quaternary
\definecolor{DarkGreen}{HTML}{228B22}      % Success/positive
\definecolor{DarkRed}{HTML}{CC0000}        % Warning/negative

% ====================================
% BEAMER COLOR CONFIGURATION
% ====================================
\setbeamercolor{structure}{fg=PureBlack}
\setbeamercolor{frametitle}{fg=PureBlack,bg=white}
\setbeamercolor{title}{fg=PureBlack,bg=white}
\setbeamercolor{subtitle}{fg=DarkGray}
\setbeamercolor{author}{fg=DarkGray}
\setbeamercolor{date}{fg=DarkGray}
\setbeamercolor{institute}{fg=DarkGray}

% Blocks - no backgrounds
\setbeamercolor{block title}{fg=PureBlack,bg=white}
\setbeamercolor{block body}{fg=PureBlack,bg=white}
\setbeamercolor{block title example}{fg=DarkGreen,bg=white}
\setbeamercolor{block body example}{fg=PureBlack,bg=white}
\setbeamercolor{block title alerted}{fg=DarkRed,bg=white}
\setbeamercolor{block body alerted}{fg=PureBlack,bg=white}

% Lists
\setbeamercolor{item}{fg=DeepBlue}
\setbeamercolor{subitem}{fg=ChartBlue}
\setbeamercolor{enumerate item}{fg=DeepBlue}
\setbeamercolor{enumerate subitem}{fg=ChartBlue}

% Text
\setbeamercolor{normal text}{fg=PureBlack,bg=white}
\setbeamercolor{alerted text}{fg=DarkRed}
\setbeamercolor{example text}{fg=DarkGreen}

% Footer
\setbeamercolor{footline}{fg=DarkGray,bg=white}
\setbeamercolor{page number in head/foot}{fg=DarkGray}

% ====================================
% FONT CONFIGURATION
% ====================================
\setbeamerfont{normal text}{size=\normalsize}
\setbeamerfont{frametitle}{size=\Large,series=\bfseries}
\setbeamerfont{title}{size=\huge,series=\bfseries}
\setbeamerfont{subtitle}{size=\large}
\setbeamerfont{author}{size=\normalsize}
\setbeamerfont{date}{size=\small}
\setbeamerfont{institute}{size=\small}

% ====================================
% REQUIRED PACKAGES
% ====================================
\usepackage{tikz}
\usepackage{amsmath}
\usepackage{amssymb}
\usepackage{booktabs}
\usepackage{graphicx}
\usepackage{array}
\usepackage{listings}
\usepackage{algorithm2e}
\usepackage{xcolor}
\usepackage{tabularx}
\usepackage{multirow}
\usepackage{subcaption}

% ====================================
% CUSTOM COMMANDS
% ====================================
% Text highlighting commands
\newcommand{\highlight}[1]{\textcolor{DeepBlue}{\textbf{#1}}}
\newcommand{\secondary}[1]{\textcolor{DarkGray}{#1}}
\newcommand{\success}[1]{\textcolor{DarkGreen}{#1}}
\newcommand{\warning}[1]{\textcolor{DarkRed}{#1}}
\newcommand{\data}[1]{\textcolor{ChartBlue}{#1}}
\newcommand{\dataalt}[1]{\textcolor{ChartOrange}{#1}}

% Mathematical notation
\newcommand{\given}{\mid}
\newcommand{\prob}[1]{P(#1)}
\newcommand{\argmax}{\operatorname*{argmax}}
\newcommand{\argmin}{\operatorname*{argmin}}
\newcommand{\softmax}{\operatorname{softmax}}

% Box commands for emphasis
\newcommand{\keypoint}[1]{%
  \begin{center}
  \fbox{\parbox{0.9\textwidth}{\centering\textbf{#1}}}
  \end{center}
}

\newcommand{\formula}[1]{%
  \begin{center}
  \colorbox{LightGray}{\parbox{0.8\textwidth}{\centering$\displaystyle #1$}}
  \end{center}
}

% ====================================
% LISTINGS CONFIGURATION
% ====================================
\lstset{
  basicstyle=\ttfamily\small,
  keywordstyle=\color{DeepBlue}\bfseries,
  commentstyle=\color{DarkGray}\itshape,
  stringstyle=\color{ChartOrange},
  numbers=left,
  numberstyle=\tiny\color{DarkGray},
  stepnumber=1,
  numbersep=5pt,
  backgroundcolor=\color{white},
  showspaces=false,
  showstringspaces=false,
  showtabs=false,
  frame=single,
  frameround=tttt,
  rulecolor=\color{LightGray},
  tabsize=2,
  captionpos=b,
  breaklines=true,
  breakatwhitespace=true,
  language=Python,
  escapeinside={(*@}{@*)},
  morekeywords={self, yield, assert, with, as}
}

% ====================================
% STANDARD SLIDE LAYOUTS
% ====================================

% Two-column layout with title
\newcommand{\twocolslide}[3]{%
  \begin{frame}{#1}
  \begin{columns}[T]
  \column{0.48\textwidth}
  #2
  \column{0.48\textwidth}
  #3
  \end{columns}
  \end{frame}
}

% Three-column layout
\newcommand{\threecolslide}[4]{%
  \begin{frame}{#1}
  \begin{columns}[T]
  \column{0.32\textwidth}
  #2
  \column{0.32\textwidth}
  #3
  \column{0.32\textwidth}
  #4
  \end{columns}
  \end{frame}
}

% Chart slide with caption
\newcommand{\chartslide}[3]{%
  \begin{frame}{#1}
  \begin{center}
  \includegraphics[width=#2\textwidth]{#3}
  \end{center}
  \end{frame}
}

% Full chart slide - optimized to 0.85 for proper margins
\newcommand{\fullchartslide}[2]{%
  \begin{frame}{#1}
  \begin{center}
  \includegraphics[width=0.85\textwidth]{#2}
  \end{center}
  \end{frame}
}

% Code slide
\newcommand{\codeslide}[3]{%
  \begin{frame}[fragile]{#1}
  \begin{lstlisting}[language=#2]
#3
  \end{lstlisting}
  \end{frame}
}

% Concept slide with figure
\newcommand{\conceptslide}[3]{%
  \begin{frame}{#1}
  \begin{columns}[T]
  \column{0.6\textwidth}
  #2
  \column{0.35\textwidth}
  \begin{center}
  \includegraphics[width=0.95\textwidth]{#3}
  \end{center}
  \end{columns}
  \end{frame}
}

% Table slide
\newcommand{\tableslide}[2]{%
  \begin{frame}{#1}
  \begin{center}
  \Large
  \renewcommand{\arraystretch}{1.5}
  #2
  \end{center}
  \end{frame}
}

% Summary slide
\newcommand{\summaryslide}[2]{%
  \begin{frame}{#1}
  \begin{center}
  \Large
  #2
  \end{center}
  \vfill
  \begin{center}
  \keypoint{Key Takeaway}
  \end{center}
  \end{frame}
}

% ====================================
% REMOVE DECORATIONS
% ====================================
\setbeamertemplate{blocks}[default]
\setbeamertemplate{title page}[default][colsep=-4bp,rounded=false]
\setbeamertemplate{itemize items}[circle]
\setbeamertemplate{enumerate items}[default]
\setbeamertemplate{section in toc}[sections numbered]
\setbeamertemplate{subsection in toc}[subsections numbered]

% ====================================
% PAGE NUMBERING
% ====================================
\setbeamertemplate{footline}{
  \leavevmode%
  \hbox{%
  \begin{beamercolorbox}[wd=.333333\paperwidth,ht=2.25ex,dp=1ex,center]{author in head/foot}%
    \usebeamerfont{author in head/foot}\secondary{\insertshortauthor}
  \end{beamercolorbox}%
  \begin{beamercolorbox}[wd=.333333\paperwidth,ht=2.25ex,dp=1ex,center]{title in head/foot}%
    \usebeamerfont{title in head/foot}\secondary{\insertshorttitle}
  \end{beamercolorbox}%
  \begin{beamercolorbox}[wd=.333333\paperwidth,ht=2.25ex,dp=1ex,right]{date in head/foot}%
    \usebeamerfont{date in head/foot}\secondary{\insertshortdate{}\hspace*{2em}
    \insertframenumber{} / \inserttotalframenumber\hspace*{2ex}}
  \end{beamercolorbox}}%
  \vskip0pt%
}

% ====================================
% TABLE OF CONTENTS STYLE
% ====================================
\setbeamertemplate{section in toc}{%
  \leavevmode\leftskip=1.5em%
  \llap{%
    \usebeamerfont{section in toc}%
    \usebeamercolor[fg]{section in toc}%
    \inserttocsectionnumber.%
  }%
  \usebeamerfont{section in toc}%
  \usebeamercolor[fg]{section in toc}%
  \inserttocsection\par%
}

\setbeamertemplate{subsection in toc}{%
  \leavevmode\leftskip=3em%
  \llap{%
    \usebeamerfont{subsection in toc}%
    \usebeamercolor[fg]{subsection in toc}%
    \inserttocsectionnumber.\inserttocsubsectionnumber%
  }%
  \usebeamerfont{subsection in toc}%
  \usebeamercolor[fg]{subsection in toc}%
  \inserttocsubsection\par%
}

% ====================================
% END OF MASTER TEMPLATE
% ====================================

% Define bottomnote command for slide annotations
\newcommand{\bottomnote}[1]{%
    \vspace{0.2cm}
    \begin{center}
    \footnotesize\secondary{#1}
    \end{center}
}

\title{Foundations of NLP}
\subtitle{\secondary{Week 1 - From Dice to Text Prediction}}
\author{NLP Course 2025}
\date{October 26, 2025}

\begin{document}

% ===== OPENING SEQUENCE (Slides 1-5) =====

% Slide 1: Title
\begin{frame}
\titlepage
\vfill
\begin{center}
\secondary{\footnotesize BSc Discovery-Based Presentation}
\end{center}
\end{frame}

% Slide 2: Hook - Same Model, Different Quality
\begin{frame}[t]{Same Model, Different Text Quality}
\vspace{-0.3cm}
\begin{center}
\includegraphics[width=0.75\textwidth]{../figures/text_quality_progression_bsc.pdf}
\end{center}

\begin{center}
\textbf{Key Insight}: Better models predict next words more accurately
\end{center}

\vspace{0.2cm}
\bottomnote{Your phone keyboard uses these models - but how do they work?}
\end{frame}

% Slide 3: Discovery Question
\begin{frame}[t]{A Thought Experiment}
\begin{columns}[T]
\column{0.48\textwidth}
\textbf{Rolling a Die}

\vspace{3mm}
\begin{itemize}
\item Six possible outcomes: 1, 2, 3, 4, 5, 6
\item Each has probability: $\frac{1}{6} = 0.167$
\item No memory: Previous rolls don't matter
\item Fair die: All outcomes equally likely
\end{itemize}

\vspace{5mm}
\textbf{Question}: Can we predict the next roll?

\column{0.48\textwidth}
\textbf{Predicting Words}

\vspace{3mm}
\begin{itemize}
\item Thousands of possible words
\item Each has \textit{different} probability
\item \highlight{Memory matters}: Previous words help
\item Not equally likely: ``the'' more common than ``xylophone''
\end{itemize}

\vspace{5mm}
\textbf{Question}: Can we predict the next word?
\end{columns}

\vspace{0.2cm}
\bottomnote{Both are probability problems - but text prediction is harder}
\end{frame}

% Slide 4: Visual - Dice Probability
\begin{frame}[t]{Dice Probability: The Foundation}
\vspace{-0.3cm}
\begin{center}
\includegraphics[width=0.65\textwidth]{../figures/dice_probability_bsc.pdf}
\end{center}

\begin{center}
\textbf{Key Insight}: When all outcomes equally likely, $P(outcome) = \frac{1}{\text{number of outcomes}}$
\end{center}

\vspace{0.2cm}
\bottomnote{This is the simplest case - text is more complex}
\end{frame}

% Slide 5: Bridge to Text
\begin{frame}[t]{From Dice to Text Prediction}
\small
\begin{columns}[T]
\column{0.48\textwidth}
\raggedright
\textbf{What We Know from Dice}:
\begin{itemize}
\item Probability quantifies uncertainty
\item Sum of all probabilities = 1
\item More data $\rightarrow$ better estimates
\end{itemize}

\vspace{5mm}
\textbf{New Challenges for Text}:
\begin{itemize}
\item Outcomes NOT equally likely
\item Context matters (``the cat'' vs ``the xylophone'')
\item How to estimate probabilities?
\end{itemize}

\column{0.48\textwidth}
\raggedright
\textbf{Our Goal}:

Predict next word given previous words

\vspace{3mm}
\textbf{Example}:
\begin{itemize}
\item Given: ``The cat sat on the''
\item Predict: ``mat'' (likely) or ``xylophone'' (unlikely)
\end{itemize}

\vspace{5mm}
\textbf{How}?

Use \highlight{conditional probability} based on what we've seen before

\end{columns}

\vspace{0.2cm}
\bottomnote{Next: Build conditional probability from first principles}
\end{frame}

% ===== FOUNDATION BUILDING (Slides 6-15) =====

% Slide 6: Visual - Conditional Probability with Dice
\begin{frame}[t]{Conditional Probability: A Simple Example}
\vspace{-0.3cm}
\begin{center}
\includegraphics[width=0.7\textwidth]{../figures/conditional_probability_tree_bsc.pdf}
\end{center}

\begin{center}
\textbf{Key Insight}: Knowing previous outcomes changes our predictions
\end{center}

\vspace{0.2cm}
\bottomnote{P(next $|$ previous) reads as ``probability of next GIVEN previous''}
\end{frame}

% Slide 7: Detail - Conditional Probability Mathematics
\begin{frame}[t]{Conditional Probability: The Mathematics}
\small
\begin{columns}[T]
\column{0.48\textwidth}
\raggedright
\textbf{Definition}:

$$P(A|B) = \frac{P(A \cap B)}{P(B)}$$

Reads: ``Probability of A given B''

\vspace{3mm}
\textbf{Worked Example with Cards}:

Given: Drew a face card (J, Q, K)

Question: What's probability it's a King?

\vspace{3mm}
\begin{itemize}
\item Total face cards: 12 (4 Jacks, 4 Queens, 4 Kings)
\item Kings among face cards: 4
\item $P(\text{King}|\text{Face}) = \frac{4}{12} = \frac{1}{3}$
\end{itemize}

\column{0.48\textwidth}
\raggedright
\textbf{For Text Prediction}:

$$P(w_n | w_1, w_2, ..., w_{n-1})$$

Reads: ``Probability of word $n$ given all previous words''

\vspace{3mm}
\textbf{Example}:

Given: ``The cat sat on the''

Question: What's $P(\text{mat}| \text{the cat sat on the})$?

\vspace{3mm}
\textbf{Answer}: Count how many times we've seen this pattern!

\end{columns}

\vspace{0.2cm}
\bottomnote{Conditional probability is the foundation of language modeling}
\end{frame}

% Slide 8: Visual - Text as Sequence
\begin{frame}[t]{Text as a Sequence of Decisions}
\vspace{-0.3cm}
\begin{center}
\includegraphics[width=0.75\textwidth]{../figures/text_sequence_bsc.pdf}
\end{center}

\begin{center}
\textbf{Key Insight}: Each word is a prediction given all previous words
\end{center}

\vspace{0.2cm}
\bottomnote{But how do we get these probabilities?}
\end{frame}

% Slide 9: Detail - Word Prediction as Probability
\begin{frame}[t]{Word Prediction: The Core Problem}
\small
\begin{columns}[T]
\column{0.48\textwidth}
\raggedright
\textbf{The Chain Rule}:

For a sequence $w_1, w_2, ..., w_n$:

\begin{align*}
P(w_1, w_2, ..., w_n) &= P(w_1) \\
&\times P(w_2|w_1) \\
&\times P(w_3|w_1, w_2) \\
&\times ... \\
&\times P(w_n|w_1, ..., w_{n-1})
\end{align*}

\textbf{Challenge}: How to estimate $P(w_n|w_1, ..., w_{n-1})$?

\column{0.48\textwidth}
\raggedright
\textbf{Example: ``The cat sat''}:

\begin{align*}
P(\text{the, cat, sat}) &= P(\text{the}) \\
&\times P(\text{cat}|\text{the}) \\
&\times P(\text{sat}|\text{the, cat})
\end{align*}

\vspace{3mm}
\textbf{Problem}:

Infinitely many possible histories!

\vspace{3mm}
\textbf{Solution}:

Make a simplifying assumption (coming next)

\end{columns}

\vspace{0.2cm}
\bottomnote{We need a practical way to estimate these probabilities}
\end{frame}

% Slide 10: Visual - Counting Co-occurrences
\begin{frame}[t]{The Solution: Count What We've Seen}
\vspace{-0.3cm}
\begin{center}
\includegraphics[width=0.65\textwidth]{../figures/corpus_statistics_bsc.pdf}
\end{center}

\begin{center}
\textbf{Key Insight}: Probability $\approx$ Frequency in large corpus
\end{center}

\vspace{0.2cm}
\bottomnote{This is Maximum Likelihood Estimation}
\end{frame}

% Slide 11: Detail - Maximum Likelihood Estimation
\begin{frame}[t]{Maximum Likelihood Estimation (MLE)}
\small
\begin{columns}[T]
\column{0.48\textwidth}
\raggedright
\textbf{The Idea}:

Estimate probabilities from observed counts

\vspace{3mm}
\textbf{Formula}:

$$P(w|context) = \frac{\text{count}(context, w)}{\text{count}(context)}$$

\vspace{3mm}
\textbf{Example from Shakespeare}:

\begin{itemize}
\item count(``to be'') = 150 times
\item count(``to be or'') = 42 times
\item $P(\text{or}|\text{to be}) = \frac{42}{150} = 0.28$
\end{itemize}

\column{0.48\textwidth}
\raggedright
\textbf{Why This Works}:

\begin{itemize}
\item Law of Large Numbers
\item As corpus size $\rightarrow \infty$, relative frequency $\rightarrow$ true probability
\item Real corpora: millions/billions of words
\end{itemize}

\vspace{5mm}
\textbf{When to Use}:

\begin{itemize}
\item Have large text corpus
\item Want simple, interpretable model
\item Need fast training and inference
\end{itemize}

\end{columns}

\vspace{0.2cm}
\bottomnote{MLE is simple but effective - used in production systems}
\end{frame}

% Slide 12: Visual - N-gram Context Windows
\begin{frame}[t]{N-gram Models: The Markov Assumption}
\vspace{-0.3cm}
\begin{center}
\includegraphics[width=0.75\textwidth]{../figures/ngram_context_windows_bsc.pdf}
\end{center}

\begin{center}
\textbf{Key Insight}: Only last $n-1$ words matter for prediction
\end{center}

\vspace{0.2cm}
\bottomnote{This is the Markov assumption - limited memory}
\end{frame}

% Slide 13: Detail - N-gram Formula and Markov Assumption
\begin{frame}[t]{N-gram Models: Making It Practical}
\small
\begin{columns}[T]
\column{0.48\textwidth}
\raggedright
\textbf{The Markov Assumption}:

$$P(w_i|w_1, ..., w_{i-1}) \approx P(w_i|w_{i-n+1}, ..., w_{i-1})$$

Only last $n-1$ words matter

\vspace{3mm}
\textbf{Different N-gram Models}:

\begin{itemize}
\item \textbf{Unigram} ($n=1$): $P(w_i)$
\item \textbf{Bigram} ($n=2$): $P(w_i|w_{i-1})$
\item \textbf{Trigram} ($n=3$): $P(w_i|w_{i-2}, w_{i-1})$
\end{itemize}

\column{0.48\textwidth}
\raggedright
\textbf{Why This Helps}:

\begin{itemize}
\item Reduces number of parameters
\item More data per pattern
\item Tractable computation
\end{itemize}

\vspace{3mm}
\textbf{Trade-off}:

\begin{itemize}
\item Small $n$: Fast, robust, but misses long-range dependencies
\item Large $n$: Captures context, but sparse data
\end{itemize}

\vspace{3mm}
\textbf{Typical Choice}: $n=3$ (trigram) balances both

\end{columns}

\vspace{0.2cm}
\bottomnote{N-grams are the workhorse of statistical language modeling}
\end{frame}

% Slide 14: Worked Example - Computing Bigram Probabilities
\begin{frame}[t]{Worked Example: Bigram Probabilities}
\small
\textbf{Given Shakespeare corpus extract}:

``To be or not to be that is the question whether''

\vspace{3mm}
\textbf{Task}: Compute $P(\text{be}|\text{to})$

\vspace{3mm}
\textbf{Step 1}: Count all bigrams starting with ``to''

\begin{center}
\begin{tabular}{ll}
Bigram & Count \\
\hline
(to, be) & 2 \\
\end{tabular}
\end{center}

\textbf{Step 2}: Count total occurrences of ``to''

count(to) = 2

\vspace{3mm}
\textbf{Step 3}: Apply MLE formula

$$P(\text{be}|\text{to}) = \frac{\text{count}(\text{to, be})}{\text{count}(\text{to})} = \frac{2}{2} = 1.0$$

\vspace{3mm}
\textbf{Interpretation}: In this tiny corpus, ``to'' is always followed by ``be''!

\vspace{0.2cm}
\bottomnote{Real corpora give more nuanced probabilities}
\end{frame}

% Slide 15: Checkpoint Quiz
\begin{frame}[t]{Checkpoint: Test Your Understanding}
\begin{center}
\textbf{Quick Quiz}
\end{center}
\vspace{3mm}

\begin{columns}[T]
\column{0.48\textwidth}
\raggedright
\textbf{Question 1}:

Given corpus: ``the cat sat on the mat the dog''

What is $P(\text{cat}|\text{the})$?

\vspace{3mm}
\textbf{A)} 1/3 \\
\textbf{B)} 1/2 \\
\textbf{C)} 2/3 \\
\textbf{D)} 1/8

\vspace{5mm}
\textbf{Question 2}:

Why do we use the Markov assumption?

\vspace{3mm}
\textbf{A)} It's always correct \\
\textbf{B)} Reduces parameters \\
\textbf{C)} Improves accuracy \\
\textbf{D)} Runs faster

\column{0.48\textwidth}
\raggedright
\textbf{Answer 1}: \textbf{A) 1/3}

\begin{itemize}
\item count(the) = 3
\item count(the, cat) = 1
\item $P(\text{cat}|\text{the}) = 1/3$
\end{itemize}

\vspace{5mm}
\textbf{Answer 2}: \textbf{B) Reduces parameters}

\begin{itemize}
\item Without it: Infinite histories
\item With it: Tractable parameter count
\item Trade-off: Lose long-range info
\end{itemize}

\end{columns}

\vspace{0.2cm}
\bottomnote{Understanding these foundations is critical for what comes next}
\end{frame}

% ===== TAXONOMY: N-GRAM TYPES (Slides 16-23) =====

% Slide 16: Visual - Unigram Model
\begin{frame}[t]{Unigram Model: The Simplest Baseline}
\vspace{-0.3cm}
\begin{center}
\includegraphics[width=0.65\textwidth]{../figures/unigram_frequencies_bsc.pdf}
\end{center}

\begin{center}
\textbf{Key Insight}: Each word has fixed probability regardless of context
\end{center}

\vspace{0.2cm}
\bottomnote{Simple but ignores all context - like random word sampling}
\end{frame}

% Slide 17: Detail - Unigram Model
\begin{frame}[t]{Unigram Model: When Context Doesn't Matter}
\small
\begin{columns}[T]
\column{0.48\textwidth}
\raggedright
\textbf{Formula}:

$$P(w_1, w_2, ..., w_n) = \prod_{i=1}^{n} P(w_i)$$

Each word independent

\vspace{3mm}
\textbf{MLE Estimation}:

$$P(w) = \frac{\text{count}(w)}{\text{total words}}$$

\vspace{3mm}
\textbf{Example}:

\begin{itemize}
\item Total words: 1,000,000
\item count(``the'') = 70,000
\item $P(\text{the}) = 0.07$
\end{itemize}

\column{0.48\textwidth}
\raggedright
\textbf{When to Use}:

\begin{itemize}
\item Baseline comparison
\item Document classification
\item Bag-of-words features
\item When word order truly doesn't matter
\end{itemize}

\vspace{3mm}
\textbf{Limitations}:

\begin{itemize}
\item Generates nonsense: ``the the the cat dog''
\item No grammar
\item No meaning
\item High perplexity
\end{itemize}

\end{columns}

\vspace{0.2cm}
\bottomnote{Unigrams are weak for generation but useful for other NLP tasks}
\end{frame}

% Slide 18: Visual - Bigram Model
\begin{frame}[t]{Bigram Model: One Word of Memory}
\vspace{-0.3cm}
\begin{center}
\includegraphics[width=0.75\textwidth]{../figures/bigram_context_bsc.pdf}
\end{center}

\begin{center}
\textbf{Key Insight}: Previous word dramatically narrows possibilities
\end{center}

\vspace{0.2cm}
\bottomnote{``the'' is often followed by nouns, rarely by verbs}
\end{frame}

% Slide 19: Detail - Bigram Model
\begin{frame}[t]{Bigram Model: The Sweet Spot}
\small
\begin{columns}[T]
\column{0.48\textwidth}
\raggedright
\textbf{Formula}:

$$P(w_i|w_{i-1})$$

Condition on previous word only

\vspace{3mm}
\textbf{MLE Estimation}:

$$P(w_2|w_1) = \frac{\text{count}(w_1, w_2)}{\text{count}(w_1)}$$

\vspace{3mm}
\textbf{Worked Example}:

\begin{itemize}
\item count(``the'') = 70,000
\item count(``the cat'') = 850
\item $P(\text{cat}|\text{the}) = \frac{850}{70000} = 0.012$
\end{itemize}

\column{0.48\textwidth}
\raggedright
\textbf{When to Use}:

\begin{itemize}
\item Real-time applications (autocomplete)
\item Limited memory/computation
\item Reasonable text quality needed
\item Standard baseline for comparison
\end{itemize}

\vspace{3mm}
\textbf{Performance}:

\begin{itemize}
\item Captures local grammar
\item Generates coherent phrases
\item Moderate perplexity (100-150)
\item Still misses long-range dependencies
\end{itemize}

\end{columns}

\vspace{0.2cm}
\bottomnote{Bigrams are widely used - good balance of simplicity and performance}
\end{frame}

% Slide 20: Visual - Trigram Model
\begin{frame}[t]{Trigram Model: Two Words of Memory}
\vspace{-0.3cm}
\begin{center}
\includegraphics[width=0.75\textwidth]{../figures/trigram_context_bsc.pdf}
\end{center}

\begin{center}
\textbf{Key Insight}: Two-word context captures richer patterns
\end{center}

\vspace{0.2cm}
\bottomnote{``on the'' strongly suggests location noun: mat, floor, table}
\end{frame}

% Slide 21: Detail - Trigram Model and Performance
\begin{frame}[t]{Trigram Model: More Context, Better Predictions}
\small
\begin{columns}[T]
\column{0.48\textwidth}
\raggedright
\textbf{Formula}:

$$P(w_i|w_{i-2}, w_{i-1})$$

Condition on two previous words

\vspace{3mm}
\textbf{MLE Estimation}:

$$P(w_3|w_1, w_2) = \frac{\text{count}(w_1, w_2, w_3)}{\text{count}(w_1, w_2)}$$

\vspace{3mm}
\textbf{Example}:

\begin{itemize}
\item count(``on the'') = 5,200
\item count(``on the mat'') = 127
\item $P(\text{mat}|\text{on the}) = \frac{127}{5200} = 0.024$
\end{itemize}

\column{0.48\textwidth}
\raggedright
\textbf{Comparison to Bigram}:

\begin{center}
\small
\begin{tabular}{lcc}
Model & Perplexity & Quality \\
\hline
Bigram & 125 & Good \\
Trigram & 78 & Better \\
\end{tabular}
\end{center}

\vspace{3mm}
\textbf{When to Use}:

\begin{itemize}
\item Sufficient training data
\item Quality matters more than speed
\item Speech recognition, translation
\end{itemize}

\vspace{3mm}
\textbf{Limitation}: Data sparsity increases

\end{columns}

\vspace{0.2cm}
\bottomnote{Trigrams are standard in production systems when data permits}
\end{frame}

% Slide 22: Visual - Higher-Order N-grams
\begin{frame}[t]{Higher-Order N-grams: The Sparsity Wall}
\vspace{-0.3cm}
\begin{center}
\includegraphics[width=0.65\textwidth]{../figures/parameter_growth_bsc.pdf}
\end{center}

\begin{center}
\textbf{Key Insight}: Parameters explode exponentially with $n$
\end{center}

\vspace{0.2cm}
\bottomnote{This is why we rarely go beyond trigrams}
\end{frame}

% Slide 23: Detail - Context vs Sparsity Tradeoff
\begin{frame}[t]{The Fundamental Tradeoff}
\small
\begin{columns}[T]
\column{0.48\textwidth}
\raggedright
\textbf{Parameter Count}:

For vocabulary size $V$:

\begin{itemize}
\item Unigram: $V$ (e.g., 50,000)
\item Bigram: $V^2$ (2.5 billion)
\item Trigram: $V^3$ (125 trillion!)
\item 4-gram: $V^4$ (6,250,000 trillion!)
\end{itemize}

\vspace{3mm}
\textbf{Reality}: Most combinations never seen

\column{0.48\textwidth}
\raggedright
\textbf{The Dilemma}:

\begin{itemize}
\item More context $\rightarrow$ Better predictions
\item More context $\rightarrow$ More parameters
\item More parameters $\rightarrow$ Sparse data
\item Sparse data $\rightarrow$ Poor estimates
\end{itemize}

\vspace{3mm}
\textbf{Practical Choice}:

\begin{itemize}
\item $n=2$ (bigram): Fast, robust
\item $n=3$ (trigram): Standard
\item $n \geq 4$: Rarely used
\end{itemize}

\end{columns}

\vspace{0.2cm}
\bottomnote{This limitation motivates smoothing (coming next) and neural models (Week 2)}
\end{frame}

% ===== PROBLEM-SOLUTION: SPARSITY (Slides 24-32) =====

% Slide 24: Problem Quantification - Unseen N-grams
\begin{frame}[t]{The Sparsity Problem: Quantified}
\small
\textbf{Experiment}: Train on 1 million words, test on held-out data

\vspace{3mm}
\begin{center}
\begin{tabular}{lccc}
Model & Seen in Training & Unseen in Test & OOV Rate \\
\hline
Unigram & 45,000 words & 2,300 words & 5\% \\
Bigram & 523,000 pairs & 87,000 pairs & 14\% \\
Trigram & 891,000 triples & 203,000 triples & 19\% \\
4-gram & 958,000 quads & 347,000 quads & 27\% \\
\end{tabular}
\end{center}

\vspace{5mm}
\textbf{Pattern}: As $n$ increases, more unseen combinations

\vspace{3mm}
\textbf{Problem}: MLE assigns $P = 0$ to unseen n-grams

\vspace{3mm}
\textbf{Consequence}: If any n-gram has $P=0$, entire sentence gets $P=0$!

\vspace{0.2cm}
\bottomnote{We need a way to handle unseen n-grams without destroying sentence probabilities}
\end{frame}

% Slide 25: Failure Case
\begin{frame}[t]{Why Zero Probability is Catastrophic}
\small
\textbf{Example Sentence}: ``The cat sat on the xylophone''

\vspace{3mm}
\textbf{Using Bigram Model}:

\begin{align*}
P(\text{sentence}) &= P(\text{the}) \times P(\text{cat}|\text{the}) \times P(\text{sat}|\text{cat}) \\
&\times P(\text{on}|\text{sat}) \times P(\text{the}|\text{on}) \times P(\text{xylophone}|\text{the})
\end{align*}

\vspace{3mm}
\textbf{Problem}:

If ``the xylophone'' never appeared in training:

$$P(\text{xylophone}|\text{the}) = \frac{0}{70000} = 0$$

\vspace{3mm}
Therefore: $P(\text{entire sentence}) = 0$

\vspace{5mm}
\textbf{Consequence}:

\begin{itemize}
\item Can't rank this sentence
\item Can't generate it
\item Perplexity becomes infinite!
\end{itemize}

\vspace{0.2cm}
\bottomnote{A single unseen n-gram destroys everything - we must fix this}
\end{frame}

% Slide 26: Root Cause Diagnosis
\begin{frame}[t]{Root Cause: The Sparse Data Curse}
\small
\begin{columns}[T]
\column{0.48\textwidth}
\raggedright
\textbf{What We Have}:

\begin{itemize}
\item Training corpus: 1M words
\item Vocabulary: 50K words
\item Possible bigrams: $50K^2 = 2.5$ billion
\item Observed bigrams: ~500K
\end{itemize}

\vspace{3mm}
\textbf{Coverage}: $\frac{500K}{2.5B} = 0.02\%$

\vspace{3mm}
We've seen only 0.02\% of possible bigrams!

\column{0.48\textwidth}
\raggedright
\textbf{The Mathematics}:

\begin{itemize}
\item No smoothing: $P = 0$ for 99.98\% of bigrams
\item Sentence of length 10: $P(\text{sentence}) = 0$ in 95\% of cases
\item Useless for real applications
\end{itemize}

\vspace{3mm}
\textbf{Root Cause}:

MLE assumes if we haven't seen it, it's impossible

\vspace{3mm}
\textbf{Reality}:

Unseen $\neq$ Impossible

\end{columns}

\vspace{0.2cm}
\bottomnote{We need to reserve some probability mass for unseen events}
\end{frame}

% Slide 27: Visual - Smoothing Techniques Comparison
\begin{frame}[t]{The Solution: Smoothing}
\vspace{-0.3cm}
\begin{center}
\includegraphics[width=0.75\textwidth]{../figures/smoothing_comparison_bsc.pdf}
\end{center}

\begin{center}
\textbf{Key Insight}: Steal probability from seen events, give to unseen
\end{center}

\vspace{0.2cm}
\bottomnote{Many smoothing methods - we'll cover the most important}
\end{frame}

% Slide 28: Detail - Laplace Smoothing (Add-1)
\begin{frame}[t]{Laplace Smoothing: The Simplest Fix}
\small
\begin{columns}[T]
\column{0.48\textwidth}
\raggedright
\textbf{The Idea}:

Pretend we've seen every n-gram one extra time

\vspace{3mm}
\textbf{Formula (Bigram)}:

$$P_{smooth}(w_2|w_1) = \frac{\text{count}(w_1, w_2) + 1}{\text{count}(w_1) + V}$$

where $V$ = vocabulary size

\vspace{3mm}
\textbf{Worked Example}:

\begin{itemize}
\item count(``the'') = 70,000
\item count(``the cat'') = 850
\item $V$ = 50,000
\end{itemize}

$$P(\text{cat}|\text{the}) = \frac{850 + 1}{70000 + 50000} = \frac{851}{120000} = 0.0071$$

\column{0.48\textwidth}
\raggedright
\textbf{For Unseen N-gram}:

count(``the xylophone'') = 0

$$P(\text{xylophone}|\text{the}) = \frac{0 + 1}{70000 + 50000} = \frac{1}{120000} = 8.3 \times 10^{-6}$$

\vspace{3mm}
Small but non-zero!

\vspace{5mm}
\textbf{Properties}:

\begin{itemize}
\item \textcolor{green}{Simple to implement}
\item \textcolor{green}{Never gives $P=0$}
\item \textcolor{red}{Too much mass to rare events}
\item \textcolor{red}{Hurts frequent events}
\end{itemize}

\end{columns}

\vspace{0.2cm}
\bottomnote{Laplace smoothing is too aggressive for language - we need something better}
\end{frame}

% Slide 29: Visual - Add-k Smoothing
\begin{frame}[t]{Add-k Smoothing: A Better Balance}
\vspace{-0.3cm}
\begin{center}
\includegraphics[width=0.65\textwidth]{../figures/addk_smoothing_bsc.pdf}
\end{center}

\begin{center}
\textbf{Key Insight}: Add $k < 1$ instead of 1 to balance probability redistribution
\end{center}

\vspace{0.2cm}
\bottomnote{Typical values: $k = 0.01$ to $k = 0.5$}
\end{frame}

% Slide 30: Detail - Advanced Smoothing (Concept)
\begin{frame}[t]{Advanced Smoothing: Kneser-Ney}
\small
\begin{columns}[T]
\column{0.48\textwidth}
\raggedright
\textbf{The Insight}:

Not all words are equally likely in new contexts

\vspace{3mm}
\textbf{Example}:

\begin{itemize}
\item ``Francisco'' appears often
\item But only after ``San''
\item Shouldn't get high probability after other words!
\end{itemize}

\vspace{3mm}
\textbf{Solution}:

Count \textit{how many different contexts} a word appears in, not just total frequency

\column{0.48\textwidth}
\raggedright
\textbf{Kneser-Ney Benefits}:

\begin{itemize}
\item State-of-the-art for n-grams
\item 15-20\% perplexity improvement
\item Used in production systems
\end{itemize}

\vspace{3mm}
\textbf{Complexity}:

\begin{itemize}
\item More sophisticated than add-k
\item Requires continuation counts
\item Backoff to shorter n-grams
\end{itemize}

\vspace{3mm}
\textbf{When to Use}:

When quality matters most

\end{columns}

\vspace{0.2cm}
\bottomnote{Full Kneser-Ney derivation beyond BSc scope - but know it exists}
\end{frame}

% Slide 31: Validation - Perplexity Comparison
\begin{frame}[t]{Validation: Smoothing Improves Performance}
\small
\textbf{Experimental Setup}:

Train on 1M words, test on 100K held-out words, trigram model

\vspace{5mm}
\begin{center}
\begin{tabular}{lcc}
Smoothing Method & Perplexity & Unseen N-gram Handling \\
\hline
None (MLE) & $\infty$ & \textcolor{red}{Fails} \\
Add-1 (Laplace) & 245 & \textcolor{orange}{Poor} \\
Add-0.1 & 156 & \textcolor{green}{Good} \\
Add-0.01 & 132 & \textcolor{green}{Good} \\
Kneser-Ney & 98 & \textcolor{green}{Best} \\
\end{tabular}
\end{center}

\vspace{5mm}
\textbf{Pattern}:

\begin{itemize}
\item Smaller $k$ better for large corpora
\item Kneser-Ney best overall
\item 35\% improvement over add-1
\end{itemize}

\vspace{3mm}
\textbf{Takeaway}: Smoothing is not optional - it's essential

\vspace{0.2cm}
\bottomnote{Modern systems use sophisticated smoothing as standard}
\end{frame}

% Slide 32: Implementation - Python Code
\begin{frame}[fragile, t]{Implementation: Add-k Smoothing}
\small
\begin{columns}[T]
\column{0.55\textwidth}
\begin{lstlisting}[language=Python]
class SmoothBigram:
    def __init__(self, k=0.01):
        self.k = k
        self.counts = defaultdict(
            lambda: defaultdict(int)
        )
        self.vocab = set()

    def train(self, text):
        words = text.split()
        for i in range(len(words)-1):
            w1, w2 = words[i], words[i+1]
            self.counts[w1][w2] += 1
            self.vocab.update([w1, w2])

    def probability(self, w1, w2):
        V = len(self.vocab)
        numerator = self.counts[w1][w2] + self.k
        denominator = sum(
            self.counts[w1].values()
        ) + self.k * V
        return numerator / denominator
\end{lstlisting}

\column{0.42\textwidth}
\raggedright
\textbf{Key Changes from MLE}:

\begin{itemize}
\item Add $k$ to numerator
\item Add $k \times V$ to denominator
\item Never returns 0!
\end{itemize}

\vspace{5mm}
\textbf{Usage}:

\begin{lstlisting}[language=Python]
model = SmoothBigram(k=0.01)
model.train(corpus)
p = model.probability(
    "the", "xylophone"
)
# Returns small non-zero value
\end{lstlisting}

\end{columns}

\vspace{0.2cm}
\bottomnote{Just 4 lines different from MLE - huge impact on performance}
\end{frame}

% ===== EVALUATION (Slides 33-37) =====

% Slide 33: Visual - Perplexity Comparison
\begin{frame}[t]{Evaluation: How Good is Our Model?}
\vspace{-0.3cm}
\begin{center}
\includegraphics[width=0.75\textwidth]{../figures/perplexity_comparison_bsc.pdf}
\end{center}

\begin{center}
\textbf{Key Insight}: Lower perplexity = better predictions
\end{center}

\vspace{0.2cm}
\bottomnote{Perplexity is the standard metric for language models}
\end{frame}

% Slide 34: Detail - Perplexity Formula and Interpretation
\begin{frame}[t]{Perplexity: Measuring Uncertainty}
\small
\begin{columns}[T]
\column{0.48\textwidth}
\raggedright
\textbf{Formula}:

$$PP(W) = P(w_1, w_2, ..., w_N)^{-\frac{1}{N}}$$

or equivalently:

$$PP(W) = \sqrt[N]{\frac{1}{P(w_1, ..., w_N)}}$$

\vspace{3mm}
\textbf{Logarithmic Form}:

$$\log_2 PP(W) = -\frac{1}{N} \sum_{i=1}^{N} \log_2 P(w_i|context)$$

\column{0.48\textwidth}
\raggedright
\textbf{Interpretation}:

``On average, how many equally likely words could come next?''

\vspace{3mm}
\begin{itemize}
\item PP = 100: Choosing from 100 words
\item PP = 50: Model is twice as confident
\item PP = 10: Very good model
\item PP = 1000: Poor model
\end{itemize}

\vspace{3mm}
\textbf{Properties}:

\begin{itemize}
\item Lower is better
\item Minimized by true distribution
\item Inverse of geometric mean probability
\end{itemize}

\end{columns}

\vspace{0.2cm}
\bottomnote{Perplexity connects probability to human intuition about uncertainty}
\end{frame}

% Slide 35: Visual - Information Theory Basics
\begin{frame}[t]{Information Theory: The Foundations}
\vspace{-0.3cm}
\begin{center}
\includegraphics[width=0.65\textwidth]{../figures/information_theory_bsc.pdf}
\end{center}

\begin{center}
\textbf{Key Insight}: Rare events carry more information than common events
\end{center}

\vspace{0.2cm}
\bottomnote{Claude Shannon founded information theory in 1948}
\end{frame}

% Slide 36: Detail - Shannon's Information Theory
\begin{frame}[t]{Shannon's Information Theory}
\small
\begin{columns}[T]
\column{0.48\textwidth}
\raggedright
\textbf{Entropy} (Uncertainty):

$$H(X) = -\sum_{x} P(x) \log_2 P(x)$$

Measures average information per symbol

\vspace{3mm}
\textbf{Cross-Entropy}:

$$H(P, Q) = -\sum_{x} P(x) \log_2 Q(x)$$

Measures information loss when using $Q$ to approximate $P$

\vspace{3mm}
\textbf{Relationship to Perplexity}:

$$PP = 2^{H(P, Q)}$$

\column{0.48\textwidth}
\raggedright
\textbf{Example - Coin Flip}:

Fair coin: $P(\text{heads}) = 0.5$

$$H = -0.5 \log_2(0.5) - 0.5 \log_2(0.5) = 1 \text{ bit}$$

\vspace{3mm}
\textbf{Example - English Text}:

Entropy $\approx$ 1-2 bits per character

\vspace{3mm}
\textbf{Why This Matters}:

\begin{itemize}
\item Fundamental limits on compression
\item Optimal encoding
\item Neural models minimize cross-entropy
\end{itemize}

\end{columns}

\vspace{0.2cm}
\bottomnote{Information theory is the mathematical foundation of all NLP}
\end{frame}

% Slide 37: Worked Example - Computing Perplexity
\begin{frame}[t]{Worked Example: Computing Perplexity}
\small
\textbf{Given}: Bigram model, test sentence: ``the cat sat''

\vspace{3mm}
\textbf{Model Probabilities}:

\begin{itemize}
\item $P(\text{the}) = 0.07$
\item $P(\text{cat}|\text{the}) = 0.012$
\item $P(\text{sat}|\text{cat}) = 0.08$
\end{itemize}

\vspace{3mm}
\textbf{Step 1}: Compute sentence probability

\begin{align*}
P(\text{the, cat, sat}) &= P(\text{the}) \times P(\text{cat}|\text{the}) \times P(\text{sat}|\text{cat}) \\
&= 0.07 \times 0.012 \times 0.08 \\
&= 0.0000672
\end{align*}

\vspace{3mm}
\textbf{Step 2}: Apply perplexity formula

$$PP = (0.0000672)^{-\frac{1}{3}} = (0.0000672)^{-0.333} = 126.7$$

\vspace{3mm}
\textbf{Interpretation}: Model is as uncertain as picking from ~127 equally likely words

\vspace{0.2cm}
\bottomnote{Lower perplexity means higher confidence in predictions}
\end{frame}

% ===== META-KNOWLEDGE (Slides 38-39) =====

% Slide 38: When NOT to Use N-grams
\begin{frame}[t]{When NOT to Use N-grams}
\small
\begin{columns}[T]
\column{0.48\textwidth}
\raggedright
\textbf{Failure Cases}:

\begin{itemize}
\item \textbf{Long-range dependencies}:

  ``The author, who wrote several books about..., was awarded a prize.''

  (``author'' $\rightarrow$ ``was'', but 10+ words apart)

\item \textbf{Semantic understanding}:

  ``The bank is closed'' vs ``The river bank''

  (Same n-grams, different meanings)

\item \textbf{Rare but valid constructions}:

  Creative language, poetry, technical jargon
\end{itemize}

\column{0.48\textwidth}
\raggedright
\textbf{Better Alternatives}:

\begin{itemize}
\item \textbf{Neural language models} (Week 2):

  Learn distributed representations

\item \textbf{RNN/LSTM} (Week 3):

  Unbounded context window

\item \textbf{Transformers} (Week 5):

  Direct long-range connections
\end{itemize}

\vspace{5mm}
\textbf{When to Stick with N-grams}:

\begin{itemize}
\item Speed critical
\item Interpretability needed
\item Limited data
\item Baseline comparison
\end{itemize}

\end{columns}

\vspace{0.2cm}
\bottomnote{Know your model's limitations - choose the right tool for the job}
\end{frame}

% Slide 39: Common Pitfalls
\begin{frame}[t]{Common Pitfalls and How to Avoid Them}
\small
\begin{columns}[T]
\column{0.48\textwidth}
\raggedright
\textbf{Pitfall 1: Choosing n}

\begin{itemize}
\item Too small: Misses context
\item Too large: Data sparsity
\item \textcolor{green}{Solution}: Start with trigrams, validate on held-out data
\end{itemize}

\vspace{3mm}
\textbf{Pitfall 2: Forgetting smoothing}

\begin{itemize}
\item MLE gives zero probabilities
\item \textcolor{green}{Solution}: Always use smoothing in production
\end{itemize}

\vspace{3mm}
\textbf{Pitfall 3: Train/test contamination}

\begin{itemize}
\item Testing on training data
\item \textcolor{green}{Solution}: Strict data splits
\end{itemize}

\column{0.48\textwidth}
\raggedright
\textbf{Pitfall 4: Vocabulary mismatch}

\begin{itemize}
\item Test words not in training vocab
\item \textcolor{green}{Solution}: UNK token for rare words
\end{itemize}

\vspace{3mm}
\textbf{Pitfall 5: Computational explosion}

\begin{itemize}
\item Storing all n-grams
\item \textcolor{green}{Solution}: Pruning, hashing tricks
\end{itemize}

\vspace{3mm}
\textbf{Pitfall 6: Overfitting to corpus}

\begin{itemize}
\item Model learns corpus-specific patterns
\item \textcolor{green}{Solution}: Large, diverse training data
\end{itemize}

\end{columns}

\vspace{0.2cm}
\bottomnote{Awareness of pitfalls is half the battle - test rigorously}
\end{frame}

% ===== SUMMARY (Slides 40-42) =====

% Slide 40: Unified View
\begin{frame}[t]{Unified View: Everything is Conditional Probability}
\begin{center}
\includegraphics[width=0.7\textwidth]{../figures/unified_framework_bsc.pdf}
\end{center}

\vspace{3mm}
\begin{center}
\textbf{The Core Principle}:

Language modeling = Estimating $P(\text{next}|\text{previous})$ from data

\vspace{3mm}
\begin{itemize}
\item \textbf{Challenge}: Infinite possible histories
\item \textbf{Solution}: Markov assumption (limited context)
\item \textbf{Method}: Count and normalize (MLE)
\item \textbf{Fix}: Smoothing for unseen events
\item \textbf{Evaluation}: Perplexity (lower is better)
\end{itemize}
\end{center}

\vspace{0.2cm}
\bottomnote{This foundation supports all language models, including neural networks}
\end{frame}

% Slide 41: Key Takeaways
\begin{frame}[t]{Key Takeaways}
\begin{enumerate}
\item \textbf{Conditional probability is the foundation}

  Text prediction is estimating $P(w_n|w_1, ..., w_{n-1})$

\item \textbf{N-grams make it practical}

  Markov assumption limits context to last $n-1$ words

\item \textbf{MLE estimates from counts}

  $P(w|context) = \frac{\text{count}(context, w)}{\text{count}(context)}$

\item \textbf{Smoothing is essential}

  Unseen n-grams get non-zero probability

\item \textbf{Perplexity measures quality}

  Lower perplexity = better predictions

\item \textbf{Trade-offs everywhere}

  Context vs sparsity, speed vs quality, simplicity vs sophistication
\end{enumerate}

\vspace{0.2cm}
\bottomnote{Master these foundations - they underpin all of NLP}
\end{frame}

% Slide 42: Lab Preview
\begin{frame}[t]{Next: Hands-On Practice}
\begin{columns}[T]
\column{0.48\textwidth}
\raggedright
\textbf{Lab Notebook Activities}:

\begin{enumerate}
\item Build unigram model from scratch
\item Implement bigram with MLE
\item Add smoothing (compare methods)
\item Generate text and compare quality
\item Compute perplexity on test data
\item Visualize n-gram distributions
\item Experiment with different $n$ values
\end{enumerate}

\column{0.48\textwidth}
\raggedright
\textbf{What You'll Learn}:

\begin{itemize}
\item Hands-on probability estimation
\item See smoothing's impact
\item Debug zero probability issues
\item Understand perplexity intuitively
\item Reproduce presentation charts
\end{itemize}

\vspace{5mm}
\textbf{Corpus}: Shakespeare sonnets (interesting patterns!)

\end{columns}

\vspace{5mm}
\begin{center}
\Large
\textbf{Let's build some language models!}
\end{center}

\vspace{0.2cm}
\bottomnote{The lab cements understanding - theory alone is not enough}
\end{frame}

\end{document}
