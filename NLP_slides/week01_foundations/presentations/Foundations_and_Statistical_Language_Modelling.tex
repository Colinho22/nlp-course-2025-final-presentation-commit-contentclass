\documentclass[8pt,aspectratio=169]{beamer}
\usepackage[utf8]{inputenc}
\usepackage{graphicx}
\usepackage{amsmath,amssymb}
\usepackage{algorithm2e}
\usepackage{listings}
\usepackage{xcolor}
\usepackage{tikz}
\usepackage{pgfplots}
\pgfplotsset{compat=1.17}
\usepackage{subfigure}
\usepackage{hyperref}
\usepackage{multicol}
\usepackage{tcolorbox}
\usepackage{booktabs}
\usepackage{adjustbox}

% Theme settings
\usetheme{Frankfurt}
\usecolortheme{seahorse}
\setbeamertemplate{navigation symbols}{}
\setbeamertemplate{footline}[frame number]

% Custom colors
\definecolor{darkblue}{RGB}{0,51,102}
\definecolor{lightblue}{RGB}{173,216,230}
\definecolor{codegreen}{RGB}{0,128,0}
\definecolor{codegray}{RGB}{150,150,150}
\definecolor{codepurple}{RGB}{128,0,128}
\definecolor{backcolor}{RGB}{245,245,245}
\definecolor{correctgreen}{RGB}{0,150,0}
\definecolor{incorrectred}{RGB}{200,0,0}

% Code listing settings
\lstset{
    backgroundcolor=\color{backcolor},
    basicstyle=\ttfamily\footnotesize,
    breakatwhitespace=false,
    breaklines=true,
    captionpos=b,
    commentstyle=\color{codegreen},
    keywordstyle=\color{blue},
    numberstyle=\tiny\color{codegray},
    stringstyle=\color{codepurple},
    showstringspaces=false,
    frame=single,
    numbers=left,
    language=Python
}

% Include slide layouts
% SLIDE LAYOUT TEMPLATES FOR NLP COURSE
% Three consistent layouts used throughout the course

% ==============================================================
% LAYOUT 1: CONCEPT SLIDE
% Used for: Theory, definitions, mathematical concepts
% Features: Title, main content area with bullets/equations, optional figure
% ==============================================================

\newcommand{\conceptslide}[3]{ % #1: title, #2: content, #3: optional figure
\begin{frame}[t]{#1}
    \begin{columns}[T]
        \begin{column}{0.6\textwidth}
            #2
        \end{column}
        \begin{column}{0.35\textwidth}
            \centering
            #3
        \end{column}
    \end{columns}
\end{frame}
}

% ==============================================================
% LAYOUT 2: CODE & IMPLEMENTATION SLIDE
% Used for: Code examples, algorithms, implementation details
% Features: Title, code block, explanation text
% ==============================================================

% Note: For code slides, use \begin{frame}[fragile] manually

\newcommand{\codeexplanation}[1]{%
    \vspace{0.5em}
    \small
    #1
}

\newcommand{\codeblock}[2][Python]{%
    \begin{lstlisting}[language=#1, basicstyle=\ttfamily\tiny]
#2
    \end{lstlisting}
}

\newcommand{\codeslide}[3]{%
    \begin{frame}[fragile]{#1}
    \begin{columns}[T]
        \column{0.55\textwidth}
        \codeblock{#2}
        \column{0.43\textwidth}
        \codeexplanation{#3}
    \end{columns}
    \end{frame}
}

% ==============================================================
% LAYOUT 3: RESULTS & VISUALIZATION SLIDE
% Used for: Experimental results, charts, comparisons
% Features: Title, large visualization area, key insights
% ==============================================================

\newcommand{\resultslide}[3]{ % #1: title, #2: visualization, #3: insights
\begin{frame}[t]{#1}
    \centering
    \vspace{-0.5em}
    #2
    \vspace{0.5em}
    
    \begin{block}{Key Insights}
        #3
    \end{block}
\end{frame}
}

% Additional utility commands for consistency

% Highlighted text
\newcommand{\highlight}[1]{\textcolor{blue}{\textbf{#1}}}

% Math notation shortcuts
\newcommand{\prob}[1]{P(#1)}
\newcommand{\given}{\mid}
\newcommand{\argmax}{\operatorname{argmax}}
\newcommand{\softmax}{\operatorname{softmax}}

% Standard figure size
\newcommand{\stdfigsizeinchins}{0.4\textwidth}

% For code blocks, use lstlisting directly:
% \begin{lstlisting}[language=Python]
% ... code ...
% \end{lstlisting}

% Equation highlight box
\newcommand{\eqbox}[1]{
    \begin{center}
    \colorbox{lightblue!20}{
        \parbox{0.8\textwidth}{
            \vspace{0.3em}
            \centering
            #1
            \vspace{0.3em}
        }
    }
    \end{center}
}

% Custom commands for BSc level
\newcommand{\checkpoint}[1]{
    \begin{tcolorbox}[colback=yellow!10,colframe=orange!50!black,title=Checkpoint]
        #1
    \end{tcolorbox}
}

\newcommand{\prereq}[1]{
    \begin{tcolorbox}[colback=blue!5,colframe=blue!50!black,title=Prerequisite]
        #1
    \end{tcolorbox}
}

\newcommand{\misconception}[1]{
    \begin{tcolorbox}[colback=red!5,colframe=red!50!black,title=Common Misconception]
        #1
    \end{tcolorbox}
}

\title[Week 1: Foundations]{Natural Language Processing Course}
\subtitle{Week 1: Foundations and Statistical Language Models}
\author{Joerg R. Osterrieder \\ \url{www.joergosterrieder.com}}
\date{BSc Computer Science}

\begin{document}

% Title page
\begin{frame}
    \titlepage
\end{frame}

% Table of Contents
\begin{frame}{Contents}
    \tableofcontents
\end{frame}

\section{Part 1: Introduction and Motivation}

% Interactive Hook Slide
\begin{frame}[t]{Interactive Exercise: Word Prediction}
    \centering
    {\Large Complete this sentence:}
    
    \vspace{1cm}
    {\huge ``The cat sat on the \_\_\_\_\_''}
    
    \vspace{1cm}
    \pause
    
    \begin{columns}[T]
        \begin{column}{0.3\textwidth}
            \centering
            \textcolor{correctgreen}{\textbf{mat}} \\
            65\% of you
        \end{column}
        \begin{column}{0.3\textwidth}
            \centering
            \textcolor{correctgreen}{\textbf{floor}} \\
            20\% of you
        \end{column}
        \begin{column}{0.3\textwidth}
            \centering
            \textcolor{correctgreen}{\textbf{couch}} \\
            15\% of you
        \end{column}
    \end{columns}
    
    \vspace{1cm}
    \pause
    \checkpoint{
        You predicted based on patterns seen thousands of times.
    }
\end{frame}

% Real-world Impact with Statistics
\begin{frame}[t]{This Technology Is Everywhere}
    \begin{columns}[T]
        \begin{column}{0.5\textwidth}
            \includegraphics[width=\textwidth]{../figures/smartphone_typing.pdf}
        \end{column}
        \begin{column}{0.5\textwidth}
            \textbf{Every day, you use language models:}
            \begin{itemize}
                \item \textbf{5 hours:} Average smartphone use\footnotemark
                \item \textbf{36.2 wpm:} Mobile typing speed\footnotemark
                \item \textbf{25-30\%:} Keystrokes saved by prediction\footnotemark
                \item \textbf{10,000+:} Words typed monthly
            \end{itemize}
            
            \vspace{0.5em}
        \end{column}
    \end{columns}
    
    \footnotetext[1]{DataReportal Global Digital Report, 2024}
    \footnotetext[2]{Cambridge typing speed study, 2019}
    \footnotetext[3]{Google Keyboard team, 2023}
\end{frame}

% Learning Objectives with Difficulty Indicators
\begin{frame}[t]{Learning Objectives}
    \begin{columns}[T]
        \begin{column}{0.6\textwidth}
            \textbf{Core Skills:}
            \begin{itemize}
                \item[\checkmark] \textcolor{correctgreen}{[Easy]} Count word patterns in text
                \item[\checkmark] \textcolor{correctgreen}{[Easy]} Calculate simple probabilities
                \item[\checkmark] \textcolor{orange}{[Medium]} Build a text predictor
                \item[\checkmark] \textcolor{orange}{[Medium]} Handle unseen words
                \item[\checkmark] \textcolor{red}{[Challenge]} Implement smoothing
            \end{itemize}
            
            \vspace{0.5em}
            \textbf{Applications:}
            \begin{itemize}
                \item Build autocomplete
                \item Create spell checker
                \item Understand ChatGPT foundations
            \end{itemize}
        \end{column}
        \begin{column}{0.35\textwidth}
            \prereq{
                \textbf{You should know:}
                \begin{itemize}
                    \item Basic probability
                    \item Python basics
                    \item What a dictionary is
                \end{itemize}
                
            }
        \end{column}
    \end{columns}
\end{frame}

% Prerequisite Self-Check
\begin{frame}[fragile,t]{Quick Self-Assessment}
    \textbf{Can you answer these? (Solutions on next slide)}
    
    \begin{enumerate}
        \item If you flip a fair coin twice, what's P(two heads)?
        \item What does this Python code do?
        \begin{lstlisting}[language=Python]
counts = {}
for word in text.split():
    counts[word] = counts.get(word, 0) + 1
        \end{lstlisting}
        \item If P(rain) = 0.3 and P(clouds|rain) = 0.9, what is P(rain AND clouds)?
    \end{enumerate}
    
    \vspace{0.5em}
    \pause
    
    \checkpoint{
        \textbf{Solutions:}
        \begin{enumerate}
            \item 0.5 × 0.5 = 0.25
            \item Counts word frequencies in text
            \item P(rain) × P(clouds|rain) = 0.3 × 0.9 = 0.27
        \end{enumerate}
        
        \textbf{Scoring:} 2+ correct = ready to proceed; <2 = review recommended
    }
\end{frame}

\section{Part 2: Core Concepts}

% Building Intuition - Human Pattern Recognition
\begin{frame}[t]{Human Text Prediction}
    \textbf{Human pattern recognition:}
    
    \vspace{0.5em}
    Try to complete these:
    \begin{itemize}
        \item ``Once upon a \_\_\_\_'' \pause $\rightarrow$ \textcolor{correctgreen}{time}
        \pause
        \item ``Thank you very \_\_\_\_'' \pause $\rightarrow$ \textcolor{correctgreen}{much}
        \pause
        \item ``To be or not to \_\_\_\_'' \pause $\rightarrow$ \textcolor{correctgreen}{be}
    \end{itemize}
    
    \vspace{0.5em}
    \textbf{Mechanism:}
    \begin{itemize}
        \item You've seen these phrases hundreds of times
        \item Your brain stored the patterns
        \item You can estimate likelihood from experience
    \end{itemize}
    
    \vspace{0.5em}
    \checkpoint{
        Computers replicate this process through counting.
    }
\end{frame}

% The Counting Approach - Visual
\begin{frame}[t]{The Counting Approach}
    \begin{columns}[T]
        \begin{column}{0.5\textwidth}
            \textbf{Step 1: Read lots of text}
            \begin{itemize}
                \item ``the cat sat''
                \item ``the cat ran''
                \item ``the dog sat''
                \item ``the dog ran''
                \item ``the cat sat''
            \end{itemize}
            
            \vspace{0.5em}
            \textbf{Step 2: Count patterns}
            \begin{tabular}{lr}
                \toprule
                Pattern & Count \\
                \midrule
                ``the cat'' & 3 \\
                ``the dog'' & 2 \\
                ``cat sat'' & 2 \\
                ``cat ran'' & 1 \\
                ``dog sat'' & 1 \\
                ``dog ran'' & 1 \\
                \bottomrule
            \end{tabular}
        \end{column}
        \begin{column}{0.5\textwidth}
            \textbf{Step 3: Calculate probabilities}
            
            After ``the'':
            \begin{itemize}
                \item P(cat) = 3/5 = 0.6
                \item P(dog) = 2/5 = 0.4
            \end{itemize}
            
            After ``cat'':
            \begin{itemize}
                \item P(sat) = 2/3 = 0.67
                \item P(ran) = 1/3 = 0.33
            \end{itemize}
            
            \vspace{0.5em}
            \includegraphics[width=\textwidth]{../figures/ngram_probabilities.pdf}
        \end{column}
    \end{columns}
\end{frame}

% Probability Basics - Gentle Introduction
\begin{frame}[t]{Probability Refresher (5 minutes)}
    \begin{columns}[T]
        \begin{column}{0.5\textwidth}
            \textbf{What is probability?}
            \begin{itemize}
                \item Chance of something happening
                \item Always between 0 and 1
                \item 0 = impossible, 1 = certain
            \end{itemize}
            
            \vspace{0.5em}
            \textbf{Language example:}
            \begin{itemize}
                \item Total times we saw ``the'': 100
                \item Times followed by ``cat'': 30
                \item P(cat|the) = 30/100 = 0.3
            \end{itemize}
        \end{column}
        \begin{column}{0.5\textwidth}
            \textbf{Conditional Probability:}
            
            $P(B\given A)$ = ``Probability of B given A''
            
            \vspace{0.5em}
            \includegraphics[width=\textwidth]{../figures/probability_tree.pdf}
            
            \vspace{0.5em}
            \textbf{In our context:}
            \begin{itemize}
                \item A = previous word(s)
                \item B = next word
                \item P(next|previous) is the target
            \end{itemize}
        \end{column}
    \end{columns}
\end{frame}

% N-gram Models - Progressive Complexity
\begin{frame}[t]{N-gram Models: From Simple to Complex}
    \textbf{The ``N'' in N-gram = how many words we look at}
    
    \vspace{0.5em}
    \begin{columns}[T]
        \begin{column}{0.33\textwidth}
            \textbf{1-gram (Unigram)}
            \begin{itemize}
                \item Look at: current word only
                \item Ignore: all context
                \item P(cat) = count(cat)/total
            \end{itemize}
            
            Example output:
            ``cat the dog sat ran''
            
            \textcolor{incorrectred}{Random word sequence}
        \end{column}
        \begin{column}{0.33\textwidth}
            \textbf{2-gram (Bigram)}
            \begin{itemize}
                \item Look at: previous word
                \item Remember: last word
                \item P(cat|the) = count(the cat)/count(the)
            \end{itemize}
            
            Example output:
            ``the cat sat on''
            
            \textcolor{correctgreen}{More coherent}
        \end{column}
        \begin{column}{0.33\textwidth}
            \textbf{3-gram (Trigram)}
            \begin{itemize}
                \item Look at: two previous words
                \item Remember: more context
                \item P(sat|the cat)
            \end{itemize}
            
            Example output:
            ``the cat sat on the mat''
            
            \textcolor{correctgreen}{Better coherence}
        \end{column}
    \end{columns}
    
    \vspace{0.5em}
    \includegraphics[width=\textwidth]{../figures/ngram_sliding_window.pdf}
\end{frame}

% Interactive Class Example
\begin{frame}[t]{Class Example: Building a Bigram Model}
    \textbf{Training data:} ``I love cats. I love dogs. Dogs love bones.''
    
    \vspace{0.5em}
    \textbf{Step 1: Split into words}
    \begin{itemize}
        \item Words: [I, love, cats, I, love, dogs, dogs, love, bones]
    \end{itemize}
    
    \pause
    \vspace{0.5em}
    \textbf{Step 2: Count bigrams (pairs)}
    \begin{columns}[T]
        \begin{column}{0.3\textwidth}
            \begin{tabular}{lr}
                \toprule
                Bigram & Count \\
                \midrule
                (I, love) & 2 \\
                (love, cats) & 1 \\
                (love, dogs) & 1 \\
                (dogs, love) & 1 \\
                (love, bones) & 1 \\
                \bottomrule
            \end{tabular}
        \end{column}
        \pause
        \begin{column}{0.3\textwidth}
            \textbf{Step 3: Count words}
            \begin{tabular}{lr}
                \toprule
                Word & Count \\
                \midrule
                I & 2 \\
                love & 3 \\
                cats & 1 \\
                dogs & 2 \\
                bones & 1 \\
                \bottomrule
            \end{tabular}
        \end{column}
        \pause
        \begin{column}{0.3\textwidth}
            \textbf{Step 4: Calculate}
            
            $P(\text{love}\given\text{I}) = 2/2 = 1.0$
            
            $P(\text{cats}\given\text{love}) = 1/3 = 0.33$
            
            $P(\text{dogs}\given\text{love}) = 1/3 = 0.33$
            
            $P(\text{bones}\given\text{love}) = 1/3 = 0.33$
        \end{column}
    \end{columns}
\end{frame}

% Checkpoint: Understanding Check
\begin{frame}[t]{Checkpoint: Test Your Understanding}
    \checkpoint{
        \textbf{Quick Quiz:} Given the text ``the cat sat the cat ran the dog sat''
        
        \begin{enumerate}
            \item What is P(cat|the)?
            \item What is P(sat|cat)?
            \item What word most likely follows ``the''?
        \end{enumerate}
        
        \textbf{Calculate, then verify}
    }
    
    \pause
    \vspace{0.5em}
    
    \textbf{Solution:}
    \begin{itemize}
        \item Count: ``the cat'' appears 2 times, ``the dog'' appears 1 time
        \item Total ``the'': 3 times
        \item P(cat|the) = 2/3 = 0.67
        \item P(sat|cat) = 1/2 = 0.5 (``cat sat'' once, ``cat ran'' once)
        \item Most likely after ``the'': ``cat'' (67\% vs 33\%)
    \end{itemize}
    
    \textcolor{correctgreen}{Correct answers indicate understanding of n-grams}
\end{frame}

% Enhanced Implementation with Documentation
\begin{frame}[fragile]{Implementation: Clean Code with Type Hints}
\begin{columns}[T]
\column{0.55\textwidth}

\begin{lstlisting}[language=Python]
from collections import defaultdict
from typing import Dict, List, Tuple

class BigramModel:
    """A simple bigram language model."""
    
    def __init__(self):
        """Initialize with empty count dictionaries."""
        # Count of word pairs: (word1, word2) -> count
        self.bigram_counts: Dict[Tuple[str, str], int] = defaultdict(int)
        # Count of individual words: word -> count
        self.word_counts: Dict[str, int] = defaultdict(int)
        
    def train(self, sentences: List[str]) -> None:
        """Train model on list of sentences.
        
        Args:
            sentences: List of text strings to train on
            
        Example:
            >>> model = BigramModel()
            >>> model.train(["the cat sat", "the dog ran"])
        """
        for sentence in sentences:
            # Add start/end markers for sentence boundaries
            words = ['<START>'] + sentence.split() + ['<END>']
            
            # Count all adjacent word pairs
            for i in range(len(words) - 1):
                self.bigram_counts[(words[i], words[i+1])] += 1
                self.word_counts[words[i]] += 1
\end{lstlisting}

\column{0.43\textwidth}
\codeexplanation{
    \textbf{Key Improvements:}
    \begin{itemize}
        \item Type hints for clarity
        \item Docstrings with examples
        \item Clear variable names
        \item Comments for complex parts
    \end{itemize}
    
    \vspace{0.5em}
    \textbf{Why defaultdict?}
    \begin{itemize}
        \item Avoids KeyError
        \item Auto-initializes to 0
        \item Cleaner than if/else
    \end{itemize}
    
    \vspace{0.5em}
    \textbf{Why START/END?}
    \begin{itemize}
        \item Handle first word
        \item Know when to stop
        \item Improve generation
    \end{itemize}
}
\end{columns}
\end{frame}

\section{Part 3: Challenges and Solutions}

% The Zero Problem - Dramatic Reveal
\begin{frame}[t]{The Disaster: When Your Model Breaks}
    \textbf{You trained on 1 billion words from Wikipedia...}
    
    \vspace{0.5em}
    Then someone types: ``I love my new \textcolor{red}{iPhone}''
    
    \pause
    \vspace{0.5em}
    \textbf{Problem:} ``iPhone'' wasn't invented when Wikipedia was written
    \begin{itemize}
        \item count(iPhone) = 0
        \item P(anything|iPhone) = 0/0 = \textcolor{red}{UNDEFINED}
        \item Application crashes
    \end{itemize}
    
    \pause
    \vspace{0.5em}
    \misconception{
        ``Just use more training data'' - \textcolor{incorrectred}{Incorrect}
        
        You'll ALWAYS have unseen words:
        \begin{itemize}
            \item New words: COVID, cryptocurrency, TikTok
            \item Typos: teh, recieve, occured
            \item Names: Khaleesi, Hermione (before the books)
            \item Made-up words: hangry, adulting, selfie
        \end{itemize}
    }
    
    \vspace{0.5em}
    \textbf{Fact:} 50\% of word types appear only once in any text
\end{frame}

% Smoothing Intuition - Restaurant Analogy
\begin{frame}[t]{The Solution: Smoothing (Restaurant Analogy)}
    \begin{columns}[T]
        \begin{column}{0.5\textwidth}
            \textbf{Imagine a new restaurant:}
            \begin{itemize}
                \item Day 1: 10 people order pizza, 5 order pasta
                \item Day 2: Someone asks for salad
                \item Problem: P(salad) = 0?
            \end{itemize}
            
            \vspace{0.5em}
            \textbf{Bad solution:} ``We don't serve salad''
            
            \textbf{Good solution:} ``We might have salad, let me check''
            
            \vspace{0.5em}
            \textbf{In probability terms:}
            \begin{itemize}
                \item Reserve some probability for unseen items
                \item Take a little from seen items
                \item Give to unseen possibilities
            \end{itemize}
        \end{column}
        \begin{column}{0.5\textwidth}
            \textbf{Add-One Smoothing:}
            
            Pretend you saw everything once more:
            
            \vspace{0.5em}
            Original:
            \begin{itemize}
                \item Pizza: 10/15 = 0.67
                \item Pasta: 5/15 = 0.33
                \item Salad: 0/15 = 0.00
            \end{itemize}
            
            \vspace{0.5em}
            With smoothing (+1 to all):
            \begin{itemize}
                \item Pizza: 11/18 = 0.61
                \item Pasta: 6/18 = 0.33
                \item Salad: 1/18 = 0.06
            \end{itemize}
            
            \textcolor{correctgreen}{All words now have non-zero probability}
        \end{column}
    \end{columns}
\end{frame}

% Smoothing Methods Comparison
\begin{frame}[t]{Smoothing Methods: From Simple to Sophisticated}
    \includegraphics[width=0.8\textwidth]{../figures/smoothing_comparison.pdf}
    
    \vspace{0.5em}
    \begin{columns}[T]
        \begin{column}{0.33\textwidth}
            \textbf{Add-One (Laplace)}
            \begin{itemize}
                \item[+] Dead simple
                \item[+] Always works
                \item[-] Too generous to rare
                \item Use: Quick prototypes
            \end{itemize}
        \end{column}
        \begin{column}{0.33\textwidth}
            \textbf{Good-Turing}
            \begin{itemize}
                \item[+] Theoretically sound
                \item[+] Used in Enigma code-breaking
                \item[-] Complex math
                \item Use: When you have time
            \end{itemize}
        \end{column}
        \begin{column}{0.33\textwidth}
            \textbf{Kneser-Ney}
            \begin{itemize}
                \item[+] Best performance
                \item[+] Context-aware
                \item[-] Hard to implement
                \item Use: Production systems
            \end{itemize}
        \end{column}
    \end{columns}
\end{frame}

\section{Part 4: Applications and Looking Forward}

% Success Stories Timeline
\begin{frame}[t]{N-grams Changed the World}
    \centering
    \includegraphics[width=0.9\textwidth]{../figures/ngram_timeline.pdf}
    
    \vspace{0.5em}
    \begin{itemize}
        \item \textbf{1948:} Shannon invents the concept
        \item \textbf{1970s:} First speech recognition systems (IBM)
        \item \textbf{1990s:} Statistical machine translation revolution
        \item \textbf{2000s:} Google search suggestions
        \item \textbf{2007:} iPhone predictive text
        \item \textbf{2010s:} Still in every smartphone keyboard
        \item \textbf{Today:} Foundation for understanding transformers
    \end{itemize}
\end{frame}

% Interactive Failure Examples
\begin{frame}[t]{Try This: Where N-grams Fail}
    \textbf{Testing n-gram limitations:}
    
    \vspace{0.5em}
    \textbf{Test 1: Long-distance dependencies}
    \begin{itemize}
        \item ``The key to the cabinet that was in the kitchen \underline{\hspace{2cm}}''
        \item N-gram sees: ``kitchen \_\_''
        \item Should see: ``The key ... is''
    \end{itemize}
    
    \pause
    \vspace{0.5em}
    \textbf{Test 2: Common sense}
    \begin{itemize}
        \item ``I put milk in the fridge. I put the car in the \underline{\hspace{2cm}}''
        \item N-gram might say: ``fridge'' (high frequency)
        \item Should say: ``garage''
    \end{itemize}
    
    \pause
    \vspace{0.5em}
    \textbf{Test 3: Context switching}
    \begin{itemize}
        \item ``The bank of the river'' vs ``The bank account''
        \item Same word, different meaning
        \item N-grams can't tell the difference
    \end{itemize}
    
    \checkpoint{
        These limitations motivated neural approaches
    }
\end{frame}

% Bridge to Next Week
\begin{frame}[t]{Next Week: The Neural Revolution}
    \begin{columns}[T]
        \begin{column}{0.5\textwidth}
            \textbf{N-grams (This Week):}
            \begin{itemize}
                \item Words are just symbols
                \item Count and divide
                \item No understanding
                \item Limited context
                \item 156 perplexity
            \end{itemize}
            
            \vspace{0.5em}
            \textbf{Strengths:}
            \begin{itemize}
                \item Simple
                \item Fast
                \item Interpretable
                \item Low memory
            \end{itemize}
        \end{column}
        \begin{column}{0.5\textwidth}
            \textbf{Neural Models (Next Week):}
            \begin{itemize}
                \item Words have meaning
                \item Learn representations
                \item Understand similarity
                \item Unlimited context
                \item 12 perplexity!
            \end{itemize}
            
            \vspace{0.5em}
            \textbf{Preview:}
            \begin{itemize}
                \item ``King - Man + Woman = ?''
                \item Word vectors
                \item Semantic understanding
                \item Foundation of ChatGPT
            \end{itemize}
        \end{column}
    \end{columns}
    
    \vspace{0.5em}
    \centering
    \colorbox{lightblue!30}{
        \parbox{0.8\textwidth}{
            \centering
            \textbf{Key Question:} What if words weren't just strings, but had meaning?
        }
    }
\end{frame}

% Summary with Key Formulas
\begin{frame}[t]{Week 1 Summary: Your Toolkit}
    \begin{columns}[T]
        \begin{column}{0.5\textwidth}
            \textbf{Key Concepts:}
            \begin{itemize}
                \item Language modeling = prediction
                \item N-grams = counting patterns
                \item Probability from frequency
                \item Smoothing for unknowns
            \end{itemize}
            
            \vspace{0.5em}
            \textbf{Key Skills:}
            \begin{itemize}
                \item Build bigram model
                \item Calculate probabilities
                \item Handle unseen words
                \item Evaluate with perplexity
            \end{itemize}
        \end{column}
        \begin{column}{0.5\textwidth}
            \textbf{Essential Formulas:}
            
            \eqbox{
                Bigram: $P(w_2|w_1) = \frac{C(w_1, w_2)}{C(w_1)}$
            }
            
            \eqbox{
                Add-One: $P(w_2|w_1) = \frac{C(w_1, w_2) + 1}{C(w_1) + V}$
            }
            
            \eqbox{
                Perplexity: $PP = 2^{-\frac{1}{N}\sum\log_2 P(w_i|w_{i-1})}$
            }
        \end{column}
    \end{columns}
    
    \vspace{0.5em}
    \checkpoint{
        \textbf{Remember:} Simple counting goes surprisingly far in NLP!
    }
\end{frame}

% Homework Assignment
\begin{frame}[t]{Homework: Build Your Own Spell Checker}
    \textbf{Your Mission:} Create a spell checker like Google's
    
    \vspace{0.5em}
    \begin{columns}[T]
        \begin{column}{0.5\textwidth}
            \textbf{Part 1: Basic (60\%)}
            \begin{itemize}
                \item Load provided corpus
                \item Build character trigram model
                \item Generate corrections for:
                    \begin{itemize}
                        \item ``teh'' $\rightarrow$ ``the''
                        \item ``speling'' $\rightarrow$ ``spelling''
                    \end{itemize}
            \end{itemize}
            
            \vspace{0.5em}
            \textbf{Part 2: Enhanced (30\%)}
            \begin{itemize}
                \item Rank by probability
                \item Handle word boundaries
                \item Compare smoothing methods
            \end{itemize}
        \end{column}
        \begin{column}{0.5\textwidth}
            \textbf{Part 3: Analysis (10\%)}
            \begin{itemize}
                \item Test on 100 misspellings
                \item Report accuracy
                \item Explain failures
            \end{itemize}
            
            \vspace{0.5em}
            \textbf{Starter Code Provided:}
            \begin{itemize}
                \item Data loading function
                \item Evaluation metrics
                \item Test cases
            \end{itemize}
            
            \vspace{0.5em}
            \textbf{Due:} Next week before class
        \end{column}
    \end{columns}
\end{frame}

% Self-Assessment Quiz
\begin{frame}[t]{Self-Assessment: How Well Did You Learn?}
    \textbf{Test yourself (answers in appendix):}
    
    \begin{enumerate}
        \item What's the difference between unigram and bigram models?
        \item Why do we need smoothing?
        \item Given ``the cat'' appears 10 times and ``the'' appears 20 times, what is P(cat|the)?
        \item What's the main limitation of n-gram models?
        \item When would you use trigrams over bigrams?
    \end{enumerate}
    
    \vspace{0.5em}
    \checkpoint{
        \textbf{Score yourself:}
        \begin{itemize}
            \item 5/5: Ready for next week!
            \item 3-4/5: Review smoothing section
            \item Less than 3/5: Office hours recommended
        \end{itemize}
    }
\end{frame}

\appendix
\section{Appendix A: Prerequisites}

% Probability Review
\begin{frame}[t]{Appendix A.1: Probability Basics Review}
    \begin{columns}[T]
        \begin{column}{0.5\textwidth}
            \textbf{Core Concepts:}
            
            \textbf{1. Basic Probability}
            \begin{itemize}
                \item $P(A) = \frac{\text{favorable outcomes}}{\text{total outcomes}}$
                \item Always: $0 \leq P(A) \leq 1$
                \item $P(\text{not A}) = 1 - P(A)$
            \end{itemize}
            
            \vspace{0.5em}
            \textbf{2. Joint Probability}
            \begin{itemize}
                \item $P(A \text{ and } B) = P(A) \times P(B|A)$
                \item If independent: $P(A,B) = P(A) \times P(B)$
            \end{itemize}
            
            \vspace{0.5em}
            \textbf{3. Conditional Probability}
            \begin{itemize}
                \item $P(B|A) = \frac{P(A,B)}{P(A)}$
                \item Read as: ``B given A''
            \end{itemize}
        \end{column}
        \begin{column}{0.5\textwidth}
            \textbf{Language Examples:}
            
            \textbf{Example 1:}
            \begin{itemize}
                \item 1000 words in text
                \item ``the'' appears 100 times
                \item $P(\text{the}) = 100/1000 = 0.1$
            \end{itemize}
            
            \vspace{0.5em}
            \textbf{Example 2:}
            \begin{itemize}
                \item ``the cat'' appears 20 times
                \item ``the'' appears 100 times
                \item $P(\text{cat}|\text{the}) = 20/100 = 0.2$
            \end{itemize}
            
            \vspace{0.5em}
            \textbf{Example 3:}
            \begin{itemize}
                \item P(sunny) = 0.7
                \item P(happy|sunny) = 0.9
                \item P(sunny and happy) = 0.7 × 0.9 = 0.63
            \end{itemize}
        \end{column}
    \end{columns}
\end{frame}

% Python Prerequisites
\begin{frame}[fragile]{Appendix A.2: Python Essentials for NLP}
\begin{columns}[T]
\column{0.5\textwidth}
\textbf{Dictionaries (for counting):}
\begin{lstlisting}[language=Python]
# Creating and using dictionaries
counts = {}
counts['cat'] = 5
counts['dog'] = 3

# Default values
counts.get('bird', 0)  # Returns 0 if not found

# Counting pattern
word = 'cat'
counts[word] = counts.get(word, 0) + 1
\end{lstlisting}

\textbf{Lists and Loops:}
\begin{lstlisting}[language=Python]
# Splitting text
text = "the cat sat"
words = text.split()  # ['the', 'cat', 'sat']

# Iterating with index
for i in range(len(words) - 1):
    current = words[i]
    next_word = words[i + 1]
    print(f"{current} -> {next_word}")
\end{lstlisting}

\column{0.5\textwidth}
\textbf{Collections module:}
\begin{lstlisting}[language=Python]
from collections import defaultdict, Counter

# defaultdict: never KeyError
bigrams = defaultdict(int)
bigrams[('the', 'cat')] += 1

# Counter: automatic counting
words = ['cat', 'dog', 'cat', 'bird']
word_counts = Counter(words)
# Counter({'cat': 2, 'dog': 1, 'bird': 1})

# Most common
top_2 = word_counts.most_common(2)
# [('cat', 2), ('dog', 1)]
\end{lstlisting}

\textbf{File I/O:}
\begin{lstlisting}[language=Python]
# Reading text file
with open('corpus.txt', 'r') as f:
    text = f.read()
    
# Line by line
with open('corpus.txt', 'r') as f:
    for line in f:
        process(line.strip())
\end{lstlisting}
\end{columns}
\end{frame}

% Notation Guide
\begin{frame}[t]{Appendix A.3: Mathematical Notation Guide}
    \begin{columns}[T]
        \begin{column}{0.5\textwidth}
            \textbf{Symbols:}
            \begin{tabular}{ll}
                \toprule
                Symbol & Meaning \\
                \midrule
                $P(A)$ & Probability of A \\
                $P(A|B)$ & Probability of A given B \\
                $C(w)$ & Count of word w \\
                $V$ & Vocabulary size \\
                $N$ & Number of words \\
                $\sum$ & Sum over all values \\
                $\prod$ & Product over all values \\
                $\log$ & Logarithm (usually base 2) \\
                $\approx$ & Approximately equal \\
                $\propto$ & Proportional to \\
                \bottomrule
            \end{tabular}
        \end{column}
        \begin{column}{0.5\textwidth}
            \textbf{Common Formulas:}
            
            \textbf{N-gram probability:}
            $$P(w_n|w_1...w_{n-1})$$
            
            \textbf{Chain rule:}
            $$P(w_1, w_2, w_3) = P(w_1) \cdot P(w_2|w_1) \cdot P(w_3|w_1,w_2)$$
            
            \textbf{Maximum likelihood:}
            $$P_{ML}(w_i|w_{i-1}) = \frac{C(w_{i-1}, w_i)}{C(w_{i-1})}$$
            
            \textbf{Log probability (avoid underflow):}
            $$\log P(sentence) = \sum_i \log P(w_i|w_{i-1})$$
        \end{column}
    \end{columns}
\end{frame}

\section{Appendix B: Extended Examples}

% Shakespeare Generation
\begin{frame}[fragile]{Appendix B.1: Generate Shakespeare with N-grams}
    \textbf{Let's build a Shakespeare text generator!}
    
\begin{lstlisting}[language=Python]
import random
from collections import defaultdict

class ShakespeareGenerator:
    def __init__(self, n=2):
        self.n = n  # Use bigrams by default
        self.model = defaultdict(list)
        
    def train(self, text):
        words = text.split()
        for i in range(len(words) - self.n):
            context = tuple(words[i:i+self.n])
            next_word = words[i+self.n]
            self.model[context].append(next_word)
    
    def generate(self, length=20, start=None):
        if start is None:
            start = random.choice(list(self.model.keys()))
        
        result = list(start)
        context = start
        
        for _ in range(length):
            if context in self.model:
                next_word = random.choice(self.model[context])
                result.append(next_word)
                # Slide the context window
                context = tuple(list(context)[1:] + [next_word])
            else:
                break  # No continuation found
                
        return ' '.join(result)
\end{lstlisting}
\end{frame}

% WhatsApp Analysis
\begin{frame}[fragile]{Appendix B.2: Analyze Your WhatsApp Messages}
    \textbf{Find your most predictable phrases:}
    
\begin{lstlisting}[language=Python]
def analyze_whatsapp_chat(filename):
    """Find most common patterns in your chats."""
    
    # Parse WhatsApp export
    messages = []
    with open(filename, 'r', encoding='utf-8') as f:
        for line in f:
            # Extract just the message text
            if ': ' in line:
                message = line.split(': ', 1)[1].strip()
                messages.append(message)
    
    # Build trigram model
    trigrams = defaultdict(int)
    for message in messages:
        words = message.lower().split()
        for i in range(len(words) - 2):
            trigram = (words[i], words[i+1], words[i+2])
            trigrams[trigram] += 1
    
    # Find your catchphrases
    print("Your most common phrases:")
    for trigram, count in sorted(trigrams.items(), 
                                  key=lambda x: x[1], 
                                  reverse=True)[:10]:
        print(f"  '{' '.join(trigram)}' - {count} times")
    
    return trigrams
\end{lstlisting}
\end{frame}

% Autocomplete Implementation
\begin{frame}[fragile]{Appendix B.3: Build Autocomplete from Scratch}
\begin{columns}[T]
\column{0.55\textwidth}
\begin{lstlisting}[language=Python]
class Autocomplete:
    def __init__(self):
        self.bigrams = defaultdict(Counter)
        
    def train(self, corpus):
        """Train on text corpus."""
        for sentence in corpus:
            words = ['<S>'] + sentence.split()
            for i in range(len(words)-1):
                self.bigrams[words[i]][words[i+1]] += 1
    
    def predict(self, context, n=3):
        """Get top n predictions."""
        if context not in self.bigrams:
            return []
        
        # Get all possibilities with counts
        candidates = self.bigrams[context]
        
        # Sort by frequency
        predictions = candidates.most_common(n)
        
        # Convert to probabilities
        total = sum(candidates.values())
        result = []
        for word, count in predictions:
            prob = count / total
            result.append({
                'word': word,
                'probability': prob,
                'confidence': self.confidence(prob)
            })
        
        return result
\end{lstlisting}

\column{0.43\textwidth}
\begin{lstlisting}[language=Python]
    def confidence(self, prob):
        """Convert probability to confidence level."""
        if prob > 0.5:
            return "high"
        elif prob > 0.2:
            return "medium"
        else:
            return "low"
    
    def interactive_demo(self):
        """Run interactive autocomplete."""
        print("Type a word, I'll predict the next!")
        
        while True:
            word = input("\nEnter word: ").strip()
            if word == 'quit':
                break
                
            predictions = self.predict(word)
            
            if predictions:
                print("Predictions:")
                for i, pred in enumerate(predictions, 1):
                    print(f"{i}. {pred['word']} "
                          f"({pred['probability']:.1%}, "
                          f"{pred['confidence']})")
            else:
                print("No predictions available")

# Usage
auto = Autocomplete()
auto.train(sentences)
auto.interactive_demo()
\end{lstlisting}
\end{columns}
\end{frame}

\section{Appendix C: Advanced Topics}

% Kneser-Ney Mathematics
\begin{frame}[t]{Appendix C.1: Kneser-Ney Smoothing Details}
    \textbf{The Problem with Simple Counting:}
    
    ``San Francisco'' is common, but ``Francisco'' alone is rare.
    Simple smoothing would overestimate P(Francisco|new context).
    
    \vspace{0.5em}
    \textbf{Kneser-Ney Solution:}
    
    Consider word ``versatility'' - how many different contexts it appears in.
    
    \vspace{0.5em}
    \textbf{The Mathematics:}
    
    \begin{align}
        P_{KN}(w_i|w_{i-1}) &= \frac{\max(C(w_{i-1}, w_i) - d, 0)}{C(w_{i-1})} + \lambda(w_{i-1}) P_{continuation}(w_i)
    \end{align}
    
    Where:
    \begin{itemize}
        \item $d$ = discount parameter (typically 0.75)
        \item $\lambda(w_{i-1})$ = normalization to ensure probabilities sum to 1
        \item $P_{continuation}(w_i)$ = probability based on number of unique contexts
    \end{itemize}
    
    \vspace{0.5em}
    \textbf{Continuation Probability:}
    $$P_{continuation}(w) = \frac{|\{v : C(v, w) > 0\}|}{|\{(u, v) : C(u, v) > 0\}|}$$
    
    This counts unique bigram types, not tokens!
\end{frame}

% Perplexity Deep Dive
\begin{frame}[t]{Appendix C.2: Understanding Perplexity}
    \textbf{What is Perplexity?}
    
    ``How surprised is the model by the test data?''
    
    \vspace{0.5em}
    \begin{columns}[T]
        \begin{column}{0.5\textwidth}
            \textbf{Formal Definition:}
            
            $$PP(W) = P(w_1, w_2, ..., w_N)^{-\frac{1}{N}}$$
            
            \textbf{Using Chain Rule:}
            
            $$PP = \sqrt[N]{\frac{1}{\prod_{i=1}^N P(w_i|w_1...w_{i-1})}}$$
            
            \textbf{In Practice (using logs):}
            
            $$PP = 2^{-\frac{1}{N}\sum_{i=1}^N \log_2 P(w_i|w_{i-1})}$$
        \end{column}
        \begin{column}{0.5\textwidth}
            \textbf{Intuition:}
            \begin{itemize}
                \item PP = 10: Model is as confused as choosing from 10 equally likely options
                \item PP = 100: Like choosing from 100 options
                \item Lower is better!
            \end{itemize}
            
            \vspace{0.5em}
            \textbf{Typical Values:}
            \begin{itemize}
                \item Random: 10,000+
                \item Unigram: 1,000
                \item Bigram: 200
                \item Trigram: 100
                \item Neural: 20-50
                \item Human: 10-20
            \end{itemize}
            
            \vspace{0.5em}
            \textbf{Relationship to Entropy:}
            $$H = \log_2(PP)$$
        \end{column}
    \end{columns}
\end{frame}

% Information Theory Connection
\begin{frame}[t]{Appendix C.3: Connection to Information Theory}
    \textbf{Shannon's Fundamental Question:}
    
    How much information is in English text?
    
    \vspace{0.5em}
    \begin{columns}[T]
        \begin{column}{0.5\textwidth}
            \textbf{Information Content:}
            
            Surprising events = more information
            
            $$I(w) = -\log_2 P(w)$$
            
            \vspace{0.5em}
            \textbf{Examples:}
            \begin{itemize}
                \item P(``the'') = 0.1 $\rightarrow$ 3.3 bits
                \item P(``dog'') = 0.01 $\rightarrow$ 6.6 bits
                \item P(``aardvark'') = 0.00001 $\rightarrow$ 16.6 bits
            \end{itemize}
            
            Rare words carry more information!
        \end{column}
        \begin{column}{0.5\textwidth}
            \textbf{Entropy of Language:}
            
            Average information per word:
            
            $$H(L) = -\sum_{w} P(w) \log_2 P(w)$$
            
            \vspace{0.5em}
            \textbf{Shannon's Experiments:}
            \begin{itemize}
                \item Random letters: 4.7 bits/char
                \item English letters: 4.14 bits/char
                \item With context: ~1 bit/char
            \end{itemize}
            
            \vspace{0.5em}
            \textbf{Implications:}
            \begin{itemize}
                \item English is ~75\% redundant
                \item Compression possible!
                \item Prediction possible!
            \end{itemize}
        \end{column}
    \end{columns}
    
    \vspace{0.5em}
    \centering
    \colorbox{lightblue!30}{
        \parbox{0.8\textwidth}{
            \centering
            N-gram models approximate the true entropy of language
        }
    }
\end{frame}

% Quiz Answers
\begin{frame}[t]{Self-Assessment Answers}
    \textbf{Quiz Answers:}
    
    \begin{enumerate}
        \item \textbf{Difference between unigram and bigram:}
        \begin{itemize}
            \item Unigram: considers single words independently
            \item Bigram: considers pairs of adjacent words
            \item Bigram uses context, unigram doesn't
        \end{itemize}
        
        \item \textbf{Why we need smoothing:}
        \begin{itemize}
            \item Handle unseen word combinations
            \item Avoid zero probabilities
            \item Prevent model crashes/errors
        \end{itemize}
        
        \item \textbf{Calculate P(cat|the):}
        \begin{itemize}
            \item P(cat|the) = count(the cat) / count(the)
            \item P(cat|the) = 10 / 20 = 0.5
        \end{itemize}
        
        \item \textbf{Main limitation of n-grams:}
        \begin{itemize}
            \item No semantic understanding
            \item Limited context window
            \item Treat words as meaningless symbols
        \end{itemize}
        
        \item \textbf{When to use trigrams over bigrams:}
        \begin{itemize}
            \item When you have lots of training data
            \item When longer context matters
            \item When bigram accuracy isn't sufficient
        \end{itemize}
    \end{enumerate}
\end{frame}

% References
\begin{frame}[t]{References and Further Reading}
    \textbf{Core Papers:}
    \begin{itemize}
        \item Shannon, C.E. (1948). ``A Mathematical Theory of Communication''
        \item Kneser, R. \& Ney, H. (1995). ``Improved backing-off for M-gram language modeling''
        \item Chen, S.F. \& Goodman, J. (1999). ``An empirical study of smoothing techniques''
    \end{itemize}
    
    \textbf{Textbooks:}
    \begin{itemize}
        \item Jurafsky \& Martin (2024). ``Speech and Language Processing'' Chapter 3
        \item Manning \& Schütze (1999). ``Foundations of Statistical NLP''
    \end{itemize}
    
    \textbf{Online Resources:}
    \begin{itemize}
        \item Peter Norvig's ``How to Write a Spelling Corrector'' (2007)
        \item Google's ``The Unreasonable Effectiveness of Data'' (2009)
        \item Michael Collins' NLP course notes (Columbia)
    \end{itemize}
    
    \textbf{Implementation Resources:}
    \begin{itemize}
        \item NLTK library: \url{https://www.nltk.org/}
        \item KenLM (fast n-gram toolkit): \url{https://kheafield.com/code/kenlm/}
    \end{itemize}
\end{frame}

\end{document}