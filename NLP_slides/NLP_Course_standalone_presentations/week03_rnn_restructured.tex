\documentclass[8pt,aspectratio=169]{beamer}
\usepackage[utf8]{inputenc}
\usepackage{graphicx}
\usepackage{amsmath,amssymb}
\usepackage{algorithm2e}
\usepackage{listings}
\usepackage{xcolor}
\usepackage{tikz}
\usepackage{pgfplots}
\pgfplotsset{compat=1.17}
\usepackage{subfigure}
\usepackage{hyperref}
\usepackage{tcolorbox}

% Theme settings
\usetheme{Frankfurt}
\usecolortheme{seahorse}
\setbeamertemplate{navigation symbols}{}
\setbeamertemplate{footline}[frame number]

% Custom colors
\definecolor{darkblue}{RGB}{0,51,102}
\definecolor{lightblue}{RGB}{173,216,230}
\definecolor{codegreen}{RGB}{0,128,0}
\definecolor{codegray}{RGB}{150,150,150}
\definecolor{codepurple}{RGB}{128,0,128}
\definecolor{backcolor}{RGB}{245,245,245}

% Code listing settings
\lstset{
    backgroundcolor=\color{backcolor},
    basicstyle=\ttfamily\footnotesize,
    breakatwhitespace=false,
    breaklines=true,
    captionpos=b,
    commentstyle=\color{codegreen},
    keywordstyle=\color{blue},
    numberstyle=\tiny\color{codegray},
    stringstyle=\color{codepurple},
    showstringspaces=false,
    frame=single,
    numbers=left,
    language=Python
}

% Custom commands
\newcommand{\highlight}[1]{\textcolor{blue}{\textbf{#1}}}
\newcommand{\eqbox}[1]{\begin{tcolorbox}[colback=blue!5!white,colframe=blue!75!black]#1\end{tcolorbox}}
\newcommand{\given}{\mid}
\newcommand{\prob}[1]{P(#1)}

% Include slide layouts
% % SLIDE LAYOUT TEMPLATES FOR NLP COURSE
% Three consistent layouts used throughout the course

% ==============================================================
% LAYOUT 1: CONCEPT SLIDE
% Used for: Theory, definitions, mathematical concepts
% Features: Title, main content area with bullets/equations, optional figure
% ==============================================================

\newcommand{\conceptslide}[3]{ % #1: title, #2: content, #3: optional figure
\begin{frame}[t]{#1}
    \begin{columns}[T]
        \begin{column}{0.6\textwidth}
            #2
        \end{column}
        \begin{column}{0.35\textwidth}
            \centering
            #3
        \end{column}
    \end{columns}
\end{frame}
}

% ==============================================================
% LAYOUT 2: CODE & IMPLEMENTATION SLIDE
% Used for: Code examples, algorithms, implementation details
% Features: Title, code block, explanation text
% ==============================================================

% Note: For code slides, use \begin{frame}[fragile] manually

\newcommand{\codeexplanation}[1]{%
    \vspace{0.5em}
    \small
    #1
}

\newcommand{\codeblock}[2][Python]{%
    \begin{lstlisting}[language=#1, basicstyle=\ttfamily\tiny]
#2
    \end{lstlisting}
}

\newcommand{\codeslide}[3]{%
    \begin{frame}[fragile]{#1}
    \begin{columns}[T]
        \column{0.55\textwidth}
        \codeblock{#2}
        \column{0.43\textwidth}
        \codeexplanation{#3}
    \end{columns}
    \end{frame}
}

% ==============================================================
% LAYOUT 3: RESULTS & VISUALIZATION SLIDE
% Used for: Experimental results, charts, comparisons
% Features: Title, large visualization area, key insights
% ==============================================================

\newcommand{\resultslide}[3]{ % #1: title, #2: visualization, #3: insights
\begin{frame}[t]{#1}
    \centering
    \vspace{-0.5em}
    #2
    \vspace{0.5em}
    
    \begin{block}{Key Insights}
        #3
    \end{block}
\end{frame}
}

% Additional utility commands for consistency

% Highlighted text
\newcommand{\highlight}[1]{\textcolor{blue}{\textbf{#1}}}

% Math notation shortcuts
\newcommand{\prob}[1]{P(#1)}
\newcommand{\given}{\mid}
\newcommand{\argmax}{\operatorname{argmax}}
\newcommand{\softmax}{\operatorname{softmax}}

% Standard figure size
\newcommand{\stdfigsizeinchins}{0.4\textwidth}

% For code blocks, use lstlisting directly:
% \begin{lstlisting}[language=Python]
% ... code ...
% \end{lstlisting}

% Equation highlight box
\newcommand{\eqbox}[1]{
    \begin{center}
    \colorbox{lightblue!20}{
        \parbox{0.8\textwidth}{
            \vspace{0.3em}
            \centering
            #1
            \vspace{0.3em}
        }
    }
    \end{center}
}  % Commented out - commands defined above

\title[Week 3: RNNs]{Natural Language Processing Course}
\subtitle{Week 3: Recurrent Neural Networks\\Teaching Networks to Remember}
\author{Joerg R. Osterrieder}
\date{}

\begin{document}

% Title page
\begin{frame}
    \titlepage
\end{frame}

% Course Overview Reminder
\begin{frame}[t]{Course Journey: Where We Are}
    \centering
    \begin{tikzpicture}[scale=0.9]
        % Timeline
        \draw[thick,->] (0,0) -- (12,0) node[right] {Complexity};
        
        % Week markers
        \node[below] at (2,0) {Week 1};
        \node[below] at (4,0) {Week 2};
        \node[above,red,thick] at (6,0) {\textbf{Week 3}};
        \node[below] at (8,0) {Week 4};
        \node[below] at (10,0) {Week 5};
        
        % Content
        \node[above] at (2,0.3) {N-grams};
        \node[above] at (4,0.3) {Word2Vec};
        \node[below,red] at (6,-0.5) {\textbf{RNNs}};
        \node[above] at (8,0.3) {Seq2Seq};
        \node[above] at (10,0.3) {Transformers};
        
        % Progress arrow
        \draw[red,thick,->] (0,-1) -- (6,-1) node[midway,below] {You are here};
    \end{tikzpicture}
    
    \vspace{1em}
    \textbf{Journey so far:}
    \begin{itemize}
        \item Week 1: Words as counts (n-grams)
        \item Week 2: Words as vectors (embeddings)
        \item \highlight{Week 3: Processing sequences with memory}
    \end{itemize}
\end{frame}

% Learning Objectives
\begin{frame}[t]{Week 3: Learning Objectives}
    \textbf{By the end of this week, you will:}
    \begin{enumerate}
        \item \textbf{Understand} why sequential processing matters in language
        \item \textbf{Build} intuition for how RNNs maintain memory
        \item \textbf{Implement} a vanilla RNN from scratch
        \item \textbf{Diagnose} the vanishing gradient problem
        \item \textbf{Master} LSTM gates through hands-on coding
        \item \textbf{Choose} between RNN, LSTM, and GRU for your tasks
    \end{enumerate}
    
    \vspace{1em}
    \textbf{Practical Skills:}
    \begin{itemize}
        \item Build a character-level text generator
        \item Debug gradient flow in sequential models
        \item Implement LSTM gates step-by-step
        \item Know when RNNs beat Transformers (yes, it happens!)
    \end{itemize}
\end{frame}

%=====================================
% PART 1: THE SEQUENTIAL NATURE OF LANGUAGE
%=====================================

\section{Part 1: The Sequential Nature of Language}

\begin{frame}
    \centering
    \vspace{2cm}
    {\Large \textbf{Part 1}}\\
    \vspace{0.5cm}
    {\huge \textbf{The Sequential Nature of Language}}\\
    \vspace{1cm}
    {\large Why Order Matters}
\end{frame}

% Evolution from bag of words
\begin{frame}[t]{Evolution: From Bag of Words to Sequential Processing}
    \begin{columns}[T]
        \begin{column}{0.5\textwidth}
            \textbf{The Bag of Words Era:}
            \begin{itemize}
                \item Words as unordered sets
                \item "dog bites man" = \{dog, bites, man\}
                \item Lost all sequential information
                \item Still used in: Document classification
            \end{itemize}
            
            \vspace{0.5em}
            \textbf{Word Embeddings (Week 2):}
            \begin{itemize}
                \item Words as vectors
                \item Captured meaning
                \item Still no sequence modeling
            \end{itemize}
        \end{column}
        \begin{column}{0.5\textwidth}
            \textbf{The Sequential Revolution:}
            \begin{itemize}
                \item Process words in order
                \item Maintain memory of past
                \item Understand context
                \item Handle variable lengths
            \end{itemize}
            
            \vspace{0.5em}
            \colorbox{yellow!20}{
                \parbox{\textwidth}{
                    \centering
                    Language is not a bag of words.\\
                    It's a carefully ordered sequence!
                }
            }
        \end{column}
    \end{columns}
\end{frame}

% Why order matters
\begin{frame}[t]{Why Order Matters: Simple but Profound Examples}
    \textbf{Example 1: Same words, opposite meanings}
    \begin{itemize}
        \item "Dog bites man" → Normal event
        \item "Man bites dog" → News headline!
    \end{itemize}
    
    \vspace{0.5em}
    \textbf{Example 2: Negation changes everything}
    \begin{itemize}
        \item "This movie is not bad" → Positive
        \item "This movie is bad, not good" → Negative
    \end{itemize}
    
    \vspace{0.5em}
    \textbf{Example 3: Questions vs statements}
    \begin{itemize}
        \item "Can you help?" → Request
        \item "You can help" → Statement
    \end{itemize}
    
    \vspace{0.5em}
    \textbf{Example 4: Time and causality}
    \begin{itemize}
        \item "I ate because I was hungry" → Cause after effect
        \item "Because I was hungry, I ate" → Cause before effect
    \end{itemize}
    
    \vspace{0.5em}
    \begin{center}
    \colorbox{red!20}{
        Word embeddings see these as identical - they have the same vectors!
    }
    \end{center}
\end{frame}

% Memory in human reading
\begin{frame}[t]{How Humans Read: Natural Sequential Processing}
    \textbf{Reading this sentence:} "The student who studied hard..."
    
    \vspace{0.5em}
    \begin{enumerate}
        \item Read "The" → \textit{Expect noun}
        \item Read "student" → \textit{Remember: subject is student}
        \item Read "who" → \textit{Expect description of student}
        \item Read "studied" → \textit{Remember: student studied}
        \item Read "hard" → \textit{Remember: studied hard}
        \item Expect: → \textit{What happened to this student?}
    \end{enumerate}
    
    \vspace{0.5em}
    \textbf{Your brain maintains:}
    \begin{itemize}
        \item Subject tracking (who/what)
        \item Verb tracking (actions)
        \item Modifier accumulation
        \item Expectation updating
    \end{itemize}
    
    \vspace{0.5em}
    \colorbox{lightblue!30}{
        This is exactly what RNNs do - maintain and update memory!
    }
\end{frame}

% Real-world failures
\begin{frame}[t]{When Systems Forget: Real Voice Assistant Failures}
    \textbf{Actual conversation with Alexa (2024):}
    
    \begin{tcolorbox}[colback=gray!10]
    \textbf{You:} "Set a timer for 10 minutes"\\
    \textbf{Alexa:} "Timer set for 10 minutes"\\
    \textbf{You:} "Actually, make it 15"\\
    \textbf{Alexa:} "I'm not sure what you want me to make"
    \end{tcolorbox}
    
    \vspace{0.5em}
    \textbf{The problem:} No memory of "timer" from previous turn
    
    \vspace{0.5em}
    \textbf{More examples of context loss:}
    \begin{itemize}
        \item "Play some Beatles... actually, make it Queen" → Confused
        \item "What's the weather?... And tomorrow?" → Lost context
        \item "Call mom... wait, call dad instead" → Doesn't understand "instead"
    \end{itemize}
    
    \vspace{0.5em}
    \textbf{Solution needed:} Networks that remember previous inputs!
\end{frame}

% The fundamental problem
\begin{frame}[t]{The Fundamental Problem: Networks Need Memory}
    \begin{columns}[T]
        \begin{column}{0.5\textwidth}
            \textbf{Feed-forward networks:}
            \begin{itemize}
                \item Process each input independently
                \item No connection between time steps
                \item Can't handle sequences
                \item Fixed input size
            \end{itemize}
            
            \vspace{0.5em}
            \textbf{What we need:}
            \begin{itemize}
                \item Connect information across time
                \item Variable sequence lengths
                \item Memory of previous inputs
                \item Context-aware predictions
            \end{itemize}
        \end{column}
        \begin{column}{0.5\textwidth}
            \textbf{Mathematical requirement:}
            
            \eqbox{
                $h_t = f(x_t, h_{t-1})$
            }
            
            Where:
            \begin{itemize}
                \item $h_t$ = hidden state (memory)
                \item $x_t$ = current input
                \item $h_{t-1}$ = previous memory
            \end{itemize}
            
            \vspace{0.5em}
            \textbf{This recursion is the key insight!}
        \end{column}
    \end{columns}
    
    \vspace{0.5em}
    \begin{center}
    \colorbox{yellow!20}{
        RNN = Neural Network + Memory
    }
    \end{center}
\end{frame}

%=====================================
% PART 2: SIMPLE RNNs - NETWORKS WITH MEMORY
%=====================================

\section{Part 2: Simple RNNs - Networks with Memory}

\begin{frame}
    \centering
    \vspace{2cm}
    {\Large \textbf{Part 2}}\\
    \vspace{0.5cm}
    {\huge \textbf{Simple RNNs}}\\
    \vspace{1cm}
    {\large Networks with Memory}
\end{frame}

% The brilliant idea
\begin{frame}[t]{The Brilliant Idea: Hidden States as Memory}
    \textbf{How to give a network memory?}
    
    \vspace{0.5em}
    \begin{columns}[T]
        \begin{column}{0.5\textwidth}
            \textbf{The RNN approach:}
            \begin{enumerate}
                \item Start with empty memory $h_0 = 0$
                \item Read first word $x_1$
                \item Update memory: $h_1 = f(x_1, h_0)$
                \item Read second word $x_2$
                \item Update memory: $h_2 = f(x_2, h_1)$
                \item Continue...
            \end{enumerate}
            
            \vspace{0.5em}
            Memory accumulates information!
        \end{column}
        \begin{column}{0.5\textwidth}
            \textbf{Processing "The cat sat":}
            \begin{itemize}
                \item $h_0 = [0, 0, 0, ...]$ (blank)
                \item $h_1 = [0.2, -0.1, ...]$ ("The")
                \item $h_2 = [0.3, 0.5, ...]$ ("The cat")
                \item $h_3 = [0.1, 0.8, ...]$ ("The cat sat")
            \end{itemize}
            
            \vspace{0.5em}
            \colorbox{lightblue!30}{
                $h_3$ contains compressed history!
            }
        \end{column}
    \end{columns}
\end{frame}

% Mathematical formulation
\begin{frame}[t]{RNN Mathematics: Just Two Equations!}
    \textbf{The entire RNN in two lines:}
    
    \eqbox{
        $\begin{aligned}
        h_t &= \tanh(W_h h_{t-1} + W_x x_t + b_h) & \text{Update memory}\\
        y_t &= W_y h_t + b_y & \text{Generate output}
        \end{aligned}$
    }
    
    \vspace{0.5em}
    \textbf{Concrete example with actual numbers:}
    
    Let's say hidden size = 3, input size = 2:
    \begin{itemize}
        \item $x_t = [0.5, -0.3]$ (current word embedding)
        \item $h_{t-1} = [0.1, 0.2, -0.1]$ (previous memory)
        \item $W_x$ is 3×2, $W_h$ is 3×3, $W_y$ is vocab\_size×3
    \end{itemize}
    
    \vspace{0.5em}
    \textbf{Step-by-step computation:}
    \begin{enumerate}
        \item $W_x x_t = [0.2, -0.1, 0.3]$ (transform input)
        \item $W_h h_{t-1} = [0.1, 0.0, -0.2]$ (transform memory)
        \item Sum + bias = $[0.3, -0.1, 0.1]$
        \item $h_t = \tanh([0.3, -0.1, 0.1]) = [0.29, -0.10, 0.10]$
    \end{enumerate}
\end{frame}

% Unfolding through time
\begin{frame}[t]{Visualizing RNNs: Unfolding Through Time}
    \begin{center}
    \textbf{Compact view:}
    
    \begin{tikzpicture}[scale=0.8]
        \node[draw,circle,minimum size=1.5cm] (rnn) at (0,0) {RNN};
        \draw[->] (-2,0) -- (rnn) node[midway,above] {$x_t$};
        \draw[->] (rnn) -- (2,0) node[midway,above] {$y_t$};
        \draw[->] (rnn.north) .. controls (0,1.5) and (0,1.5) .. (rnn.east) 
            node[midway,right] {$h_t$};
    \end{tikzpicture}
    
    \vspace{1em}
    \textbf{Unfolded view for "The cat sat":}
    
    \begin{tikzpicture}[scale=0.7]
        % Time step 1
        \node[draw,circle] (rnn1) at (0,0) {RNN};
        \draw[->] (-1,0) -- (rnn1) node[midway,above] {"The"};
        \draw[->] (rnn1) -- (0,-1.5) node[midway,right] {$y_1$};
        
        % Time step 2
        \node[draw,circle] (rnn2) at (3,0) {RNN};
        \draw[->] (rnn1) -- (rnn2) node[midway,above] {$h_1$};
        \draw[->] (2,0.5) -- (rnn2) node[midway,above] {"cat"};
        \draw[->] (rnn2) -- (3,-1.5) node[midway,right] {$y_2$};
        
        % Time step 3
        \node[draw,circle] (rnn3) at (6,0) {RNN};
        \draw[->] (rnn2) -- (rnn3) node[midway,above] {$h_2$};
        \draw[->] (5,0.5) -- (rnn3) node[midway,above] {"sat"};
        \draw[->] (rnn3) -- (6,-1.5) node[midway,right] {$y_3$};
        
        \draw[->] (rnn3) -- (7.5,0) node[midway,above] {$h_3$};
    \end{tikzpicture}
    \end{center}
    
    \vspace{0.5em}
    \colorbox{yellow!20}{
        Same weights (RNN box) used at every time step!
    }
\end{frame}

% Implementation from scratch
\begin{frame}[fragile]{Building an RNN: Complete Implementation}
\begin{lstlisting}[language=Python]
import numpy as np

class SimpleRNN:
    def __init__(self, input_size, hidden_size, output_size):
        # Initialize weights randomly
        self.Wxh = np.random.randn(hidden_size, input_size) * 0.01
        self.Whh = np.random.randn(hidden_size, hidden_size) * 0.01
        self.Why = np.random.randn(output_size, hidden_size) * 0.01
        self.bh = np.zeros((hidden_size, 1))
        self.by = np.zeros((output_size, 1))
        
    def step(self, x, h_prev):
        """Process one time step"""
        # Update hidden state (memory)
        h = np.tanh(self.Wxh @ x + self.Whh @ h_prev + self.bh)
        
        # Compute output
        y = self.Why @ h + self.by
        
        return y, h
    
    def forward(self, inputs):
        """Process entire sequence"""
        h = np.zeros((self.Whh.shape[0], 1))  # Initial memory
        outputs = []
        
        for x in inputs:
            y, h = self.step(x, h)
            outputs.append(y)
            
        return outputs, h
\end{lstlisting}
\end{frame}

% Training with BPTT
\begin{frame}[t]{Training RNNs: Backpropagation Through Time (BPTT)}
    \textbf{The challenge:} Error must flow back through all time steps
    
    \vspace{0.5em}
    \begin{center}
    \begin{tikzpicture}[scale=0.7]
        % Forward pass
        \node[draw,circle] (rnn1) at (0,0) {RNN};
        \node[draw,circle] (rnn2) at (3,0) {RNN};
        \node[draw,circle] (rnn3) at (6,0) {RNN};
        
        \draw[->] (rnn1) -- (rnn2) node[midway,above] {$h_1$};
        \draw[->] (rnn2) -- (rnn3) node[midway,above] {$h_2$};
        
        % Losses
        \node[red] (l1) at (0,-1.5) {$L_1$};
        \node[red] (l2) at (3,-1.5) {$L_2$};
        \node[red] (l3) at (6,-1.5) {$L_3$};
        
        \draw[->] (rnn1) -- (l1);
        \draw[->] (rnn2) -- (l2);
        \draw[->] (rnn3) -- (l3);
        
        % Backward pass
        \draw[<-,red,thick] (rnn3.south) -- (l3) node[midway,right] {$\nabla L_3$};
        \draw[<-,red,thick] (rnn2.south) -- (l2) node[midway,right] {$\nabla L_2$};
        \draw[<-,red,thick] (rnn1.south) -- (l1) node[midway,right] {$\nabla L_1$};
        
        \draw[<-,red,dashed] (rnn2) -- (rnn3) node[midway,below] {gradient};
        \draw[<-,red,dashed] (rnn1) -- (rnn2) node[midway,below] {gradient};
    \end{tikzpicture}
    \end{center}
    
    \vspace{0.5em}
    \textbf{BPTT algorithm:}
    \begin{enumerate}
        \item Forward pass: Compute all hidden states and outputs
        \item Compute loss at each time step
        \item Backward pass: Accumulate gradients through time
        \item Update weights using total gradient
    \end{enumerate}
    
    \textbf{Key insight:} Gradient has to flow through many matrix multiplications!
\end{frame}

% Example application
\begin{frame}[fragile]{Example: Character-Level Text Generation}
    \textbf{Task:} Train RNN to generate "hello"
    
\begin{lstlisting}[language=Python]
# Training data
text = "hello"
chars = list(set(text))  # unique chars: h,e,l,o
char_to_idx = {ch:i for i,ch in enumerate(chars)}

# Prepare training sequences
# Input:  h e l l
# Target: e l l o

# After training, generate:
def generate(rnn, seed_char='h', length=10):
    h = np.zeros((hidden_size, 1))
    x = one_hot(seed_char)
    result = seed_char
    
    for _ in range(length):
        y, h = rnn.step(x, h)
        next_char = sample(softmax(y))
        result += next_char
        x = one_hot(next_char)
    
    return result

# Output: "hellohelllo..." (learns the pattern!)
\end{lstlisting}
\end{frame}

%=====================================
% PART 3: THE VANISHING GRADIENT PROBLEM  
%=====================================

\section{Part 3: The Vanishing Gradient Problem}

\begin{frame}
    \centering
    \vspace{2cm}
    {\Large \textbf{Part 3}}\\
    \vspace{0.5cm}
    {\huge \textbf{The Vanishing Gradient Problem}}\\
    \vspace{1cm}
    {\large Why Simple RNNs Fail}
\end{frame}

% Why RNNs fail
\begin{frame}[t]{The Fatal Flaw: Gradients Vanish Over Time}
    \textbf{Try to process this sentence:}
    
    "The student who the professor who won the Nobel Prize taught \highlight{was} brilliant"
    
    \vspace{0.5em}
    \begin{itemize}
        \item "was" agrees with "student" (15 words back!)
        \item RNN must remember "student" is singular
        \item But after 15 steps, memory is almost gone
    \end{itemize}
    
    \vspace{0.5em}
    \textbf{Mathematical explanation:}
    
    Gradient flow: $\frac{\partial L}{\partial h_0} = \frac{\partial L}{\partial h_T} \prod_{t=1}^{T} \frac{\partial h_t}{\partial h_{t-1}}$
    
    \vspace{0.5em}
    Each $\frac{\partial h_t}{\partial h_{t-1}} = W_h^T \cdot \text{diag}(\tanh'(h_{t-1}))$
    
    \vspace{0.5em}
    Since $|\tanh'(x)| \leq 1$ and typical $||W_h|| < 1$:
    
    \colorbox{red!20}{
        After T steps: gradient $\approx (0.9)^T$ → exponentially small!
    }
\end{frame}

% Visual demonstration
\begin{frame}[t]{Gradient Vanishing: A Visual Demonstration}
    \textbf{Gradient magnitude after T time steps:}
    
    \begin{center}
    \begin{tikzpicture}[scale=0.8]
        \begin{axis}[
            xlabel={Time Steps},
            ylabel={Gradient Magnitude},
            grid=major,
            width=10cm,
            height=6cm,
            ymode=log,
            legend pos=north east
        ]
        
        % Vanishing gradient
        \addplot[blue, thick] coordinates {
            (1, 0.9) (2, 0.81) (3, 0.729) (4, 0.656) 
            (5, 0.590) (10, 0.349) (15, 0.206) (20, 0.122)
        };
        \addlegendentry{Gradient ($0.9^t$)}
        
        % Threshold
        \addplot[red, dashed, thick] coordinates {
            (0, 0.001) (20, 0.001)
        };
        \addlegendentry{Effective Zero}
        
        \end{axis}
    \end{tikzpicture}
    \end{center}
    
    \textbf{Consequence:} After ~15 steps, gradient $\approx 0$ → No learning!
\end{frame}

% Real examples where RNNs fail
\begin{frame}[fragile,t]{Where Simple RNNs Break: Real Examples}
    \textbf{1. Long-range dependencies:}
    \begin{itemize}
        \item "The keys \underline{that I left on the table in the kitchen yesterday} are missing"
        \item RNN forgets "keys" are plural by the time it reaches "are"
    \end{itemize}
    
    \vspace{0.5em}
    \textbf{2. Question answering:}
    \begin{itemize}
        \item "What did John give Mary? ... [long story] ... John gave Mary a book"
        \item RNN can't connect answer to question after long context
    \end{itemize}
    
    \vspace{0.5em}
    \textbf{3. Code understanding:}
    \begin{lstlisting}[language=Python, basicstyle=\tiny]
    def complex_function(x):
        # ... 20 lines of code ...
        return result  # What is result? RNN forgot!
    \end{lstlisting}
    
    \vspace{0.5em}
    \textbf{4. Document summarization:}
    \begin{itemize}
        \item Introduction mentions key point
        \item 10 paragraphs later, conclusion should reference it
        \item RNN has completely forgotten the introduction
    \end{itemize}
\end{frame}

% The need for gates
\begin{frame}[t]{The Solution: Gated Memory (Preview of LSTM)}
    \textbf{The key insight:} Create shortcuts for gradient flow!
    
    \vspace{0.5em}
    \begin{columns}[T]
        \begin{column}{0.5\textwidth}
            \textbf{Simple RNN:}
            \begin{itemize}
                \item Gradient must flow through tanh
                \item Multiplied by weights each step
                \item Exponential decay
                \item No control over memory
            \end{itemize}
        \end{column}
        \begin{column}{0.5\textwidth}
            \textbf{LSTM Solution:}
            \begin{itemize}
                \item Add "highway" for gradients
                \item Gates control information flow
                \item Can preserve gradient for 100+ steps
                \item Selective memory
            \end{itemize}
        \end{column}
    \end{columns}
    
    \vspace{0.5em}
    \begin{center}
    \begin{tikzpicture}[scale=0.8]
        % RNN path
        \draw[thick,->] (0,0) -- (1,0) -- (2,0) -- (3,0) -- (4,0);
        \node[above] at (2,0) {RNN: Many operations};
        
        % LSTM path
        \draw[thick,blue,->] (0,-1.5) -- (4,-1.5);
        \node[below] at (2,-1.5) {LSTM: Direct gradient highway};
        
        % Gates
        \foreach \x in {1,2,3} {
            \draw[blue] (\x,-1.3) -- (\x,-1.7);
            \node[blue] at (\x,-2) {gate};
        }
    \end{tikzpicture}
    \end{center}
\end{frame}

%=====================================
% PART 4: LSTM & GRU - THE SOLUTION
%=====================================

\section{Part 4: LSTM \& GRU - The Solution}

\begin{frame}
    \centering
    \vspace{2cm}
    {\Large \textbf{Part 4}}\\
    \vspace{0.5cm}
    {\huge \textbf{LSTM \& GRU}}\\
    \vspace{1cm}
    {\large The Gated Solution}
\end{frame}

% LSTM intuition
\begin{frame}[t]{LSTM: The Breakthrough (1997)}
    \textbf{Long Short-Term Memory by Hochreiter \& Schmidhuber}
    
    \vspace{0.5em}
    \textbf{The genius idea:} Separate memory into two parts:
    \begin{enumerate}
        \item \textbf{Cell state ($c_t$):} Long-term memory highway
        \item \textbf{Hidden state ($h_t$):} Short-term working memory
    \end{enumerate}
    
    \vspace{0.5em}
    \textbf{Three gates control memory:}
    \begin{itemize}
        \item \textbf{Forget gate:} What to discard from memory
        \item \textbf{Input gate:} What new information to store
        \item \textbf{Output gate:} What to output now
    \end{itemize}
    
    \vspace{0.5em}
    \textbf{Analogy: Email inbox management}
    \begin{itemize}
        \item Forget gate = Delete spam
        \item Input gate = Save important emails
        \item Output gate = What to reply to now
    \end{itemize}
\end{frame}

% LSTM architecture with numbers
\begin{frame}[t]{LSTM Architecture: Gates Explained with Numbers}
    \textbf{Processing "The cat sat on the mat"}
    
    At time step for "sat":
    
    \vspace{0.5em}
    \begin{enumerate}
        \item \textbf{Forget gate:} $f_t = \sigma(W_f \cdot [h_{t-1}, x_t])$
        \begin{itemize}
            \item Output: $f_t = 0.9$ → Keep 90\% of previous memory
            \item "The cat" is still relevant
        \end{itemize}
        
        \item \textbf{Input gate:} $i_t = \sigma(W_i \cdot [h_{t-1}, x_t])$
        \begin{itemize}
            \item Output: $i_t = 0.8$ → Store 80\% of new info
            \item "sat" is important action
        \end{itemize}
        
        \item \textbf{Candidate values:} $\tilde{c}_t = \tanh(W_c \cdot [h_{t-1}, x_t])$
        \begin{itemize}
            \item New information to potentially store
        \end{itemize}
        
        \item \textbf{Update cell state:} $c_t = f_t * c_{t-1} + i_t * \tilde{c}_t$
        \begin{itemize}
            \item Combines old (90\%) and new (80\%) memory
        \end{itemize}
        
        \item \textbf{Output gate:} $o_t = \sigma(W_o \cdot [h_{t-1}, x_t])$
        \begin{itemize}
            \item Output: $o_t = 0.7$ → Use 70\% for current output
        \end{itemize}
    \end{enumerate}
\end{frame}

% LSTM implementation
\begin{frame}[fragile]{LSTM Implementation: Complete Code}
\begin{lstlisting}[language=Python, basicstyle=\tiny]
import numpy as np

def sigmoid(x):
    return 1 / (1 + np.exp(-x))

class LSTM:
    def __init__(self, input_size, hidden_size):
        # Weight matrices for each gate
        self.Wf = np.random.randn(hidden_size, input_size + hidden_size) * 0.01
        self.Wi = np.random.randn(hidden_size, input_size + hidden_size) * 0.01
        self.Wc = np.random.randn(hidden_size, input_size + hidden_size) * 0.01
        self.Wo = np.random.randn(hidden_size, input_size + hidden_size) * 0.01
        
    def step(self, x, h_prev, c_prev):
        # Concatenate input and previous hidden state
        combined = np.concatenate([h_prev, x])
        
        # Forget gate: decide what to forget
        f_t = sigmoid(self.Wf @ combined)
        
        # Input gate: decide what to store
        i_t = sigmoid(self.Wi @ combined)
        
        # Candidate values: create new information
        c_tilde = np.tanh(self.Wc @ combined)
        
        # Update cell state (long-term memory)
        c_t = f_t * c_prev + i_t * c_tilde
        
        # Output gate: decide what to output
        o_t = sigmoid(self.Wo @ combined)
        
        # Hidden state (short-term memory)
        h_t = o_t * np.tanh(c_t)
        
        return h_t, c_t
\end{lstlisting}
\end{frame}

% GRU simplification
\begin{frame}[t]{GRU: The Simplified Alternative (2014)}
    \textbf{Gated Recurrent Unit - Fewer parameters, similar performance}
    
    \vspace{0.5em}
    \textbf{Simplifications from LSTM:}
    \begin{itemize}
        \item Combines forget and input gates → Update gate
        \item No separate cell state (only hidden state)
        \item 25\% fewer parameters
    \end{itemize}
    
    \vspace{0.5em}
    \textbf{GRU equations:}
    \eqbox{
        $\begin{aligned}
        z_t &= \sigma(W_z \cdot [h_{t-1}, x_t]) & \text{Update gate}\\
        r_t &= \sigma(W_r \cdot [h_{t-1}, x_t]) & \text{Reset gate}\\
        \tilde{h}_t &= \tanh(W \cdot [r_t * h_{t-1}, x_t]) & \text{Candidate}\\
        h_t &= (1-z_t) * h_{t-1} + z_t * \tilde{h}_t & \text{Final output}
        \end{aligned}$
    }
    
    \vspace{0.5em}
    \textbf{When to use GRU vs LSTM:}
    \begin{itemize}
        \item GRU: Smaller datasets, faster training needed
        \item LSTM: Complex long-range dependencies, more data available
    \end{itemize}
\end{frame}

% Comparison table
\begin{frame}[t]{RNN vs LSTM vs GRU: Complete Comparison}
    \begin{center}
    \begin{tabular}{|l|c|c|c|}
    \hline
    \textbf{Aspect} & \textbf{RNN} & \textbf{LSTM} & \textbf{GRU} \\
    \hline\hline
    Year introduced & 1986 & 1997 & 2014 \\
    \hline
    Parameters & $O(h^2)$ & $O(4h^2)$ & $O(3h^2)$ \\
    \hline
    Gates & 0 & 3 & 2 \\
    \hline
    Memory & Short & Long & Long \\
    \hline
    Gradient flow & Poor & Excellent & Good \\
    \hline
    Training speed & Fast & Slow & Medium \\
    \hline
    Max sequence & ~10-20 & 100+ & 100+ \\
    \hline
    \end{tabular}
    \end{center}
    
    \vspace{0.5em}
    \textbf{Practical recommendations:}
    \begin{itemize}
        \item \textbf{Start with:} GRU (good balance)
        \item \textbf{If underfitting:} Switch to LSTM
        \item \textbf{If overfitting:} Try simpler RNN
        \item \textbf{For production:} LSTM (most proven)
    \end{itemize}
\end{frame}

% When RNNs still win
\begin{frame}[t]{When RNNs Beat Transformers (Yes, in 2024!)}
    \textbf{RNNs are not dead! They excel at:}
    
    \vspace{0.5em}
    \begin{enumerate}
        \item \textbf{Streaming/Online processing:}
        \begin{itemize}
            \item Process one token at a time
            \item O(1) memory per step
            \item Perfect for live transcription
        \end{itemize}
        
        \item \textbf{Time series with clear patterns:}
        \begin{itemize}
            \item Stock prices, weather data
            \item Energy consumption
            \item IoT sensor streams
        \end{itemize}
        
        \item \textbf{Resource-constrained devices:}
        \begin{itemize}
            \item Mobile phones (Gboard uses RNN-T)
            \item Embedded systems
            \item Edge AI applications
        \end{itemize}
        
        \item \textbf{Variable-length generation:}
        \begin{itemize}
            \item Music generation (MuseNet uses LSTM)
            \item Handwriting synthesis
            \item Speech synthesis
        \end{itemize}
    \end{enumerate}
    
    \colorbox{yellow!20}{
        Rule: Use RNNs when sequential processing is natural and efficient
    }
\end{frame}

%=====================================
% SUMMARY AND EXERCISES
%=====================================

\section{Summary and Exercises}

% Key takeaways
\begin{frame}[t]{Key Takeaways: Week 3}
    \textbf{What you've learned:}
    \begin{enumerate}
        \item \textbf{Sequential processing} is fundamental to language
        \item \textbf{RNNs} add memory through hidden states
        \item \textbf{Vanishing gradients} limit simple RNNs to ~20 steps
        \item \textbf{LSTM gates} create gradient highways for long-range dependencies
        \item \textbf{GRU} simplifies LSTM with similar performance
    \end{enumerate}
    
    \vspace{0.5em}
    \textbf{Key formulas to remember:}
    \begin{itemize}
        \item RNN: $h_t = \tanh(W_h h_{t-1} + W_x x_t)$
        \item LSTM: $c_t = f_t * c_{t-1} + i_t * \tilde{c}_t$
        \item Gradient vanishing: $(0.9)^T \rightarrow 0$
    \end{itemize}
    
    \vspace{0.5em}
    \textbf{Next week:} Sequence-to-Sequence models (Translation, Summarization)
\end{frame}

% References
\begin{frame}[t]{References and Further Reading}
    \textbf{Original Papers:}
    \begin{itemize}
        \item Rumelhart et al. (1986): "Learning representations by back-propagating errors"
        \item Hochreiter \& Schmidhuber (1997): "Long Short-Term Memory"
        \item Cho et al. (2014): "GRU - Gated Recurrent Unit"
    \end{itemize}
    
    \vspace{0.5em}
    \textbf{Recommended Tutorials:}
    \begin{itemize}
        \item Karpathy (2015): "The Unreasonable Effectiveness of RNNs"
        \item Olah (2015): "Understanding LSTM Networks" (colah.github.io)
        \item PyTorch RNN Tutorial: pytorch.org/tutorials
    \end{itemize}
    
    \vspace{0.5em}
    \textbf{Practice Resources:}
    \begin{itemize}
        \item Week 3 Lab: Character-level language model
        \item Assignment: Build LSTM from scratch
        \item Challenge: Implement attention mechanism
    \end{itemize}
\end{frame}

%=====================================
% APPENDIX A: MATHEMATICAL DEEP DIVE
%=====================================

\appendix
\section{Appendix A: Mathematical Deep Dive}

\begin{frame}[t]{Complete LSTM Mathematics}
    \textbf{Forward pass equations:}
    
    \small
    \begin{align}
        f_t &= \sigma(W_f \cdot [h_{t-1}, x_t] + b_f) & \text{Forget gate}\\
        i_t &= \sigma(W_i \cdot [h_{t-1}, x_t] + b_i) & \text{Input gate}\\
        \tilde{c}_t &= \tanh(W_c \cdot [h_{t-1}, x_t] + b_c) & \text{Candidate}\\
        c_t &= f_t \odot c_{t-1} + i_t \odot \tilde{c}_t & \text{Cell state}\\
        o_t &= \sigma(W_o \cdot [h_{t-1}, x_t] + b_o) & \text{Output gate}\\
        h_t &= o_t \odot \tanh(c_t) & \text{Hidden state}
    \end{align}
    
    \textbf{Gradient flow analysis:}
    
    $\frac{\partial c_t}{\partial c_{t-1}} = f_t$ (can be close to 1!)
    
    This allows gradient to flow unchanged if forget gate $\approx$ 1
\end{frame}

\begin{frame}[t]{Computational Complexity Analysis}
    \textbf{Time complexity per time step:}
    
    \begin{center}
    \begin{tabular}{|l|c|c|}
    \hline
    \textbf{Model} & \textbf{Operations} & \textbf{Memory} \\
    \hline
    RNN & $O(h^2 + hx)$ & $O(h)$ \\
    LSTM & $O(4(h^2 + hx))$ & $O(2h)$ \\
    GRU & $O(3(h^2 + hx))$ & $O(h)$ \\
    Transformer & $O(n^2d)$ & $O(n^2)$ \\
    \hline
    \end{tabular}
    \end{center}
    
    Where:
    \begin{itemize}
        \item $h$ = hidden size (typically 256-1024)
        \item $x$ = input size
        \item $n$ = sequence length
        \item $d$ = model dimension
    \end{itemize}
    
    \textbf{Key insight:} RNNs are O(n) in sequence length, Transformers are O($n^2$)
\end{frame}

%=====================================
% APPENDIX B: PRACTICAL APPLICATIONS
%=====================================

\section{Appendix B: Practical Applications}

\begin{frame}[t]{Real-World RNN Applications (2024)}
    \textbf{1. Speech Recognition:}
    \begin{itemize}
        \item Google's RNN-T on Pixel phones
        \item Apple's on-device Siri processing
        \item Works offline, low latency
    \end{itemize}
    
    \textbf{2. Time Series Forecasting:}
    \begin{itemize}
        \item Stock price prediction (LSTM dominant)
        \item Weather forecasting
        \item Traffic flow prediction
    \end{itemize}
    
    \textbf{3. Music Generation:}
    \begin{itemize}
        \item MuseNet (OpenAI) uses LSTM
        \item Magenta project (Google)
        \item Real-time composition
    \end{itemize}
    
    \textbf{4. Healthcare:}
    \begin{itemize}
        \item ECG anomaly detection
        \item Patient monitoring
        \item Drug discovery sequences
    \end{itemize}
\end{frame}

\end{document}