% Week 7: Advanced Transformers
% BSc Discovery-Based Presentation - Two-Tier Structure
% 20 concise main + 15 deep appendix focused on scaling laws

% Master Template for NLP Course
% Optimal Readability Layout Standard
% All presentations should include this template

\documentclass[8pt,aspectratio=169]{beamer}
\usetheme{Madrid}
\setbeamertemplate{navigation symbols}{}

% ====================================
% OPTIMAL READABILITY COLOR PALETTE
% ====================================
\definecolor{PureBlack}{HTML}{000000}      % Main text (21:1 contrast)
\definecolor{DeepBlue}{HTML}{003D7A}       % Primary accent (12.6:1 contrast)
\definecolor{DarkGray}{HTML}{4A4A4A}       % Secondary text (9.7:1 contrast)
\definecolor{LightGray}{HTML}{E5E5E5}      % Borders and grids
\definecolor{ChartBlue}{HTML}{0066CC}      % Chart primary
\definecolor{ChartOrange}{HTML}{FF8800}    % Chart secondary
\definecolor{ChartTeal}{HTML}{00A0A0}      % Chart tertiary
\definecolor{ChartPurple}{HTML}{8B4789}    % Chart quaternary
\definecolor{DarkGreen}{HTML}{228B22}      % Success/positive
\definecolor{DarkRed}{HTML}{CC0000}        % Warning/negative

% ====================================
% BEAMER COLOR CONFIGURATION
% ====================================
\setbeamercolor{structure}{fg=PureBlack}
\setbeamercolor{frametitle}{fg=PureBlack,bg=white}
\setbeamercolor{title}{fg=PureBlack,bg=white}
\setbeamercolor{subtitle}{fg=DarkGray}
\setbeamercolor{author}{fg=DarkGray}
\setbeamercolor{date}{fg=DarkGray}
\setbeamercolor{institute}{fg=DarkGray}

% Blocks - no backgrounds
\setbeamercolor{block title}{fg=PureBlack,bg=white}
\setbeamercolor{block body}{fg=PureBlack,bg=white}
\setbeamercolor{block title example}{fg=DarkGreen,bg=white}
\setbeamercolor{block body example}{fg=PureBlack,bg=white}
\setbeamercolor{block title alerted}{fg=DarkRed,bg=white}
\setbeamercolor{block body alerted}{fg=PureBlack,bg=white}

% Lists
\setbeamercolor{item}{fg=DeepBlue}
\setbeamercolor{subitem}{fg=ChartBlue}
\setbeamercolor{enumerate item}{fg=DeepBlue}
\setbeamercolor{enumerate subitem}{fg=ChartBlue}

% Text
\setbeamercolor{normal text}{fg=PureBlack,bg=white}
\setbeamercolor{alerted text}{fg=DarkRed}
\setbeamercolor{example text}{fg=DarkGreen}

% Footer
\setbeamercolor{footline}{fg=DarkGray,bg=white}
\setbeamercolor{page number in head/foot}{fg=DarkGray}

% ====================================
% FONT CONFIGURATION
% ====================================
\setbeamerfont{normal text}{size=\normalsize}
\setbeamerfont{frametitle}{size=\Large,series=\bfseries}
\setbeamerfont{title}{size=\huge,series=\bfseries}
\setbeamerfont{subtitle}{size=\large}
\setbeamerfont{author}{size=\normalsize}
\setbeamerfont{date}{size=\small}
\setbeamerfont{institute}{size=\small}

% ====================================
% REQUIRED PACKAGES
% ====================================
\usepackage{tikz}
\usepackage{amsmath}
\usepackage{amssymb}
\usepackage{booktabs}
\usepackage{graphicx}
\usepackage{array}
\usepackage{listings}
\usepackage{algorithm2e}
\usepackage{xcolor}
\usepackage{tabularx}
\usepackage{multirow}
\usepackage{subcaption}

% ====================================
% CUSTOM COMMANDS
% ====================================
% Text highlighting commands
\newcommand{\highlight}[1]{\textcolor{DeepBlue}{\textbf{#1}}}
\newcommand{\secondary}[1]{\textcolor{DarkGray}{#1}}
\newcommand{\success}[1]{\textcolor{DarkGreen}{#1}}
\newcommand{\warning}[1]{\textcolor{DarkRed}{#1}}
\newcommand{\data}[1]{\textcolor{ChartBlue}{#1}}
\newcommand{\dataalt}[1]{\textcolor{ChartOrange}{#1}}

% Mathematical notation
\newcommand{\given}{\mid}
\newcommand{\prob}[1]{P(#1)}
\newcommand{\argmax}{\operatorname*{argmax}}
\newcommand{\argmin}{\operatorname*{argmin}}
\newcommand{\softmax}{\operatorname{softmax}}

% Box commands for emphasis
\newcommand{\keypoint}[1]{%
  \begin{center}
  \fbox{\parbox{0.9\textwidth}{\centering\textbf{#1}}}
  \end{center}
}

\newcommand{\formula}[1]{%
  \begin{center}
  \colorbox{LightGray}{\parbox{0.8\textwidth}{\centering$\displaystyle #1$}}
  \end{center}
}

% ====================================
% LISTINGS CONFIGURATION
% ====================================
\lstset{
  basicstyle=\ttfamily\small,
  keywordstyle=\color{DeepBlue}\bfseries,
  commentstyle=\color{DarkGray}\itshape,
  stringstyle=\color{ChartOrange},
  numbers=left,
  numberstyle=\tiny\color{DarkGray},
  stepnumber=1,
  numbersep=5pt,
  backgroundcolor=\color{white},
  showspaces=false,
  showstringspaces=false,
  showtabs=false,
  frame=single,
  frameround=tttt,
  rulecolor=\color{LightGray},
  tabsize=2,
  captionpos=b,
  breaklines=true,
  breakatwhitespace=true,
  language=Python,
  escapeinside={(*@}{@*)},
  morekeywords={self, yield, assert, with, as}
}

% ====================================
% STANDARD SLIDE LAYOUTS
% ====================================

% Two-column layout with title
\newcommand{\twocolslide}[3]{%
  \begin{frame}{#1}
  \begin{columns}[T]
  \column{0.48\textwidth}
  #2
  \column{0.48\textwidth}
  #3
  \end{columns}
  \end{frame}
}

% Three-column layout
\newcommand{\threecolslide}[4]{%
  \begin{frame}{#1}
  \begin{columns}[T]
  \column{0.32\textwidth}
  #2
  \column{0.32\textwidth}
  #3
  \column{0.32\textwidth}
  #4
  \end{columns}
  \end{frame}
}

% Chart slide with caption
\newcommand{\chartslide}[3]{%
  \begin{frame}{#1}
  \begin{center}
  \includegraphics[width=#2\textwidth]{#3}
  \end{center}
  \end{frame}
}

% Full chart slide - optimized to 0.85 for proper margins
\newcommand{\fullchartslide}[2]{%
  \begin{frame}{#1}
  \begin{center}
  \includegraphics[width=0.85\textwidth]{#2}
  \end{center}
  \end{frame}
}

% Code slide
\newcommand{\codeslide}[3]{%
  \begin{frame}[fragile]{#1}
  \begin{lstlisting}[language=#2]
#3
  \end{lstlisting}
  \end{frame}
}

% Concept slide with figure
\newcommand{\conceptslide}[3]{%
  \begin{frame}{#1}
  \begin{columns}[T]
  \column{0.6\textwidth}
  #2
  \column{0.35\textwidth}
  \begin{center}
  \includegraphics[width=0.95\textwidth]{#3}
  \end{center}
  \end{columns}
  \end{frame}
}

% Table slide
\newcommand{\tableslide}[2]{%
  \begin{frame}{#1}
  \begin{center}
  \Large
  \renewcommand{\arraystretch}{1.5}
  #2
  \end{center}
  \end{frame}
}

% Summary slide
\newcommand{\summaryslide}[2]{%
  \begin{frame}{#1}
  \begin{center}
  \Large
  #2
  \end{center}
  \vfill
  \begin{center}
  \keypoint{Key Takeaway}
  \end{center}
  \end{frame}
}

% ====================================
% REMOVE DECORATIONS
% ====================================
\setbeamertemplate{blocks}[default]
\setbeamertemplate{title page}[default][colsep=-4bp,rounded=false]
\setbeamertemplate{itemize items}[circle]
\setbeamertemplate{enumerate items}[default]
\setbeamertemplate{section in toc}[sections numbered]
\setbeamertemplate{subsection in toc}[subsections numbered]

% ====================================
% PAGE NUMBERING
% ====================================
\setbeamertemplate{footline}{
  \leavevmode%
  \hbox{%
  \begin{beamercolorbox}[wd=.333333\paperwidth,ht=2.25ex,dp=1ex,center]{author in head/foot}%
    \usebeamerfont{author in head/foot}\secondary{\insertshortauthor}
  \end{beamercolorbox}%
  \begin{beamercolorbox}[wd=.333333\paperwidth,ht=2.25ex,dp=1ex,center]{title in head/foot}%
    \usebeamerfont{title in head/foot}\secondary{\insertshorttitle}
  \end{beamercolorbox}%
  \begin{beamercolorbox}[wd=.333333\paperwidth,ht=2.25ex,dp=1ex,right]{date in head/foot}%
    \usebeamerfont{date in head/foot}\secondary{\insertshortdate{}\hspace*{2em}
    \insertframenumber{} / \inserttotalframenumber\hspace*{2ex}}
  \end{beamercolorbox}}%
  \vskip0pt%
}

% ====================================
% TABLE OF CONTENTS STYLE
% ====================================
\setbeamertemplate{section in toc}{%
  \leavevmode\leftskip=1.5em%
  \llap{%
    \usebeamerfont{section in toc}%
    \usebeamercolor[fg]{section in toc}%
    \inserttocsectionnumber.%
  }%
  \usebeamerfont{section in toc}%
  \usebeamercolor[fg]{section in toc}%
  \inserttocsection\par%
}

\setbeamertemplate{subsection in toc}{%
  \leavevmode\leftskip=3em%
  \llap{%
    \usebeamerfont{subsection in toc}%
    \usebeamercolor[fg]{subsection in toc}%
    \inserttocsectionnumber.\inserttocsubsectionnumber%
  }%
  \usebeamerfont{subsection in toc}%
  \usebeamercolor[fg]{subsection in toc}%
  \inserttocsubsection\par%
}

% ====================================
% END OF MASTER TEMPLATE
% ====================================

% Define bottomnote
\newcommand{\bottomnote}[1]{%
    \vspace{0.2cm}
    \begin{center}
    \footnotesize\secondary{#1}
    \end{center}
}

\title{Advanced Transformers}
\subtitle{\secondary{Week 7 - The Predictable Path to Intelligence}}
\author{NLP Course 2025}
\date{October 27, 2025}

\begin{document}

% ===== MAIN PRESENTATION (20 SLIDES) =====

% Slide 1: Title
\begin{frame}
\titlepage
\vfill
\begin{center}
\secondary{\footnotesize Two-Tier BSc Discovery Presentation}
\end{center}
\end{frame}

% ===== OPENING: SCALING LAWS PARADOX (Slides 2-3) =====

% Slide 2: Discovery Hook
\begin{frame}[t]{The Scaling Laws Paradox}
\vspace{-0.3cm}
\begin{center}
\includegraphics[width=0.7\textwidth]{../figures/scaling_laws_loglog_bsc.pdf}
\end{center}

\begin{center}
\Large
\textbf{Why is AI progress so predictable?}
\end{center}

\vspace{3mm}
\begin{center}
\textbf{Key Insight}: Performance follows power laws - we can predict the future
\end{center}

\vspace{0.2cm}
\bottomnote{Bigger models are better in predictable, measurable ways}
\end{frame}

% Slide 3: Three Paths Forward
\begin{frame}[t]{Three Paths to Better Transformers}
\small
\begin{columns}[T]
\column{0.32\textwidth}
\raggedright
\textbf{Path 1: Bigger}

\vspace{3mm}
GPT-3 (2020)

\begin{itemize}
\item 175B parameters
\item $1000\times$ GPT-1
\item Emergent abilities
\item \$4.6M training cost
\end{itemize}

\vspace{3mm}
\textbf{Bet}: Scale alone works

\column{0.32\textwidth}
\raggedright
\textbf{Path 2: Smarter}

\vspace{3mm}
Mixture of Experts

\begin{itemize}
\item 1.6T parameters
\item Only 10B active
\item Sparse activation
\item Same cost as 100B
\end{itemize}

\vspace{3mm}
\textbf{Bet}: Efficiency matters

\column{0.32\textwidth}
\raggedright
\textbf{Path 3: Efficient}

\vspace{3mm}
Reformer, Linformer

\begin{itemize}
\item Reduce $O(n^2)$
\item Long sequences
\item Lower memory
\item Practical deployment
\end{itemize}

\vspace{3mm}
\textbf{Bet}: Algorithms matter

\end{columns}

\vspace{5mm}
\begin{center}
\textbf{All three work!} Next 17 slides explore each path
\end{center}

\vspace{0.2cm}
\bottomnote{Different paths for different constraints - no single winner}
\end{frame}

% ===== SCALING LAWS (Slides 4-8) =====

% Slide 4: Visual - Kaplan Scaling Laws
\begin{frame}[t]{The Kaplan Scaling Laws (2020)}
\vspace{-0.3cm}
\begin{center}
\includegraphics[width=0.65\textwidth]{../figures/kaplan_scaling_laws_bsc.pdf}
\end{center}

\begin{center}
\textbf{Key Insight}: Loss decreases as power law with parameters, data, compute
\end{center}

\vspace{0.2cm}
\bottomnote{Three independent power laws - all perfectly smooth}
\end{frame}

% Slide 5: Detail - Power Law Mathematics
\begin{frame}[t]{Scaling Laws: The Mathematics}
\small
\begin{columns}[T]
\column{0.48\textwidth}
\raggedright
\textbf{The Three Power Laws}:

$$L(N) = (N_c / N)^{\alpha_N}$$
$$L(D) = (D_c / D)^{\alpha_D}$$
$$L(C) = (C_c / C)^{\alpha_C}$$

where:
\begin{itemize}
\item $N$: Number of parameters
\item $D$: Dataset size (tokens)
\item $C$: Compute (FLOPs)
\item $\alpha_N \approx 0.076$
\item $\alpha_D \approx 0.095$
\item $\alpha_C \approx 0.050$
\end{itemize}

\vspace{3mm}
\textbf{Key Discovery}:

Smooth, predictable improvement

\column{0.48\textwidth}
\raggedright
\textbf{Worked Example}:

GPT-3: $N = 175$B, $L = 2.1$ nats

\vspace{3mm}
Predict GPT-4 ($N = 500$B hypothetical):

\begin{align*}
L(500B) &= L(175B) \times (175B / 500B)^{0.076} \\
&= 2.1 \times (0.35)^{0.076} \\
&= 2.1 \times 0.92 \\
&= 1.93 \text{ nats}
\end{align*}

\vspace{3mm}
\textbf{Interpretation}:

8\% loss reduction from 3x parameters

Matches observed scaling!

\end{columns}

\vspace{0.2cm}
\bottomnote{These laws held from 1M to 175B parameters - remarkable consistency}
\end{frame}

% Slide 6: Visual - Emergent Abilities
\begin{frame}[t]{Emergent Abilities at Scale}
\vspace{-0.3cm}
\begin{center}
\includegraphics[width=0.7\textwidth]{../figures/emergent_abilities_bsc.pdf}
\end{center}

\begin{center}
\textbf{Key Insight}: New capabilities appear suddenly at threshold model sizes
\end{center}

\vspace{0.2cm}
\bottomnote{Not gradual improvement - discontinuous jumps in ability}
\end{frame}

% Slide 7: Detail - Compute-Optimal Training
\begin{frame}[t]{Chinchilla: Most Models Are Undertrained}
\small
\begin{columns}[T]
\column{0.48\textwidth}
\raggedright
\textbf{The Discovery (2022)}:

GPT-3: 175B params, 300B tokens

\vspace{3mm}
Chinchilla: 70B params, 1.4T tokens

\vspace{3mm}
\textbf{Result}: Chinchilla beats GPT-3!

\vspace{5mm}
\textbf{The Rule}:

$$\text{Optimal tokens} \approx 20 \times \text{parameters}$$

\vspace{3mm}
GPT-3 should have seen:

$175B \times 20 = 3.5T$ tokens

Actually saw: 300B (9\% of optimal!)

\column{0.48\textwidth}
\raggedright
\textbf{Why This Matters}:

\begin{itemize}
\item Most large models undertrained
\item Training longer is cheaper than bigger model
\item Compute allocation matters
\end{itemize}

\vspace{3mm}
\textbf{Practical Impact}:

\begin{itemize}
\item LLaMA-2 (70B): Trained on 2T tokens
\item Follows Chinchilla scaling
\item Outperforms GPT-3 with 2.5x fewer params
\item Cheaper to run, same quality
\end{itemize}

\vspace{3mm}
\textbf{Lesson}:

Don't just make models bigger - train them longer!

\end{columns}

\vspace{0.2cm}
\bottomnote{Chinchilla changed how we think about model training}
\end{frame}

% Slide 8: Worked Example - Compute-Optimal Sizing
\begin{frame}[t]{Worked Example: Compute-Optimal Model}
\small
\textbf{Given}: Fixed compute budget $C = 10^{23}$ FLOPs

\vspace{3mm}
\textbf{Question}: How many parameters $N$ and tokens $D$ to use?

\vspace{5mm}
\textbf{Chinchilla Formula}:

For optimal allocation:

$$N_{opt} \approx 0.73 \times C^{0.37}$$
$$D_{opt} \approx 1.45 \times C^{0.37}$$

\vspace{5mm}
\textbf{Calculate}:

\begin{align*}
N_{opt} &= 0.73 \times (10^{23})^{0.37} = 0.73 \times 1.86 \times 10^{8} \\
&\approx 136M \text{ parameters}
\end{align*}

\begin{align*}
D_{opt} &= 1.45 \times (10^{23})^{0.37} \approx 270M \text{ tokens}
\end{align*}

\vspace{3mm}
\textbf{Verification}: $D_{opt} / N_{opt} = 270M / 136M \approx 2$ (roughly $20\times$ rule)

\vspace{0.2cm}
\bottomnote{Optimal compute allocation: Balance parameters and training data}
\end{frame}

% ===== GPT-3 SECTION (Slides 9-13) =====

% Slide 9: Visual - GPT-3 Scale
\begin{frame}[t]{GPT-3: The Scale Breakthrough}
\vspace{-0.3cm}
\begin{center}
\includegraphics[width=0.7\textwidth]{../figures/gpt3_scale_bsc.pdf}
\end{center}

\begin{center}
\textbf{Key Insight}: 175 billion parameters - 1000× GPT-1
\end{center}

\vspace{0.2cm}
\bottomnote{Training: 300B tokens, 3640 petaflop-days, \$4.6M cost}
\end{frame}

% Slide 10: Detail - GPT-3 Architecture
\begin{frame}[t]{GPT-3 Architecture Details}
\small
\begin{columns}[T]
\column{0.48\textwidth}
\raggedright
\textbf{Specifications}:

\begin{itemize}
\item \textbf{Layers}: 96 decoder layers
\item \textbf{Hidden size}: 12,288 dimensions
\item \textbf{Attention heads}: 96 (128 dim each)
\item \textbf{Context window}: 2048 tokens
\item \textbf{Parameters}: 175 billion
\item \textbf{Vocabulary}: 50,257 (BPE)
\end{itemize}

\vspace{3mm}
\textbf{Why So Big}:

Scale enables emergent abilities

\column{0.48\textwidth}
\raggedright
\textbf{Training}:

\begin{itemize}
\item \textbf{Dataset}: Common Crawl (570GB)
\item \textbf{Tokens}: 300 billion
\item \textbf{Batch size}: 3.2M tokens
\item \textbf{Hardware}: 10,000+ GPUs
\item \textbf{Time}: Several months
\item \textbf{Cost}: \$4.6 million
\end{itemize}

\vspace{3mm}
\textbf{Notable}:

8 different model sizes tested (125M to 175B)

All follow same scaling law!

\end{columns}

\vspace{0.2cm}
\bottomnote{Largest dense transformer ever trained (as of 2020)}
\end{frame}

% Slide 11: Visual - Few-Shot Emergence
\begin{frame}[t]{Few-Shot Learning: Emergent at Scale}
\vspace{-0.3cm}
\begin{center}
\includegraphics[width=0.7\textwidth]{../figures/fewshot_emergence_bsc.pdf}
\end{center}

\begin{center}
\textbf{Key Insight}: Capabilities emerge suddenly above threshold sizes
\end{center}

\vspace{0.2cm}
\bottomnote{Few-shot learning doesn't work below ~13B parameters - then suddenly does}
\end{frame}

% Slide 12: Detail - Cost-Performance Tradeoffs
\begin{frame}[t]{The Cost of Scale}
\small
\begin{columns}[T]
\column{0.48\textwidth}
\raggedright
\textbf{Training Cost Escalation}:

\begin{center}
\begin{tabular}{lr}
Model & Cost \\
\hline
GPT-1 (117M) & \$50K \\
GPT-2 (1.5B) & \$500K \\
GPT-3 (175B) & \$4.6M \\
GPT-4 (est 1.7T) & \$50M+ \\
\end{tabular}
\end{center}

\vspace{3mm}
100× parameters $\approx$ 1000× cost

\vspace{5mm}
\textbf{Diminishing Returns}:

Each 10× scale:
\begin{itemize}
\item Cost: $10\times$ increase
\item Loss: $0.076$ decrease (8\%)
\item Linear cost, log gains
\end{itemize}

\column{0.48\textwidth}
\raggedright
\textbf{Why Continue Scaling}:

\begin{itemize}
\item Emergent abilities worth the cost
\item Few-shot learning transforms UX
\item Reasoning capabilities
\item Multimodal integration (GPT-4)
\end{itemize}

\vspace{3mm}
\textbf{Alternative Strategies}:

\begin{itemize}
\item Chinchilla: Train longer, not bigger
\item MoE: Sparse activation
\item Efficient: Reduce $O(n^2)$
\item Distillation: Compress after training
\end{itemize}

\vspace{3mm}
\textbf{The Tension}:

Better models vs affordable training

\end{columns}

\vspace{0.2cm}
\bottomnote{Scaling works but is expensive - motivates smarter approaches}
\end{frame}

% Slide 13: Worked Example - Scaling Prediction
\begin{frame}[t]{Worked Example: Predicting Future Performance}
\small
\textbf{Given}: GPT-3 at 175B params has loss $L = 2.1$ nats

\vspace{3mm}
\textbf{Question}: What loss would 500B parameter model achieve?

\vspace{5mm}
\textbf{Using Scaling Law}: $L(N) = L_0 \times (N_0 / N)^{\alpha}$ where $\alpha = 0.076$

\vspace{3mm}
\textbf{Step 1}: Set up equation

$$L(500B) = 2.1 \times (175B / 500B)^{0.076}$$

\vspace{3mm}
\textbf{Step 2}: Calculate ratio

$$\frac{175}{500} = 0.35$$

\vspace{3mm}
\textbf{Step 3}: Apply exponent

$$0.35^{0.076} \approx 0.92$$

\vspace{3mm}
\textbf{Step 4}: Final loss

$$L(500B) = 2.1 \times 0.92 = 1.93 \text{ nats}$$

\vspace{5mm}
\textbf{Interpretation}: 3× parameters → 8\% loss reduction (diminishing returns!)

\vspace{0.2cm}
\bottomnote{We can predict GPT-5 performance before training it}
\end{frame}

% ===== MoE SECTION (Slides 14-17) =====

% Slide 14: Visual - MoE Concept
\begin{frame}[t]{Mixture of Experts: Sparse Activation}
\vspace{-0.3cm}
\begin{center}
\includegraphics[width=0.7\textwidth]{../figures/moe_architecture_bsc.pdf}
\end{center}

\begin{center}
\textbf{Key Insight}: Train 1T parameters, activate only 10B per example
\end{center}

\vspace{0.2cm}
\bottomnote{Sparse models: Capacity of large, cost of small}
\end{frame}

% Slide 15: Detail - MoE Architecture
\begin{frame}[t]{Mixture of Experts: How It Works}
\small
\begin{columns}[T]
\column{0.48\textwidth}
\raggedright
\textbf{Components}:

\begin{itemize}
\item \textbf{Router}: Gating network (small NN)
\item \textbf{Experts}: $E$ specialist FFNs
\item \textbf{Top-k selection}: Use best $k$ experts
\end{itemize}

\vspace{3mm}
\textbf{Forward Pass}:

\begin{enumerate}
\item Router computes scores for all experts
\item Select top-$k$ (typically $k=2$)
\item Activate only selected experts
\item Weighted sum of expert outputs
\end{enumerate}

\vspace{3mm}
\textbf{Example}:

\begin{itemize}
\item 128 experts total
\item Top-2 selection
\item Activate: 2/128 = 1.6\%
\end{itemize}

\column{0.48\textwidth}
\raggedright
\textbf{Gating Function}:

$$G(x) = \softmax(\text{KeepTopK}(W_g \cdot x, k))$$

\vspace{3mm}
\textbf{Output}:

$$y = \sum_{i \in \text{Top-k}} G(x)_i \cdot E_i(x)$$

\vspace{3mm}
\textbf{Load Balancing}:

Auxiliary loss ensures experts used equally

$$L_{aux} = \alpha \times \sum_{i=1}^{E} f_i \times P_i$$

where $f_i$ = fraction routed to expert $i$

\vspace{3mm}
\textbf{Benefits}:

\begin{itemize}
\item $100\times$ parameters at $2\times$ cost
\item Experts specialize (syntax, semantics, etc.)
\item Conditional computation
\end{itemize}

\end{columns}

\vspace{0.2cm}
\bottomnote{MoE is how to scale beyond dense models}
\end{frame}

% Slide 16: Worked Example - MoE Routing
\begin{frame}[t]{Worked Example: MoE Router Computation}
\small
\textbf{Given}: Input token $x$, 8 experts, top-2 selection

\vspace{3mm}
\textbf{Step 1}: Router computes logits

$$\text{logits} = W_g \cdot x = [2.1, 0.3, 1.8, 0.1, 3.2, 0.5, 1.1, 0.8]$$

\vspace{3mm}
\textbf{Step 2}: Select top-2

Top-2 experts: Expert 5 (3.2) and Expert 1 (2.1)

\vspace{3mm}
\textbf{Step 3}: Softmax over top-2 only

$$G_5 = \frac{\exp(3.2)}{\exp(3.2) + \exp(2.1)} = \frac{24.5}{32.6} = 0.75$$

$$G_1 = \frac{\exp(2.1)}{\exp(3.2) + \exp(2.1)} = \frac{8.17}{32.6} = 0.25$$

\vspace{3mm}
\textbf{Step 4}: Weighted sum

$$y = 0.75 \times E_5(x) + 0.25 \times E_1(x)$$

\vspace{5mm}
\textbf{Sparsity}: Only 2/8 experts activated = 75\% reduction in computation!

\vspace{0.2cm}
\bottomnote{Router learns which experts are relevant for which inputs}
\end{frame}

% Slide 17: Visual - Switch Transformer Results
\begin{frame}[t]{Switch Transformer: 1.6 Trillion Parameters}
\vspace{-0.3cm}
\begin{center}
\includegraphics[width=0.65\textwidth]{../figures/switch_transformer_results_bsc.pdf}
\end{center}

\begin{center}
\textbf{Key Insight}: MoE (1.6T params) beats dense (100B) at same compute cost
\end{center}

\vspace{0.2cm}
\bottomnote{Google's Switch Transformer (2021) - largest model at the time}
\end{frame}

% ===== EFFICIENT TRANSFORMERS (Slides 18-20) =====

% Slide 18: Visual - Attention Complexity Crisis
\begin{frame}[t]{The Attention Complexity Problem}
\vspace{-0.3cm}
\begin{center}
\includegraphics[width=0.7\textwidth]{../figures/attention_complexity_bsc.pdf}
\end{center}

\begin{center}
\textbf{Key Insight}: Standard attention is $O(n^2)$ - memory explodes for long sequences
\end{center}

\vspace{0.2cm}
\bottomnote{$n=512$ fine, $n=8192$ impossible on most GPUs}
\end{frame}

% Slide 19: Detail - Efficient Attention Variants
\begin{frame}[t]{Efficient Transformers: Reducing Complexity}
\small
\begin{columns}[T]
\column{0.48\textwidth}
\raggedright
\textbf{Reformer} (LSH Attention):

\begin{itemize}
\item Locality-Sensitive Hashing
\item Group similar queries/keys
\item Attend only within groups
\item Complexity: $O(n \log n)$
\end{itemize}

\vspace{3mm}
\textbf{Linformer}:

\begin{itemize}
\item Low-rank projection
\item Project $n \times n$ to $n \times k$
\item $k \ll n$ (e.g., 256)
\item Complexity: $O(n)$
\end{itemize}

\column{0.48\textwidth}
\raggedright
\textbf{Linear Attention}:

\begin{itemize}
\item Kernel trick
\item Reorder operations
\item Never compute $n \times n$ matrix
\item Complexity: $O(n)$
\end{itemize}

\vspace{3mm}
\textbf{When to Use}:

\begin{itemize}
\item \textbf{Long documents}: $n > 4096$
\item \textbf{Limited memory}: Consumer GPUs
\item \textbf{Real-time}: Latency critical
\item \textbf{Edge deployment}: Mobile, IoT
\end{itemize}

\end{columns}

\vspace{0.2cm}
\bottomnote{Multiple approaches to same problem - choose based on use case}
\end{frame}

% Slide 20: Worked Example - Complexity Savings
\begin{frame}[t]{Worked Example: Memory Savings from Efficient Attention}
\small
\textbf{Given}: Sequence length $n = 4096$, hidden dim $d = 512$

\vspace{5mm}
\textbf{Standard Attention Memory}:

Attention matrix: $n \times n = 4096 \times 4096 = 16.8M$ floats

Memory: $16.8M \times 4$ bytes $= 67$ MB per attention head

With 16 heads: $67 \times 16 = 1$ GB just for attention!

\vspace{5mm}
\textbf{Linformer Memory} ($k = 256$):

Projected matrix: $n \times k = 4096 \times 256 = 1M$ floats

Memory: $1M \times 4$ bytes $= 4$ MB per head

With 16 heads: $4 \times 16 = 64$ MB

\vspace{5mm}
\textbf{Savings}: $\frac{1GB}{64MB} = 16\times$ memory reduction!

\vspace{5mm}
\textbf{Impact}: Can process 4x longer sequences on same hardware

\vspace{0.2cm}
\bottomnote{Efficient attention enables long-context applications}
\end{frame}

% ===== TECHNICAL APPENDIX (15 SLIDES) =====

\begin{frame}[t]{}
\begin{center}
\Huge\textbf{Technical Appendix}

\vspace{5mm}
\Large\secondary{Deep Dive: GPT-3, MoE, Efficient Transformers}
\end{center}
\end{frame}

% ===== GPT-3 TECHNICAL (A1-A5) =====

% Slide A1: GPT-3 Complete Specifications
\begin{frame}[t]{Appendix A1: GPT-3 Complete Specifications}
\small
\begin{center}
\begin{tabular}{lcccc}
\toprule
\textbf{Component} & \textbf{Small} & \textbf{Medium} & \textbf{Large} & \textbf{175B} \\
\midrule
Parameters & 125M & 350M & 1.3B & 175B \\
Layers & 12 & 24 & 24 & 96 \\
Hidden Size & 768 & 1024 & 2048 & 12,288 \\
Heads & 12 & 16 & 16 & 96 \\
Head Dimension & 64 & 64 & 128 & 128 \\
Context & 2048 & 2048 & 2048 & 2048 \\
Batch Size & 0.5M & 0.5M & 1M & 3.2M \\
Learning Rate & 6e-4 & 3e-4 & 2.5e-4 & 1.2e-4 \\
\midrule
\textbf{Performance} & & & & \\
LAMBADA (acc) & 42.7 & 54.3 & 63.6 & 76.2 \\
HellaSwag (acc) & 43.6 & 54.7 & 67.4 & 78.9 \\
\bottomrule
\end{tabular}
\end{center}

\vspace{3mm}
\textbf{Key Pattern}: Consistent scaling across all metrics

\vspace{0.2cm}
\bottomnote{Every metric improves smoothly with size}
\end{frame}

% Slide A2: GPT-3 Training Infrastructure
\begin{frame}[t]{Appendix A2: GPT-3 Training Infrastructure}
\small
\textbf{Hardware}:

\begin{itemize}
\item 10,000+ NVIDIA V100 GPUs (32GB each)
\item 285,000 CPU cores
\item 400 Gbps network interconnect
\item Custom Microsoft Azure infrastructure
\end{itemize}

\vspace{5mm}
\textbf{Training Details}:

\begin{itemize}
\item Distributed across thousands of nodes
\item Model parallelism: Split model across GPUs
\item Data parallelism: Different batches per GPU
\item Pipeline parallelism: Layer-wise distribution
\item Mixed precision training (FP16/FP32)
\end{itemize}

\vspace{5mm}
\textbf{Challenges}:

\begin{itemize}
\item Synchronization overhead
\item Gradient accumulation across devices
\item Fault tolerance (GPU failures during training)
\item Checkpointing (model state = 350GB)
\end{itemize}

\vspace{3mm}
\textbf{Training Time}: Several months wall-clock time

\vspace{0.2cm}
\bottomnote{Training GPT-3 requires infrastructure beyond most organizations}
\end{frame}

% Slide A3: Few-Shot Prompting Techniques
\begin{frame}[t]{Appendix A3: Few-Shot Prompting Best Practices}
\small
\textbf{Prompt Engineering for GPT-3}:

\vspace{3mm}
\textbf{Zero-Shot}:

\texttt{Translate to French: Hello} → Bonjour

\vspace{3mm}
\textbf{One-Shot}:

\texttt{English: Hello, French: Bonjour}

\texttt{English: Goodbye, French:} → Au revoir

\vspace{3mm}
\textbf{Few-Shot (Optimal)}:

\texttt{Translate English to French:}

\texttt{English: Hello / French: Bonjour}

\texttt{English: Goodbye / French: Au revoir}

\texttt{English: Thank you / French: Merci}

\texttt{English: Please / French:} → S'il vous plaît

\vspace{5mm}
\textbf{Best Practices}:

\begin{itemize}
\item Use 3-10 examples (more doesn't always help)
\item Examples should be diverse
\item Format consistency critical
\item Order matters (recency bias)
\item Few-shot $>$ fine-tuning for small datasets (< 100 examples)
\end{itemize}

\vspace{0.2cm}
\bottomnote{Prompting is an art - small changes yield large effects}
\end{frame}

% Slide A4: GPT-3 API Usage
\begin{frame}[t]{Appendix A4: GPT-3 API and Pricing}
\small
\textbf{API Access}:

\begin{itemize}
\item No model download (too large)
\item API calls only (OpenAI servers)
\item Multiple model sizes available
\end{itemize}

\vspace{5mm}
\textbf{Pricing (2024)}:

\begin{center}
\begin{tabular}{lrr}
Model & Input (per 1M tokens) & Output \\
\hline
GPT-3.5-turbo & \$0.50 & \$1.50 \\
GPT-4-turbo & \$10.00 & \$30.00 \\
GPT-4 (8K) & \$30.00 & \$60.00 \\
\end{tabular}
\end{center}

\vspace{5mm}
\textbf{Rate Limits}:

\begin{itemize}
\item Free tier: 3 requests/minute
\item Paid tier: 3500 requests/minute
\item Tokens per minute: 90K-2M depending on tier
\end{itemize}

\vspace{3mm}
\textbf{Business Model}: Amortize \$4.6M training cost across millions of users

\vspace{0.2cm}
\bottomnote{API pricing reflects compute cost and value delivered}
\end{frame}

% Slide A5: GPT-4 Evolution
\begin{frame}[t]{Appendix A5: GPT-3 to GPT-4 Evolution}
\small
\textbf{GPT-4 Improvements (2023)}:

\begin{itemize}
\item \textbf{Multimodal}: Images + text input
\item \textbf{Larger context}: 32K tokens (16× GPT-3)
\item \textbf{Better reasoning}: Chain-of-thought, mathematical
\item \textbf{RLHF}: Reinforcement learning from human feedback
\item \textbf{Parameters}: Rumored 1.7T (unconfirmed)
\end{itemize}

\vspace{5mm}
\textbf{Performance Gains}:

\begin{itemize}
\item Bar exam: 10th percentile → 90th percentile
\item Coding: HumanEval 48\% → 67\%
\item MMLU (general knowledge): 70\% → 86\%
\item Reduced hallucinations by 40\%
\end{itemize}

\vspace{5mm}
\textbf{Architecture Speculation}:

\begin{itemize}
\item Likely MoE (not confirmed)
\item Multiple expert models combined
\item Compute-optimal training (Chinchilla-informed)
\end{itemize}

\vspace{0.2cm}
\bottomnote{GPT-4 shows continued scaling + better training}
\end{frame}

% ===== MoE DEEP DIVE (A6-A10) =====

% Slide A6: Router Network Mathematics
\begin{frame}[t]{Appendix A6: MoE Router Network Mathematics}
\small
\textbf{Router Architecture}:

$$h = W_g x \in \mathbb{R}^E$$

where $E$ = number of experts

\vspace{5mm}
\textbf{Top-k Gating}:

$$G(x) = \softmax(\text{KeepTopK}(h, k))$$

KeepTopK sets all but top-$k$ values to $-\infty$ before softmax

\vspace{5mm}
\textbf{Sparse Gating Properties}:

\begin{itemize}
\item Only $k$ experts get non-zero weight
\item Weights sum to 1 (softmax)
\item Differentiable (can train with backprop)
\item Gradient flows only through selected experts
\end{itemize}

\vspace{5mm}
\textbf{Sparsity Benefit}:

$k=2$, $E=128$: Activate $2/128 = 1.6\%$ of parameters

$100\times$ parameters at $2\times$ compute!

\vspace{0.2cm}
\bottomnote{Sparsity is key - conditional computation based on input}
\end{frame}

% Slide A7: Load Balancing Algorithms
\begin{frame}[t]{Appendix A7: Load Balancing in MoE}
\small
\textbf{Problem}: Some experts get all traffic, others unused

\vspace{5mm}
\textbf{Auxiliary Loss}:

$$L_{aux} = \alpha \times \text{CV}(f_1, f_2, ..., f_E)^2$$

where $\text{CV}$ = coefficient of variation, $f_i$ = fraction routed to expert $i$

\vspace{3mm}
Penalizes imbalanced routing

\vspace{5mm}
\textbf{Capacity Factor}:

Limit tokens per expert:

$$\text{capacity}_i = \frac{\text{total tokens}}{E} \times \text{capacity factor}$$

Typical capacity factor = 1.25

\vspace{5mm}
\textbf{Expert Choice Routing} (alternative):

Experts choose tokens instead of tokens choosing experts

Better load balance, more stable training

\vspace{0.2cm}
\bottomnote{Load balancing critical for effective expert utilization}
\end{frame}

% Slide A8: Switch Transformer Architecture
\begin{frame}[t]{Appendix A8: Switch Transformer Details}
\small
\textbf{Google's Switch Transformer (2021)}:

\begin{itemize}
\item 1.6 trillion parameters (largest at time)
\item Top-1 routing (simplest form of MoE)
\item 2048 experts per layer
\item Trained on C4 dataset (750GB)
\end{itemize}

\vspace{5mm}
\textbf{Key Innovation}: Simplified MoE

\begin{itemize}
\item Top-1 instead of top-2 (simpler)
\item Expert capacity (limit tokens per expert)
\item Smaller routers (reduced overhead)
\item Better scaling than previous MoE
\end{itemize}

\vspace{5mm}
\textbf{Results}:

\begin{itemize}
\item 4× faster pre-training than T5-XXL (same quality)
\item 7× faster fine-tuning
\item Outperforms dense models at matched compute
\end{itemize}

\vspace{0.2cm}
\bottomnote{Largest model trained demonstrates MoE viability}
\end{frame}

% Slide A9: MoE Training Dynamics
\begin{frame}[t]{Appendix A9: Training Dynamics and Expert Specialization}
\small
\textbf{What Do Experts Learn}?

Empirical findings:

\begin{itemize}
\item Some experts specialize by syntax
\item Some by semantics
\item Some by domain (code, math, language)
\item Specialization emerges during training
\end{itemize}

\vspace{5mm}
\textbf{Training Challenges}:

\begin{itemize}
\item \textbf{Mode collapse}: All traffic to few experts
\item \textbf{Instability}: Router can oscillate
\item \textbf{Expert imbalance}: Uneven utilization
\item \textbf{Gradient noise}: Discrete routing not smooth
\end{itemize}

\vspace{5mm}
\textbf{Solutions}:

\begin{itemize}
\item Strong auxiliary losses
\item Careful initialization
\item Dropout on router
\item Expert capacity constraints
\end{itemize}

\vspace{0.2cm}
\bottomnote{MoE training trickier than dense - but rewards are huge}
\end{frame}

% Slide A10: MoE vs Dense Comparison
\begin{frame}[t]{Appendix A10: MoE vs Dense - Complete Comparison}
\small
\begin{center}
\begin{tabular}{lcc}
\toprule
\textbf{Aspect} & \textbf{Dense (GPT-3)} & \textbf{MoE (Switch)} \\
\midrule
Active Parameters & 175B & 10B \\
Total Parameters & 175B & 1600B \\
Sparsity & 0\% & 99.4\% \\
Memory (inference) & 350GB & 20GB \\
Training Time & Baseline & 4× faster \\
Fine-tuning Time & Baseline & 7× faster \\
Quality (matched compute) & Good & Better \\
Implementation & Simpler & Complex \\
Deployment & Standard GPUs & Needs coordination \\
\bottomrule
\end{tabular}
\end{center}

\vspace{5mm}
\textbf{When to Use MoE}:

\begin{itemize}
\item Want massive capacity
\item Have coordination infrastructure
\item Training cost constrained
\item Serving many requests (can batch)
\end{itemize}

\vspace{3mm}
\textbf{When to Use Dense}:

\begin{itemize}
\item Simpler deployment
\item Low latency critical
\item Small-scale serving
\end{itemize}

\vspace{0.2cm}
\bottomnote{MoE trades implementation complexity for computational efficiency}
\end{frame}

% ===== EFFICIENT TRANSFORMERS (A11-A15) =====

% Slide A11: Reformer - LSH Attention
\begin{frame}[t]{Appendix A11: Reformer and LSH Attention}
\small
\textbf{Locality-Sensitive Hashing (LSH)}:

\begin{itemize}
\item Hash queries and keys to buckets
\item Similar vectors $\rightarrow$ same bucket (high probability)
\item Attend only within bucket
\end{itemize}

\vspace{5mm}
\textbf{LSH Function}:

$$h(x) = \argmax([x \cdot r_1, x \cdot r_2, ..., x \cdot r_b])$$

where $r_i$ are random projection vectors

\vspace{5mm}
\textbf{Complexity}:

Standard: $O(n^2)$

Reformer: $O(n \log n)$

\vspace{5mm}
\textbf{Trade-offs}:

\begin{itemize}
\item \textcolor{green}{Much faster for $n > 4096$}
\item \textcolor{red}{Approximate (misses some attention pairs)}
\item \textcolor{green}{Enables sequences up to 64K tokens}
\item \textcolor{red}{More complex implementation}
\end{itemize}

\vspace{0.2cm}
\bottomnote{Reformer enables long-context applications (books, long documents)}
\end{frame}

% Slide A12: Linformer - Low-Rank Attention
\begin{frame}[t]{Appendix A12: Linformer Low-Rank Approximation}
\small
\textbf{Key Idea}: Attention matrix is approximately low-rank

\vspace{5mm}
\textbf{Standard Attention}:

$$\text{Attention}(Q, K, V) = \softmax(\frac{QK^T}{\sqrt{d}})V$$

where $QK^T \in \mathbb{R}^{n \times n}$

\vspace{5mm}
\textbf{Linformer Modification}:

Project keys and values to lower dimension $k$:

$$K' = KE_k \in \mathbb{R}^{n \times k}$$
$$V' = VF_k \in \mathbb{R}^{n \times k}$$

\vspace{3mm}
Then: $QK'^T \in \mathbb{R}^{n \times k}$ instead of $\mathbb{R}^{n \times n}$

\vspace{5mm}
\textbf{Complexity}:

Standard: $O(n^2 d)$

Linformer: $O(nkd)$ where $k=256$ typical

\vspace{3mm}
For $n=8192$, $k=256$: $32\times$ speedup!

\vspace{0.2cm}
\bottomnote{Simple idea, dramatic impact - linear complexity}
\end{frame}

% Slide A13: Linear Attention - Kernel Methods
\begin{frame}[t]{Appendix A13: Linear Attention via Kernel Trick}
\small
\textbf{Kernel Reformulation}:

Standard: $\text{Attention}(Q,K,V) = \softmax(QK^T)V$

\vspace{3mm}
Rewrite softmax as kernel: $\phi(q)^T \phi(k)$

\vspace{5mm}
\textbf{Key Trick}: Reorder operations

$$\text{Attention} = \phi(Q)(\phi(K)^T V)$$

Compute $\phi(K)^T V$ first! This is $k \times d$ (small)

\vspace{5mm}
\textbf{Complexity}:

\begin{itemize}
\item Standard: $(n \times n) \times (n \times d) = O(n^2 d)$
\item Linear: $(n \times k) \times (k \times d) = O(nkd)$
\item With $k=d$: $O(nd^2)$ - linear in $n$!
\end{itemize}

\vspace{5mm}
\textbf{Approximation Quality}:

Not exact, but empirically good

Works well for long sequences

\vspace{0.2cm}
\bottomnote{Kernel trick enables true linear attention}
\end{frame}

% Slide A14: FlashAttention - IO Optimization
\begin{frame}[t]{Appendix A14: FlashAttention - GPU Optimization}
\small
\textbf{Problem}: Standard attention is IO-bound, not compute-bound

\vspace{3mm}
GPU memory hierarchy:
\begin{itemize}
\item SRAM (on-chip): Fast but tiny (20MB)
\item HBM (GPU RAM): Slow but large (40GB)
\end{itemize}

\vspace{5mm}
\textbf{Standard Attention IO}:

\begin{enumerate}
\item Load $Q, K$ from HBM
\item Compute $S = QK^T$ (write to HBM)
\item Load $S$ from HBM
\item Apply softmax (write to HBM)
\item Load $P, V$ from HBM
\item Compute $PV$ (write to HBM)
\end{enumerate}

Multiple slow HBM reads/writes!

\vspace{3mm}
\textbf{FlashAttention Optimization}:

\begin{itemize}
\item Fuse operations (compute in SRAM)
\item Tiling (process in blocks)
\item Never materialize full $n \times n$ matrix
\item 2-4× faster, less memory
\end{itemize}

\vspace{0.2cm}
\bottomnote{Algorithm innovation - same math, better hardware utilization}
\end{frame}

% Slide A15: Efficient Variants Comparison
\begin{frame}[t]{Appendix A15: Choosing Efficient Transformer Variant}
\small
\begin{center}
\begin{tabular}{lcccc}
\toprule
\textbf{Variant} & \textbf{Complexity} & \textbf{Exact} & \textbf{Max n} & \textbf{Use Case} \\
\midrule
Standard & $O(n^2)$ & Yes & 2K-4K & Default \\
Reformer & $O(n \log n)$ & No & 64K & Long docs \\
Linformer & $O(n)$ & No & 32K & Fast training \\
Linear Attn & $O(n)$ & No & 16K & Real-time \\
FlashAttention & $O(n^2)$ & Yes & 8K & GPU-optimized \\
\midrule
\textbf{Best for:} & & & & \\
Books (100K+) & Reformer & & & \\
Articles (8-16K) & Linformer & & & \\
Production (2-4K) & FlashAttn & & & \\
Mobile/Edge & Linear Attn & & & \\
\bottomrule
\end{tabular}
\end{center}

\vspace{5mm}
\textbf{Recommendation}:

\begin{itemize}
\item Start with FlashAttention (better standard)
\item Use Linformer if need $n > 4K$
\item Reformer for extreme lengths ($n > 32K$)
\item Linear attention for edge deployment
\end{itemize}

\vspace{0.2cm}
\bottomnote{Choose based on sequence length and deployment constraints}
\end{frame}

\end{document}
