\documentclass[8pt,aspectratio=169,8pt]{beamer}
\usepackage[utf8]{inputenc}
\usepackage{graphicx}
\usepackage{amsmath,amssymb}
\usepackage{algorithm2e}
\usepackage{listings}
\usepackage{xcolor}
\usepackage{tikz}
\usepackage{pgfplots}
\pgfplotsset{compat=1.17}
\usepackage{subfigure}
\usepackage{hyperref}

% Theme settings
\usetheme{Frankfurt}
\usecolortheme{seahorse}
\setbeamertemplate{navigation symbols}{}
\setbeamertemplate{footline}[frame number]

% Custom colors
\definecolor{darkblue}{RGB}{0,51,102}
\definecolor{lightblue}{RGB}{173,216,230}
\definecolor{codegreen}{RGB}{0,128,0}
\definecolor{codegray}{RGB}{150,150,150}
\definecolor{codepurple}{RGB}{128,0,128}
\definecolor{backcolor}{RGB}{245,245,245}

% Code listing settings
\lstset{
    backgroundcolor=\color{backcolor},
    basicstyle=\ttfamily\footnotesize,
    breakatwhitespace=false,
    breaklines=true,
    captionpos=b,
    commentstyle=\color{codegreen},
    keywordstyle=\color{blue},
    numberstyle=\tiny\color{codegray},
    stringstyle=\color{codepurple},
    showstringspaces=false,
    frame=single,
    numbers=left,
    language=Python
}

% Include slide layouts
% SLIDE LAYOUT TEMPLATES FOR NLP COURSE
% Three consistent layouts used throughout the course

% ==============================================================
% LAYOUT 1: CONCEPT SLIDE
% Used for: Theory, definitions, mathematical concepts
% Features: Title, main content area with bullets/equations, optional figure
% ==============================================================

\newcommand{\conceptslide}[3]{ % #1: title, #2: content, #3: optional figure
\begin{frame}[t]{#1}
    \begin{columns}[T]
        \begin{column}{0.6\textwidth}
            #2
        \end{column}
        \begin{column}{0.35\textwidth}
            \centering
            #3
        \end{column}
    \end{columns}
\end{frame}
}

% ==============================================================
% LAYOUT 2: CODE & IMPLEMENTATION SLIDE
% Used for: Code examples, algorithms, implementation details
% Features: Title, code block, explanation text
% ==============================================================

% Note: For code slides, use \begin{frame}[fragile] manually

\newcommand{\codeexplanation}[1]{%
    \vspace{0.5em}
    \small
    #1
}

\newcommand{\codeblock}[2][Python]{%
    \begin{lstlisting}[language=#1, basicstyle=\ttfamily\tiny]
#2
    \end{lstlisting}
}

\newcommand{\codeslide}[3]{%
    \begin{frame}[fragile]{#1}
    \begin{columns}[T]
        \column{0.55\textwidth}
        \codeblock{#2}
        \column{0.43\textwidth}
        \codeexplanation{#3}
    \end{columns}
    \end{frame}
}

% ==============================================================
% LAYOUT 3: RESULTS & VISUALIZATION SLIDE
% Used for: Experimental results, charts, comparisons
% Features: Title, large visualization area, key insights
% ==============================================================

\newcommand{\resultslide}[3]{ % #1: title, #2: visualization, #3: insights
\begin{frame}[t]{#1}
    \centering
    \vspace{-0.5em}
    #2
    \vspace{0.5em}
    
    \begin{block}{Key Insights}
        #3
    \end{block}
\end{frame}
}

% Additional utility commands for consistency

% Highlighted text
\newcommand{\highlight}[1]{\textcolor{blue}{\textbf{#1}}}

% Math notation shortcuts
\newcommand{\prob}[1]{P(#1)}
\newcommand{\given}{\mid}
\newcommand{\argmax}{\operatorname{argmax}}
\newcommand{\softmax}{\operatorname{softmax}}

% Standard figure size
\newcommand{\stdfigsizeinchins}{0.4\textwidth}

% For code blocks, use lstlisting directly:
% \begin{lstlisting}[language=Python]
% ... code ...
% \end{lstlisting}

% Equation highlight box
\newcommand{\eqbox}[1]{
    \begin{center}
    \colorbox{lightblue!20}{
        \parbox{0.8\textwidth}{
            \vspace{0.3em}
            \centering
            #1
            \vspace{0.3em}
        }
    }
    \end{center}
}

\title[Week 5: Transformers]{Natural Language Processing Course}
\subtitle{Week 5: The Transformer Architecture}
\author{Joerg R. Osterrieder \\ \url{www.joergosterrieder.com}}
\date{}

\begin{document}

% Title page
\begin{frame}
    \titlepage
\end{frame}

\section{Week 5: The Transformer Revolution}

% Title slide
\begin{frame}
    \centering
    \vspace{2cm}
    {\Large \textbf{Week 5}}\\
    \vspace{0.5cm}
    {\huge \textbf{The Transformer}}\\
    \vspace{1cm}
    {\large Attention Is All You Need}
\end{frame}

% Motivation: The parallelization problem
\begin{frame}[t]{Why Google Couldn't Scale Translation Fast Enough}
    \textbf{The RNN bottleneck (2016):}
    
    \vspace{0.5em}
    To translate "I love machine learning":
    \begin{enumerate}
        \item Process "I" → wait → 
        \item Process "love" → wait →
        \item Process "machine" → wait →
        \item Process "learning" → done
    \end{enumerate}
    
    \vspace{0.5em}
    \begin{center}
    \colorbox{red!20}{
        \parbox{0.8\textwidth}{
            \centering
            RNNs must process words one at a time - can't parallelize!
        }
    }
    \end{center}
    
    \vspace{0.5em}
    \textbf{The cost:}
    \begin{itemize}
        \item Training large models: Weeks to months\footnotemark
        \item Can't use modern GPUs effectively (built for parallel computation)
        \item Google needed 8,000 TPUs for production\footnotemark
    \end{itemize}
    
    \footnotetext[1]{Original transformer trained in 3.5 days vs weeks for RNNs}
    \footnotetext[2]{Wu et al. (2016) Google NMT system requirements}
\end{frame}

% The radical idea
\begin{frame}[t]{A Radical Idea: What If We Remove RNNs Entirely?}
    \textbf{The 2017 breakthrough:}\footnotemark
    
    \vspace{0.5em}
    "What if we use ONLY attention mechanisms?"
    
    \vspace{0.5em}
    \textbf{Revolutionary insights:}
    \begin{enumerate}
        \item Attention can capture all relationships directly
        \item No sequential processing needed
        \item Every word can look at every other word simultaneously
        \item Parallelization becomes trivial!
    \end{enumerate}
    
    \vspace{0.5em}
    \textbf{The impact:}
    \begin{itemize}
        \item Training time: Weeks → Days
        \item Model quality: BLEU 41.8 → 28.4 (EN-DE)\footnotemark
        \item Spawned GPT, BERT, and all modern LLMs
    \end{itemize}
    
    \vspace{0.5em}
    \begin{center}
    \colorbox{green!20}{
        \parbox{0.8\textwidth}{
            \centering
            The Transformer: Process all words in parallel using attention
        }
    }
    \end{center}
    
    \footnotetext[1]{Vaswani et al. (2017). "Attention Is All You Need", NeurIPS}
    \footnotetext[2]{New state-of-the-art on WMT 2014 English-German}
\end{frame}

% Real-world impact
\begin{frame}[t]{Transformers Power Everything You Use (2024)}
    \begin{columns}[T]
        \begin{column}{0.5\textwidth}
            \textbf{Language Models:}
            \begin{itemize}
                \item ChatGPT (GPT-4): 1.76T params\footnotemark
                \item Google Bard (Gemini)
                \item Claude (Anthropic)
                \item GitHub Copilot
            \end{itemize}
            
            \vspace{0.5em}
            \textbf{Search \& Translation:}
            \begin{itemize}
                \item Google Search (BERT)
                \item DeepL Translator
                \item Microsoft Translator
                \item Every modern NMT system
            \end{itemize}
        \end{column}
        \begin{column}{0.5\textwidth}
            \textbf{Multimodal AI:}
            \begin{itemize}
                \item DALL-E (text → image)
                \item Whisper (speech → text)
                \item CLIP (vision-language)
                \item Flamingo (image understanding)
            \end{itemize}
            
            \vspace{0.5em}
            \textbf{Key Advantages:}
            \begin{itemize}
                \item 100x faster training\footnotemark
                \item Better long-range dependencies
                \item Transfer learning revolution
                \item Scale to trillions of parameters
            \end{itemize}
        \end{column}
    \end{columns}
    
    \vspace{0.5em}
    \begin{center}
    \colorbox{lightblue!30}{
        \parbox{0.8\textwidth}{
            \centering
            98\% of state-of-the-art NLP uses transformers (2024)
        }
    }
    \end{center}
    
    \footnotetext[1]{Estimated from performance characteristics}
    \footnotetext[2]{Compared to equivalent RNN models}
\end{frame}

% Learning objectives
\begin{frame}[t]{Week 5: What You'll Master}
    \textbf{By the end of this week, you will:}
    \begin{itemize}
        \item \textbf{Understand} why parallelization changes everything
        \item \textbf{Build} intuition for self-attention mechanism
        \item \textbf{Implement} a complete transformer from scratch
        \item \textbf{Master} positional encodings and multi-head attention
        \item \textbf{Create} your own mini-GPT
    \end{itemize}
    
    \vspace{0.5em}
    \begin{center}
    \colorbox{lightblue!30}{
        \parbox{0.8\textwidth}{
            \centering
            \textbf{Core Insight:} Let every word attend to every other word directly
        }
    }
    \end{center}
\end{frame}

% Self-attention intuition
\begin{frame}[t]{The Genius of Self-Attention}
    \textbf{How humans read "The cat sat on the mat":}
    
    \vspace{0.5em}
    When we see "sat", we instantly know:
    \begin{itemize}
        \item WHO sat? → look at "cat"
        \item WHERE? → look at "mat"
        \item No need to process sequentially!
    \end{itemize}
    
    \vspace{0.5em}
    \textbf{Self-attention does exactly this:}
    \begin{enumerate}
        \item Each word asks: "Who should I pay attention to?"
        \item Computes attention scores with all other words
        \item Creates weighted combination of relevant words
        \item All happening simultaneously!
    \end{enumerate}
    
    \vspace{0.5em}
    \textbf{Example: "The student who studied hard passed"}
    \begin{itemize}
        \item "passed" attends strongly to "student" (not "hard")
        \item "hard" attends to "studied"
        \item All connections computed in parallel
    \end{itemize}
    
    \vspace{0.5em}
    \begin{center}
    \colorbox{yellow!20}{
        \parbox{0.8\textwidth}{
            \centering
            Self-attention = Each word decides what's relevant to it
        }
    }
    \end{center}
\end{frame}

% Visual self-attention
\begin{frame}[t]{Visualizing Self-Attention}
    \centering
    \includegraphics[width=0.9\textwidth]{../figures/self_attention_visualization.pdf}
    
    \vspace{0.5em}
    \textbf{Key insights:}
    \begin{itemize}
        \item Every word connects to every other word
        \item Attention weights learned during training
        \item No fixed left-to-right processing
        \item Captures long-range dependencies naturally
    \end{itemize}
\end{frame}

% Mathematical formulation
\conceptslide{Self-Attention Mathematics: Elegantly Simple}{
    \textbf{The attention formula:}
    
    \vspace{0.5em}
    \eqbox{
        \[\text{Attention}(Q, K, V) = \text{softmax}\left(\frac{QK^T}{\sqrt{d_k}}\right)V\]
    }
    
    \vspace{0.5em}
    Where for each word:
    \begin{itemize}
        \item $Q$ (Query): "What am I looking for?"
        \item $K$ (Key): "What do I contain?"
        \item $V$ (Value): "What information do I provide?"
    \end{itemize}
    
    \vspace{0.5em}
    \textbf{In plain English:}
    \begin{enumerate}
        \item Compare my query with all keys (dot product)
        \item Scale by $\sqrt{d_k}$ to prevent saturation
        \item Apply softmax to get attention weights
        \item Weighted sum of values
    \end{enumerate}
}{
    \textbf{Why this works:}
    \begin{itemize}
        \item Dot product measures similarity
        \item Softmax creates probability distribution
        \item Fully differentiable
        \item Parallelizable!
    \end{itemize}
}

% Implementation
\begin{frame}[fragile]{Building Self-Attention: Complete Implementation}
    \begin{columns}[T]
\column{0.55\textwidth}

\begin{lstlisting}[language=Python,basicstyle=\ttfamily\tiny]
import torch
import torch.nn as nn
import torch.nn.functional as F
import math

class SelfAttention(nn.Module):
    def __init__(self, embed_size, heads=8):
        """Multi-head self-attention"""
        super().__init__()
        self.embed_size = embed_size
        self.heads = heads
        self.head_dim = embed_size // heads
        
        assert self.head_dim * heads == embed_size
        
        # Linear projections for Q, K, V
        self.queries = nn.Linear(embed_size, embed_size)
        self.keys = nn.Linear(embed_size, embed_size)
        self.values = nn.Linear(embed_size, embed_size)
        self.fc_out = nn.Linear(embed_size, embed_size)
        
    def forward(self, x, mask=None):
        """Compute multi-head attention"""
        N, seq_len, _ = x.shape
        
        # Project to Q, K, V
        Q = self.queries(x)
        K = self.keys(x)
        V = self.values(x)
        
        # Reshape for multi-head attention
        Q = Q.reshape(N, seq_len, self.heads, self.head_dim)
        K = K.reshape(N, seq_len, self.heads, self.head_dim)
        V = V.reshape(N, seq_len, self.heads, self.head_dim)
        
        # Compute attention scores
        energy = torch.einsum("nqhd,nkhd->nhqk", [Q, K])
        
        if mask is not None:
            energy = energy.masked_fill(mask == 0, float("-1e20"))
            
        attention = F.softmax(energy / math.sqrt(self.head_dim), dim=3)
        
        # Apply attention to values
        out = torch.einsum("nhql,nlhd->nqhd", [attention, V])
        out = out.reshape(N, seq_len, self.embed_size)
        
        return self.fc_out(out)
\end{lstlisting}
\column{0.43\textwidth}
        \codeexplanation{
            \textbf{Design Choices:}
            \begin{itemize}
                \item 8 heads typical (parallel attention)\footnotemark
                \item Head dim = 64 (512 / 8)
                \item Scaling prevents gradient issues
            \end{itemize}
            
            \vspace{0.3em}
            \textbf{Multi-Head Benefits:}
            \begin{itemize}
                \item Different heads learn different relationships
                \item One head: syntax
                \item Another: semantics
                \item Another: position
            \end{itemize}
            
            \footnotetext{Original paper used 8 heads}
        }
    \end{columns}
\end{frame}

% Positional encoding
\begin{frame}[t]{The Position Problem: Order Still Matters!}
    \textbf{Self-attention has no notion of position!}
    
    \vspace{0.5em}
    These are identical to self-attention:
    \begin{itemize}
        \item "The cat sat on the mat"
        \item "Mat the on sat the cat"
        \item "Cat mat the the on sat"
    \end{itemize}
    
    \vspace{0.5em}
    \textbf{The solution: Positional Encoding}\footnotemark
    
    Add position information to each word embedding:
    
    \vspace{0.5em}
    \eqbox{
        \[\begin{aligned}
        PE_{(pos,2i)} &= \sin(pos/10000^{2i/d}) \\
        PE_{(pos,2i+1)} &= \cos(pos/10000^{2i/d})
        \end{aligned}\]
    }
    
    \vspace{0.5em}
    \textbf{Why sinusoids?}
    \begin{itemize}
        \item Unique pattern for each position
        \item Can extrapolate to longer sequences
        \item Relative positions have consistent patterns
    \end{itemize}
    
    \vspace{0.5em}
    \begin{center}
    \colorbox{lightblue!30}{
        \parbox{0.8\textwidth}{
            \centering
            Positional encoding = GPS coordinates for words
        }
    }
    \end{center}
    
    \footnotetext{Many alternatives explored: learned, RoPE, ALiBi}
\end{frame}

% Complete transformer block
\begin{frame}[fragile]{The Transformer Block: Putting It Together}
    \begin{columns}[T]
        \column{0.55\textwidth}

\begin{lstlisting}[language=Python,basicstyle=\ttfamily\tiny]
class TransformerBlock(nn.Module):
    def __init__(self, embed_size, heads, dropout, forward_expansion):
        """One transformer encoder block"""
        super().__init__()
        self.attention = SelfAttention(embed_size, heads)
        self.norm1 = nn.LayerNorm(embed_size)
        self.norm2 = nn.LayerNorm(embed_size)
        
        self.feed_forward = nn.Sequential(
            nn.Linear(embed_size, forward_expansion * embed_size),
            nn.ReLU(),
            nn.Linear(forward_expansion * embed_size, embed_size)
        )
        self.dropout = nn.Dropout(dropout)
        
    def forward(self, x, mask=None):
        """Forward pass with residual connections"""
        # Self-attention with residual
        attention = self.attention(x, mask)
        x = self.dropout(self.norm1(attention + x))
        
        # Feed-forward with residual
        forward = self.feed_forward(x)
        out = self.dropout(self.norm2(forward + x))
        
        return out

class Transformer(nn.Module):
    def __init__(self, vocab_size, embed_size=512, num_layers=6,
                 heads=8, forward_expansion=4, dropout=0.1, max_len=5000):
        """Complete transformer model"""
        super().__init__()
        self.embed_size = embed_size
        self.word_embedding = nn.Embedding(vocab_size, embed_size)
        self.position_embedding = nn.Embedding(max_len, embed_size)
        
        self.layers = nn.ModuleList([
            TransformerBlock(embed_size, heads, dropout, forward_expansion)
            for _ in range(num_layers)
        ])
        
        self.dropout = nn.Dropout(dropout)
        self.fc_out = nn.Linear(embed_size, vocab_size)
\end{lstlisting}
        \column{0.45\textwidth}

        \codeexplanation{
            \textbf{Architecture (Base):}
            \begin{itemize}
                \item 6 layers deep\footnotemark
                \item 512 embedding dimension
                \item 2048 feed-forward dimension
                \item Residual connections crucial
            \end{itemize}
            
            \vspace{0.3em}
            \textbf{Why Residuals?}
            \begin{itemize}
                \item Enable deep networks
                \item Gradient flow preservation
                \item Each layer learns refinement
            \end{itemize}
            
            \footnotetext{GPT-3 has 96 layers!}
        }
    \end{columns}
\end{frame}

% Training improvements
\begin{frame}[t]{Why Transformers Train So Fast}
    \centering
    \includegraphics[width=0.55\textwidth]{../figures/transformer_vs_rnn_speed.pdf}
    
    \vspace{0.5em}
    \textbf{The parallelization advantage:}
    \begin{itemize}
        \item RNN: Must wait for each step (sequential)
        \item Transformer: All positions computed simultaneously
        \item GPU utilization: 15\% → 90\%\footnotemark
        \item Training time: Weeks → Days
    \end{itemize}
    
    \footnotetext{Typical GPU utilization improvements reported}
\end{frame}

% Results comparison
\resultslide{Transformer Impact: Immediate Dominance}{
    \centering
    \includegraphics[width=0.55\textwidth]{../figures/transformer_benchmarks.pdf}
}{
    \begin{itemize}
        \item WMT'14 EN-DE: 28.4 BLEU (previous best: 25.2)
        \item WMT'14 EN-FR: 41.8 BLEU (previous best: 37.0)
        \item Training: 3.5 days on 8 GPUs (vs weeks)
        \item Started the pre-training era (BERT, GPT)
        \item Every subsequent breakthrough built on transformers
    \end{itemize}
}

% Modern variants
\begin{frame}[t]{The Transformer Family Tree (2024)}
    \begin{columns}[T]
        \begin{column}{0.5\textwidth}
            \textbf{Encoder-Only (BERT-style):}
            \begin{itemize}
                \item BERT: Bidirectional understanding
                \item RoBERTa: Better training
                \item DeBERTa: Disentangled attention
                \item Used for: Classification, NER, QA
            \end{itemize}
            
            \vspace{0.5em}
            \textbf{Decoder-Only (GPT-style):}
            \begin{itemize}
                \item GPT-4: 1.76T parameters\footnotemark
                \item Claude: Constitutional AI
                \item LLaMA: Efficient architecture
                \item Used for: Generation, chat, code
            \end{itemize}
        \end{column}
        \begin{column}{0.5\textwidth}
            \textbf{Encoder-Decoder (T5-style):}
            \begin{itemize}
                \item T5: Text-to-text unified
                \item BART: Denoising approach
                \item mT5: Multilingual
                \item Used for: Translation, summarization
            \end{itemize}
            
            \vspace{0.5em}
            \textbf{Efficient Variants:}
            \begin{itemize}
                \item FlashAttention: 2-3x faster\footnotemark
                \item Linformer: Linear complexity
                \item Performer: Kernel approximation
                \item Used for: Long sequences
            \end{itemize}
        \end{column}
    \end{columns}
    
    \vspace{0.5em}
    \begin{center}
    \colorbox{yellow!20}{
        \parbox{0.8\textwidth}{
            \centering
            All modern LLMs are transformer variants!
        }
    }
    \end{center}
    
    \footnotetext[1]{Estimated from capabilities}
    \footnotetext[2]{Dao et al. (2022) FlashAttention}
\end{frame}

% Common pitfalls
\begin{frame}[t]{Transformer Gotchas and Solutions}
    \textbf{1. Attention is Quadratic}
    \begin{itemize}
        \item Problem: $O(n^2)$ memory for sequence length $n$
        \item Solution: Sparse attention, sliding windows
        \item Example: GPT-3 uses sparse patterns
    \end{itemize}
    
    \vspace{0.5em}
    \textbf{2. Position Extrapolation}
    \begin{itemize}
        \item Problem: Fails on sequences longer than training
        \item Solution: ALiBi, RoPE, or relative encodings
        \item Example: LLaMA uses RoPE for 100k+ context
    \end{itemize}
    
    \vspace{0.5em}
    \textbf{3. Training Instability}
    \begin{itemize}
        \item Problem: Large models diverge easily
        \item Solution: Learning rate warmup, careful initialization
        \item Example: GPT-3 took months of tuning
    \end{itemize}
    
    \vspace{0.5em}
    \textbf{Real Example - ChatGPT:}
    \begin{itemize}
        \item Uses modified attention (sparse + dense)
        \item Special position encodings for long context
        \item Extensive stability modifications
    \end{itemize}
\end{frame}

% Exercise
\begin{frame}[t]{Week 5 Exercise: Build Your Own Mini-GPT}
    \textbf{Your Mission:} Create a character-level GPT for text generation
    
    \vspace{0.5em}
    \textbf{Implementation Steps:}
    \begin{enumerate}
        \item Implement multi-head self-attention
        \item Add positional encodings
        \item Stack 6 transformer blocks
        \item Train on Shakespeare/your favorite text
        \item Generate new text autoregressively
    \end{enumerate}
    
    \vspace{0.5em}
    \textbf{Key Experiments:}
    \begin{itemize}
        \item Compare 1 vs 8 vs 16 attention heads
        \item Try with/without positional encoding
        \item Measure GPU utilization vs RNN
        \item Visualize attention patterns
    \end{itemize}
    
    \vspace{0.5em}
    \textbf{Bonus Challenges:}
    \begin{itemize}
        \item Implement sparse attention for longer sequences
        \item Add beam search for better generation
        \item Try different position encoding schemes
        \item Build a simple chatbot interface
    \end{itemize}
    
    \vspace{0.5em}
    \textbf{You'll discover:} Why transformers took over the world!
\end{frame}

% Summary
\begin{frame}[t]{Key Takeaways: The Attention Revolution}
    \textbf{What we learned:}
    \begin{itemize}
        \item Sequential processing was the bottleneck
        \item Self-attention enables full parallelization
        \item Every word can attend to every other word
        \item Position encodings restore order information
        \item Transformers scale to trillions of parameters
    \end{itemize}
    
    \vspace{0.5em}
    \textbf{The evolution:}
    \begin{center}
    \colorbox{lightblue!30}{
        \parbox{0.8\textwidth}{
            \centering
            Sequential (RNN) $\rightarrow$ Parallel (Transformer) $\rightarrow$ Scale (GPT/BERT)
        }
    }
    \end{center}
    
    \vspace{0.5em}
    \textbf{Why it matters:}
    \begin{itemize}
        \item Enabled training on internet-scale data
        \item Made transfer learning practical
        \item Started the LLM revolution
    \end{itemize}
    
    \vspace{0.5em}
    \textbf{Next week: Pre-trained Language Models}
    
    How do we use transformers to learn from all of human knowledge?
\end{frame}

% References
\begin{frame}[t]{References and Further Reading}
    \textbf{Foundational Papers:}
    \begin{itemize}
        \item Vaswani et al. (2017). "Attention Is All You Need", NeurIPS
        \item Devlin et al. (2019). "BERT: Pre-training of Deep Bidirectional Transformers"
        \item Radford et al. (2018). "Improving Language Understanding by Generative Pre-Training"
    \end{itemize}
    
    \textbf{Implementation Resources:}
    \begin{itemize}
        \item "The Illustrated Transformer" by Jay Alammar
        \item "The Annotated Transformer" (Harvard NLP)
        \item Hugging Face Transformers library
    \end{itemize}
    
    \textbf{Recent Advances:}
    \begin{itemize}
        \item FlashAttention: Making attention practical
        \item Scaling laws for neural language models
        \item Efficient transformers survey (2022)
    \end{itemize}
\end{frame}

\end{document}
