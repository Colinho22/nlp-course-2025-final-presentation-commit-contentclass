% Chapter 5: The Transformer Architecture - Comprehensive Presentation
% NLP Course 2025
% Created: 2025-09-24

% Master Template for NLP Course
% Optimal Readability Layout Standard
% All presentations should include this template

\documentclass[8pt,aspectratio=169]{beamer}
\usetheme{Madrid}
\setbeamertemplate{navigation symbols}{}

% ====================================
% OPTIMAL READABILITY COLOR PALETTE
% ====================================
\definecolor{PureBlack}{HTML}{000000}      % Main text (21:1 contrast)
\definecolor{DeepBlue}{HTML}{003D7A}       % Primary accent (12.6:1 contrast)
\definecolor{DarkGray}{HTML}{4A4A4A}       % Secondary text (9.7:1 contrast)
\definecolor{LightGray}{HTML}{E5E5E5}      % Borders and grids
\definecolor{ChartBlue}{HTML}{0066CC}      % Chart primary
\definecolor{ChartOrange}{HTML}{FF8800}    % Chart secondary
\definecolor{ChartTeal}{HTML}{00A0A0}      % Chart tertiary
\definecolor{ChartPurple}{HTML}{8B4789}    % Chart quaternary
\definecolor{DarkGreen}{HTML}{228B22}      % Success/positive
\definecolor{DarkRed}{HTML}{CC0000}        % Warning/negative

% ====================================
% BEAMER COLOR CONFIGURATION
% ====================================
\setbeamercolor{structure}{fg=PureBlack}
\setbeamercolor{frametitle}{fg=PureBlack,bg=white}
\setbeamercolor{title}{fg=PureBlack,bg=white}
\setbeamercolor{subtitle}{fg=DarkGray}
\setbeamercolor{author}{fg=DarkGray}
\setbeamercolor{date}{fg=DarkGray}
\setbeamercolor{institute}{fg=DarkGray}

% Blocks - no backgrounds
\setbeamercolor{block title}{fg=PureBlack,bg=white}
\setbeamercolor{block body}{fg=PureBlack,bg=white}
\setbeamercolor{block title example}{fg=DarkGreen,bg=white}
\setbeamercolor{block body example}{fg=PureBlack,bg=white}
\setbeamercolor{block title alerted}{fg=DarkRed,bg=white}
\setbeamercolor{block body alerted}{fg=PureBlack,bg=white}

% Lists
\setbeamercolor{item}{fg=DeepBlue}
\setbeamercolor{subitem}{fg=ChartBlue}
\setbeamercolor{enumerate item}{fg=DeepBlue}
\setbeamercolor{enumerate subitem}{fg=ChartBlue}

% Text
\setbeamercolor{normal text}{fg=PureBlack,bg=white}
\setbeamercolor{alerted text}{fg=DarkRed}
\setbeamercolor{example text}{fg=DarkGreen}

% Footer
\setbeamercolor{footline}{fg=DarkGray,bg=white}
\setbeamercolor{page number in head/foot}{fg=DarkGray}

% ====================================
% FONT CONFIGURATION
% ====================================
\setbeamerfont{normal text}{size=\normalsize}
\setbeamerfont{frametitle}{size=\Large,series=\bfseries}
\setbeamerfont{title}{size=\huge,series=\bfseries}
\setbeamerfont{subtitle}{size=\large}
\setbeamerfont{author}{size=\normalsize}
\setbeamerfont{date}{size=\small}
\setbeamerfont{institute}{size=\small}

% ====================================
% REQUIRED PACKAGES
% ====================================
\usepackage{tikz}
\usepackage{amsmath}
\usepackage{amssymb}
\usepackage{booktabs}
\usepackage{graphicx}
\usepackage{array}
\usepackage{listings}
\usepackage{algorithm2e}
\usepackage{xcolor}
\usepackage{tabularx}
\usepackage{multirow}
\usepackage{subcaption}

% ====================================
% CUSTOM COMMANDS
% ====================================
% Text highlighting commands
\newcommand{\highlight}[1]{\textcolor{DeepBlue}{\textbf{#1}}}
\newcommand{\secondary}[1]{\textcolor{DarkGray}{#1}}
\newcommand{\success}[1]{\textcolor{DarkGreen}{#1}}
\newcommand{\warning}[1]{\textcolor{DarkRed}{#1}}
\newcommand{\data}[1]{\textcolor{ChartBlue}{#1}}
\newcommand{\dataalt}[1]{\textcolor{ChartOrange}{#1}}

% Mathematical notation
\newcommand{\given}{\mid}
\newcommand{\prob}[1]{P(#1)}
\newcommand{\argmax}{\operatorname*{argmax}}
\newcommand{\argmin}{\operatorname*{argmin}}
\newcommand{\softmax}{\operatorname{softmax}}

% Box commands for emphasis
\newcommand{\keypoint}[1]{%
  \begin{center}
  \fbox{\parbox{0.9\textwidth}{\centering\textbf{#1}}}
  \end{center}
}

\newcommand{\formula}[1]{%
  \begin{center}
  \colorbox{LightGray}{\parbox{0.8\textwidth}{\centering$\displaystyle #1$}}
  \end{center}
}

% ====================================
% LISTINGS CONFIGURATION
% ====================================
\lstset{
  basicstyle=\ttfamily\small,
  keywordstyle=\color{DeepBlue}\bfseries,
  commentstyle=\color{DarkGray}\itshape,
  stringstyle=\color{ChartOrange},
  numbers=left,
  numberstyle=\tiny\color{DarkGray},
  stepnumber=1,
  numbersep=5pt,
  backgroundcolor=\color{white},
  showspaces=false,
  showstringspaces=false,
  showtabs=false,
  frame=single,
  frameround=tttt,
  rulecolor=\color{LightGray},
  tabsize=2,
  captionpos=b,
  breaklines=true,
  breakatwhitespace=true,
  language=Python,
  escapeinside={(*@}{@*)},
  morekeywords={self, yield, assert, with, as}
}

% ====================================
% STANDARD SLIDE LAYOUTS
% ====================================

% Two-column layout with title
\newcommand{\twocolslide}[3]{%
  \begin{frame}{#1}
  \begin{columns}[T]
  \column{0.48\textwidth}
  #2
  \column{0.48\textwidth}
  #3
  \end{columns}
  \end{frame}
}

% Three-column layout
\newcommand{\threecolslide}[4]{%
  \begin{frame}{#1}
  \begin{columns}[T]
  \column{0.32\textwidth}
  #2
  \column{0.32\textwidth}
  #3
  \column{0.32\textwidth}
  #4
  \end{columns}
  \end{frame}
}

% Chart slide with caption
\newcommand{\chartslide}[3]{%
  \begin{frame}{#1}
  \begin{center}
  \includegraphics[width=#2\textwidth]{#3}
  \end{center}
  \end{frame}
}

% Full chart slide - optimized to 0.85 for proper margins
\newcommand{\fullchartslide}[2]{%
  \begin{frame}{#1}
  \begin{center}
  \includegraphics[width=0.85\textwidth]{#2}
  \end{center}
  \end{frame}
}

% Code slide
\newcommand{\codeslide}[3]{%
  \begin{frame}[fragile]{#1}
  \begin{lstlisting}[language=#2]
#3
  \end{lstlisting}
  \end{frame}
}

% Concept slide with figure
\newcommand{\conceptslide}[3]{%
  \begin{frame}{#1}
  \begin{columns}[T]
  \column{0.6\textwidth}
  #2
  \column{0.35\textwidth}
  \begin{center}
  \includegraphics[width=0.95\textwidth]{#3}
  \end{center}
  \end{columns}
  \end{frame}
}

% Table slide
\newcommand{\tableslide}[2]{%
  \begin{frame}{#1}
  \begin{center}
  \Large
  \renewcommand{\arraystretch}{1.5}
  #2
  \end{center}
  \end{frame}
}

% Summary slide
\newcommand{\summaryslide}[2]{%
  \begin{frame}{#1}
  \begin{center}
  \Large
  #2
  \end{center}
  \vfill
  \begin{center}
  \keypoint{Key Takeaway}
  \end{center}
  \end{frame}
}

% ====================================
% REMOVE DECORATIONS
% ====================================
\setbeamertemplate{blocks}[default]
\setbeamertemplate{title page}[default][colsep=-4bp,rounded=false]
\setbeamertemplate{itemize items}[circle]
\setbeamertemplate{enumerate items}[default]
\setbeamertemplate{section in toc}[sections numbered]
\setbeamertemplate{subsection in toc}[subsections numbered]

% ====================================
% PAGE NUMBERING
% ====================================
\setbeamertemplate{footline}{
  \leavevmode%
  \hbox{%
  \begin{beamercolorbox}[wd=.333333\paperwidth,ht=2.25ex,dp=1ex,center]{author in head/foot}%
    \usebeamerfont{author in head/foot}\secondary{\insertshortauthor}
  \end{beamercolorbox}%
  \begin{beamercolorbox}[wd=.333333\paperwidth,ht=2.25ex,dp=1ex,center]{title in head/foot}%
    \usebeamerfont{title in head/foot}\secondary{\insertshorttitle}
  \end{beamercolorbox}%
  \begin{beamercolorbox}[wd=.333333\paperwidth,ht=2.25ex,dp=1ex,right]{date in head/foot}%
    \usebeamerfont{date in head/foot}\secondary{\insertshortdate{}\hspace*{2em}
    \insertframenumber{} / \inserttotalframenumber\hspace*{2ex}}
  \end{beamercolorbox}}%
  \vskip0pt%
}

% ====================================
% TABLE OF CONTENTS STYLE
% ====================================
\setbeamertemplate{section in toc}{%
  \leavevmode\leftskip=1.5em%
  \llap{%
    \usebeamerfont{section in toc}%
    \usebeamercolor[fg]{section in toc}%
    \inserttocsectionnumber.%
  }%
  \usebeamerfont{section in toc}%
  \usebeamercolor[fg]{section in toc}%
  \inserttocsection\par%
}

\setbeamertemplate{subsection in toc}{%
  \leavevmode\leftskip=3em%
  \llap{%
    \usebeamerfont{subsection in toc}%
    \usebeamercolor[fg]{subsection in toc}%
    \inserttocsectionnumber.\inserttocsubsectionnumber%
  }%
  \usebeamerfont{subsection in toc}%
  \usebeamercolor[fg]{subsection in toc}%
  \inserttocsubsection\par%
}

% ====================================
% END OF MASTER TEMPLATE
% ====================================

\title{The Transformer Architecture}
\subtitle{\secondary{Chapter 5: Attention Is All You Need - Complete Guide}}
\author{NLP Course 2025}
\date{\today}

\begin{document}

% ============================================
% PART 1: INTRODUCTION & MOTIVATION
% ============================================

% Slide 1: Title
\begin{frame}
\titlepage
\vfill
\begin{center}
\secondary{\footnotesize Comprehensive Coverage of Transformer Architecture}\\
\secondary{\footnotesize 45 slides covering theory, implementation, and applications}
\end{center}
\end{frame}

% Slide 2: Learning Objectives
\begin{frame}{Learning Objectives}
\begin{center}
{\Large \textbf{What You Will Master Today}}
\end{center}

\vspace{8mm}

\begin{columns}[T]
\column{0.48\textwidth}
\textbf{Theoretical Understanding}
\begin{itemize}
\item Self-attention mechanism
\item Multi-head attention
\item Positional encoding
\item Layer normalization
\item Complete architecture
\end{itemize}

\column{0.48\textwidth}
\textbf{Practical Skills}
\begin{itemize}
\item Implement attention from scratch
\item Build a mini-transformer
\item Debug attention weights
\item Optimize for performance
\item Apply to real problems
\end{itemize}
\end{columns}

\vspace{8mm}
\keypoint{By the end: You'll understand how ChatGPT, BERT, and all modern LLMs work}
\end{frame}

% Slide 3: The Journey So Far
\begin{frame}{The Journey to Transformers}
\begin{center}
\includegraphics[width=0.9\textwidth]{../figures/architecture_evolution.pdf}
\end{center}

\vspace{5mm}
\begin{itemize}
\item \textbf{1990s}: Simple RNNs - vanishing gradients
\item \textbf{1997}: LSTMs - better memory but still sequential
\item \textbf{2014}: Seq2Seq - encoder-decoder paradigm
\item \textbf{2015}: Attention mechanism - focus on relevance
\item \textbf{2017}: \highlight{Transformers} - parallel processing revolution
\item \textbf{2018+}: BERT, GPT - the LLM era begins
\end{itemize}

\secondary{\footnotesize Each step solved a problem but created new limitations}
\end{frame}

% Slide 4: The RNN Bottleneck
\begin{frame}{The Sequential Processing Problem}
\textbf{Why RNNs Hit a Wall:}

\vspace{8mm}
\begin{columns}[T]
\column{0.48\textwidth}
\textbf{Sequential Processing}
\begin{itemize}
\item Must process word 1 before word 2
\item Can't parallelize across sequence
\item Training time: $O(T)$ where $T$ = length
\item GPUs sit idle most of the time
\end{itemize}

\vspace{5mm}
\warning{Result: Weeks to train large models}

\column{0.48\textwidth}
\textbf{Information Bottleneck}
\begin{itemize}
\item All history compressed into one vector
\item Long sequences lose information
\item Gradient vanishing/exploding
\item Can't capture long-range dependencies
\end{itemize}

\vspace{5mm}
\warning{Result: Poor performance on long texts}
\end{columns}

\vspace{8mm}
\begin{center}
\keypoint{We needed a completely different approach}
\end{center}
\end{frame}

% Slide 5: The Transformer Impact
\begin{frame}{The Transformer Revolution (2017-2024)}
\begin{center}
{\Large \textbf{One Architecture Changed Everything}}
\end{center}

\vspace{8mm}
\begin{columns}[T]
\column{0.32\textwidth}
\textbf{Training Speed}
\begin{itemize}
\item \success{100x faster} than RNNs
\item Full parallelization
\item Days not weeks
\item Scales with hardware
\end{itemize}

\column{0.32\textwidth}
\textbf{Performance}
\begin{itemize}
\item SOTA on all benchmarks
\item Better long-range deps
\item Transfer learning works
\item Multimodal capabilities
\end{itemize}

\column{0.32\textwidth}
\textbf{Applications}
\begin{itemize}
\item ChatGPT (175B params)
\item Google Search (BERT)
\item GitHub Copilot
\item DALL-E, Whisper, etc.
\end{itemize}
\end{columns}

\vspace{10mm}
\begin{center}
\includegraphics[width=0.7\textwidth]{../figures/week5_benchmark_timeline.pdf}
\end{center}

\secondary{\footnotesize Transformers now power 98\% of state-of-the-art NLP systems}
\end{frame}

% ============================================
% PART 2: CORE CONCEPTS
% ============================================

% Slide 6: Section Divider
\begin{frame}
\begin{center}
{\Huge \textbf{Part 2}}\\
\vspace{5mm}
{\Large Core Concepts}\\
\vspace{10mm}
{\large Understanding Self-Attention}
\end{center}
\end{frame}

% Slide 7: Self-Attention Intuition
\begin{frame}{Self-Attention: The Core Innovation}
\textbf{The Brilliant Insight:}

\vspace{5mm}
"Let every word directly look at every other word to decide what's relevant"

\vspace{10mm}
\textbf{Example:} "The cat sat on the mat because it was tired"

\vspace{8mm}
When processing "it":
\begin{itemize}
\item Traditional RNN: Only sees previous hidden states sequentially
\item \highlight{Self-attention}: Directly examines all words:
\begin{itemize}
\item "it" strongly attends to "cat" (0.7 weight)
\item "it" weakly attends to "mat" (0.1 weight)
\item "it" moderately attends to "tired" (0.2 weight)
\end{itemize}
\end{itemize}

\vspace{8mm}
\keypoint{Every word builds its representation by looking at all other words}

\secondary{\footnotesize This happens for ALL words SIMULTANEOUSLY - that's the magic!}
\end{frame}

% Slide 8: Mathematical Foundation
\begin{frame}{The Attention Formula}
\begin{center}
{\Large \textbf{Scaled Dot-Product Attention}}
\end{center}

\vspace{8mm}
\formula{\text{Attention}(Q, K, V) = \text{softmax}\left(\frac{QK^T}{\sqrt{d_k}}\right)V}

\vspace{10mm}
\textbf{Step-by-step computation:}
\begin{enumerate}
\item \textbf{Create Q, K, V:} Linear projections of input embeddings
\[Q = XW^Q, \quad K = XW^K, \quad V = XW^V\]
\item \textbf{Compute attention scores:} How much should word $i$ attend to word $j$?
\[\text{score}(i,j) = \frac{q_i \cdot k_j}{\sqrt{d_k}}\]
\item \textbf{Apply softmax:} Convert scores to probability distribution
\[\alpha_{ij} = \frac{\exp(\text{score}(i,j))}{\sum_k \exp(\text{score}(i,k))}\]
\item \textbf{Weighted sum:} Combine values using attention weights
\[\text{output}_i = \sum_j \alpha_{ij} v_j\]
\end{enumerate}

\secondary{\footnotesize The $\sqrt{d_k}$ scaling prevents softmax saturation for large dimensions}
\end{frame}

% Slide 9: Query, Key, Value Explained
\begin{frame}{Understanding Query, Key, and Value}
\textbf{The Information Retrieval Analogy:}

\vspace{8mm}
\begin{columns}[T]
\column{0.32\textwidth}
\textbf{Query (Q)}
\begin{itemize}
\item "What am I looking for?"
\item The current word's search vector
\item Dimension: $d_k$
\item Learned through $W^Q$
\end{itemize}

\vspace{5mm}
\secondary{\small Example: "bank" asks "Am I financial or river-related?"}

\column{0.32\textwidth}
\textbf{Key (K)}
\begin{itemize}
\item "What do I contain?"
\item Each word's content descriptor
\item Dimension: $d_k$
\item Learned through $W^K$
\end{itemize}

\vspace{5mm}
\secondary{\small Example: "river" advertises "I'm about water"}

\column{0.32\textwidth}
\textbf{Value (V)}
\begin{itemize}
\item "What information do I provide?"
\item The actual content to aggregate
\item Dimension: $d_v$
\item Learned through $W^V$
\end{itemize}

\vspace{5mm}
\secondary{\small Example: "river" provides its semantic content}
\end{columns}

\vspace{10mm}
\keypoint{Q and K determine attention weights; V provides the actual information}
\end{frame}

% Slide 10: Attention Matrix Visualization
\begin{frame}{Visualizing Attention Patterns}
\begin{center}
\textbf{Attention Weights for: "The cat sat on the mat"}
\end{center}

\vspace{5mm}
\begin{columns}[T]
\column{0.5\textwidth}
\begin{center}
\includegraphics[width=\textwidth]{../figures/week5_attention_matrix.pdf}
\end{center}

\column{0.45\textwidth}
\textbf{What the Matrix Shows:}
\begin{itemize}
\item Darker = stronger attention
\item Row $i$ = where word $i$ looks
\item Column $j$ = who attends to word $j$
\end{itemize}

\vspace{5mm}
\textbf{Key Observations:}
\begin{itemize}
\item "cat" and "sat" strongly connected
\item "on" attends to "mat"
\item Articles attend broadly
\item Diagonal = self-attention
\end{itemize}
\end{columns}

\vspace{8mm}
\keypoint{Attention patterns reveal semantic relationships}

\secondary{\footnotesize Real transformers have multiple attention heads capturing different relationships}
\end{frame}

% Slide 11: Multi-Head Attention
\begin{frame}{Multi-Head Attention: Multiple Perspectives}
\textbf{Why One Head Isn't Enough:}

\vspace{5mm}
Different heads capture different relationships:
\begin{itemize}
\item Head 1: Syntactic dependencies (subject-verb)
\item Head 2: Coreference resolution (pronouns)
\item Head 3: Semantic similarity
\item Head 4: Positional patterns
\item ...
\end{itemize}

\vspace{8mm}
\formula{\text{MultiHead}(Q,K,V) = \text{Concat}(\text{head}_1, ..., \text{head}_h)W^O}

where each head is:
\[\text{head}_i = \text{Attention}(QW_i^Q, KW_i^K, VW_i^V)\]

\vspace{8mm}
\begin{columns}[T]
\column{0.48\textwidth}
\textbf{Typical Configuration:}
\begin{itemize}
\item 8-16 heads in practice
\item Each head: $d_k = d_{model}/h$
\item Parallel computation
\item Concatenated and projected
\end{itemize}

\column{0.48\textwidth}
\textbf{Benefits:}
\begin{itemize}
\item Richer representations
\item Redundancy and robustness
\item Different abstraction levels
\item Better generalization
\end{itemize}
\end{columns}
\end{frame}

% Slide 12: Why Multiple Heads Matter
\begin{frame}{Multi-Head Attention in Action}
\begin{center}
\includegraphics[width=0.9\textwidth]{../figures/multihead_attention.pdf}
\end{center}

\textbf{Example:} "The bank by the river"

\vspace{5mm}
\begin{columns}[T]
\column{0.24\textwidth}
\textbf{Head 1}\\
\secondary{\small Syntax}
\begin{itemize}
\item bank → the
\item river → the
\item by → bank
\end{itemize}

\column{0.24\textwidth}
\textbf{Head 2}\\
\secondary{\small Semantics}
\begin{itemize}
\item bank → river
\item river → bank
\item Strong connection
\end{itemize}

\column{0.24\textwidth}
\textbf{Head 3}\\
\secondary{\small Position}
\begin{itemize}
\item Sequential focus
\item Local context
\item Adjacent words
\end{itemize}

\column{0.24\textwidth}
\textbf{Head 4}\\
\secondary{\small Global}
\begin{itemize}
\item Broad attention
\item Sentence-level
\item Context gathering
\end{itemize}
\end{columns}

\vspace{8mm}
\keypoint{Different heads learn complementary attention patterns}
\end{frame}

% Slide 13: Positional Encoding Necessity
\begin{frame}{The Position Problem}
\textbf{Attention Has No Notion of Order!}

\vspace{8mm}
Without position information, these are identical to self-attention:
\begin{itemize}
\item "The cat chased the mouse"
\item "The mouse chased the cat"
\item "Cat the chased mouse the"
\end{itemize}

\vspace{8mm}
\warning{Self-attention is permutation invariant - order doesn't matter!}

\vspace{10mm}
\textbf{The Solution: Positional Encoding}
\begin{itemize}
\item Add position information to embeddings
\item Must be deterministic and smooth
\item Should generalize to unseen lengths
\item Preserve relative position information
\end{itemize}

\vspace{8mm}
\keypoint{We inject position information before attention computation}
\end{frame}

% Slide 14: Positional Encoding Mathematics
\begin{frame}{Positional Encoding: Sinusoidal Functions}
\textbf{The Elegant Solution:}

\vspace{8mm}
\formula{
\begin{aligned}
PE_{(pos, 2i)} &= \sin\left(\frac{pos}{10000^{2i/d_{model}}}\right)\\
PE_{(pos, 2i+1)} &= \cos\left(\frac{pos}{10000^{2i/d_{model}}}\right)
\end{aligned}
}

\vspace{8mm}
where:
\begin{itemize}
\item $pos$ = position in sequence (0, 1, 2, ...)
\item $i$ = dimension index
\item $d_{model}$ = model dimension (e.g., 512)
\end{itemize}

\vspace{8mm}
\begin{columns}[T]
\column{0.48\textwidth}
\textbf{Why Sinusoids?}
\begin{itemize}
\item Unique pattern for each position
\item Smooth interpolation
\item Can extrapolate to longer sequences
\item Relative positions computable
\end{itemize}

\column{0.48\textwidth}
\begin{center}
\includegraphics[width=\textwidth]{../figures/positional_encoding.pdf}
\end{center}
\end{columns}

\secondary{\footnotesize Modern models sometimes learn positional embeddings instead}
\end{frame}

% Slide 15: Layer Normalization
\begin{frame}{Layer Normalization: Stabilizing Training}
\textbf{Why Normalization is Critical:}

\vspace{8mm}
\begin{itemize}
\item Deep networks suffer from internal covariate shift
\item Attention can produce varying magnitude outputs
\item Gradients can vanish or explode
\item Training becomes unstable
\end{itemize}

\vspace{10mm}
\textbf{Layer Normalization:}
\formula{\text{LayerNorm}(x) = \gamma \cdot \frac{x - \mu}{\sigma + \epsilon} + \beta}

where:
\begin{itemize}
\item $\mu$ = mean across features for each sample
\item $\sigma$ = standard deviation across features
\item $\gamma, \beta$ = learned scale and shift parameters
\item $\epsilon$ = small constant for numerical stability
\end{itemize}

\vspace{8mm}
\keypoint{Applied after each sub-layer (attention and feed-forward)}
\end{frame}

% Slide 16: Residual Connections
\begin{frame}[fragile]{Residual Connections: Highway for Gradients}
\textbf{The Deep Network Challenge:}

\vspace{5mm}
Transformers are DEEP (12-24+ layers). Without residuals:
\begin{itemize}
\item Gradients vanish exponentially
\item Information gets lost
\item Training becomes impossible
\end{itemize}

\vspace{10mm}
\textbf{Residual Connections to the Rescue:}
\formula{\text{Output} = \text{LayerNorm}(x + \text{Sublayer}(x))}

\vspace{8mm}
\begin{columns}[T]
\column{0.48\textwidth}
\textbf{Benefits:}
\begin{itemize}
\item Direct gradient path
\item Preserves information
\item Enables very deep networks
\item Faster convergence
\end{itemize}

\column{0.48\textwidth}
\textbf{Implementation:}
\begin{lstlisting}[language=Python]
# Attention with residual
attn_out = attention(x)
x = layer_norm(x + attn_out)

# FFN with residual
ffn_out = feed_forward(x)
x = layer_norm(x + ffn_out)
\end{lstlisting}
\end{columns}

\secondary{\footnotesize Every sublayer in the transformer has a residual connection}
\end{frame}

% Slide 17: Complete Transformer Architecture
\begin{frame}{The Complete Transformer Architecture}
\begin{center}
\includegraphics[width=0.85\textwidth]{../figures/architecture_evolution.pdf}
\end{center}

\begin{columns}[T]
\column{0.48\textwidth}
\textbf{Encoder (Left):}
\begin{itemize}
\item 6 identical layers
\item Multi-head self-attention
\item Position-wise feed-forward
\item Residuals \& LayerNorm
\end{itemize}

\column{0.48\textwidth}
\textbf{Decoder (Right):}
\begin{itemize}
\item 6 identical layers
\item Masked self-attention
\item Encoder-decoder attention
\item Position-wise feed-forward
\end{itemize}
\end{columns}

\keypoint{Most modern models use only encoder (BERT) or decoder (GPT)}
\end{frame}

% Slide 18: Encoder Stack Details
\begin{frame}[fragile]{Inside the Encoder}
\textbf{Each Encoder Layer Contains:}

\vspace{8mm}
\begin{enumerate}
\item \textbf{Multi-Head Self-Attention}
\begin{itemize}
\item All positions attend to all positions
\item Parallel computation for all words
\item 8-16 attention heads typically
\end{itemize}

\vspace{5mm}
\item \textbf{Position-wise Feed-Forward Network}
\begin{lstlisting}[language=Python]
FFN(x) = max(0, xW1 + b1)W2 + b2
\end{lstlisting}
\begin{itemize}
\item Applied to each position separately
\item Hidden dimension typically 4x model dimension
\item ReLU or GELU activation
\end{itemize}

\vspace{5mm}
\item \textbf{Residual Connections} around each sublayer
\item \textbf{Layer Normalization} after each sublayer
\end{enumerate}

\vspace{8mm}
\textbf{Dimensions (BERT-base example):}
\begin{itemize}
\item Model dimension: 768
\item Feed-forward dimension: 3072
\item Attention heads: 12
\item Layers: 12
\end{itemize}
\end{frame}

% Slide 19: Decoder Stack Details
\begin{frame}{Inside the Decoder}
\textbf{Three Sublayers per Decoder Layer:}

\vspace{8mm}
\begin{enumerate}
\item \textbf{Masked Multi-Head Self-Attention}
\begin{itemize}
\item Prevents looking at future tokens
\item Ensures autoregressive property
\item Critical for generation tasks
\end{itemize}

\vspace{5mm}
\item \textbf{Encoder-Decoder Attention}
\begin{itemize}
\item Queries from decoder
\item Keys and Values from encoder output
\item Allows decoder to focus on relevant input
\end{itemize}

\vspace{5mm}
\item \textbf{Position-wise Feed-Forward}
\begin{itemize}
\item Same as encoder FFN
\item Independent processing per position
\end{itemize}
\end{enumerate}

\vspace{8mm}
\warning{Masking ensures we can't "cheat" during training by looking ahead}

\secondary{\footnotesize GPT models are decoder-only with masked self-attention}
\end{frame}

% Slide 20: Encoder-Decoder Attention
\begin{frame}{Cross-Attention: Connecting Encoder and Decoder}
\textbf{How Decoder Attends to Encoder:}

\vspace{8mm}
\begin{columns}[T]
\column{0.48\textwidth}
\textbf{The Mechanism:}
\begin{itemize}
\item \highlight{Q} comes from decoder
\item \highlight{K, V} come from encoder
\item Each decoder position can attend to all encoder positions
\item No masking needed here
\end{itemize}

\vspace{8mm}
\formula{\text{CrossAttn}(Q_{dec}, K_{enc}, V_{enc})}

\column{0.48\textwidth}
\textbf{Translation Example:}
\begin{itemize}
\item Input (English): "I love cats"
\item Generating (French): "J'aime les..."
\item When generating "chats":
\begin{itemize}
\item Q from "les" position
\item Attends strongly to "cats"
\item Weakly to other words
\end{itemize}
\end{itemize}
\end{columns}

\vspace{10mm}
\keypoint{Cross-attention allows decoder to focus on relevant input parts}

\secondary{\footnotesize This is how the model aligns source and target sequences}
\end{frame}

% ============================================
% PART 3: IMPLEMENTATION DETAILS
% ============================================

% Slide 21: Section Divider
\begin{frame}
\begin{center}
{\Huge \textbf{Part 3}}\\
\vspace{5mm}
{\Large Implementation Details}\\
\vspace{10mm}
{\large Building Transformers in Practice}
\end{center}
\end{frame}

% Slide 22: Self-Attention Implementation
\begin{frame}[fragile]{Implementing Self-Attention from Scratch}
\begin{lstlisting}[language=Python]
import torch
import torch.nn.functional as F

class SelfAttention(torch.nn.Module):
    def __init__(self, d_model, d_k):
        super().__init__()
        self.W_q = torch.nn.Linear(d_model, d_k)
        self.W_k = torch.nn.Linear(d_model, d_k)
        self.W_v = torch.nn.Linear(d_model, d_k)
        self.scale = d_k ** 0.5

    def forward(self, x):
        # x shape: (batch_size, seq_len, d_model)
        Q = self.W_q(x)  # (batch, seq_len, d_k)
        K = self.W_k(x)
        V = self.W_v(x)

        # Compute attention scores
        scores = torch.matmul(Q, K.transpose(-2, -1)) / self.scale
        # scores shape: (batch, seq_len, seq_len)

        # Apply softmax to get weights
        attn_weights = F.softmax(scores, dim=-1)

        # Apply weights to values
        output = torch.matmul(attn_weights, V)
        return output, attn_weights
\end{lstlisting}

\secondary{\footnotesize Key insight: Matrix multiplication enables parallel computation}
\end{frame}

% Slide 23: Multi-Head Attention Implementation
\begin{frame}[fragile]{Implementing Multi-Head Attention}
\begin{lstlisting}[language=Python]
class MultiHeadAttention(torch.nn.Module):
    def __init__(self, d_model, n_heads):
        super().__init__()
        assert d_model % n_heads == 0
        self.d_k = d_model // n_heads
        self.n_heads = n_heads

        # Linear projections for all heads at once
        self.W_q = torch.nn.Linear(d_model, d_model)
        self.W_k = torch.nn.Linear(d_model, d_model)
        self.W_v = torch.nn.Linear(d_model, d_model)
        self.W_o = torch.nn.Linear(d_model, d_model)

    def forward(self, x, mask=None):
        batch_size, seq_len, d_model = x.shape

        # Project and reshape for multi-head
        Q = self.W_q(x).view(batch_size, seq_len, self.n_heads, self.d_k)
        K = self.W_k(x).view(batch_size, seq_len, self.n_heads, self.d_k)
        V = self.W_v(x).view(batch_size, seq_len, self.n_heads, self.d_k)

        # Transpose for attention: (batch, n_heads, seq_len, d_k)
        Q = Q.transpose(1, 2)
        K = K.transpose(1, 2)
        V = V.transpose(1, 2)

        # Attention for all heads in parallel
        attn_output = scaled_dot_product_attention(Q, K, V, mask)

        # Concatenate heads and project
        concat = attn_output.transpose(1, 2).reshape(batch_size, seq_len, d_model)
        output = self.W_o(concat)
        return output
\end{lstlisting}
\end{frame}

% Slide 24: Positional Encoding Implementation
\begin{frame}[fragile]{Implementing Positional Encoding}
\begin{lstlisting}[language=Python]
import numpy as np

class PositionalEncoding(torch.nn.Module):
    def __init__(self, d_model, max_seq_len=5000):
        super().__init__()

        # Create positional encoding matrix
        pe = torch.zeros(max_seq_len, d_model)
        position = torch.arange(0, max_seq_len).unsqueeze(1).float()

        # Create div_term for the sinusoidal pattern
        div_term = torch.exp(torch.arange(0, d_model, 2).float() *
                           -(np.log(10000.0) / d_model))

        # Apply sin to even indices
        pe[:, 0::2] = torch.sin(position * div_term)
        # Apply cos to odd indices
        pe[:, 1::2] = torch.cos(position * div_term)

        # Register as buffer (not a parameter)
        self.register_buffer('pe', pe.unsqueeze(0))

    def forward(self, x):
        # x shape: (batch_size, seq_len, d_model)
        seq_len = x.size(1)
        # Add positional encoding
        x = x + self.pe[:, :seq_len]
        return x
\end{lstlisting}

\secondary{\footnotesize The sinusoidal pattern allows the model to extrapolate to longer sequences}
\end{frame}

% Slide 25: Feed-Forward Network
\begin{frame}[fragile]{The Feed-Forward Network}
\textbf{Position-wise FFN: Simple but Crucial}

\vspace{8mm}
\begin{lstlisting}[language=Python]
class FeedForward(torch.nn.Module):
    def __init__(self, d_model, d_ff, dropout=0.1):
        super().__init__()
        self.linear1 = torch.nn.Linear(d_model, d_ff)
        self.linear2 = torch.nn.Linear(d_ff, d_model)
        self.dropout = torch.nn.Dropout(dropout)
        self.activation = torch.nn.ReLU()

    def forward(self, x):
        # Expand to higher dimension
        x = self.linear1(x)      # (batch, seq_len, d_ff)
        x = self.activation(x)    # Non-linearity
        x = self.dropout(x)       # Regularization
        # Project back to model dimension
        x = self.linear2(x)       # (batch, seq_len, d_model)
        return x
\end{lstlisting}

\vspace{8mm}
\textbf{Why Important:}
\begin{itemize}
\item Adds non-linearity (attention is linear)
\item Increases model capacity
\item Applied independently to each position
\item Typically $d_{ff} = 4 \times d_{model}$
\end{itemize}
\end{frame}

% Slide 26: Training Considerations
\begin{frame}{Training Transformers: Key Considerations}
\textbf{Optimization Challenges:}

\vspace{8mm}
\begin{columns}[T]
\column{0.48\textwidth}
\textbf{Learning Rate Schedule}
\begin{itemize}
\item Warmup is critical
\item Linear increase for first steps
\item Then decay by $1/\sqrt{step}$
\item Prevents early instability
\end{itemize}

\vspace{8mm}
\formula{lr = d_{model}^{-0.5} \cdot \min(step^{-0.5}, step \cdot warmup^{-1.5})}

\column{0.48\textwidth}
\textbf{Regularization}
\begin{itemize}
\item Dropout (typically 0.1)
\item Weight decay
\item Label smoothing
\item Gradient clipping
\end{itemize}

\vspace{8mm}
\textbf{Batch Size}
\begin{itemize}
\item Larger is better (if fits)
\item Gradient accumulation
\item Typically 4k-64k tokens
\end{itemize}
\end{columns}

\vspace{10mm}
\warning{Training is unstable without proper warmup and initialization}
\end{frame}

% Slide 27: Computational Complexity
\begin{frame}{Computational Complexity Analysis}
\textbf{Time and Space Complexity:}

\vspace{8mm}
\begin{center}
\begin{tabular}{|l|c|c|c|}
\hline
\textbf{Component} & \textbf{Time} & \textbf{Space} & \textbf{Bottleneck} \\
\hline
Self-Attention & $O(n^2 \cdot d)$ & $O(n^2)$ & Quadratic in length \\
Feed-Forward & $O(n \cdot d^2)$ & $O(1)$ & Linear in length \\
Multi-Head Proj & $O(n \cdot d^2)$ & $O(1)$ & Model dimension \\
\hline
Total per Layer & $O(n^2 \cdot d + n \cdot d^2)$ & $O(n^2)$ & \\
\hline
\end{tabular}
\end{center}

where $n$ = sequence length, $d$ = model dimension

\vspace{10mm}
\textbf{Practical Implications:}
\begin{itemize}
\item Memory grows quadratically with sequence length
\item Long sequences (>2k tokens) become problematic
\item Modern solutions: Sparse attention, Flash Attention
\item Trade-off: Parallelization vs memory usage
\end{itemize}

\vspace{8mm}
\keypoint{The $O(n^2)$ attention is both the strength and limitation}
\end{frame}

% Slide 28: Memory Requirements
\begin{frame}{Memory Requirements in Practice}
\textbf{Where Does the Memory Go?}

\vspace{8mm}
For BERT-base (110M parameters):
\begin{itemize}
\item \textbf{Model Parameters:} 440 MB (fp32)
\item \textbf{Optimizer States:} 880 MB (Adam - 2x params)
\item \textbf{Gradients:} 440 MB
\item \textbf{Activations:} Depends on batch size and sequence length
\end{itemize}

\vspace{10mm}
\textbf{Activation Memory for Batch Size 32, Seq Length 512:}
\begin{itemize}
\item Attention matrices: $32 \times 12 \times 512 \times 512 \times 4$ bytes = 402 MB
\item FFN activations: $32 \times 512 \times 3072 \times 4$ bytes = 201 MB
\item Per layer: ~600 MB
\item 12 layers: ~7.2 GB just for activations!
\end{itemize}

\vspace{8mm}
\warning{Memory, not compute, is often the limiting factor}

\secondary{\footnotesize Techniques: Gradient checkpointing, mixed precision, model parallelism}
\end{frame}

% Slide 29: Attention Patterns Visualization
\begin{frame}{Debugging Attention Patterns}
\textbf{What Should Attention Look Like?}

\vspace{5mm}
\begin{center}
\includegraphics[width=0.9\textwidth]{../figures/attention_visualization.pdf}
\end{center}

\vspace{5mm}
\textbf{Common Patterns to Check:}
\begin{columns}[T]
\column{0.48\textwidth}
\textbf{Good Patterns:}
\begin{itemize}
\item Diagonal for local context
\item Specific word alignments
\item Syntax dependencies visible
\item Diverse across heads
\end{itemize}

\column{0.48\textwidth}
\textbf{Problems to Watch:}
\begin{itemize}
\item Uniform attention (not learning)
\item Only attending to [CLS]/[SEP]
\item Completely random patterns
\item All heads look identical
\end{itemize}
\end{columns}

\keypoint{Visualizing attention helps debug training issues}
\end{frame}

% Slide 30: Common Implementation Pitfalls
\begin{frame}{Common Implementation Pitfalls}
\textbf{Mistakes That Cost Days of Debugging:}

\vspace{8mm}
\begin{enumerate}
\item \textbf{Forgetting the Scaling Factor}
\begin{itemize}
\item Without $1/\sqrt{d_k}$: Softmax saturates
\item Gradients vanish, model doesn't learn
\end{itemize}

\vspace{5mm}
\item \textbf{Wrong Mask Implementation}
\begin{itemize}
\item Must add -inf BEFORE softmax
\item Boolean mask vs additive mask confusion
\end{itemize}

\vspace{5mm}
\item \textbf{Positional Encoding Bugs}
\begin{itemize}
\item Not adding to embeddings
\item Wrong dimension ordering
\item Not handling variable lengths
\end{itemize}

\vspace{5mm}
\item \textbf{Layer Norm Placement}
\begin{itemize}
\item Pre-norm vs post-norm matters
\item Modern models use pre-norm for stability
\end{itemize}
\end{enumerate}

\vspace{8mm}
\warning{Always verify shapes and attention patterns early in training}
\end{frame}

% ============================================
% PART 4: APPLICATIONS & IMPACT
% ============================================

% Slide 31: Section Divider
\begin{frame}
\begin{center}
{\Huge \textbf{Part 4}}\\
\vspace{5mm}
{\Large Applications \& Impact}\\
\vspace{10mm}
{\large From BERT to GPT-4}
\end{center}
\end{frame}

% Slide 32: BERT Architecture
\begin{frame}{BERT: Bidirectional Encoder Representations}
\textbf{The Encoder-Only Revolution (2018):}

\vspace{8mm}
\begin{columns}[T]
\column{0.48\textwidth}
\textbf{Architecture:}
\begin{itemize}
\item Encoder-only transformer
\item Bidirectional context
\item 12/24 layers (Base/Large)
\item 110M/340M parameters
\end{itemize}

\vspace{8mm}
\textbf{Training Objective:}
\begin{itemize}
\item Masked Language Modeling (MLM)
\item Randomly mask 15\% of tokens
\item Predict masked tokens
\item Next Sentence Prediction (NSP)
\end{itemize}

\column{0.48\textwidth}
\textbf{Key Innovations:}
\begin{itemize}
\item Bidirectional pre-training
\item Fine-tuning paradigm
\item [CLS] token for classification
\item Segment embeddings
\end{itemize}

\vspace{8mm}
\textbf{Impact:}
\begin{itemize}
\item SOTA on 11 NLP tasks
\item Powers Google Search
\item Started fine-tuning era
\item 50,000+ citations
\end{itemize}
\end{columns}

\vspace{8mm}
\keypoint{BERT showed that pre-training + fine-tuning is incredibly powerful}
\end{frame}

% Slide 33: GPT Architecture
\begin{frame}{GPT: The Decoder-Only Approach}
\textbf{Generative Pre-trained Transformer Evolution:}

\vspace{8mm}
\begin{columns}[T]
\column{0.48\textwidth}
\textbf{Architecture:}
\begin{itemize}
\item Decoder-only (masked attention)
\item Autoregressive generation
\item Left-to-right only
\item Scales to extreme sizes
\end{itemize}

\vspace{8mm}
\textbf{Model Progression:}
\begin{itemize}
\item GPT (2018): 117M params
\item GPT-2 (2019): 1.5B params
\item GPT-3 (2020): 175B params
\item GPT-4 (2023): ~1.7T params*
\end{itemize}

\column{0.48\textwidth}
\textbf{Training Approach:}
\begin{itemize}
\item Next token prediction
\item Massive text corpora
\item No fine-tuning needed
\item In-context learning
\end{itemize}

\vspace{8mm}
\textbf{Capabilities:}
\begin{itemize}
\item Text generation
\item Few-shot learning
\item Code generation
\item Reasoning tasks
\end{itemize}
\end{columns}

\vspace{8mm}
\keypoint{GPT proved that scale + simple objective = emergent abilities}

\secondary{\footnotesize *Estimated; OpenAI hasn't disclosed GPT-4 size}
\end{frame}

% Slide 34: Encoder vs Decoder Comparison
\begin{frame}{Encoder-Only vs Decoder-Only vs Both}
\begin{center}
\textbf{Which Architecture When?}
\end{center}

\vspace{8mm}
\begin{tabular}{|l|c|c|c|}
\hline
\textbf{Aspect} & \textbf{Encoder-Only} & \textbf{Decoder-Only} & \textbf{Both} \\
\hline
Example & BERT, RoBERTa & GPT, LLaMA & T5, BART \\
\hline
Context & Bidirectional & Left-to-right & Flexible \\
Best for & Understanding & Generation & Translation \\
Training & MLM & Next token & Varies \\
Fine-tuning & Required & Optional & Optional \\
Speed & Fast inference & Slower (sequential) & Slowest \\
Parameters & Smaller & Can be huge & Medium \\
\hline
\end{tabular}

\vspace{10mm}
\textbf{Use Cases:}
\begin{columns}[T]
\column{0.32\textwidth}
\textbf{Encoder-Only:}
\begin{itemize}
\item Classification
\item NER, POS tagging
\item Sentiment analysis
\item Search/retrieval
\end{itemize}

\column{0.32\textwidth}
\textbf{Decoder-Only:}
\begin{itemize}
\item Text generation
\item Dialogue systems
\item Code completion
\item Creative writing
\end{itemize}

\column{0.32\textwidth}
\textbf{Encoder-Decoder:}
\begin{itemize}
\item Translation
\item Summarization
\item Question answering
\item Text-to-text tasks
\end{itemize}
\end{columns}
\end{frame}

% Slide 35: Vision Transformers
\begin{frame}{Beyond Text: Vision Transformers (ViT)}
\textbf{Transformers Conquer Computer Vision (2020):}

\vspace{8mm}
\begin{columns}[T]
\column{0.48\textwidth}
\textbf{Key Idea:}
\begin{itemize}
\item Divide image into patches (16x16)
\item Treat patches as "tokens"
\item Add positional embeddings
\item Standard transformer encoder
\item [CLS] token for classification
\end{itemize}

\vspace{8mm}
\textbf{Results:}
\begin{itemize}
\item Beats CNNs at scale
\item ImageNet: 88.5\% accuracy
\item Needs lots of data
\item Better transfer learning
\end{itemize}

\column{0.48\textwidth}
\textbf{Why It Works:}
\begin{itemize}
\item Global receptive field from start
\item Flexible attention patterns
\item No inductive bias of convolution
\item Scales better than CNNs
\end{itemize}

\vspace{8mm}
\textbf{Extensions:}
\begin{itemize}
\item CLIP: Vision + Language
\item DALL-E: Text to Image
\item Flamingo: Visual QA
\item SAM: Segmentation
\end{itemize}
\end{columns}

\vspace{8mm}
\keypoint{Transformers are becoming the universal architecture for all modalities}
\end{frame}

% Slide 36: Multimodal Transformers
\begin{frame}{Multimodal Transformers: Unified Architecture}
\textbf{One Architecture to Process Everything:}

\vspace{8mm}
\begin{center}
\includegraphics[width=0.8\textwidth]{../figures/speed_comparison.pdf}
\end{center}

\vspace{8mm}
\textbf{Current Multimodal Models:}
\begin{columns}[T]
\column{0.32\textwidth}
\textbf{CLIP (OpenAI)}
\begin{itemize}
\item Text + Image
\item Contrastive learning
\item Zero-shot classification
\end{itemize}

\column{0.32\textwidth}
\textbf{Flamingo (DeepMind)}
\begin{itemize}
\item Vision + Language
\item Few-shot learning
\item Visual QA
\end{itemize}

\column{0.32\textwidth}
\textbf{Gato (DeepMind)}
\begin{itemize}
\item Text, images, games
\item Single model
\item General agent
\end{itemize}
\end{columns}

\vspace{8mm}
\keypoint{Transformers enable true multimodal AI systems}
\end{frame}

% Slide 37: Performance Benchmarks
\begin{frame}{Performance Evolution: Transformers Dominate}
\begin{center}
\includegraphics[width=0.9\textwidth]{../figures/week5_benchmark_timeline.pdf}
\end{center}

\textbf{Key Benchmarks Conquered:}
\begin{itemize}
\item \textbf{GLUE/SuperGLUE:} Human performance surpassed (2019)
\item \textbf{SQuAD 2.0:} Reading comprehension solved (2019)
\item \textbf{WMT Translation:} Near-human quality (2020)
\item \textbf{ImageNet:} ViT beats ResNet (2021)
\item \textbf{HumanEval:} Code generation >80\% (2023)
\end{itemize}

\secondary{\footnotesize Every major NLP benchmark is now led by transformer models}
\end{frame}

% Slide 38: Scaling Laws
\begin{frame}{Scaling Laws: Bigger is (Usually) Better}
\textbf{The Kaplan Scaling Laws (2020):}

\vspace{8mm}
\formula{\text{Loss} = aN^{-\alpha} + bD^{-\beta} + c}

where $N$ = parameters, $D$ = data, $\alpha \approx 0.076$, $\beta \approx 0.095$

\vspace{10mm}
\begin{columns}[T]
\column{0.48\textwidth}
\textbf{Key Findings:}
\begin{itemize}
\item Performance improves predictably
\item No plateau in sight
\item Data and compute equally important
\item Optimal model size grows with budget
\end{itemize}

\column{0.48\textwidth}
\begin{center}
\includegraphics[width=\textwidth]{../figures/week5_efficiency_comparison.pdf}
\end{center}
\end{columns}

\vspace{8mm}
\keypoint{Scaling laws justify the race to build larger models}

\secondary{\footnotesize Chinchilla showed we also need proportionally more data}
\end{frame}

% Slide 39: Modern Variants
\begin{frame}{Modern Transformer Variants}
\textbf{Innovations to Address Limitations:}

\vspace{8mm}
\begin{columns}[T]
\column{0.48\textwidth}
\textbf{Efficiency Improvements:}
\begin{itemize}
\item \textbf{Flash Attention:} 2-4x faster, less memory
\item \textbf{Sparse Transformers:} $O(n\sqrt{n})$ complexity
\item \textbf{Linformer:} $O(n)$ attention
\item \textbf{Performer:} Kernel-based attention
\end{itemize}

\vspace{8mm}
\textbf{Architectural Changes:}
\begin{itemize}
\item \textbf{Mixture of Experts:} Conditional computation
\item \textbf{Retrieval-Augmented:} External memory
\item \textbf{Toolformer:} API calling ability
\end{itemize}

\column{0.48\textwidth}
\textbf{Training Innovations:}
\begin{itemize}
\item \textbf{RoPE:} Better positional encoding
\item \textbf{ALiBi:} Extrapolate to longer sequences
\item \textbf{FlashAttention-2:} Even faster
\item \textbf{GQA:} Grouped query attention
\end{itemize}

\vspace{8mm}
\textbf{Current State-of-the-Art:}
\begin{itemize}
\item LLaMA 2 architecture
\item RMSNorm instead of LayerNorm
\item SwiGLU activation
\item Rotary embeddings
\end{itemize}
\end{columns}

\vspace{8mm}
\keypoint{Innovation continues at breakneck pace}
\end{frame}

% Slide 40: Real-World Applications
\begin{frame}{Real-World Impact: Transformers Everywhere}
\textbf{Applications Powered by Transformers (2024):}

\vspace{8mm}
\begin{columns}[T]
\column{0.24\textwidth}
\textbf{Language:}
\begin{itemize}
\item ChatGPT
\item Google Bard
\item Claude
\item DeepL
\item Grammarly
\end{itemize}

\column{0.24\textwidth}
\textbf{Code:}
\begin{itemize}
\item GitHub Copilot
\item Replit AI
\item Amazon CodeWhisperer
\item Tabnine
\end{itemize}

\column{0.24\textwidth}
\textbf{Creative:}
\begin{itemize}
\item DALL-E 3
\item Midjourney
\item Stable Diffusion
\item MuseNet
\end{itemize}

\column{0.24\textwidth}
\textbf{Science:}
\begin{itemize}
\item AlphaFold
\item ESMFold
\item Galactica
\item BioGPT
\end{itemize}
\end{columns}

\vspace{10mm}
\textbf{Industry Adoption:}
\begin{itemize}
\item \textbf{Microsoft:} Copilot in Office 365
\item \textbf{Google:} Bard, Search, Workspace
\item \textbf{Meta:} LLaMA, Make-A-Video
\item \textbf{Every Tech Company:} Building transformer-based products
\end{itemize}

\vspace{8mm}
\keypoint{Transformers are the foundation of the AI revolution}
\end{frame}

% ============================================
% PART 5: HANDS-ON & SUMMARY
% ============================================

% Slide 41: Section Divider
\begin{frame}
\begin{center}
{\Huge \textbf{Part 5}}\\
\vspace{5mm}
{\Large Hands-On \& Summary}\\
\vspace{10mm}
{\large Bringing It All Together}
\end{center}
\end{frame}

% Slide 42: Building a Mini-Transformer
\begin{frame}[fragile]{Let's Build a Complete Mini-Transformer}
\begin{lstlisting}[language=Python, basicstyle=\tiny]
class TransformerBlock(nn.Module):
    def __init__(self, d_model, n_heads, d_ff, dropout=0.1):
        super().__init__()
        self.attention = MultiHeadAttention(d_model, n_heads)
        self.norm1 = nn.LayerNorm(d_model)
        self.ff = FeedForward(d_model, d_ff, dropout)
        self.norm2 = nn.LayerNorm(d_model)
        self.dropout = nn.Dropout(dropout)

    def forward(self, x, mask=None):
        # Self-attention with residual
        attn_out = self.attention(x, mask)
        x = self.norm1(x + self.dropout(attn_out))
        # Feed-forward with residual
        ff_out = self.ff(x)
        x = self.norm2(x + self.dropout(ff_out))
        return x

class MiniTransformer(nn.Module):
    def __init__(self, vocab_size, d_model, n_layers, n_heads, max_len=5000):
        super().__init__()
        self.embedding = nn.Embedding(vocab_size, d_model)
        self.pos_encoding = PositionalEncoding(d_model, max_len)
        self.blocks = nn.ModuleList([
            TransformerBlock(d_model, n_heads, d_model*4)
            for _ in range(n_layers)
        ])
        self.ln_f = nn.LayerNorm(d_model)
        self.output = nn.Linear(d_model, vocab_size)

    def forward(self, x):
        x = self.embedding(x)
        x = self.pos_encoding(x)
        for block in self.blocks:
            x = block(x)
        x = self.ln_f(x)
        return self.output(x)
\end{lstlisting}

\secondary{\footnotesize This is essentially GPT-2 architecture!}
\end{frame}

% Slide 43: Key Takeaways
\begin{frame}{Key Takeaways}
\begin{center}
{\Large \textbf{What to Remember}}
\end{center}

\vspace{10mm}
\begin{enumerate}
\item \textbf{Core Innovation:} Self-attention enables parallel processing
\vspace{3mm}
\item \textbf{Key Components:}
\begin{itemize}
\item Multi-head attention (different relationships)
\item Positional encoding (order information)
\item FFN (non-linearity and capacity)
\item Residuals \& LayerNorm (training stability)
\end{itemize}
\vspace{3mm}
\item \textbf{Advantages:} Parallelization, long-range dependencies, transfer learning
\vspace{3mm}
\item \textbf{Limitations:} Quadratic complexity, memory intensive
\vspace{3mm}
\item \textbf{Impact:} Powers essentially all modern NLP and beyond
\vspace{3mm}
\item \textbf{Future:} Longer context, efficiency, multimodal, reasoning
\end{enumerate}

\vspace{10mm}
\keypoint{Transformers are the foundation you need to understand modern AI}
\end{frame}

% Slide 44: Common Misconceptions
\begin{frame}{Common Misconceptions}
\textbf{Let's Clear These Up:}

\vspace{10mm}
\begin{enumerate}
\item \misconception{"Transformers are just attention"}\\
\success{Reality:} FFN, residuals, and layer norm are equally critical

\vspace{5mm}
\item \misconception{"Bigger is always better"}\\
\success{Reality:} Depends on task, data, and computational budget

\vspace{5mm}
\item \misconception{"Transformers understand language"}\\
\success{Reality:} They learn statistical patterns, not meaning

\vspace{5mm}
\item \misconception{"Attention weights show what the model thinks"}\\
\success{Reality:} They're one part of a complex computation

\vspace{5mm}
\item \misconception{"Transformers made RNNs obsolete"}\\
\success{Reality:} RNNs still useful for streaming, small models
\end{enumerate}

\vspace{10mm}
\warning{Understanding limitations is as important as understanding capabilities}
\end{frame}

% Slide 45: Practice Problems
\begin{frame}{Practice Problems}
\textbf{Test Your Understanding:}

\vspace{8mm}
\begin{enumerate}
\item \textbf{Computation:} For sequence length 100, model dimension 512, 8 heads:
\begin{itemize}
\item What's the dimension of each head?
\item How many parameters in multi-head attention?
\item Memory for attention matrices?
\end{itemize}

\vspace{5mm}
\item \textbf{Architecture:} Why do we need positional encoding? What happens without it?

\vspace{5mm}
\item \textbf{Implementation:} Write the attention masking for decoder self-attention

\vspace{5mm}
\item \textbf{Analysis:} Why is the scaling factor $\sqrt{d_k}$ and not $d_k$?

\vspace{5mm}
\item \textbf{Design:} How would you modify transformers for 10k+ token sequences?
\end{enumerate}

\vspace{10mm}
\secondary{\footnotesize Solutions will be discussed in the lab session}
\end{frame}

% Slide 46: Resources and References
\begin{frame}{Resources for Deep Dive}
\textbf{Essential Papers:}
\begin{itemize}
\item Vaswani et al. (2017): "Attention Is All You Need" - The original
\item Devlin et al. (2018): "BERT: Pre-training of Deep Bidirectional Transformers"
\item Radford et al. (2019): "Language Models are Unsupervised Multitask Learners" (GPT-2)
\item Brown et al. (2020): "Language Models are Few-Shot Learners" (GPT-3)
\item Dosovitskiy et al. (2020): "An Image is Worth 16x16 Words" (ViT)
\end{itemize}

\vspace{8mm}
\textbf{Implementation Resources:}
\begin{itemize}
\item \textbf{The Annotated Transformer:} Harvard NLP's line-by-line implementation
\item \textbf{HuggingFace Transformers:} Industry standard library
\item \textbf{nanoGPT:} Karpathy's minimal GPT implementation
\item \textbf{x-transformers:} Cutting-edge variants
\end{itemize}

\vspace{8mm}
\textbf{Visualization Tools:}
\begin{itemize}
\item BertViz: Attention visualization
\item Tensor2Tensor: Interactive transformer
\item Papers with Code: Benchmark tracking
\end{itemize}
\end{frame}

% Final slide
\begin{frame}
\begin{center}
{\Huge \textbf{Thank You!}}\\
\vspace{10mm}
{\Large Questions?}\\
\vspace{15mm}
{\large Next: Lab Session - Implementing Transformers}\\
\vspace{5mm}
\secondary{\footnotesize We'll build and train a transformer from scratch}
\end{center}
\end{frame}

\end{document}