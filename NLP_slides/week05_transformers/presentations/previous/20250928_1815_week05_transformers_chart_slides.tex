% Week 5: Transformers - The Speed Revolution (Individual Chart Slides)
% NLP Course 2025
% Created: 2025-09-28 18:15
% Each conceptual chart on its own slide with brief explanation

\documentclass[8pt,aspectratio=169]{beamer}
\usetheme{Madrid}
\usecolortheme{seahorse}
\setbeamertemplate{navigation symbols}{}

% Dynamic color scheme for each act
\definecolor{act1red}{RGB}{214,39,40}
\definecolor{act1orange}{RGB}{255,127,14}
\definecolor{act2blue}{RGB}{31,119,180}
\definecolor{act2gray}{RGB}{150,150,150}
\definecolor{act3green}{RGB}{44,160,44}
\definecolor{act3teal}{RGB}{23,190,207}
\definecolor{act4gold}{RGB}{255,215,0}
\definecolor{act4purple}{RGB}{148,0,211}

% Apply Madrid theme customization
\setbeamercolor{palette primary}{bg=act3teal!30,fg=black}
\setbeamercolor{palette secondary}{bg=act3teal!20,fg=black}
\setbeamercolor{palette tertiary}{bg=act3teal!40,fg=white}
\setbeamercolor{palette quaternary}{bg=act3teal,fg=white}
\setbeamercolor{structure}{fg=act3teal!80!black}
\setbeamercolor{frametitle}{fg=white,bg=act3teal!80!black}

% Custom commands
\newcommand{\conceptnote}[1]{%
\vspace{3mm}
\begin{center}
\colorbox{yellow!20}{\parbox{0.8\textwidth}{\centering\small\textbf{#1}}}
\end{center}
}

% Math and code support
\usepackage{amsmath,amssymb}
\usepackage{graphicx}
\usepackage{listings}

\title{Natural Language Processing}
\subtitle{Week 5: The Speed Revolution - Conceptual Journey}
\author{From Sequential Bottlenecks to Parallel Breakthroughs}
\date{NLP Course 2025}

\begin{document}

% Title slide
\begin{frame}
\titlepage
\end{frame}

% Overview slide
\begin{frame}{Visual Journey Through the Transformer Revolution}
\textbf{12 Conceptual Visualizations in 4 Acts}

\vspace{5mm}
\begin{columns}
\column{0.48\textwidth}
\textbf{Act 1: The Waiting Game}
\begin{itemize}
\item Chart 1: Domino Effect
\item Chart 2: Traffic Jam
\item Chart 3: Assembly Line
\end{itemize}

\vspace{3mm}
\textbf{Act 2: The Disappointment}
\begin{itemize}
\item Chart 4: Memory Maze
\item Chart 5: Broken Telegraph
\item Chart 6: Computational Quicksand
\end{itemize}

\column{0.48\textwidth}
\textbf{Act 3: The Breakthrough}
\begin{itemize}
\item Chart 7: Attention Theatre
\item Chart 8: Circuit Board
\item Chart 9: Parallel Universe
\end{itemize}

\vspace{3mm}
\textbf{Act 4: The Impact}
\begin{itemize}
\item Chart 10: Language Galaxy
\item Chart 11: Evolution Tree
\item Chart 12: Scaling Rocket
\end{itemize}
\end{columns}
\end{frame}

% ============================================
% ACT 1: THE WAITING GAME
% ============================================

\section{Act 1: The Waiting Game}

% Act 1 Title Slide
\begin{frame}
\vfill
\centering
\Huge\textcolor{act1red}{\textbf{Act 1: The Waiting Game}}
\vfill
\Large\textit{When Sequential Processing Becomes the Bottleneck}
\vfill
\end{frame}

% Chart 1: Domino Effect
\begin{frame}{Chart 1: The Domino Effect}
\centering
\includegraphics[width=0.9\textwidth]{../figures/new_01_domino_effect.pdf}

\conceptnote{RNNs process tokens like falling dominos - each must wait for the previous. Transformers process all tokens simultaneously, like dominos standing independently.}
\end{frame}

% Chart 2: Traffic Jam
\begin{frame}{Chart 2: Traffic Jam Visualization}
\centering
\includegraphics[width=0.8\textwidth]{../figures/new_02_traffic_visualization.pdf}

\conceptnote{RNNs = Single lane highway causing bottlenecks. Transformers = 8-lane highway with parallel flow. Processing speed increases from 1 token/cycle to 8 tokens/cycle!}
\end{frame}

% Chart 3: Assembly Line
\begin{frame}{Chart 3: The Assembly Line Problem}
\centering
\includegraphics[width=0.85\textwidth]{../figures/new_03_assembly_line.pdf}

\conceptnote{Traditional assembly line (RNN) has one worker processing tokens sequentially. Modern parallel factory (Transformer) has multiple workers processing all tokens simultaneously.}
\end{frame}

% ============================================
% ACT 2: THE DISAPPOINTMENT
% ============================================

\section{Act 2: The Disappointment}

% Act 2 Title Slide
\begin{frame}
\vfill
\centering
\Huge\textcolor{act2blue}{\textbf{Act 2: The Disappointment}}
\vfill
\Large\textit{Why Sequential Models Fail at Scale}
\vfill
\end{frame}

% Chart 4: Memory Maze
\begin{frame}{Chart 4: The Memory Maze}
\centering
\includegraphics[width=0.75\textwidth]{../figures/new_04_memory_maze.pdf}

\conceptnote{Information gets lost navigating through sequential layers. The longer the path, the more the signal degrades - a fundamental limitation of RNNs.}
\end{frame}

% Chart 5: Broken Telegraph
\begin{frame}{Chart 5: The Broken Telegraph}
\centering
\includegraphics[width=0.9\textwidth]{../figures/new_05_broken_telegraph.pdf}

\conceptnote{Like a game of telephone, sequential processing accumulates errors. "The quick brown fox" becomes noise after passing through multiple RNN layers.}
\end{frame}

% Chart 6: Computational Quicksand
\begin{frame}{Chart 6: Computational Quicksand}
\centering
\includegraphics[width=0.7\textwidth]{../figures/new_06_computational_quicksand.pdf}

\conceptnote{Vanishing gradients: The deeper you go, the weaker the signal. By layer 10, gradient magnitude drops to 0.007 - learning becomes impossible.}
\end{frame}

% ============================================
% ACT 3: THE BREAKTHROUGH
% ============================================

\section{Act 3: The Breakthrough}

% Act 3 Title Slide
\begin{frame}
\vfill
\centering
\Huge\textcolor{act3green}{\textbf{Act 3: The Breakthrough}}
\vfill
\Large\textit{Parallel Attention Changes Everything}
\vfill
\end{frame}

% Chart 7: Attention Theatre
\begin{frame}{Chart 7: The Attention Spotlight Theatre}
\centering
\includegraphics[width=0.85\textwidth]{../figures/new_07_attention_theatre.pdf}

\conceptnote{Multi-head attention is like multiple spotlights on a stage. Each head focuses on different relationships - subject, verb, object - all simultaneously.}
\end{frame}

% Chart 8: Circuit Board
\begin{frame}{Chart 8: The Neural Network Circuit Board}
\centering
\includegraphics[width=0.9\textwidth]{../figures/new_08_circuit_board.pdf}

\conceptnote{RNNs = Serial circuit with sequential resistance. Transformers = Parallel circuit with direct connections. Information flows without bottlenecks.}
\end{frame}

% Chart 9: Parallel Universe
\begin{frame}{Chart 9: The Parallel Universe Portal}
\centering
\includegraphics[width=0.85\textwidth]{../figures/new_09_parallel_universe.pdf}

\conceptnote{Time dilation effect: Sequential processing in O(n) time vs parallel processing in O(1). What takes 8 seconds sequentially happens instantly in parallel!}
\end{frame}

% ============================================
% ACT 4: THE IMPACT
% ============================================

\section{Act 4: The Impact}

% Act 4 Title Slide
\begin{frame}
\vfill
\centering
\Huge\textcolor{act4gold}{\textbf{Act 4: The Impact}}
\vfill
\Large\textit{How Transformers Changed Everything}
\vfill
\end{frame}

% Chart 10: Language Galaxy
\begin{frame}{Chart 10: The Language Galaxy}
\centering
\includegraphics[width=0.75\textwidth]{../figures/new_10_language_galaxy.pdf}

\conceptnote{Transformers create universal bridges between all languages. Attention mechanisms connect distant language "star systems" enabling unprecedented multilingual understanding.}
\end{frame}

% Chart 11: Evolution Tree
\begin{frame}{Chart 11: The AI Evolution Tree}
\centering
\includegraphics[width=0.8\textwidth]{../figures/new_11_evolution_tree.pdf}

\conceptnote{From the Transformer trunk (2017), an entire ecosystem emerged: BERT, GPT, T5, PaLM. Each branch represents billions of parameters and new capabilities.}
\end{frame}

% Chart 12: Scaling Rocket
\begin{frame}{Chart 12: The Scaling Rocket}
\centering
\includegraphics[width=0.75\textwidth]{../figures/new_12_scaling_rocket.pdf}

\conceptnote{Exponential growth trajectory: 65M (2017) → 340M (2018) → 1.5B (2019) → 175B (2020) → 1.7T (2023). The rocket keeps accelerating!}
\end{frame}

% Summary Slide
\begin{frame}{The Speed Revolution: Summary}
\textbf{From Sequential Bottlenecks to Parallel Breakthroughs}

\vspace{5mm}
\begin{columns}
\column{0.48\textwidth}
\textbf{The Problem (Acts 1-2)}
\begin{itemize}
\item Sequential processing = waiting game
\item Single lane bottlenecks
\item Information degradation
\item Vanishing gradients
\item O(n) complexity
\end{itemize}

\column{0.48\textwidth}
\textbf{The Solution (Acts 3-4)}
\begin{itemize}
\item Parallel attention = instant processing
\item Multi-lane information highways
\item Direct connections preserve signal
\item Stable gradient flow
\item O(1) complexity
\end{itemize}
\end{columns}

\vspace{5mm}
\begin{center}
\colorbox{act3green!20}{\parbox{0.8\textwidth}{\centering\textbf{Result: 100x speedup enabled ChatGPT, Claude, and modern AI}}}
\end{center}
\end{frame}

% Technical Details Slide
\begin{frame}{Want the Technical Details?}
\textbf{Key Transformer Innovations}

\vspace{3mm}
\begin{enumerate}
\item \textbf{Self-Attention Mechanism}
   \begin{itemize}
   \item Query, Key, Value matrices
   \item Attention scores: $\text{Attention}(Q,K,V) = \text{softmax}\left(\frac{QK^T}{\sqrt{d_k}}\right)V$
   \end{itemize}

\item \textbf{Multi-Head Attention}
   \begin{itemize}
   \item 8 parallel attention operations
   \item Different representation subspaces
   \end{itemize}

\item \textbf{Positional Encoding}
   \begin{itemize}
   \item Sine/cosine waves: $PE_{(pos,2i)} = \sin(pos/10000^{2i/d})$
   \item Adds position information to embeddings
   \end{itemize}

\item \textbf{Parallelization}
   \begin{itemize}
   \item All positions processed simultaneously
   \item GPU utilization: 2\% → 92\%
   \end{itemize}
\end{enumerate}

\vspace{3mm}
\centering
\textit{But the conceptual understanding comes first!}
\end{frame}

% Final Slide
\begin{frame}
\centering
\vfill
\Huge\textbf{Questions?}
\vfill
\Large\textit{The revolution wasn't just technical -\\it was conceptual}
\vfill
\normalsize
These visualizations demonstrate how thinking differently\\
about the problem led to a 100x speedup
\vfill
\end{frame}

\end{document}