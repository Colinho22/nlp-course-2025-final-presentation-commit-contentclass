% Week 6: Pre-trained Models (BERT & GPT)
% BSc Discovery-Based Presentation using Educational Framework
% Complete rewrite with $1M cost revolution hook and problem-driven narrative

% Master Template for NLP Course
% Optimal Readability Layout Standard
% All presentations should include this template

\documentclass[8pt,aspectratio=169]{beamer}
\usetheme{Madrid}
\setbeamertemplate{navigation symbols}{}

% ====================================
% OPTIMAL READABILITY COLOR PALETTE
% ====================================
\definecolor{PureBlack}{HTML}{000000}      % Main text (21:1 contrast)
\definecolor{DeepBlue}{HTML}{003D7A}       % Primary accent (12.6:1 contrast)
\definecolor{DarkGray}{HTML}{4A4A4A}       % Secondary text (9.7:1 contrast)
\definecolor{LightGray}{HTML}{E5E5E5}      % Borders and grids
\definecolor{ChartBlue}{HTML}{0066CC}      % Chart primary
\definecolor{ChartOrange}{HTML}{FF8800}    % Chart secondary
\definecolor{ChartTeal}{HTML}{00A0A0}      % Chart tertiary
\definecolor{ChartPurple}{HTML}{8B4789}    % Chart quaternary
\definecolor{DarkGreen}{HTML}{228B22}      % Success/positive
\definecolor{DarkRed}{HTML}{CC0000}        % Warning/negative

% ====================================
% BEAMER COLOR CONFIGURATION
% ====================================
\setbeamercolor{structure}{fg=PureBlack}
\setbeamercolor{frametitle}{fg=PureBlack,bg=white}
\setbeamercolor{title}{fg=PureBlack,bg=white}
\setbeamercolor{subtitle}{fg=DarkGray}
\setbeamercolor{author}{fg=DarkGray}
\setbeamercolor{date}{fg=DarkGray}
\setbeamercolor{institute}{fg=DarkGray}

% Blocks - no backgrounds
\setbeamercolor{block title}{fg=PureBlack,bg=white}
\setbeamercolor{block body}{fg=PureBlack,bg=white}
\setbeamercolor{block title example}{fg=DarkGreen,bg=white}
\setbeamercolor{block body example}{fg=PureBlack,bg=white}
\setbeamercolor{block title alerted}{fg=DarkRed,bg=white}
\setbeamercolor{block body alerted}{fg=PureBlack,bg=white}

% Lists
\setbeamercolor{item}{fg=DeepBlue}
\setbeamercolor{subitem}{fg=ChartBlue}
\setbeamercolor{enumerate item}{fg=DeepBlue}
\setbeamercolor{enumerate subitem}{fg=ChartBlue}

% Text
\setbeamercolor{normal text}{fg=PureBlack,bg=white}
\setbeamercolor{alerted text}{fg=DarkRed}
\setbeamercolor{example text}{fg=DarkGreen}

% Footer
\setbeamercolor{footline}{fg=DarkGray,bg=white}
\setbeamercolor{page number in head/foot}{fg=DarkGray}

% ====================================
% FONT CONFIGURATION
% ====================================
\setbeamerfont{normal text}{size=\normalsize}
\setbeamerfont{frametitle}{size=\Large,series=\bfseries}
\setbeamerfont{title}{size=\huge,series=\bfseries}
\setbeamerfont{subtitle}{size=\large}
\setbeamerfont{author}{size=\normalsize}
\setbeamerfont{date}{size=\small}
\setbeamerfont{institute}{size=\small}

% ====================================
% REQUIRED PACKAGES
% ====================================
\usepackage{tikz}
\usepackage{amsmath}
\usepackage{amssymb}
\usepackage{booktabs}
\usepackage{graphicx}
\usepackage{array}
\usepackage{listings}
\usepackage{algorithm2e}
\usepackage{xcolor}
\usepackage{tabularx}
\usepackage{multirow}
\usepackage{subcaption}

% ====================================
% CUSTOM COMMANDS
% ====================================
% Text highlighting commands
\newcommand{\highlight}[1]{\textcolor{DeepBlue}{\textbf{#1}}}
\newcommand{\secondary}[1]{\textcolor{DarkGray}{#1}}
\newcommand{\success}[1]{\textcolor{DarkGreen}{#1}}
\newcommand{\warning}[1]{\textcolor{DarkRed}{#1}}
\newcommand{\data}[1]{\textcolor{ChartBlue}{#1}}
\newcommand{\dataalt}[1]{\textcolor{ChartOrange}{#1}}

% Mathematical notation
\newcommand{\given}{\mid}
\newcommand{\prob}[1]{P(#1)}
\newcommand{\argmax}{\operatorname*{argmax}}
\newcommand{\argmin}{\operatorname*{argmin}}
\newcommand{\softmax}{\operatorname{softmax}}

% Box commands for emphasis
\newcommand{\keypoint}[1]{%
  \begin{center}
  \fbox{\parbox{0.9\textwidth}{\centering\textbf{#1}}}
  \end{center}
}

\newcommand{\formula}[1]{%
  \begin{center}
  \colorbox{LightGray}{\parbox{0.8\textwidth}{\centering$\displaystyle #1$}}
  \end{center}
}

% ====================================
% LISTINGS CONFIGURATION
% ====================================
\lstset{
  basicstyle=\ttfamily\small,
  keywordstyle=\color{DeepBlue}\bfseries,
  commentstyle=\color{DarkGray}\itshape,
  stringstyle=\color{ChartOrange},
  numbers=left,
  numberstyle=\tiny\color{DarkGray},
  stepnumber=1,
  numbersep=5pt,
  backgroundcolor=\color{white},
  showspaces=false,
  showstringspaces=false,
  showtabs=false,
  frame=single,
  frameround=tttt,
  rulecolor=\color{LightGray},
  tabsize=2,
  captionpos=b,
  breaklines=true,
  breakatwhitespace=true,
  language=Python,
  escapeinside={(*@}{@*)},
  morekeywords={self, yield, assert, with, as}
}

% ====================================
% STANDARD SLIDE LAYOUTS
% ====================================

% Two-column layout with title
\newcommand{\twocolslide}[3]{%
  \begin{frame}{#1}
  \begin{columns}[T]
  \column{0.48\textwidth}
  #2
  \column{0.48\textwidth}
  #3
  \end{columns}
  \end{frame}
}

% Three-column layout
\newcommand{\threecolslide}[4]{%
  \begin{frame}{#1}
  \begin{columns}[T]
  \column{0.32\textwidth}
  #2
  \column{0.32\textwidth}
  #3
  \column{0.32\textwidth}
  #4
  \end{columns}
  \end{frame}
}

% Chart slide with caption
\newcommand{\chartslide}[3]{%
  \begin{frame}{#1}
  \begin{center}
  \includegraphics[width=#2\textwidth]{#3}
  \end{center}
  \end{frame}
}

% Full chart slide - optimized to 0.85 for proper margins
\newcommand{\fullchartslide}[2]{%
  \begin{frame}{#1}
  \begin{center}
  \includegraphics[width=0.85\textwidth]{#2}
  \end{center}
  \end{frame}
}

% Code slide
\newcommand{\codeslide}[3]{%
  \begin{frame}[fragile]{#1}
  \begin{lstlisting}[language=#2]
#3
  \end{lstlisting}
  \end{frame}
}

% Concept slide with figure
\newcommand{\conceptslide}[3]{%
  \begin{frame}{#1}
  \begin{columns}[T]
  \column{0.6\textwidth}
  #2
  \column{0.35\textwidth}
  \begin{center}
  \includegraphics[width=0.95\textwidth]{#3}
  \end{center}
  \end{columns}
  \end{frame}
}

% Table slide
\newcommand{\tableslide}[2]{%
  \begin{frame}{#1}
  \begin{center}
  \Large
  \renewcommand{\arraystretch}{1.5}
  #2
  \end{center}
  \end{frame}
}

% Summary slide
\newcommand{\summaryslide}[2]{%
  \begin{frame}{#1}
  \begin{center}
  \Large
  #2
  \end{center}
  \vfill
  \begin{center}
  \keypoint{Key Takeaway}
  \end{center}
  \end{frame}
}

% ====================================
% REMOVE DECORATIONS
% ====================================
\setbeamertemplate{blocks}[default]
\setbeamertemplate{title page}[default][colsep=-4bp,rounded=false]
\setbeamertemplate{itemize items}[circle]
\setbeamertemplate{enumerate items}[default]
\setbeamertemplate{section in toc}[sections numbered]
\setbeamertemplate{subsection in toc}[subsections numbered]

% ====================================
% PAGE NUMBERING
% ====================================
\setbeamertemplate{footline}{
  \leavevmode%
  \hbox{%
  \begin{beamercolorbox}[wd=.333333\paperwidth,ht=2.25ex,dp=1ex,center]{author in head/foot}%
    \usebeamerfont{author in head/foot}\secondary{\insertshortauthor}
  \end{beamercolorbox}%
  \begin{beamercolorbox}[wd=.333333\paperwidth,ht=2.25ex,dp=1ex,center]{title in head/foot}%
    \usebeamerfont{title in head/foot}\secondary{\insertshorttitle}
  \end{beamercolorbox}%
  \begin{beamercolorbox}[wd=.333333\paperwidth,ht=2.25ex,dp=1ex,right]{date in head/foot}%
    \usebeamerfont{date in head/foot}\secondary{\insertshortdate{}\hspace*{2em}
    \insertframenumber{} / \inserttotalframenumber\hspace*{2ex}}
  \end{beamercolorbox}}%
  \vskip0pt%
}

% ====================================
% TABLE OF CONTENTS STYLE
% ====================================
\setbeamertemplate{section in toc}{%
  \leavevmode\leftskip=1.5em%
  \llap{%
    \usebeamerfont{section in toc}%
    \usebeamercolor[fg]{section in toc}%
    \inserttocsectionnumber.%
  }%
  \usebeamerfont{section in toc}%
  \usebeamercolor[fg]{section in toc}%
  \inserttocsection\par%
}

\setbeamertemplate{subsection in toc}{%
  \leavevmode\leftskip=3em%
  \llap{%
    \usebeamerfont{subsection in toc}%
    \usebeamercolor[fg]{subsection in toc}%
    \inserttocsectionnumber.\inserttocsubsectionnumber%
  }%
  \usebeamerfont{subsection in toc}%
  \usebeamercolor[fg]{subsection in toc}%
  \inserttocsubsection\par%
}

% ====================================
% END OF MASTER TEMPLATE
% ====================================

% Define bottomnote command
\newcommand{\bottomnote}[1]{%
    \vspace{0.2cm}
    \begin{center}
    \footnotesize\secondary{#1}
    \end{center}
}

\title{Pre-trained Language Models}
\subtitle{\secondary{Week 6 - The \$1 Million Revolution}}
\author{NLP Course 2025}
\date{October 26, 2025}

\begin{document}

% ===== OPENING SEQUENCE (Slides 1-5) =====

% Slide 1: Title
\begin{frame}
\titlepage
\vfill
\begin{center}
\secondary{\footnotesize BSc Discovery-Based Presentation}
\end{center}
\end{frame}

% Slide 2: Discovery Hook - The $1 Million Problem
\begin{frame}[t]{The \$1 Million Problem}
\vspace{-0.3cm}
\begin{center}
\includegraphics[width=0.75\textwidth]{../figures/training_cost_comparison_bsc.pdf}
\end{center}

\begin{center}
\textbf{Key Insight}: Pre-training changes the economics of NLP completely
\end{center}

\vspace{0.2cm}
\bottomnote{Training BERT from scratch: \$1M+. Fine-tuning: \$50-500. Game changer.}
\end{frame}

% Slide 3: Before 2018 - The Old Way
\begin{frame}[t]{Before 2018: The Old Way}
\small
\begin{columns}[T]
\column{0.48\textwidth}
\raggedright
\textbf{Task-Specific Models}:

\begin{itemize}
\item \textbf{Sentiment}: Custom CNN architecture
\item \textbf{Question Answering}: BiDAF model
\item \textbf{Named Entity}: BiLSTM-CRF
\item \textbf{Translation}: Seq2Seq with attention
\item \textbf{Summarization}: Pointer-generator
\end{itemize}

\vspace{3mm}
\textbf{The Process}:

\begin{enumerate}
\item Design architecture for your task
\item Collect labeled data (10K+ examples)
\item Train from random initialization
\item Hope it works
\end{enumerate}

\column{0.48\textwidth}
\raggedright
\textbf{Limitations}:

\begin{itemize}
\item Each task starts from scratch
\item No knowledge transfer
\item Expensive data collection
\item Months per task
\item Small datasets = poor performance
\end{itemize}

\vspace{3mm}
\textbf{The Cost}:

\begin{itemize}
\item 3-6 months per task
\item 10K-100K labeled examples
\item \$50K-200K in labeling costs
\item Limited accuracy (60-75\%)
\end{itemize}

\end{columns}

\vspace{0.2cm}
\bottomnote{Every NLP task was an isolated, expensive project}
\end{frame}

% Slide 4: The Breakthrough - October 2018
\begin{frame}[t]{The Breakthrough: October 2018}
\vspace{-0.3cm}
\begin{center}
\includegraphics[width=0.75\textwidth]{../figures/2018_breakthrough_timeline_bsc.pdf}
\end{center}

\begin{center}
\textbf{Key Insight}: BERT and GPT changed everything in 4 months
\end{center}

\vspace{0.2cm}
\bottomnote{June 2018 (GPT-1) and October 2018 (BERT) - the inflection point}
\end{frame}

% Slide 5: The New Paradigm
\begin{frame}[t]{The New Paradigm}
\vspace{-0.3cm}
\begin{center}
\includegraphics[width=0.75\textwidth]{../figures/pretraining_finetuning_paradigm_bsc.pdf}
\end{center}

\begin{center}
\textbf{Key Insight}: Learn language once (expensive), apply everywhere (cheap)
\end{center}

\vspace{0.2cm}
\bottomnote{This is transfer learning - finally working for NLP}
\end{frame}

% ===== PRE-2018 LANDSCAPE (Slides 6-10) =====

% Slide 6: Visual - Task-Specific Architectures
\begin{frame}[t]{Pre-2018: Every Task Needed Its Own Model}
\vspace{-0.3cm}
\begin{center}
\includegraphics[width=0.7\textwidth]{../figures/task_specific_architectures_bsc.pdf}
\end{center}

\begin{center}
\textbf{Key Insight}: No sharing, no transfer, no efficiency
\end{center}

\vspace{0.2cm}
\bottomnote{Each task was an independent research project}
\end{frame}

% Slide 7: Detail - Why Task-Specific Failed
\begin{frame}[t]{Why This Approach Failed to Scale}
\small
\begin{columns}[T]
\column{0.48\textwidth}
\raggedright
\textbf{The Limitations}:

\begin{itemize}
\item \textbf{No transfer}: Each model learns from scratch
\item \textbf{Data hungry}: Need 10K+ labeled examples per task
\item \textbf{Expensive}: Labeling costs \$50K-200K
\item \textbf{Slow}: 3-6 months per task
\item \textbf{Brittle}: Fails on new domains
\end{itemize}

\vspace{5mm}
\textbf{Example - Sentiment Analysis}:

\begin{itemize}
\item Collect 20K movie reviews
\item Label positive/negative
\item Train custom CNN: 2-3 weeks
\item Accuracy: 82\%
\item Deploy to product reviews: Fails (65\%)!
\end{itemize}

\column{0.48\textwidth}
\raggedright
\textbf{What We Needed}:

\begin{itemize}
\item Shared language understanding
\item Transfer across tasks
\item Work with small labeled datasets
\item Fast adaptation to new tasks
\item Robust across domains
\end{itemize}

\vspace{5mm}
\textbf{The Dream}:

\begin{itemize}
\item Train once on ALL text
\item Fine-tune with 100-1000 examples
\item Days instead of months
\item State-of-art on every task
\end{itemize}

\end{columns}

\vspace{0.2cm}
\bottomnote{The question: Can we achieve this dream?}
\end{frame}

% Slide 8: Visual - Data Requirements Pre-2018
\begin{frame}[t]{The Data Bottleneck}
\vspace{-0.3cm}
\begin{center}
\includegraphics[width=0.65\textwidth]{../figures/data_requirements_pre2018_bsc.pdf}
\end{center}

\begin{center}
\textbf{Key Insight}: Performance plateaus without massive labeled datasets
\end{center}

\vspace{0.2cm}
\bottomnote{Labeled data is expensive - this limited what we could build}
\end{frame}

% Slide 9: Detail - The Transfer Learning Dream
\begin{frame}[t]{The Transfer Learning Dream}
\small
\begin{columns}[T]
\column{0.48\textwidth}
\raggedright
\textbf{Computer Vision's Success}:

\begin{itemize}
\item 2012: ImageNet pre-training (AlexNet)
\item Train on 1M images (unsupervised labels)
\item Fine-tune for any vision task
\item 10x less data needed
\item State-of-art on everything
\end{itemize}

\vspace{5mm}
\textbf{The Magic}:

\begin{itemize}
\item Low-level features shared (edges, textures)
\item High-level features shared (objects, shapes)
\item Learn once, transfer everywhere
\end{itemize}

\column{0.48\textwidth}
\raggedright
\textbf{Why NLP Lagged}:

\begin{itemize}
\item Words are discrete (images continuous)
\item Context matters more
\item Sequence length varies
\item Multiple tasks (classification, generation, QA)
\item No clear "ImageNet equivalent"
\end{itemize}

\vspace{5mm}
\textbf{The Question (2017)}:

\textit{Can we create an ImageNet moment for NLP?}

\vspace{3mm}
Answer coming in 2018...

\end{columns}

\vspace{0.2cm}
\bottomnote{Transfer learning worked for vision - could it work for language?}
\end{frame}

% Slide 10: The Question
\begin{frame}[t]{The Central Question}
\begin{center}
\Large
\textbf{Can we pre-train a language model on ALL text,}

\textbf{then fine-tune for ANY task?}
\end{center}

\vspace{10mm}

\begin{center}
\textbf{Requirements}:
\begin{itemize}
\item Unsupervised pre-training (no labels needed)
\item Massive text corpus (billions of words)
\item Learn general language understanding
\item Fast fine-tuning with small labeled data
\item Work across all NLP tasks
\end{itemize}
\end{center}

\vspace{10mm}

\begin{center}
\Large
\textbf{Answer: YES}

\vspace{3mm}
\secondary{\normalsize Next 36 slides show exactly how}
\end{center}

\vspace{0.2cm}
\bottomnote{The breakthrough: BERT and GPT (2018)}
\end{frame}

% ===== BERT SECTION (Slides 11-28) - 18 Slides =====

\begin{frame}[t]{}
\begin{center}
\Huge\textbf{Part 1: BERT}

\vspace{5mm}
\Large\secondary{Bidirectional Encoder Representations}

\vspace{5mm}
\Large\secondary{from Transformers}
\end{center}
\end{frame}

% Slides 11-12: The Fill-in-Blank Challenge (Problem-First)
\begin{frame}[t]{The Fill-in-Blank Challenge}
\vspace{-0.3cm}
\begin{center}
\includegraphics[width=0.7\textwidth]{../figures/fill_in_blank_challenge_bsc.pdf}
\end{center}

\begin{center}
\textbf{Key Insight}: Need both left AND right context to fill blanks correctly
\end{center}

\vspace{0.2cm}
\bottomnote{Left-to-right models (like GPT) can't solve this naturally}
\end{frame}

\begin{frame}[t]{Why Bidirectional Matters}
\small
\begin{columns}[T]
\column{0.48\textwidth}
\raggedright
\textbf{The Task}:

Fill in: ``The [BLANK] sat on the mat''

\vspace{3mm}
\textbf{Left-to-Right Approach}:

Only sees: ``The [BLANK]''

Cannot use: ``sat on the mat''

\vspace{3mm}
\textbf{Predictions}:
\begin{itemize}
\item ``dog'' (generic animal)
\item ``person'' (generic)
\item ``cat'' (lucky guess)
\end{itemize}

\vspace{3mm}
Accuracy: \textcolor{red}{Low - missing critical context!}

\column{0.48\textwidth}
\raggedright
\textbf{Bidirectional Approach}:

Sees both: ``The [BLANK]'' AND ``sat on the mat''

\vspace{3mm}
Uses full context:
\begin{itemize}
\item ``sat'' suggests living thing
\item ``on the mat'' suggests small animal
\item ``the'' suggests common noun
\end{itemize}

\vspace{3mm}
\textbf{Predictions}:
\begin{itemize}
\item ``cat'' (high probability)
\item ``dog'' (possible)
\item ``kitten'' (possible)
\end{itemize}

\vspace{3mm}
Accuracy: \textcolor{green}{High - full context used!}

\end{columns}

\vspace{0.2cm}
\bottomnote{This is BERT's core innovation - bidirectional understanding}
\end{frame}

% Slides 13-14: Bidirectional Context
\begin{frame}[t]{Bidirectional Context in Action}
\vspace{-0.3cm}
\begin{center}
\includegraphics[width=0.75\textwidth]{../figures/bidirectional_attention_flow_bsc.pdf}
\end{center}

\begin{center}
\textbf{Key Insight}: Every word sees every other word (no direction restriction)
\end{center}

\vspace{0.2cm}
\bottomnote{Bidirectional = encoder from Week 5 transformers}
\end{frame}

\begin{frame}[t]{How Bidirectional Helps Different Tasks}
\small
\begin{columns}[T]
\column{0.48\textwidth}
\raggedright
\textbf{Fill-in-Blank Tasks}:

\begin{itemize}
\item Masked language modeling
\item Cloze questions
\item Spell correction
\end{itemize}

\vspace{3mm}
\textbf{Classification Tasks}:

\begin{itemize}
\item Sentiment: Full sentence context
\item Spam detection: Look ahead and behind
\item Topic classification: Global understanding
\end{itemize}

\vspace{3mm}
\textbf{Question Answering}:

\begin{itemize}
\item Match question to passage
\item Find answer span
\item Use context on both sides
\end{itemize}

\column{0.48\textwidth}
\raggedright
\textbf{Why It Works}:

\begin{itemize}
\item Contextual embeddings from both sides
\item Disambiguate word meanings
\item Capture long-range dependencies
\item Understand sentence structure
\end{itemize}

\vspace{3mm}
\textbf{Example - ``bank''}:

\begin{itemize}
\item Left: ``The river''
\item Right: ``was flooding''
\item Conclusion: Water bank, not financial
\end{itemize}

\vspace{3mm}
\textbf{When Bidirectional Doesn't Help}:

\begin{itemize}
\item Text generation (can't see future!)
\item Autoregressive tasks
\end{itemize}

\end{columns}

\vspace{0.2cm}
\bottomnote{Different tasks need different architectures - BERT for understanding}
\end{frame}

% Slide 15: Visual - Masked LM Concept
\begin{frame}[t]{Masked Language Modeling: The Training Objective}
\vspace{-0.3cm}
\begin{center}
\includegraphics[width=0.75\textwidth]{../figures/masked_lm_process_bsc.pdf}
\end{center}

\begin{center}
\textbf{Key Insight}: Mask 15\% of tokens, predict them using full context
\end{center}

\vspace{0.2cm}
\bottomnote{This forces the model to learn bidirectional representations}
\end{frame}

% Slide 16: Detail - Masked LM Mathematics
\begin{frame}[t]{Masked Language Modeling: The Mathematics}
\small
\begin{columns}[T]
\column{0.48\textwidth}
\raggedright
\textbf{Objective Function}:

$$\mathcal{L}_{MLM} = -\sum_{i \in masked} \log P(w_i | context)$$

where $context$ = all other words

\vspace{3mm}
\textbf{The Process}:

\begin{enumerate}
\item Randomly mask 15\% of tokens
\item Replace with [MASK] token (80\%)
\item Replace with random word (10\%)
\item Keep unchanged (10\%)
\end{enumerate}

\vspace{3mm}
\textbf{Why the variation}?

Prevents model from just memorizing [MASK] $\rightarrow$ word

\column{0.48\textwidth}
\raggedright
\textbf{Training Example}:

Original: ``The cat sat on the mat''

\vspace{3mm}
Masked: ``The [MASK] sat on the [MASK]''

\vspace{3mm}
Model predicts:
\begin{itemize}
\item Position 2: $P(\text{cat}|\text{context})$
\item Position 6: $P(\text{mat}|\text{context})$
\end{itemize}

\vspace{3mm}
\textbf{Cross-Entropy Loss}:

$$Loss = -[\log P(\text{cat}) + \log P(\text{mat})]$$

\vspace{3mm}
Minimize this across billions of sentences

\end{columns}

\vspace{0.2cm}
\bottomnote{Masked LM is self-supervised - no labels needed!}
\end{frame}

% Slide 17: Worked Example - Masked Token Prediction
\begin{frame}[t]{Worked Example: Predicting Masked Tokens}
\small
\textbf{Given}: ``The [MASK] sat on the mat''

\vspace{3mm}
\textbf{Step 1}: Convert to token embeddings (each 768-dim vector)

Token IDs: [101, 1996, 103, 2938, 2006, 1996, 13523, 102]

\vspace{3mm}
\textbf{Step 2}: Add positional embeddings

$$E_{input} = E_{token} + E_{position}$$

\vspace{3mm}
\textbf{Step 3}: Pass through 12 transformer encoder layers

Each layer: Self-attention (bidirectional) + Feed-forward

\vspace{3mm}
\textbf{Step 4}: Get output for [MASK] position (position 2)

Output vector: $h_{[MASK]} \in \mathbb{R}^{768}$

\vspace{3mm}
\textbf{Step 5}: Project to vocabulary, apply softmax

$$P(w) = \frac{\exp(W \cdot h_{[MASK]})}{\sum_{v} \exp(W \cdot h_{[MASK]})}$$

\vspace{3mm}
\textbf{Result}: Top predictions with probabilities

\begin{itemize}
\item P(cat) = 0.73
\item P(dog) = 0.15
\item P(person) = 0.04
\end{itemize}

\vspace{0.2cm}
\bottomnote{Full bidirectional context yields confident, accurate predictions}
\end{frame}

% Slides 18-19: BERT Architecture
\begin{frame}[t]{BERT Architecture Overview}
\vspace{-0.3cm}
\begin{center}
\includegraphics[width=0.7\textwidth]{../figures/bert_architecture_bsc.pdf}
\end{center}

\begin{center}
\textbf{Key Insight}: Stack of transformer encoders - pure bidirectional attention
\end{center}

\vspace{0.2cm}
\bottomnote{This is the encoder stack from Week 5 - no decoder needed}
\end{frame}

\begin{frame}[t]{BERT Architecture: The Details}
\small
\begin{columns}[T]
\column{0.48\textwidth}
\raggedright
\textbf{BERT-Base}:

\begin{itemize}
\item Layers: 12 transformer encoders
\item Hidden size: 768 dimensions
\item Attention heads: 12 per layer
\item Parameters: 110 million
\item Max sequence: 512 tokens
\end{itemize}

\vspace{5mm}
\textbf{BERT-Large}:

\begin{itemize}
\item Layers: 24 encoders
\item Hidden size: 1024 dimensions
\item Attention heads: 16 per layer
\item Parameters: 340 million
\item Max sequence: 512 tokens
\end{itemize}

\column{0.48\textwidth}
\raggedright
\textbf{Key Components}:

\begin{itemize}
\item \textbf{Token Embeddings}: WordPiece (30K vocab)
\item \textbf{Position Embeddings}: Learned (not sinusoidal)
\item \textbf{Segment Embeddings}: Sentence A vs B
\end{itemize}

\vspace{5mm}
\textbf{Why These Choices}:

\begin{itemize}
\item 12 layers: Balance depth vs speed
\item 768 hidden: Standard transformer size
\item 12 heads: Multiple attention patterns
\item 512 max: Memory constraints
\end{itemize}

\vspace{3mm}
\textbf{Computation}:

Training BERT-base from scratch: 4 days on 64 TPUs

\end{columns}

\vspace{0.2cm}
\bottomnote{These specs became the standard for encoder-based models}
\end{frame}

% Slides 20-21: Special Tokens
\begin{frame}[t]{BERT's Special Tokens}
\vspace{-0.3cm}
\begin{center}
\includegraphics[width=0.75\textwidth]{../figures/bert_special_tokens_bsc.pdf}
\end{center}

\begin{center}
\textbf{Key Insight}: Special tokens enable multiple tasks with one model
\end{center}

\vspace{0.2cm}
\bottomnote{[CLS], [SEP], [MASK] - each serves a specific purpose}
\end{frame}

\begin{frame}[t]{Special Tokens: Purpose and Usage}
\small
\begin{columns}[T]
\column{0.48\textwidth}
\raggedright
\textbf{[CLS] - Classification Token}:

\begin{itemize}
\item Always first token
\item Aggregates sentence meaning
\item Used for classification tasks
\end{itemize}

\vspace{3mm}
Example: ``[CLS] The movie was great [SEP]''

Output of [CLS]: Sentence embedding for sentiment classification

\vspace{5mm}
\textbf{[SEP] - Separator Token}:

\begin{itemize}
\item Separates sentence pairs
\item Enables QA, entailment tasks
\end{itemize}

\vspace{3mm}
Example: ``[CLS] Question [SEP] Passage [SEP]''

\column{0.48\textwidth}
\raggedright
\textbf{[MASK] - Masked Token}:

\begin{itemize}
\item Used only during pre-training
\item Replaced with actual word during fine-tuning
\item Training signal for MLM objective
\end{itemize}

\vspace{3mm}
Example: ``The [MASK] is blue''

Model learns: $P(\text{sky} | \text{context})$

\vspace{5mm}
\textbf{[PAD] - Padding Token}:

\begin{itemize}
\item Fills sequences to same length
\item Ignored in attention
\item Enables batch processing
\end{itemize}

\end{columns}

\vspace{0.2cm}
\bottomnote{Four special tokens, each crucial for BERT's versatility}
\end{frame}

% Slides 22-23: WordPiece Tokenization
\begin{frame}[t]{WordPiece Tokenization}
\vspace{-0.3cm}
\begin{center}
\includegraphics[width=0.65\textwidth]{../figures/wordpiece_tokenization_bsc.pdf}
\end{center}

\begin{center}
\textbf{Key Insight}: Subword units handle rare words and morphology
\end{center}

\vspace{0.2cm}
\bottomnote{Full details in Week 8 - preview here for BERT context}
\end{frame}

\begin{frame}[t]{WordPiece: Subword Units}
\small
\begin{columns}[T]
\column{0.48\textwidth}
\raggedright
\textbf{The Problem}:

\begin{itemize}
\item Word-level: 100K+ vocabulary (huge)
\item Character-level: Long sequences (slow)
\item Rare words: Poor representations
\end{itemize}

\vspace{3mm}
\textbf{WordPiece Solution}:

\begin{itemize}
\item Learn 30K subword units
\item Frequent words: Single token
\item Rare words: Multiple subwords
\end{itemize}

\vspace{3mm}
\textbf{Examples}:

\begin{itemize}
\item ``playing'' $\rightarrow$ [``play'', ``\#\#ing'']
\item ``unhappiness'' $\rightarrow$ [``un'', ``\#\#happiness'']
\item ``COVID'' $\rightarrow$ [``CO'', ``\#\#VI'', ``\#\#D'']
\end{itemize}

\column{0.48\textwidth}
\raggedright
\textbf{Benefits}:

\begin{itemize}
\item Fixed 30K vocabulary
\item Handle any word (no UNK)
\item Capture morphology
\item Share representations (``play'' in ``playing'', ``player'')
\end{itemize}

\vspace{3mm}
\textbf{BERT's Vocabulary}:

\begin{itemize}
\item 30,522 WordPiece tokens
\item Covers English comprehensively
\item Trained with BPE-like algorithm
\end{itemize}

\vspace{3mm}
\textbf{Impact}:

Rare word handling improved by 40\%

\end{columns}

\vspace{0.2cm}
\bottomnote{Subword tokenization is now standard across all modern models}
\end{frame}

% Slide 24: Visual - Pre-training Data
\begin{frame}[t]{Pre-training Data: The Foundation}
\vspace{-0.3cm}
\begin{center}
\includegraphics[width=0.65\textwidth]{../figures/bert_pretraining_corpus_bsc.pdf}
\end{center}

\begin{center}
\textbf{Key Insight}: Massive unsupervised data powers general language understanding
\end{center}

\vspace{0.2cm}
\bottomnote{BooksCorpus (800M words) + Wikipedia (2.5B words) = 3.3B words}
\end{frame}

% Slide 25: Detail - Pre-training Objectives
\begin{frame}[t]{BERT's Two Pre-training Objectives}
\small
\begin{columns}[T]
\column{0.48\textwidth}
\raggedright
\textbf{Objective 1: Masked LM}:

\begin{itemize}
\item Mask 15\% of tokens
\item Predict masked words
\item Uses bidirectional context
\end{itemize}

\vspace{3mm}
Example:

\texttt{The [MASK] sat on [MASK] mat}

Predict: ``cat'' and ``the''

\vspace{5mm}
\textbf{Objective 2: Next Sentence Prediction}:

\begin{itemize}
\item Given two sentences A and B
\item Predict if B follows A
\item Binary classification (50\% real, 50\% random)
\end{itemize}

\vspace{3mm}
Example:

\texttt{A: Alice was tired.}

\texttt{B: She went to sleep.} \textcolor{green}{[True]}

\column{0.48\textwidth}
\raggedright
\textbf{Why These Objectives}:

\begin{itemize}
\item \textbf{MLM}: Learn word-level representations
\item \textbf{NSP}: Learn sentence relationships
\item Together: Comprehensive language understanding
\end{itemize}

\vspace{3mm}
\textbf{Training Details}:

\begin{itemize}
\item Batch size: 256 sequences
\item Steps: 1M (40 epochs)
\item Optimizer: Adam (lr=1e-4)
\item Time: 4 days on 64 TPUs
\item Cost: ~\$7,000 compute
\end{itemize}

\vspace{3mm}
\textbf{Result}:

General language model ready for ANY task

\end{columns}

\vspace{0.2cm}
\bottomnote{Two complementary objectives create robust representations}
\end{frame}

% Slides 26-27: Fine-tuning Mechanics
\begin{frame}[t]{Fine-tuning BERT for Your Task}
\vspace{-0.3cm}
\begin{center}
\includegraphics[width=0.75\textwidth]{../figures/bert_finetuning_process_bsc.pdf}
\end{center}

\begin{center}
\textbf{Key Insight}: Add task-specific head, train on small labeled dataset
\end{center}

\vspace{0.2cm}
\bottomnote{Days instead of months - with better results}
\end{frame}

\begin{frame}[t]{Fine-tuning: The Mechanics}
\small
\begin{columns}[T]
\column{0.48\textwidth}
\raggedright
\textbf{The Process}:

\begin{enumerate}
\item Load pre-trained BERT weights
\item Add task-specific layer on top
\item Train on labeled data (100-10K examples)
\item Use small learning rate (2e-5)
\item Train for 2-4 epochs
\end{enumerate}

\vspace{3mm}
\textbf{Task-Specific Heads}:

\begin{itemize}
\item \textbf{Classification}: Linear layer on [CLS]
\item \textbf{Token classification}: Linear on each token
\item \textbf{QA}: Span prediction (start/end)
\end{itemize}

\column{0.48\textwidth}
\raggedright
\textbf{Hyperparameters}:

\begin{itemize}
\item Learning rate: 2e-5, 3e-5, 5e-5
\item Batch size: 16 or 32
\item Epochs: 2-4
\item Warmup: 10\% of steps
\item Max sequence: 128-512
\end{itemize}

\vspace{3mm}
\textbf{Layer Freezing Options}:

\begin{itemize}
\item Freeze nothing: Full fine-tuning (best)
\item Freeze bottom 8 layers: Faster
\item Freeze all, train head only: Feature extraction
\end{itemize}

\vspace{3mm}
\textbf{Typical Results}:

1000 examples + BERT $>$ 10,000 from scratch

\end{columns}

\vspace{0.2cm}
\bottomnote{Fine-tuning is fast, cheap, and remarkably effective}
\end{frame}

% Slide 28: Checkpoint Quiz - BERT
\begin{frame}[t]{Checkpoint: Understanding BERT}
\begin{center}
\textbf{Quick Quiz}
\end{center}
\vspace{3mm}

\begin{columns}[T]
\column{0.48\textwidth}
\raggedright
\textbf{Question 1}:

Why does BERT mask 15\% of tokens?

\vspace{3mm}
\textbf{A)} It's faster \\
\textbf{B)} Balances learning vs efficiency \\
\textbf{C)} Reduces overfitting \\
\textbf{D)} Random choice

\vspace{8mm}
\textbf{Question 2}:

What makes BERT bidirectional?

\vspace{3mm}
\textbf{A)} Two LSTMs \\
\textbf{B)} Encoder allows full context \\
\textbf{C)} Reads backwards \\
\textbf{D)} Has two outputs

\column{0.48\textwidth}
\raggedright
\textbf{Answer 1}: \textbf{B) Balances learning vs efficiency}

\begin{itemize}
\item Too few: Not enough training signal
\item Too many: Model sees mostly masks
\item 15\%: Empirically optimal
\item Enough context remains unmasked
\end{itemize}

\vspace{3mm}
\textbf{Answer 2}: \textbf{B) Encoder allows full context}

\begin{itemize}
\item Transformer encoder: No causal mask
\item Each token attends to ALL others
\item Left and right context used equally
\item This is the key difference from GPT
\end{itemize}

\end{columns}

\vspace{0.2cm}
\bottomnote{Understanding these foundations is critical for using BERT effectively}
\end{frame}

% ===== GPT SECTION (Slides 29-43) - 15 Slides =====

\begin{frame}[t]{}
\begin{center}
\Huge\textbf{Part 2: GPT}

\vspace{5mm}
\Large\secondary{Generative Pre-trained Transformer}
\end{center}
\end{frame}

% Slides 29-30: The Generation Challenge (Problem-First)
\begin{frame}[t]{The Generation Challenge}
\vspace{-0.3cm}
\begin{center}
\includegraphics[width=0.7\textwidth]{../figures/text_generation_challenge_bsc.pdf}
\end{center}

\begin{center}
\textbf{Key Insight}: Generation requires predicting one word at a time, left-to-right
\end{center}

\vspace{0.2cm}
\bottomnote{Bidirectional models can't generate - they'd cheat by seeing the future}
\end{frame}

\begin{frame}[t]{Why Generation is Harder Than Classification}
\small
\begin{columns}[T]
\column{0.48\textwidth}
\raggedright
\textbf{Classification}:

\begin{itemize}
\item Input: Full sentence
\item Output: Single label
\item Can see everything
\item One prediction
\end{itemize}

\vspace{3mm}
Example: ``Great movie!'' $\rightarrow$ Positive

\vspace{5mm}
\textbf{Generation}:

\begin{itemize}
\item Input: Partial sequence
\item Output: Next word
\item Can't see future
\item Multiple sequential predictions
\end{itemize}

\vspace{3mm}
Example: ``Once upon'' $\rightarrow$ predict ``a''

Then: ``Once upon a'' $\rightarrow$ predict ``time''

\column{0.48\textwidth}
\raggedright
\textbf{The Constraint}:

\textit{During generation, you haven't written future words yet!}

\vspace{3mm}
Cannot use bidirectional model

\vspace{3mm}
Must use \highlight{causal} (left-to-right) attention

\vspace{5mm}
\textbf{Requirements}:

\begin{itemize}
\item Predict token-by-token
\item Each prediction uses only past
\item Autoregressive: Output becomes input
\item Coherent over long sequences
\end{itemize}

\vspace{3mm}
\textbf{Use Cases}:

\begin{itemize}
\item Text completion
\item Story generation
\item Code generation
\item Dialogue systems
\end{itemize}

\end{columns}

\vspace{0.2cm}
\bottomnote{Different tasks need different architectures - GPT for generation}
\end{frame}

% Slides 31-32: Autoregressive Approach
\begin{frame}[t]{Autoregressive Language Modeling}
\vspace{-0.3cm}
\begin{center}
\includegraphics[width=0.75\textwidth]{../figures/autoregressive_visual_bsc.pdf}
\end{center}

\begin{center}
\textbf{Key Insight}: Predict next token using only previous tokens (causal)
\end{center}

\vspace{0.2cm}
\bottomnote{Auto-regressive = output at time $t$ becomes input at time $t+1$}
\end{frame}

\begin{frame}[t]{Autoregressive Approach: The Mathematics}
\small
\begin{columns}[T]
\column{0.48\textwidth}
\raggedright
\textbf{Objective Function}:

$$\mathcal{L}_{AR} = -\sum_{t=1}^{T} \log P(w_t | w_1, ..., w_{t-1})$$

Maximize probability of each next word

\vspace{3mm}
\textbf{Chain Rule Decomposition}:

$$P(w_1, ..., w_T) = \prod_{t=1}^{T} P(w_t | w_1, ..., w_{t-1})$$

Exact factorization (no approximation!)

\vspace{3mm}
\textbf{Causal Constraint}:

At time $t$, can only see $w_1, ..., w_{t-1}$

Cannot see $w_t, w_{t+1}, ...$ (haven't generated yet)

\column{0.48\textwidth}
\raggedright
\textbf{Training with Teacher Forcing}:

\begin{itemize}
\item Given full sequence during training
\item At each position: Predict next
\item Use ground truth (not predictions)
\item Prevents error accumulation
\end{itemize}

\vspace{3mm}
\textbf{Example}:

Sequence: ``The cat sat''

\begin{itemize}
\item Position 1: Predict ``The'' (start)
\item Position 2: Given ``The'', predict ``cat''
\item Position 3: Given ``The cat'', predict ``sat''
\end{itemize}

\vspace{3mm}
All trained in parallel!

\end{columns}

\vspace{0.2cm}
\bottomnote{Autoregressive objective is natural for generation tasks}
\end{frame}

% Slide 33: Visual - Next Token Prediction
\begin{frame}[t]{Next Token Prediction Process}
\vspace{-0.3cm}
\begin{center}
\includegraphics[width=0.75\textwidth]{../figures/next_token_prediction_bsc.pdf}
\end{center}

\begin{center}
\textbf{Key Insight}: Each token predicted using ALL previous tokens
\end{center}

\vspace{0.2cm}
\bottomnote{This is language modeling from Week 1 - but with transformers!}
\end{frame}

% Slide 34: Detail - Autoregressive Mathematics
\begin{frame}[t]{Autoregressive Mathematics Deep Dive}
\small
\begin{columns}[T]
\column{0.48\textwidth}
\raggedright
\textbf{At Time Step $t$}:

Input: $w_1, w_2, ..., w_{t-1}$

\vspace{3mm}
\textbf{Step 1}: Embed tokens

$$E = [e_1, e_2, ..., e_{t-1}]$$

\vspace{3mm}
\textbf{Step 2}: Add positional encoding

$$H^{(0)} = E + P$$

\vspace{3mm}
\textbf{Step 3}: Pass through $L$ decoder layers

$$H^{(\ell)} = \text{TransformerDecoder}(H^{(\ell-1)})$$

\vspace{3mm}
\textbf{Step 4}: Project final layer to vocabulary

$$logits = W \cdot h_{t-1}^{(L)}$$

\textbf{Step 5}: Softmax for probabilities

$$P(w_t) = \softmax(logits)$$

\column{0.48\textwidth}
\raggedright
\textbf{Teacher Forcing}:

During training:
\begin{itemize}
\item Use ground truth $w_t$ for next step
\item Don't use model's prediction
\item Prevents compounding errors
\item Enables parallel training
\end{itemize}

\vspace{3mm}
\textbf{Inference (Generation)}:

\begin{itemize}
\item Sample from $P(w_t)$
\item Append to sequence
\item Repeat: $w_t$ $\rightarrow$ input for $w_{t+1}$
\item Stop at [END] or max length
\end{itemize}

\vspace{3mm}
\textbf{Key Difference from BERT}:

BERT: Predict masked (can see both sides)

GPT: Predict next (can only see left)

\end{columns}

\vspace{0.2cm}
\bottomnote{Causal constraint is enforced by attention masking}
\end{frame}

% Slide 35: Worked Example - Next Token Prediction
\begin{frame}[t]{Worked Example: Computing P(next word)}
\small
\textbf{Given Sequence}: ``The cat sat on''

\vspace{3mm}
\textbf{Task}: Compute $P(\text{the} | \text{The cat sat on})$

\vspace{3mm}
\textbf{Step 1}: Token IDs and embeddings

\texttt{[The=50256, cat=3797, sat=3332, on=319]} $\rightarrow$ $E \in \mathbb{R}^{4 \times 768}$

\vspace{3mm}
\textbf{Step 2}: Add positional embeddings

$$H^{(0)} = E + P$$

\vspace{3mm}
\textbf{Step 3}: 12 decoder layers with causal attention

Each token only attends to previous tokens (triangular mask)

\vspace{3mm}
\textbf{Step 4}: Final hidden state for position 4 (``on'')

$$h_4^{(12)} \in \mathbb{R}^{768}$$

\vspace{3mm}
\textbf{Step 5}: Project to vocabulary (50,257 tokens)

$$logits = W \cdot h_4^{(12)} \in \mathbb{R}^{50257}$$

\vspace{3mm}
\textbf{Step 6}: Softmax and sample

\begin{itemize}
\item $P(\text{the}) = 0.42$ \textcolor{green}{(highest!)}
\item $P(\text{a}) = 0.18$
\item $P(\text{mat}) = 0.09$
\end{itemize}

\vspace{0.2cm}
\bottomnote{Model confidently predicts ``the'' - matches English patterns}
\end{frame}

% Slides 36-37: GPT Architecture
\begin{frame}[t]{GPT Architecture Overview}
\vspace{-0.3cm}
\begin{center}
\includegraphics[width=0.7\textwidth]{../figures/gpt_architecture_bsc.pdf}
\end{center}

\begin{center}
\textbf{Key Insight}: Stack of transformer decoders with causal attention
\end{center}

\vspace{0.2cm}
\bottomnote{This is the decoder from Week 5 - no encoder needed for generation}
\end{frame}

\begin{frame}[t]{GPT Architecture: The Details}
\small
\begin{columns}[T]
\column{0.48\textwidth}
\raggedright
\textbf{GPT-1 (June 2018)}:

\begin{itemize}
\item Layers: 12 decoder layers
\item Hidden size: 768 dimensions
\item Attention heads: 12 per layer
\item Parameters: 117 million
\item Context window: 512 tokens
\end{itemize}

\vspace{5mm}
\textbf{GPT-2 (February 2019)}:

\begin{itemize}
\item Layers: 48 decoder layers
\item Hidden size: 1600 dimensions
\item Attention heads: 25 per layer
\item Parameters: 1.5 billion (13x larger!)
\item Context: 1024 tokens
\end{itemize}

\column{0.48\textwidth}
\raggedright
\textbf{GPT-3 (May 2020)}:

\begin{itemize}
\item Layers: 96 decoder layers
\item Hidden size: 12,288 dimensions
\item Attention heads: 96 per layer
\item Parameters: 175 billion (116x GPT-2!)
\item Context: 2048 tokens
\end{itemize}

\vspace{5mm}
\textbf{Training Costs}:

\begin{itemize}
\item GPT-1: \$50K (weeks on 8 GPUs)
\item GPT-2: \$500K (weeks on 256 GPUs)
\item GPT-3: \$4.6M (months on 10K GPUs)
\end{itemize}

\end{columns}

\vspace{0.2cm}
\bottomnote{Scaling drove capability emergence - few-shot learning appeared}
\end{frame}

% Slides 38-39: Causal Masking
\begin{frame}[t]{Causal Masking: Preventing Future Cheating}
\vspace{-0.3cm}
\begin{center}
\includegraphics[width=0.65\textwidth]{../figures/causal_mask_matrix_bsc.pdf}
\end{center}

\begin{center}
\textbf{Key Insight}: Lower triangular mask ensures causality (no peeking ahead)
\end{center}

\vspace{0.2cm}
\bottomnote{This is implemented in attention: masked positions get $-\infty$ before softmax}
\end{frame}

\begin{frame}[t]{Causal Masking: How It Works}
\small
\begin{columns}[T]
\column{0.48\textwidth}
\raggedright
\textbf{The Attention Mask}:

Lower triangular matrix of 1s and 0s

$$M = \begin{bmatrix}
1 & 0 & 0 & 0 \\
1 & 1 & 0 & 0 \\
1 & 1 & 1 & 0 \\
1 & 1 & 1 & 1
\end{bmatrix}$$

\vspace{3mm}
\textbf{Meaning}:

\begin{itemize}
\item 1: Can attend
\item 0: Cannot attend (masked)
\end{itemize}

\vspace{3mm}
Token 1: Sees only itself

Token 2: Sees tokens 1-2

Token 3: Sees tokens 1-3

Token 4: Sees all (tokens 1-4)

\column{0.48\textwidth}
\raggedright
\textbf{Implementation}:

Before softmax in attention:

$$\text{scores}_\text{masked} = \text{scores} + (1-M) \times (-\infty)$$

After softmax: Masked positions get probability 0

\vspace{5mm}
\textbf{Why Essential for Generation}:

\begin{itemize}
\item Training: Model can't cheat
\item Inference: Naturally left-to-right
\item Prevents data leakage
\item Ensures valid generation
\end{itemize}

\vspace{3mm}
\textbf{Contrast with BERT}:

BERT: All 1s (full bidirectional)

GPT: Lower triangular (causal)

\end{columns}

\vspace{0.2cm}
\bottomnote{Causal mask is the key difference in transformer decoder vs encoder}
\end{frame}

% Slide 40: Visual - GPT Evolution
\begin{frame}[t]{The Scaling Journey: GPT-1 to GPT-3}
\vspace{-0.3cm}
\begin{center}
\includegraphics[width=0.75\textwidth]{../figures/gpt_evolution_timeline_bsc.pdf}
\end{center}

\begin{center}
\textbf{Key Insight}: Scaling unlocked emergent capabilities (few-shot learning)
\end{center}

\vspace{0.2cm}
\bottomnote{GPT-3 can perform tasks from examples - no fine-tuning needed!}
\end{frame}

% Slide 41: Detail - Few-Shot Learning
\begin{frame}[t]{Few-Shot Learning: The Emergent Capability}
\small
\begin{columns}[T]
\column{0.48\textwidth}
\raggedright
\textbf{Definitions}:

\begin{itemize}
\item \textbf{Zero-shot}: Task description only

  ``Translate to French: Hello'' $\rightarrow$ ``Bonjour''

\item \textbf{One-shot}: One example

  ``English: Hello, French: Bonjour.

  English: Goodbye, French:'' $\rightarrow$ ``Au revoir''

\item \textbf{Few-shot}: Multiple examples (2-10)

  Show 5 translation pairs, model infers pattern
\end{itemize}

\vspace{3mm}
\textbf{What's Remarkable}:

No gradient updates! Pure inference!

\column{0.48\textwidth}
\raggedright
\textbf{How It Works}:

\begin{itemize}
\item Model learns to learn during pre-training
\item In-context learning
\item Pattern matching in prompt
\item Emerges at scale (GPT-3, not GPT-1)
\end{itemize}

\vspace{3mm}
\textbf{Example Tasks}:

\begin{itemize}
\item Translation (unseen language pairs)
\item Arithmetic (3-digit addition)
\item Programming (code generation)
\item Reasoning (logical inference)
\end{itemize}

\vspace{3mm}
\textbf{Limitations}:

\begin{itemize}
\item Inconsistent performance
\item Sensitive to prompt wording
\item Not as good as fine-tuning
\end{itemize}

\end{columns}

\vspace{0.2cm}
\bottomnote{Few-shot learning hints at artificial general intelligence}
\end{frame}

% Slide 42: Worked Example - Transfer Learning Data Efficiency
\begin{frame}[t]{Worked Example: Data Efficiency of Pre-training}
\small
\textbf{Experiment}: Sentiment classification on product reviews

\vspace{3mm}
\textbf{Approach A - Train from Scratch}:

\begin{itemize}
\item Architecture: LSTM (2 layers, 512 hidden)
\item Training data: 10,000 labeled reviews
\item Training time: 6 hours on GPU
\item Test accuracy: 78.3\%
\end{itemize}

\vspace{3mm}
\textbf{Approach B - Fine-tune GPT-2}:

\begin{itemize}
\item Base model: GPT-2 (1.5B parameters, pre-trained)
\item Training data: 100 labeled reviews (100x less!)
\item Training time: 10 minutes on GPU
\item Test accuracy: 91.7\%
\end{itemize}

\vspace{5mm}
\begin{center}
\Large
\textbf{Result}: 100 examples + GPT-2 $>$ 10,000 from scratch

\vspace{3mm}
\secondary{\normalsize 100x less data, 36x faster, 17\% better accuracy}
\end{center}

\vspace{0.2cm}
\bottomnote{This is the power of transfer learning - pre-training solves the data problem}
\end{frame}

% Slide 43: Checkpoint Quiz - GPT
\begin{frame}[t]{Checkpoint: Understanding GPT}
\begin{center}
\textbf{Quick Quiz}
\end{center}
\vspace{3mm}

\begin{columns}[T]
\column{0.48\textwidth}
\raggedright
\textbf{Question 1}:

Why is GPT autoregressive?

\vspace{3mm}
\textbf{A)} It's faster \\
\textbf{B)} Natural for generation \\
\textbf{C)} Uses less memory \\
\textbf{D)} More accurate

\vspace{8mm}
\textbf{Question 2}:

What's the key difference from BERT?

\vspace{3mm}
\textbf{A)} More parameters \\
\textbf{B)} Different dataset \\
\textbf{C)} Causal vs bidirectional \\
\textbf{D)} Slower training

\column{0.48\textwidth}
\raggedright
\textbf{Answer 1}: \textbf{B) Natural for generation}

\begin{itemize}
\item Generation = predict next word
\item Can't see future (not written yet)
\item Autoregressive = use output as next input
\item Perfect fit for the task
\end{itemize}

\vspace{3mm}
\textbf{Answer 2}: \textbf{C) Causal vs bidirectional}

\begin{itemize}
\item BERT: See full sentence (encoder)
\item GPT: See only past (decoder)
\item BERT: Better for understanding
\item GPT: Better for generation
\end{itemize}

\end{columns}

\vspace{0.2cm}
\bottomnote{Architecture follows task requirements - understand the why}
\end{frame}

% ===== INTEGRATION SECTION (Slides 44-51) - 8 Slides =====

\begin{frame}[t]{}
\begin{center}
\Huge\textbf{Part 3: Integration}

\vspace{5mm}
\Large\secondary{Comparing and Choosing}
\end{center}
\end{frame}

% Slides 44-45: BERT vs GPT Comparison
\begin{frame}[t]{BERT vs GPT: Side-by-Side}
\vspace{-0.3cm}
\begin{center}
\includegraphics[width=0.75\textwidth]{../figures/bert_vs_gpt_architecture_bsc.pdf}
\end{center}

\begin{center}
\textbf{Key Insight}: Encoder for understanding, decoder for generation
\end{center}

\vspace{0.2cm}
\bottomnote{Choose based on your task requirements}
\end{frame}

\begin{frame}[t]{BERT vs GPT: When to Use Each}
\small
\begin{columns}[T]
\column{0.48\textwidth}
\raggedright
\textbf{Use BERT When}:

\begin{itemize}
\item \textbf{Task}: Classification, QA, NER
\item \textbf{Need}: Full sentence understanding
\item \textbf{Input}: Complete text available
\item \textbf{Output}: Labels or spans
\end{itemize}

\vspace{3mm}
\textbf{BERT Strengths}:

\begin{itemize}
\item Bidirectional context
\item Best for understanding
\item [CLS] token for sentence embedding
\item Handles word sense disambiguation
\end{itemize}

\vspace{3mm}
\textbf{BERT Applications}:

\begin{itemize}
\item Search (Google uses BERT)
\item Question answering
\item Named entity recognition
\item Sentiment analysis
\end{itemize}

\column{0.48\textwidth}
\raggedright
\textbf{Use GPT When}:

\begin{itemize}
\item \textbf{Task}: Generation, completion, dialogue
\item \textbf{Need}: Coherent text generation
\item \textbf{Input}: Prompt or partial sequence
\item \textbf{Output}: Continued text
\end{itemize}

\vspace{3mm}
\textbf{GPT Strengths}:

\begin{itemize}
\item Natural text generation
\item In-context learning (few-shot)
\item Scales well (GPT-3: 175B params)
\item Handles diverse prompts
\end{itemize}

\vspace{3mm}
\textbf{GPT Applications}:

\begin{itemize}
\item ChatGPT (dialogue)
\item Code completion (Copilot)
\item Creative writing
\item Text summarization (generative)
\end{itemize}

\end{columns}

\vspace{0.2cm}
\bottomnote{Both are transformers - different objectives, different use cases}
\end{frame}

% Slides 46-47: Transfer Learning Pipeline
\begin{frame}[t]{The Complete Transfer Learning Pipeline}
\vspace{-0.3cm}
\begin{center}
\includegraphics[width=0.75\textwidth]{../figures/transfer_learning_pipeline_bsc.pdf}
\end{center}

\begin{center}
\textbf{Key Insight}: Unsupervised pre-training + supervised fine-tuning = best of both
\end{center}

\vspace{0.2cm}
\bottomnote{One-time expensive pre-training, many cheap fine-tunings}
\end{frame}

\begin{frame}[t]{Transfer Learning: Best Practices}
\small
\begin{columns}[T]
\column{0.48\textwidth}
\raggedright
\textbf{Pre-training (Done Once)}:

\begin{itemize}
\item Massive unsupervised corpus (billions of words)
\item Self-supervised objectives (MLM or AR)
\item Large compute (TPUs/GPUs, weeks)
\item Cost: \$1M-\$10M
\item Result: General language model
\end{itemize}

\vspace{3mm}
\textbf{Who Does This}:

\begin{itemize}
\item Big tech (Google, OpenAI, Meta)
\item Research labs
\item Shared publicly
\item You download, don't train
\end{itemize}

\column{0.48\textwidth}
\raggedright
\textbf{Fine-tuning (Per Task)}:

\begin{itemize}
\item Small labeled dataset (100-10K examples)
\item Task-specific head
\item Small learning rate (2e-5)
\item Short training (hours)
\item Cost: \$50-\$500
\item Result: Task-specific model
\end{itemize}

\vspace{3mm}
\textbf{Best Practices}:

\begin{itemize}
\item Start with pre-trained checkpoint
\item Use small learning rate
\item Train 2-4 epochs
\item Monitor validation loss
\item Try different layer freezing
\end{itemize}

\end{columns}

\vspace{0.2cm}
\bottomnote{This pipeline is now standard across all of NLP}
\end{frame}

% Slide 48: When NOT to Use
\begin{frame}[t]{When NOT to Use Pre-trained Models}
\small
\begin{columns}[T]
\column{0.48\textwidth}
\raggedright
\textbf{Overkill Scenarios}:

\begin{itemize}
\item Simple regex suffices
\item Rule-based system works
\item N-grams are enough
\item Tiny dataset (< 50 examples)
\end{itemize}

\vspace{3mm}
\textbf{Resource Constraints}:

\begin{itemize}
\item Limited memory (BERT needs 4GB+ GPU)
\item Strict latency requirements (< 10ms)
\item Edge deployment (phones, IoT)
\item Battery-powered devices
\end{itemize}

\column{0.48\textwidth}
\raggedright
\textbf{Domain Mismatch}:

\begin{itemize}
\item Highly specialized jargon
\item Different language structure
\item Pre-training corpus unrepresentative
\end{itemize}

\vspace{3mm}
\textbf{Better Alternatives}:

\begin{itemize}
\item DistilBERT (smaller, faster)
\item Domain-specific pre-training
\item Few-shot prompting (no fine-tuning)
\item Ensemble of simple models
\end{itemize}

\vspace{3mm}
\textbf{Cost-Benefit}:

Sometimes simple approaches better ROI

\end{columns}

\vspace{0.2cm}
\bottomnote{Know when NOT to use powerful models - engineering judgment matters}
\end{frame}

% Slide 49: Common Pitfalls
\begin{frame}[t]{Common Pitfalls in Fine-tuning}
\small
\begin{columns}[T]
\column{0.48\textwidth}
\raggedright
\textbf{Pitfall 1: Learning Rate Too High}

\begin{itemize}
\item Pre-trained weights are delicate
\item High LR destroys them (catastrophic forgetting)
\item \textcolor{green}{Solution}: Use 2e-5 to 5e-5 (100x smaller than training from scratch)
\end{itemize}

\vspace{3mm}
\textbf{Pitfall 2: Too Many Epochs}

\begin{itemize}
\item Overfits to small dataset
\item \textcolor{green}{Solution}: 2-4 epochs maximum
\end{itemize}

\vspace{3mm}
\textbf{Pitfall 3: Wrong Task Head}

\begin{itemize}
\item Architecture mismatch
\item \textcolor{green}{Solution}: Use standard heads from examples
\end{itemize}

\column{0.48\textwidth}
\raggedright
\textbf{Pitfall 4: Ignoring Validation}

\begin{itemize}
\item Train loss down, test loss up
\item \textcolor{green}{Solution}: Early stopping on validation
\end{itemize}

\vspace{3mm}
\textbf{Pitfall 5: Not Trying Layer Freezing}

\begin{itemize}
\item Small dataset: Freeze bottom layers
\item \textcolor{green}{Solution}: Experiment with freezing 0, 6, 8, or 10 layers
\end{itemize}

\vspace{3mm}
\textbf{Pitfall 6: Forgetting Preprocessing}

\begin{itemize}
\item Tokenization must match pre-training
\item \textcolor{green}{Solution}: Use model's own tokenizer
\end{itemize}

\end{columns}

\vspace{0.2cm}
\bottomnote{Most failures come from hyperparameter choices - start conservative}
\end{frame}

% Slide 50: Unified View - The Paradigm Shift
\begin{frame}[t]{The Paradigm Shift: Pre-2018 vs Post-2018}
\vspace{-0.3cm}
\begin{center}
\includegraphics[width=0.75\textwidth]{../figures/paradigm_shift_comparison_bsc.pdf}
\end{center}

\begin{center}
\textbf{The Core Principle}: Pre-training solves the data problem
\end{center}

\vspace{0.2cm}
\bottomnote{This is the foundation of modern NLP - everything changed in 2018}
\end{frame}

% Slide 51: Key Takeaways
\begin{frame}[t]{Key Takeaways}
\begin{enumerate}
\item \textbf{Pre-training on massive unlabeled data}

  Learn general language understanding without task-specific labels

\item \textbf{BERT: Bidirectional encoder for understanding}

  Masked LM objective, full context, best for classification/QA

\item \textbf{GPT: Autoregressive decoder for generation}

  Next token prediction, causal mask, best for text generation

\item \textbf{Fine-tuning adapts with small labeled data}

  100-1000 examples sufficient, days not months, state-of-art results

\item \textbf{Transfer learning finally works for NLP}

  Learn once, apply everywhere - the 2018 revolution

\item \textbf{2018 was the inflection point}

  BERT and GPT changed everything - modern NLP is post-2018

\end{enumerate}

\vspace{0.2cm}
\bottomnote{Master these concepts - they define modern NLP practice}
\end{frame}

% Slide 52: Lab Preview
\begin{frame}[t]{Next: Hands-On Feature Extraction}
\begin{columns}[T]
\column{0.48\textwidth}
\raggedright
\textbf{Lab Activities}:

\begin{enumerate}
\item Load pre-trained BERT and GPT
\item Extract embeddings from both
\item Compare representation spaces
\item Visualize attention patterns
\item Use features for classification
\item Compare BERT vs GPT effectiveness
\end{enumerate}

\column{0.48\textwidth}
\raggedright
\textbf{What You'll Learn}:

\begin{itemize}
\item Practical HuggingFace usage
\item How to extract features
\item BERT vs GPT representations
\item Attention pattern interpretation
\item Feature-based transfer learning
\end{itemize}

\vspace{5mm}
\textbf{No Fine-tuning}:

We'll use models as feature extractors (simpler, faster)

\end{columns}

\vspace{5mm}
\begin{center}
\Large
\textbf{Let's explore pre-trained models hands-on!}
\end{center}

\vspace{0.2cm}
\bottomnote{Understanding by doing - the lab makes theory concrete}
\end{frame}

% ===== TECHNICAL APPENDIX (Slides A1-A6) =====

\begin{frame}[t]{}
\begin{center}
\Huge\textbf{Technical Appendix}

\vspace{5mm}
\Large\secondary{Architecture, Training, and Deployment Details}
\end{center}
\end{frame}

% Slide A1: BERT Architecture Specifications
\begin{frame}[t]{Appendix A: BERT Architecture Specifications}
\small
\begin{center}
\begin{tabular}{lcc}
\toprule
\textbf{Component} & \textbf{BERT-Base} & \textbf{BERT-Large} \\
\midrule
Transformer Layers & 12 & 24 \\
Hidden Size & 768 & 1024 \\
Attention Heads & 12 & 16 \\
Intermediate Size (FFN) & 3072 & 4096 \\
Total Parameters & 110M & 340M \\
Max Position Embeddings & 512 & 512 \\
Vocabulary Size (WordPiece) & 30,522 & 30,522 \\
Segment Embeddings & 2 & 2 \\
\midrule
\textbf{Training Specs} & & \\
Pre-training Corpus & \multicolumn{2}{c}{BooksCorpus (800M) + Wikipedia (2.5B)} \\
Batch Size & 256 & 256 \\
Steps & 1M & 1M \\
Learning Rate & 1e-4 & 1e-4 \\
Warmup Steps & 10,000 & 10,000 \\
Hardware & 16 TPU pods & 64 TPU pods \\
Training Time & 4 days & 4 days \\
\bottomrule
\end{tabular}
\end{center}

\vspace{0.2cm}
\bottomnote{BERT-Large has 3x parameters but same training time (more parallelism)}
\end{frame}

% Slide A2: GPT Architecture Specifications
\begin{frame}[t]{Appendix B: GPT Architecture Specifications}
\small
\begin{center}
\begin{tabular}{lccc}
\toprule
\textbf{Component} & \textbf{GPT-1} & \textbf{GPT-2} & \textbf{GPT-3} \\
\midrule
Decoder Layers & 12 & 48 & 96 \\
Hidden Size & 768 & 1600 & 12,288 \\
Attention Heads & 12 & 25 & 96 \\
Context Window & 512 & 1024 & 2048 \\
Parameters & 117M & 1.5B & 175B \\
Vocabulary (BPE) & 40K & 50K & 50K \\
\midrule
\textbf{Training} & & & \\
Dataset & BooksCorpus & WebText & Common Crawl \\
Dataset Size & 5GB & 40GB & 570GB \\
Tokens & 1B & 10B & 300B \\
Batch Size & 64 & 512 & 3.2M \\
GPUs & 8 & 256 & 10,000+ \\
Training Time & Weeks & Weeks & Months \\
Cost Estimate & \$50K & \$500K & \$4.6M \\
\bottomrule
\end{tabular}
\end{center}

\vspace{0.2cm}
\bottomnote{Exponential scaling in parameters, data, and compute}
\end{frame}

% Slide A3: Pre-training Hyperparameters
\begin{frame}[t]{Appendix C: Pre-training Hyperparameters}
\small
\begin{columns}[T]
\column{0.48\textwidth}
\raggedright
\textbf{BERT Pre-training}:

\begin{itemize}
\item \textbf{Optimizer}: Adam
\item \textbf{Learning rate}: 1e-4
\item \textbf{$\beta_1, \beta_2$}: 0.9, 0.999
\item \textbf{L2 weight decay}: 0.01
\item \textbf{Warmup steps}: 10,000
\item \textbf{LR schedule}: Linear decay
\item \textbf{Dropout}: 0.1
\item \textbf{Activation}: GELU
\item \textbf{Batch size}: 256 sequences
\item \textbf{Max steps}: 1,000,000
\item \textbf{Masking}: 15\% of tokens
\end{itemize}

\column{0.48\textwidth}
\raggedright
\textbf{GPT-3 Pre-training}:

\begin{itemize}
\item \textbf{Optimizer}: Adam
\item \textbf{Learning rate}: 6e-5 (peak)
\item \textbf{$\beta_1, \beta_2$}: 0.9, 0.95
\item \textbf{Weight decay}: 0.1
\item \textbf{Gradient clipping}: 1.0
\item \textbf{LR schedule}: Cosine decay
\item \textbf{Dropout}: Varies by layer
\item \textbf{Batch size}: 3.2M tokens
\item \textbf{Tokens}: 300B total
\item \textbf{Context window}: 2048
\item \textbf{Precision}: Mixed (FP16/FP32)
\end{itemize}

\end{columns}

\vspace{5mm}
\begin{center}
\textbf{Key Insight}: These are carefully tuned over months of experimentation
\end{center}

\vspace{0.2cm}
\bottomnote{Don't change these for fine-tuning - use proven recipes}
\end{frame}

% Slide A4: Fine-tuning Recipes
\begin{frame}[t]{Appendix D: Fine-tuning Recipes by Task}
\small
\begin{center}
\begin{tabular}{llll}
\toprule
\textbf{Task} & \textbf{Learning Rate} & \textbf{Epochs} & \textbf{Strategy} \\
\midrule
\textbf{Classification} & & & \\
Sentiment & 2e-5 & 3-4 & Full fine-tuning \\
Topic & 3e-5 & 2-3 & Full fine-tuning \\
Spam & 2e-5 & 4 & Freeze bottom 6 \\
\midrule
\textbf{Question Answering} & & & \\
SQuAD & 3e-5 & 2 & Full fine-tuning \\
Custom QA & 5e-5 & 3-4 & Full fine-tuning \\
\midrule
\textbf{NER} & & & \\
Named Entities & 5e-5 & 3 & Full fine-tuning \\
Domain-specific & 2e-5 & 4-5 & Freeze bottom 8 \\
\midrule
\textbf{Generation (GPT)} & & & \\
Completion & 2e-5 & 2-3 & Full fine-tuning \\
Dialogue & 1e-5 & 3-5 & Full fine-tuning \\
Summarization & 3e-5 & 2-3 & Full fine-tuning \\
\bottomrule
\multicolumn{4}{l}{\secondary{\footnotesize Batch size: 16-32, Warmup: 10\% of steps}} \\
\end{tabular}
\end{center}

\vspace{0.2cm}
\bottomnote{Start with these proven recipes, then experiment}
\end{frame}

% Slide A5: Training Cost Analysis
\begin{frame}[t]{Appendix E: Training Cost Analysis}
\small
\begin{columns}[T]
\column{0.48\textwidth}
\raggedright
\textbf{Pre-training Costs (2018-2024)}:

\begin{center}
\begin{tabular}{lr}
Model & Cost \\
\hline
BERT-base & \$7K \\
BERT-large & \$25K \\
GPT-1 & \$50K \\
GPT-2 & \$500K \\
GPT-3 & \$4.6M \\
GPT-4 (est) & \$50M+ \\
\end{tabular}
\end{center}

\vspace{3mm}
\textbf{Why So Expensive}:

\begin{itemize}
\item Massive datasets (100B-1T tokens)
\item Large models (1B-1T parameters)
\item Weeks/months on thousands of GPUs
\item Trial and error in hyperparameters
\end{itemize}

\column{0.48\textwidth}
\raggedright
\textbf{Fine-tuning Costs (Per Task)}:

\begin{center}
\begin{tabular}{lr}
Dataset Size & Cost \\
\hline
100 examples & \$5-10 \\
1,000 examples & \$50-100 \\
10,000 examples & \$200-500 \\
\end{tabular}
\end{center}

\vspace{3mm}
\textbf{The Economics}:

\begin{itemize}
\item Pre-training: One-time investment
\item Fine-tuning: Cheap per task
\item Amortize cost across many applications
\end{itemize}

\vspace{3mm}
\textbf{Business Model}:

OpenAI, Anthropic, Google: Pay for pre-training, sell API access

\end{columns}

\vspace{0.2cm}
\bottomnote{Pre-training economics enable the modern AI industry}
\end{frame}

% Slide A6: Further Reading
\begin{frame}[t]{Appendix F: Further Reading and Resources}
\small
\begin{columns}[T]
\column{0.48\textwidth}
\raggedright
\textbf{Original Papers}:

\begin{itemize}
\item \textbf{Attention}: Vaswani et al. (2017)

  ``Attention is All You Need''

\item \textbf{GPT-1}: Radford et al. (June 2018)

  ``Improving Language Understanding by Generative Pre-Training''

\item \textbf{BERT}: Devlin et al. (October 2018)

  ``BERT: Pre-training of Deep Bidirectional Transformers''

\item \textbf{GPT-2}: Radford et al. (2019)

  ``Language Models are Unsupervised Multitask Learners''

\item \textbf{GPT-3}: Brown et al. (2020)

  ``Language Models are Few-Shot Learners''
\end{itemize}

\column{0.48\textwidth}
\raggedright
\textbf{Practical Resources}:

\begin{itemize}
\item \textbf{HuggingFace Transformers}

  Library for using pre-trained models

  https://huggingface.co/transformers

\item \textbf{Model Hub}

  Thousands of pre-trained models

  https://huggingface.co/models

\item \textbf{Fine-tuning Tutorials}

  Official guides for common tasks

\item \textbf{Papers With Code}

  Leaderboards and implementations
\end{itemize}

\vspace{5mm}
\textbf{Next Week}:

Advanced architectures (T5, GPT-4, etc.)

\end{columns}

\vspace{0.2cm}
\bottomnote{These papers are essential reading for serious NLP practitioners}
\end{frame}

\end{document}
