\documentclass[8pt,aspectratio=169]{beamer}
\usepackage[utf8]{inputenc}
\usepackage{graphicx}
\usepackage{amsmath,amssymb}
\usepackage{algorithm2e}
\usepackage{listings}
\usepackage{xcolor}
\usepackage{tikz}
\usepackage{pgfplots}
\pgfplotsset{compat=1.17}
\usepackage{subfigure}
\usepackage{hyperref}
\usepackage{tcolorbox}
\usepackage{colortbl}
\usepackage{array}
\usepackage{booktabs}

% Theme settings
\usetheme{Frankfurt}
\usecolortheme{seahorse}
\setbeamertemplate{navigation symbols}{}
\setbeamertemplate{footline}[frame number]
\setbeamertemplate{section in toc}[sections numbered]
\setbeamertemplate{subsection in toc}[subsections numbered]

% Custom colors
\definecolor{darkblue}{RGB}{0,51,102}
\definecolor{lightblue}{RGB}{173,216,230}
\definecolor{codegreen}{RGB}{0,128,0}
\definecolor{codegray}{RGB}{150,150,150}
\definecolor{codepurple}{RGB}{128,0,128}
\definecolor{backcolor}{RGB}{245,245,245}

% Pedagogical boxes
\newtcolorbox{checkpoint}{
    colback=yellow!10,
    colframe=yellow!50!black,
    title={\textbf{Checkpoint}},
    fonttitle=\bfseries
}

\newtcolorbox{intuition}{
    colback=purple!10,
    colframe=purple!50!black,
    title={\textbf{Intuition}},
    fonttitle=\bfseries
}

\newtcolorbox{keyinsight}{
    colback=blue!10,
    colframe=blue!50!black,
    title={\textbf{Key Insight}},
    fonttitle=\bfseries
}

% Code listing settings
\lstset{
    backgroundcolor=\color{backcolor},
    basicstyle=\ttfamily\footnotesize,
    breakatwhitespace=false,
    breaklines=true,
    captionpos=b,
    commentstyle=\color{codegreen},
    keywordstyle=\color{blue},
    numberstyle=\tiny\color{codegray},
    stringstyle=\color{codepurple},
    showstringspaces=false,
    frame=single,
    numbers=left,
    language=Python
}

% Include slide layouts
% SLIDE LAYOUT TEMPLATES FOR NLP COURSE
% Three consistent layouts used throughout the course

% ==============================================================
% LAYOUT 1: CONCEPT SLIDE
% Used for: Theory, definitions, mathematical concepts
% Features: Title, main content area with bullets/equations, optional figure
% ==============================================================

\newcommand{\conceptslide}[3]{ % #1: title, #2: content, #3: optional figure
\begin{frame}[t]{#1}
    \begin{columns}[T]
        \begin{column}{0.6\textwidth}
            #2
        \end{column}
        \begin{column}{0.35\textwidth}
            \centering
            #3
        \end{column}
    \end{columns}
\end{frame}
}

% ==============================================================
% LAYOUT 2: CODE & IMPLEMENTATION SLIDE
% Used for: Code examples, algorithms, implementation details
% Features: Title, code block, explanation text
% ==============================================================

% Note: For code slides, use \begin{frame}[fragile] manually

\newcommand{\codeexplanation}[1]{%
    \vspace{0.5em}
    \small
    #1
}

\newcommand{\codeblock}[2][Python]{%
    \begin{lstlisting}[language=#1, basicstyle=\ttfamily\tiny]
#2
    \end{lstlisting}
}

\newcommand{\codeslide}[3]{%
    \begin{frame}[fragile]{#1}
    \begin{columns}[T]
        \column{0.55\textwidth}
        \codeblock{#2}
        \column{0.43\textwidth}
        \codeexplanation{#3}
    \end{columns}
    \end{frame}
}

% ==============================================================
% LAYOUT 3: RESULTS & VISUALIZATION SLIDE
% Used for: Experimental results, charts, comparisons
% Features: Title, large visualization area, key insights
% ==============================================================

\newcommand{\resultslide}[3]{ % #1: title, #2: visualization, #3: insights
\begin{frame}[t]{#1}
    \centering
    \vspace{-0.5em}
    #2
    \vspace{0.5em}
    
    \begin{block}{Key Insights}
        #3
    \end{block}
\end{frame}
}

% Additional utility commands for consistency

% Highlighted text
\newcommand{\highlight}[1]{\textcolor{blue}{\textbf{#1}}}

% Math notation shortcuts
\newcommand{\prob}[1]{P(#1)}
\newcommand{\given}{\mid}
\newcommand{\argmax}{\operatorname{argmax}}
\newcommand{\softmax}{\operatorname{softmax}}

% Standard figure size
\newcommand{\stdfigsizeinchins}{0.4\textwidth}

% For code blocks, use lstlisting directly:
% \begin{lstlisting}[language=Python]
% ... code ...
% \end{lstlisting}

% Equation highlight box
\newcommand{\eqbox}[1]{
    \begin{center}
    \colorbox{lightblue!20}{
        \parbox{0.8\textwidth}{
            \vspace{0.3em}
            \centering
            #1
            \vspace{0.3em}
        }
    }
    \end{center}
}

\title[Week 6: Pre-trained Models]{Natural Language Processing}
\subtitle{Week 6: Pre-trained Language Models in 4 Parts}
\author{Joerg R. Osterrieder \\ \url{www.joergosterrieder.com}}
\date{}

\begin{document}

% Title page
\begin{frame}
    \titlepage
\end{frame}

% Table of Contents
\begin{frame}{Roadmap: Your Journey Through Pre-training}
    \tableofcontents
\end{frame}

% ===============================================
% OPENING POWER VISUALIZATION
% ===============================================

\begin{frame}[t]{The Paradigm Shift That Changed Everything}
    \begin{center}
    \includegraphics[width=0.95\textwidth]{../figures/paradigm_shift_impact.pdf}
    \end{center}
    
    \vspace{-0.5em}
    \begin{keyinsight}
    Pre-training created a 100x efficiency gain, transforming AI from elite labs to everyone's toolkit
    \end{keyinsight}
\end{frame}

% ===============================================
% PART 1: THE REVOLUTION
% ===============================================

\section{Part 1: The Revolution - Why Pre-training Changed Everything}

\begin{frame}
    \centering
    \vspace{2cm}
    {\huge \textbf{Part 1}}\\
    \vspace{0.5cm}
    {\Large \textbf{The Revolution}}\\
    \vspace{0.3cm}
    {\large Why Pre-training Changed Everything}
\end{frame}

% The Waste Problem
\begin{frame}[t]{The \$10 Million Waste Problem}
    \textbf{Before 2018: Every Team Starting from Zero}
    
    \vspace{0.5em}
    \begin{columns}
    \column{0.5\textwidth}
    \textbf{Company A: Sentiment Analysis}
    \begin{itemize}
        \item Cost: \$500,000 in compute
        \item Time: 2 weeks on 64 GPUs
        \item Learns: Grammar, syntax, sentiment
    \end{itemize}
    
    \vspace{0.5em}
    \textbf{Company B: Question Answering}
    \begin{itemize}
        \item Cost: \$500,000 in compute
        \item Time: 2 weeks on 64 GPUs
        \item Learns: Grammar, syntax... again!
    \end{itemize}
    
    \column{0.5\textwidth}
    \textbf{The Insanity:}
    \begin{itemize}
        \item 20 companies = \$10 million wasted
        \item Each re-learning what ``the'' means
        \item 90\% duplicate effort
        \item Only 10\% on actual task
    \end{itemize}
    
    \vspace{0.5em}
    \begin{tcolorbox}[colback=red!10,colframe=red!50!black]
    \textbf{Reality Check:} \\
    Imagine teaching every medical student the alphabet before medicine!
    \end{tcolorbox}
    \end{columns}
\end{frame}

% The Breakthrough Insight
\begin{frame}[t]{The Breakthrough: Learning from Computer Vision}
    \begin{columns}
    \column{0.5\textwidth}
    \textbf{2012: ImageNet Moment in Vision}
    \begin{itemize}
        \item Pre-train on millions of images
        \item Learn edges, shapes, objects
        \item Fine-tune for specific tasks
        \item 10x performance improvement
    \end{itemize}
    
    \vspace{0.5em}
    \textbf{2018: The NLP Awakening}
    \begin{itemize}
        \item ``Why not do this for language?''
        \item Pre-train on all of Wikipedia
        \item Learn grammar, facts, reasoning
        \item Fine-tune for any language task
    \end{itemize}
    
    \column{0.5\textwidth}
    \begin{center}
    \includegraphics[width=\textwidth]{../figures/knowledge_transfer_mountain.pdf}
    \end{center}
    
    \begin{intuition}
    Like learning to read once, then applying to any book!
    \end{intuition}
    \end{columns}
\end{frame}

% Timeline of Revolution
\begin{frame}[t]{Timeline: The Pre-training Revolution}
    \begin{center}
    \includegraphics[width=0.9\textwidth]{../figures/performance_timeline.pdf}
    \end{center}
    
    \textbf{Key Milestones:}
    \begin{itemize}
        \item \textbf{2018}: BERT proves the concept - beats 11 benchmarks
        \item \textbf{2019}: GPT-2 shows generation capabilities
        \item \textbf{2020}: GPT-3 demonstrates few-shot learning
        \item \textbf{2023}: ChatGPT brings AI to billions
    \end{itemize}
\end{frame}

% Scale of Change
\begin{frame}[t]{The Scale of Transformation}
    \begin{columns}
    \column{0.5\textwidth}
    \textbf{Before Pre-training (2017)}
    \begin{itemize}
        \item Training cost: \$500K per task
        \item Training time: 2 weeks
        \item Data needed: 100K+ labeled examples
        \item Performance: 70-80\% accuracy
        \item Accessibility: PhD required
        \item Deployment: Months
    \end{itemize}
    
    \column{0.5\textwidth}
    \textbf{With Pre-training (2024)}
    \begin{itemize}
        \item Training cost: \$1K per task
        \item Training time: 2 hours
        \item Data needed: 1K examples
        \item Performance: 95\%+ accuracy
        \item Accessibility: API call
        \item Deployment: Minutes
    \end{itemize}
    \end{columns}
    
    \vspace{0.5em}
    \begin{center}
    \begin{tcolorbox}[colback=green!10,colframe=green!50!black,width=0.8\textwidth]
    \centering
    \textbf{Impact: 100x cost reduction, 100x speed increase, 10x performance gain}
    \end{tcolorbox}
    \end{center}
\end{frame}

% Democratization
\begin{frame}[t]{Democratization: From Elite Labs to Everyone}
    \begin{columns}
    \column{0.6\textwidth}
    \textbf{The Old World (Pre-2018)}
    \begin{itemize}
        \item Only Google, Facebook, Microsoft
        \item Requires ML PhD team
        \item Millions in infrastructure
        \item Months to deploy
    \end{itemize}
    
    \vspace{0.5em}
    \textbf{The New World (Post-2018)}
    \begin{itemize}
        \item Any developer with an API key
        \item No ML expertise needed
        \item \$20/month subscription
        \item Deploy in minutes
    \end{itemize}
    
    \column{0.4\textwidth}
    \textbf{Who Benefits:}
    \begin{itemize}
        \item Startups
        \item Students
        \item Researchers
        \item Small businesses
        \item Non-profits
        \item Individual developers
    \end{itemize}
    
    \vspace{0.5em}
    \begin{checkpoint}
    Can you explain why pre-training is like teaching someone to read?
    \end{checkpoint}
    \end{columns}
\end{frame}

% Part 1 Summary
\begin{frame}[t]{Part 1 Summary: The Revolution}
    \textbf{Key Takeaways:}
    \begin{enumerate}
        \item \textbf{The Problem}: Every team re-learning language basics = \$10M waste
        \item \textbf{The Insight}: Pre-train once on everything, fine-tune for specific tasks
        \item \textbf{The Impact}: 100x efficiency gain across cost, time, and accessibility
        \item \textbf{The Timeline}: 2018 BERT → 2023 ChatGPT = 5 years to change the world
        \item \textbf{The Democratization}: From PhD labs to every developer
    \end{enumerate}
    
    \vspace{0.5em}
    \begin{center}
    \begin{tcolorbox}[colback=blue!10,colframe=blue!50!black,width=0.9\textwidth]
    \centering
    \textbf{Next: How do models actually learn from raw text?}
    \end{tcolorbox}
    \end{center}
\end{frame}

% ===============================================
% PART 2: UNDERSTANDING PRE-TRAINING
% ===============================================

\section{Part 2: Understanding Pre-training - How Models Learn from Raw Text}

\begin{frame}
    \centering
    \vspace{2cm}
    {\huge \textbf{Part 2}}\\
    \vspace{0.5cm}
    {\Large \textbf{Understanding Pre-training}}\\
    \vspace{0.3cm}
    {\large How Models Learn from Raw Text}
\end{frame}

% Self-Supervised Learning
\begin{frame}[t]{The Magic: Self-Supervised Learning}
    \begin{columns}
    \column{0.5\textwidth}
    \textbf{Traditional Supervised Learning}
    \begin{itemize}
        \item Need: (text, label) pairs
        \item Example: (``I love this!'', positive)
        \item Problem: Expensive to label
        \item Scale: Thousands of examples
    \end{itemize}
    
    \vspace{0.5em}
    \textbf{Self-Supervised Learning}
    \begin{itemize}
        \item Need: Just raw text
        \item Example: ``The cat sat on the [?]''
        \item Advantage: Free labels from text itself
        \item Scale: Billions of examples
    \end{itemize}
    
    \column{0.5\textwidth}
    \begin{center}
    \textbf{Creating Labels from Text}
    
    \vspace{0.5em}
    Original: ``The quick brown fox jumps''
    
    \vspace{0.3em}
    $\downarrow$ Hide words
    
    \vspace{0.3em}
    Input: ``The [MASK] brown fox jumps''
    
    \vspace{0.3em}
    Target: ``quick''
    
    \vspace{0.5em}
    \begin{intuition}
    The text teaches itself! Every sentence becomes a training example.
    \end{intuition}
    \end{center}
    \end{columns}
\end{frame}

% BERT's Approach: MLM
\begin{frame}[t]{BERT's Approach: Masked Language Modeling}
    \begin{center}
    \includegraphics[width=0.95\textwidth]{../figures/mlm_interactive_animation.pdf}
    \end{center}
    
    \textbf{The Process:}
    \begin{enumerate}
        \item Randomly mask 15\% of words
        \item Use bidirectional context to predict
        \item Learn from prediction errors
        \item Repeat billions of times
    \end{enumerate}
\end{frame}

% MLM Interactive Example
\begin{frame}[fragile,t]{Try It Yourself: Fill in the Blanks}
    \textbf{BERT's Training Game - You Try:}
    
    \begin{enumerate}
        \item The capital of France is [MASK].
        \begin{itemize}
            \item Your guess: \underline{\hspace{3cm}}
            \item Answer: Paris
        \end{itemize}
        
        \item The [MASK] rises in the east.
        \begin{itemize}
            \item Your guess: \underline{\hspace{3cm}}
            \item Answer: sun
        \end{itemize}
        
        \item She [MASK] to the store yesterday.
        \begin{itemize}
            \item Your guess: \underline{\hspace{3cm}}
            \item Answer: went
        \end{itemize}
    \end{enumerate}
    
    \vspace{0.5em}
    \begin{lstlisting}[language=Python]
# BERT learns by playing this game billions of times
def train_bert(text):
    masked_text = randomly_mask(text, rate=0.15)
    prediction = bert_model(masked_text)
    loss = compare(prediction, original_text)
    update_weights(loss)
    \end{lstlisting}
\end{frame}

% GPT's Approach
\begin{frame}[t]{GPT's Approach: Next Token Prediction}
    \begin{columns}
    \column{0.5\textwidth}
    \textbf{Autoregressive Learning}
    \begin{itemize}
        \item Predict next word given previous
        \item Left-to-right processing
        \item Natural for generation
        \item Simpler than BERT's MLM
    \end{itemize}
    
    \vspace{0.5em}
    \textbf{Training Example:}
    \begin{itemize}
        \item Input: ``The cat sat''
        \item Target: ``on''
        \item Next: ``The cat sat on''
        \item Target: ``the''
        \item Continue for all text...
    \end{itemize}
    
    \column{0.5\textwidth}
    \textbf{Interactive Example:}
    
    Complete these sentences:
    \begin{enumerate}
        \item ``Once upon a...'' $\rightarrow$ \underline{\hspace{2cm}}
        \item ``To be or not to...'' $\rightarrow$ \underline{\hspace{2cm}}
        \item ``The weather today is...'' $\rightarrow$ \underline{\hspace{2cm}}
    \end{enumerate}
    
    \vspace{0.5em}
    \begin{keyinsight}
    GPT learns to write by predicting what comes next, just like you just did!
    \end{keyinsight}
    \end{columns}
\end{frame}

% Scale Matters
\begin{frame}[t]{Scale Matters: The Power of Big}
    \begin{center}
    \includegraphics[width=0.9\textwidth]{../figures/scale_comparison_chart.pdf}
    \end{center}
    
    \textbf{Why Bigger is Better:}
    \begin{itemize}
        \item More parameters = More knowledge capacity
        \item More data = Better understanding
        \item More compute = Deeper patterns
    \end{itemize}
\end{frame}

% What Models Learn
\begin{frame}[t]{What Do Models Actually Learn?}
    \begin{columns}
    \column{0.5\textwidth}
    \textbf{Layer 1-4: Syntax \& Grammar}
    \begin{itemize}
        \item Parts of speech
        \item Word order rules
        \item Basic grammar patterns
    \end{itemize}
    
    \textbf{Layer 5-8: Semantics}
    \begin{itemize}
        \item Word meanings
        \item Concept relationships
        \item Context understanding
    \end{itemize}
    
    \textbf{Layer 9-12: High-level Reasoning}
    \begin{itemize}
        \item World knowledge
        \item Logical relationships
        \item Abstract concepts
    \end{itemize}
    
    \column{0.5\textwidth}
    \textbf{Emergent Abilities:}
    \begin{itemize}
        \item Translation (never explicitly taught!)
        \item Summarization
        \item Question answering
        \item Code generation
        \item Mathematical reasoning
        \item Creative writing
    \end{itemize}
    
    \vspace{0.5em}
    \begin{intuition}
    Like a child learning language: first sounds, then words, then meaning, then reasoning
    \end{intuition}
    \end{columns}
\end{frame}

% Part 2 Summary
\begin{frame}[t]{Part 2 Summary: Understanding Pre-training}
    \textbf{Key Concepts:}
    \begin{enumerate}
        \item \textbf{Self-Supervised}: Text provides its own labels - no manual annotation!
        \item \textbf{BERT's MLM}: Fill in the blanks using bidirectional context
        \item \textbf{GPT's Autoregressive}: Predict next word from previous words
        \item \textbf{Scale Advantage}: Billions of parameters + terabytes of text = intelligence
        \item \textbf{Emergent Learning}: Models learn tasks they were never explicitly taught
    \end{enumerate}
    
    \vspace{0.5em}
    \begin{checkpoint}
    Can you explain the difference between BERT's and GPT's training approach?
    \end{checkpoint}
    
    \vspace{0.5em}
    \begin{center}
    \begin{tcolorbox}[colback=blue!10,colframe=blue!50!black,width=0.9\textwidth]
    \centering
    \textbf{Next: Deep dive into BERT vs GPT architectures}
    \end{tcolorbox}
    \end{center}
\end{frame}

% ===============================================
% PART 3: ARCHITECTURE DEEP DIVE
% ===============================================

\section{Part 3: Architecture Deep Dive - BERT vs GPT}

\begin{frame}
    \centering
    \vspace{2cm}
    {\huge \textbf{Part 3}}\\
    \vspace{0.5cm}
    {\Large \textbf{Architecture Deep Dive}}\\
    \vspace{0.3cm}
    {\large BERT vs GPT - Two Paradigms}
\end{frame}

% BERT Architecture
\begin{frame}[t]{BERT: Bidirectional Encoder Representations}
    \begin{columns}
    \column{0.5\textwidth}
    \textbf{Architecture Components:}
    \begin{itemize}
        \item \textbf{Input}: Token + Position + Segment embeddings
        \item \textbf{Core}: 12/24 Transformer encoder layers
        \item \textbf{Attention}: Bidirectional self-attention
        \item \textbf{Output}: Contextualized representations
    \end{itemize}
    
    \vspace{0.5em}
    \textbf{Key Innovation:}
    \begin{itemize}
        \item Sees both left AND right context
        \item Example: ``The [MASK] barked loudly''
        \item Uses: ``The'' + ``barked loudly''
        \item Better understanding than left-only
    \end{itemize}
    
    \column{0.5\textwidth}
    \begin{center}
    \includegraphics[width=\textwidth]{../figures/bert_vs_gpt_architecture.pdf}
    \end{center}
    
    \textbf{BERT Sizes:}
    \begin{itemize}
        \item BERT-Base: 110M parameters
        \item BERT-Large: 340M parameters
    \end{itemize}
    \end{columns}
\end{frame}

% BERT Training Process
\begin{frame}[t]{BERT Training: Two Objectives}
    \begin{columns}
    \column{0.5\textwidth}
    \textbf{1. Masked Language Model (MLM)}
    \begin{itemize}
        \item 15\% tokens masked
        \item 80\% replaced with [MASK]
        \item 10\% replaced with random word
        \item 10\% kept unchanged
    \end{itemize}
    
    \vspace{0.5em}
    \textbf{Why the 80-10-10 split?}
    \begin{itemize}
        \item Prevents overfitting to [MASK]
        \item Forces robust representations
        \item Handles noise in real data
    \end{itemize}
    
    \column{0.5\textwidth}
    \textbf{2. Next Sentence Prediction (NSP)}
    \begin{itemize}
        \item Input: Sentence A + Sentence B
        \item Task: Are they consecutive?
        \item 50\% true next sentence
        \item 50\% random sentence
    \end{itemize}
    
    \vspace{0.5em}
    \textbf{Example:}
    \begin{itemize}
        \item A: ``The weather is nice.''
        \item B: ``Let's go for a walk.'' $\checkmark$
        \item B: ``Cats like fish.'' $\times$
    \end{itemize}
    \end{columns}
    
    \vspace{0.5em}
    \begin{keyinsight}
    MLM teaches word understanding, NSP teaches sentence relationships
    \end{keyinsight}
\end{frame}

% GPT Architecture
\begin{frame}[t]{GPT: Generative Pre-trained Transformer}
    \begin{columns}
    \column{0.5\textwidth}
    \textbf{Architecture Components:}
    \begin{itemize}
        \item \textbf{Input}: Token + Position embeddings
        \item \textbf{Core}: 12/24/48 Transformer decoder layers
        \item \textbf{Attention}: Causal (left-to-right) mask
        \item \textbf{Output}: Next token predictions
    \end{itemize}
    
    \vspace{0.5em}
    \textbf{Key Design:}
    \begin{itemize}
        \item Only sees previous tokens
        \item Natural for generation
        \item Autoregressive decoding
        \item Simpler training objective
    \end{itemize}
    
    \column{0.5\textwidth}
    \textbf{GPT Evolution:}
    \begin{itemize}
        \item GPT-1 (2018): 117M parameters
        \item GPT-2 (2019): 1.5B parameters
        \item GPT-3 (2020): 175B parameters
        \item GPT-4 (2023): 1.76T parameters*
    \end{itemize}
    
    \vspace{0.5em}
    \textbf{Training Process:}
    \begin{enumerate}
        \item Input: ``The cat sat''
        \item Predict: ``on''
        \item Input: ``The cat sat on''
        \item Predict: ``the''
        \item Continue...
    \end{enumerate}
    \end{columns}
\end{frame}

% Comparison Chart
\begin{frame}[t]{BERT vs GPT: When to Use Which?}
    \begin{center}
    \begin{tabular}{|l|c|c|}
    \hline
    \textbf{Aspect} & \textbf{BERT} & \textbf{GPT} \\
    \hline
    \hline
    Context & Bidirectional & Left-to-right \\
    \hline
    Best for & Understanding & Generation \\
    \hline
    Training & MLM + NSP & Next token \\
    \hline
    Speed & Faster inference & Slower (autoregressive) \\
    \hline
    \hline
    \multicolumn{3}{|c|}{\textbf{Use Cases}} \\
    \hline
    Classification & \textbf{Excellent} & Good \\
    \hline
    NER & \textbf{Excellent} & Good \\
    \hline
    QA & \textbf{Excellent} & Good \\
    \hline
    Generation & Poor & \textbf{Excellent} \\
    \hline
    Translation & Good & \textbf{Excellent} \\
    \hline
    Summarization & Good & \textbf{Excellent} \\
    \hline
    \end{tabular}
    \end{center}
    
    \vspace{0.5em}
    \begin{intuition}
    BERT reads the whole page to understand, GPT writes one word at a time
    \end{intuition}
\end{frame}

% Tokenization Strategies
\begin{frame}[fragile,t]{Tokenization: Breaking Text into Pieces}
    \begin{columns}
    \column{0.5\textwidth}
    \textbf{BERT: WordPiece Tokenization}
    \begin{itemize}
        \item Vocabulary: 30,000 tokens
        \item Subword units
        \item Handles unknown words
    \end{itemize}
    
    \begin{lstlisting}[basicstyle=\ttfamily\tiny]
"unbelievable" -> 
["un", "##believ", "##able"]

"COVID-19" -> 
["COVID", "-", "19"]
    \end{lstlisting}
    
    \column{0.5\textwidth}
    \textbf{GPT: Byte-Pair Encoding (BPE)}
    \begin{itemize}
        \item Vocabulary: 50,000+ tokens
        \item Learned from frequency
        \item Efficient for generation
    \end{itemize}
    
    \begin{lstlisting}[basicstyle=\ttfamily\tiny]
"unbelievable" -> 
["un", "believ", "able"]

"COVID-19" -> 
["COVID", "-19"]
    \end{lstlisting}
    \end{columns}
    
    \vspace{0.5em}
    \textbf{Why Subword Tokenization?}
    \begin{itemize}
        \item Handles any word (even made-up ones)
        \item Reduces vocabulary size
        \item Captures morphology (prefixes, suffixes)
    \end{itemize}
\end{frame}

% Input Representations
\begin{frame}[t]{Input Representations: More Than Just Words}
    \textbf{BERT's Three-Part Input:}
    
    \begin{center}
    \begin{tabular}{|c|c|c|c|c|c|}
    \hline
    \textbf{Token} & [CLS] & The & cat & sat & [SEP] \\
    \hline
    Token Emb & $E_{CLS}$ & $E_{the}$ & $E_{cat}$ & $E_{sat}$ & $E_{SEP}$ \\
    \hline
    Position Emb & $E_0$ & $E_1$ & $E_2$ & $E_3$ & $E_4$ \\
    \hline
    Segment Emb & $E_A$ & $E_A$ & $E_A$ & $E_A$ & $E_A$ \\
    \hline
    \hline
    \textbf{Final Input} & \multicolumn{5}{c|}{Sum of all three embeddings} \\
    \hline
    \end{tabular}
    \end{center}
    
    \vspace{0.5em}
    \textbf{Special Tokens:}
    \begin{itemize}
        \item [CLS]: Classification token (sentence representation)
        \item [SEP]: Separator between sentences
        \item [MASK]: Masked token for MLM
        \item [PAD]: Padding for batch processing
    \end{itemize}
    
    \vspace{0.5em}
    \begin{keyinsight}
    Position embeddings tell the model word order, segment embeddings separate sentences
    \end{keyinsight}
\end{frame}

% Part 3 Summary
\begin{frame}[t]{Part 3 Summary: Architecture Insights}
    \textbf{BERT (Bidirectional)}
    \begin{itemize}
        \item Sees full context (left + right)
        \item Best for understanding tasks
        \item Two training objectives (MLM + NSP)
        \item Cannot generate text naturally
    \end{itemize}
    
    \textbf{GPT (Autoregressive)}
    \begin{itemize}
        \item Sees only previous context
        \item Best for generation tasks
        \item Single training objective (next token)
        \item Scales to trillions of parameters
    \end{itemize}
    
    \vspace{0.5em}
    \begin{checkpoint}
    Why can't BERT generate text as naturally as GPT?
    \end{checkpoint}
    
    \vspace{0.5em}
    \begin{center}
    \begin{tcolorbox}[colback=blue!10,colframe=blue!50!black,width=0.9\textwidth]
    \centering
    \textbf{Next: How to adapt these models to your specific tasks}
    \end{tcolorbox}
    \end{center}
\end{frame}

% ===============================================
% PART 4: FINE-TUNING AND APPLICATIONS
% ===============================================

\section{Part 4: Fine-tuning and Applications}

\begin{frame}
    \centering
    \vspace{2cm}
    {\huge \textbf{Part 4}}\\
    \vspace{0.5cm}
    {\Large \textbf{Fine-tuning and Applications}}\\
    \vspace{0.3cm}
    {\large From Foundation to Task-Specific Excellence}
\end{frame}

% Fine-tuning Strategies
\begin{frame}[t]{Fine-tuning Strategy Decision Tree}
    \begin{center}
    \includegraphics[width=0.95\textwidth]{../figures/finetuning_strategy_flowchart.pdf}
    \end{center}
\end{frame}

% Task-Specific Heads
\begin{frame}[fragile,t]{Task-Specific Heads: Adapting to Your Task}
    \begin{columns}
    \column{0.5\textwidth}
    \textbf{Classification Head}
    \begin{lstlisting}[basicstyle=\ttfamily\tiny,language=Python]
# Sentiment analysis
bert_output = bert(text)
cls_token = bert_output[0]  # [CLS]
logits = linear(cls_token)
class = softmax(logits)
# Output: positive/negative
    \end{lstlisting}
    
    \textbf{Token Classification (NER)}
    \begin{lstlisting}[basicstyle=\ttfamily\tiny,language=Python]
# Named entity recognition
bert_output = bert(tokens)
token_logits = linear(bert_output)
entities = softmax(token_logits)
# Output: PER, ORG, LOC per token
    \end{lstlisting}
    
    \column{0.5\textwidth}
    \textbf{Question Answering}
    \begin{lstlisting}[basicstyle=\ttfamily\tiny,language=Python]
# Extract answer span
bert_output = bert(question, context)
start_logits = linear_start(bert_output)
end_logits = linear_end(bert_output)
answer = context[start:end]
    \end{lstlisting}
    
    \textbf{Sequence-to-Sequence}
    \begin{lstlisting}[basicstyle=\ttfamily\tiny,language=Python]
# Summarization, translation
encoder_out = bert_encoder(source)
summary = gpt_decoder(encoder_out)
# Output: Generated text
    \end{lstlisting}
    \end{columns}
    
    \vspace{0.5em}
    \begin{keyinsight}
    The pre-trained model stays mostly the same, only the task head changes!
    \end{keyinsight}
\end{frame}

% Fine-tuning Process
\begin{frame}[t]{The Fine-tuning Process: Step by Step}
    \begin{columns}
    \column{0.5\textwidth}
    \textbf{Step 1: Choose Pre-trained Model}
    \begin{itemize}
        \item BERT for understanding
        \item GPT for generation
        \item T5 for any text-to-text
    \end{itemize}
    
    \textbf{Step 2: Prepare Task Data}
    \begin{itemize}
        \item Format: (input, label) pairs
        \item Quality > Quantity
        \item 1K-10K examples usually enough
    \end{itemize}
    
    \textbf{Step 3: Add Task Head}
    \begin{itemize}
        \item Classification: Linear + Softmax
        \item NER: Token classifier
        \item QA: Span predictor
    \end{itemize}
    
    \column{0.5\textwidth}
    \textbf{Step 4: Fine-tune}
    \begin{itemize}
        \item Lower learning rate (2e-5)
        \item Few epochs (2-4)
        \item Watch for overfitting
    \end{itemize}
    
    \textbf{Step 5: Evaluate}
    \begin{itemize}
        \item Hold-out test set
        \item Task-specific metrics
        \item Compare to baseline
    \end{itemize}
    
    \vspace{0.5em}
    \begin{intuition}
    Like teaching a well-educated person a new skill - they learn fast!
    \end{intuition}
    \end{columns}
\end{frame}

% Few-shot and Zero-shot
\begin{frame}[t]{Few-shot and Zero-shot: Learning Without Training}
    \begin{columns}
    \column{0.5\textwidth}
    \textbf{Zero-shot (No Examples)}
    \begin{itemize}
        \item Just describe the task
        \item Works with large models (GPT-3+)
        \item Example prompt:
    \end{itemize}
    
    \small
    ``Classify sentiment: 'This movie is amazing!' \\
    Answer: Positive''
    
    \normalsize
    \vspace{0.5em}
    \textbf{Few-shot (1-10 Examples)}
    \begin{itemize}
        \item Provide examples in prompt
        \item No actual training
        \item In-context learning
    \end{itemize}
    
    \column{0.5\textwidth}
    \textbf{Few-shot Example:}
    
    \small
    ``Translate English to French:\\
    sea otter → loutre de mer\\
    cheese → fromage\\
    airplane → avion\\
    teacher → ?''
    
    \normalsize
    \vspace{0.5em}
    Answer: professeur
    
    \vspace{0.5em}
    \begin{keyinsight}
    Large models can learn new tasks just from instructions!
    \end{keyinsight}
    \end{columns}
\end{frame}

% Real-world Applications
\begin{frame}[t]{Real-world Success Stories}
    \begin{center}
    \includegraphics[width=0.95\textwidth]{../figures/application_ecosystem_map.pdf}
    \end{center}
\end{frame}

% Environmental Considerations
\begin{frame}[t]{Environmental Considerations}
    \begin{columns}
    \column{0.5\textwidth}
    \textbf{The Carbon Cost}
    \begin{itemize}
        \item GPT-3 training: 1,287 MWh
        \item = 552 tons CO2
        \item = 120 cars for a year
    \end{itemize}
    
    \vspace{0.5em}
    \textbf{But Consider:}
    \begin{itemize}
        \item Train once, use millions of times
        \item Replaces thousands of task-specific models
        \item Net positive if widely used
    \end{itemize}
    
    \column{0.5\textwidth}
    \textbf{Efficiency Improvements}
    \begin{itemize}
        \item LoRA: 1\% of parameters
        \item Quantization: 4-bit models
        \item Distillation: Smaller models
        \item Better hardware: TPUs, H100s
    \end{itemize}
    
    \vspace{0.5em}
    \begin{checkpoint}
    How does fine-tuning reduce environmental impact compared to training from scratch?
    \end{checkpoint}
    \end{columns}
\end{frame}

% Efficient Fine-tuning
\begin{frame}[fragile,t]{Modern Efficiency: LoRA and PEFT}
    \begin{columns}
    \column{0.5\textwidth}
    \textbf{LoRA (Low-Rank Adaptation)}
    \begin{itemize}
        \item Only train 1\% of parameters
        \item Add small matrices to layers
        \item Same performance as full fine-tuning
        \item 100x less memory
    \end{itemize}
    
    \begin{lstlisting}[basicstyle=\ttfamily\tiny,language=Python]
# Instead of updating W (d x d)
# Add low-rank matrices A (d x r) and B (r x d)
# where r << d
W_new = W + A @ B
# Only train A and B!
    \end{lstlisting}
    
    \column{0.5\textwidth}
    \textbf{Benefits:}
    \begin{itemize}
        \item Fine-tune GPT-3 on single GPU
        \item Switch tasks by swapping LoRA weights
        \item Merge multiple adaptations
        \item Deploy efficiently
    \end{itemize}
    
    \vspace{0.5em}
    \textbf{Other PEFT Methods:}
    \begin{itemize}
        \item Prefix Tuning
        \item Prompt Tuning
        \item Adapter Layers
        \item BitFit
    \end{itemize}
    \end{columns}
\end{frame}

% Part 4 Summary
\begin{frame}[t]{Part 4 Summary: Practical Applications}
    \textbf{Key Takeaways:}
    \begin{enumerate}
        \item \textbf{Fine-tuning Strategy}: Choose based on data size and compute budget
        \item \textbf{Task Heads}: Simple additions for specific tasks
        \item \textbf{Few-shot Magic}: Large models learn from just examples
        \item \textbf{Applications}: 100+ tasks enabled by pre-training
        \item \textbf{Environmental}: Consider efficiency methods like LoRA
        \item \textbf{Modern Methods}: 1\% parameters, 100\% performance
    \end{enumerate}
    
    \vspace{0.5em}
    \begin{center}
    \begin{tcolorbox}[colback=green!10,colframe=green!50!black,width=0.9\textwidth]
    \centering
    \textbf{You now understand the complete pre-training pipeline!}
    \end{tcolorbox}
    \end{center}
\end{frame}

% ===============================================
% APPENDIX
% ===============================================

\section{Appendix: Resources and Advanced Topics}

\begin{frame}
    \centering
    \vspace{2cm}
    {\huge \textbf{Appendix}}\\
    \vspace{0.5cm}
    {\Large \textbf{Resources and Advanced Topics}}\\
    \vspace{0.3cm}
    {\large Your Toolkit for Getting Started}
\end{frame}

% Model Zoo
\begin{frame}[t]{Model Zoo: Available Pre-trained Models}
    \begin{columns}
    \column{0.5\textwidth}
    \textbf{Hugging Face Hub}
    \begin{itemize}
        \item 500,000+ models
        \item All major architectures
        \item Easy to use API
        \item Community contributed
    \end{itemize}
    
    \vspace{0.5em}
    \textbf{Popular Models:}
    \begin{itemize}
        \item \textbf{BERT}: bert-base-uncased
        \item \textbf{RoBERTa}: roberta-large
        \item \textbf{GPT-2}: gpt2-medium
        \item \textbf{T5}: t5-base
        \item \textbf{BART}: facebook/bart-large
    \end{itemize}
    
    \column{0.5\textwidth}
    \textbf{Specialized Models:}
    \begin{itemize}
        \item \textbf{Code}: Codex, CodeBERT
        \item \textbf{Science}: SciBERT, BioBERT
        \item \textbf{Legal}: LegalBERT
        \item \textbf{Finance}: FinBERT
        \item \textbf{Multilingual}: mBERT, XLM-R
    \end{itemize}
    
    \vspace{0.5em}
    \textbf{Access Methods:}
    \begin{itemize}
        \item Hugging Face Transformers
        \item OpenAI API
        \item Google Vertex AI
        \item AWS SageMaker
    \end{itemize}
    \end{columns}
\end{frame}

% Computational Requirements
\begin{frame}[t]{Computational Requirements}
    \begin{center}
    \small
    \begin{tabular}{|l|c|c|c|}
    \hline
    \textbf{Model} & \textbf{Parameters} & \textbf{Memory} & \textbf{Fine-tune GPU} \\
    \hline
    \hline
    BERT-Base & 110M & 440MB & GTX 1080 (8GB) \\
    \hline
    BERT-Large & 340M & 1.3GB & RTX 3090 (24GB) \\
    \hline
    GPT-2 & 1.5B & 6GB & A100 (40GB) \\
    \hline
    GPT-3 & 175B & 700GB & 8x A100 (320GB) \\
    \hline
    \hline
    \multicolumn{4}{|c|}{\textbf{With LoRA (1\% parameters):}} \\
    \hline
    GPT-3 + LoRA & 1.75B & 7GB & RTX 4090 (24GB) \\
    \hline
    \end{tabular}
    \end{center}
    
    \vspace{0.5em}
    \textbf{Cost Estimates (2024):}
    \begin{itemize}
        \item Cloud GPU (A100): \$3-5/hour
        \item Fine-tuning BERT: \$10-50
        \item Fine-tuning GPT-3 with LoRA: \$100-500
        \item API calls: \$0.002 per 1K tokens
    \end{itemize}
\end{frame}

% Latest Developments
\begin{frame}[t]{Latest Developments (2024)}
    \begin{columns}
    \column{0.5\textwidth}
    \textbf{New Models:}
    \begin{itemize}
        \item \textbf{GPT-4}: Multimodal, 1.76T params
        \item \textbf{Claude 3}: Constitutional AI
        \item \textbf{Gemini}: Google's unified model
        \item \textbf{Llama 3}: Open-source 405B
        \item \textbf{Mistral}: Efficient 7B model
    \end{itemize}
    
    \textbf{Trends:}
    \begin{itemize}
        \item Mixture of Experts (MoE)
        \item Multimodal (text + image + audio)
        \item Longer context (1M+ tokens)
        \item Tool use and function calling
    \end{itemize}
    
    \column{0.5\textwidth}
    \textbf{Research Directions:}
    \begin{itemize}
        \item Constitutional AI
        \item Chain-of-thought reasoning
        \item Retrieval-augmented generation
        \item Sparse models
        \item Continuous learning
    \end{itemize}
    
    \vspace{0.5em}
    \textbf{Open Problems:}
    \begin{itemize}
        \item Hallucination
        \item Bias and fairness
        \item Interpretability
        \item Efficiency at scale
    \end{itemize}
    \end{columns}
\end{frame}

% Quick Implementation
\begin{frame}[fragile,t]{Quick Start: Your First Fine-tuning}
    \textbf{5-Minute Setup with Hugging Face:}
    
    \begin{lstlisting}[language=Python,basicstyle=\ttfamily\scriptsize]
from transformers import AutoModelForSequenceClassification, AutoTokenizer
from transformers import TrainingArguments, Trainer
import torch

# 1. Load pre-trained model
model = AutoModelForSequenceClassification.from_pretrained(
    "bert-base-uncased", num_labels=2
)
tokenizer = AutoTokenizer.from_pretrained("bert-base-uncased")

# 2. Prepare your data
texts = ["I love this!", "This is terrible."]
labels = [1, 0]  # positive, negative

# 3. Tokenize
inputs = tokenizer(texts, padding=True, truncation=True, return_tensors="pt")
inputs["labels"] = torch.tensor(labels)

# 4. Fine-tune
training_args = TrainingArguments(
    output_dir="./results",
    num_train_epochs=3,
    per_device_train_batch_size=16,
)
trainer = Trainer(model=model, args=training_args, train_dataset=inputs)
trainer.train()
    \end{lstlisting}
\end{frame}

% Best Practices
\begin{frame}[t]{Best Practices and Tips}
    \begin{columns}
    \column{0.5\textwidth}
    \textbf{Data Preparation:}
    \begin{itemize}
        \item Quality > Quantity
        \item Balance your classes
        \item Clean and consistent formatting
        \item Augment if needed
    \end{itemize}
    
    \textbf{Training Tips:}
    \begin{itemize}
        \item Start with small learning rate (2e-5)
        \item Use warmup steps
        \item Monitor validation loss
        \item Early stopping to prevent overfit
    \end{itemize}
    
    \column{0.5\textwidth}
    \textbf{Common Pitfalls:}
    \begin{itemize}
        \item Overfitting on small data
        \item Wrong tokenizer for model
        \item Catastrophic forgetting
        \item Ignoring class imbalance
    \end{itemize}
    
    \textbf{Debugging:}
    \begin{itemize}
        \item Start with tiny dataset
        \item Verify input shapes
        \item Check gradient flow
        \item Visualize attention weights
    \end{itemize}
    \end{columns}
\end{frame}

% Resources
\begin{frame}[t]{Learning Resources}
    \begin{columns}
    \column{0.5\textwidth}
    \textbf{Papers to Read:}
    \begin{itemize}
        \item BERT: Devlin et al. (2018)
        \item GPT-3: Brown et al. (2020)
        \item T5: Raffel et al. (2019)
        \item LoRA: Hu et al. (2021)
    \end{itemize}
    
    \textbf{Courses:}
    \begin{itemize}
        \item Hugging Face Course (free)
        \item CS224N Stanford NLP
        \item Fast.ai Practical Deep Learning
        \item Coursera NLP Specialization
    \end{itemize}
    
    \column{0.5\textwidth}
    \textbf{Tools and Libraries:}
    \begin{itemize}
        \item Hugging Face Transformers
        \item LangChain
        \item OpenAI API
        \item Weights \& Biases
    \end{itemize}
    
    \textbf{Communities:}
    \begin{itemize}
        \item Hugging Face Forums
        \item r/MachineLearning
        \item Twitter ML Community
        \item Local AI Meetups
    \end{itemize}
    \end{columns}
\end{frame}

% Quick Reference
\begin{frame}[fragile,t]{Quick Reference Card}
    \begin{columns}
    \column{0.5\textwidth}
    \textbf{Model Selection:}
    \begin{itemize}
        \item Understanding $\rightarrow$ BERT
        \item Generation $\rightarrow$ GPT
        \item Both $\rightarrow$ T5
        \item Efficiency $\rightarrow$ DistilBERT
    \end{itemize}
    
    \textbf{Data Requirements:}
    \begin{itemize}
        \item Few-shot: 1-10 examples
        \item LoRA: 100-1K examples
        \item Full fine-tune: 1K-10K examples
        \item From scratch: 100K+ examples
    \end{itemize}
    
    \column{0.5\textwidth}
    \textbf{Key Commands:}
    
    \begin{lstlisting}[basicstyle=\ttfamily\scriptsize,language=Python]
# Install
pip install transformers

# Load model
from transformers import AutoModel
model = AutoModel.from_pretrained("bert-base")

# Fine-tune
trainer.train()

# Inference
outputs = model(**inputs)
    \end{lstlisting}
    \end{columns}
\end{frame}

% Final Slide
\begin{frame}[t]{Congratulations! You're Ready to Build}
    \begin{center}
    \Large
    \textbf{What You've Learned:}
    \end{center}
    
    \begin{enumerate}
        \item Why pre-training revolutionized NLP (100x efficiency)
        \item How models learn from raw text (self-supervised)
        \item BERT vs GPT architectures (bidirectional vs autoregressive)
        \item Fine-tuning strategies (full, LoRA, few-shot)
        \item Real-world applications (100+ tasks)
    \end{enumerate}
    
    \vspace{0.5em}
    \begin{center}
    \begin{tcolorbox}[colback=green!10,colframe=green!50!black,width=0.9\textwidth]
    \centering
    \Large
    \textbf{You now have the knowledge to use pre-trained models!}\\
    \vspace{0.3em}
    \normalsize
    Start with the Jupyter notebook exercises
    \end{tcolorbox}
    \end{center}
    
    \vspace{0.5em}
    \begin{center}
    \textbf{Questions?}
    \end{center}
\end{frame}

\end{document}