\documentclass[8pt,aspectratio=169,8pt]{beamer}
\usepackage[utf8]{inputenc}
\usepackage{graphicx}
\usepackage{amsmath,amssymb}
\usepackage{algorithm2e}
\usepackage{listings}
\usepackage{xcolor}
\usepackage{tikz}
\usepackage{pgfplots}
\pgfplotsset{compat=1.17}
\usepackage{subfigure}
\usepackage{hyperref}

% Theme settings
\usetheme{Frankfurt}
\usecolortheme{seahorse}
\setbeamertemplate{navigation symbols}{}
\setbeamertemplate{footline}[frame number]

% Custom colors
\definecolor{darkblue}{RGB}{0,51,102}
\definecolor{lightblue}{RGB}{173,216,230}
\definecolor{codegreen}{RGB}{0,128,0}
\definecolor{codegray}{RGB}{150,150,150}
\definecolor{codepurple}{RGB}{128,0,128}
\definecolor{backcolor}{RGB}{245,245,245}

% Code listing settings
\lstset{
    backgroundcolor=\color{backcolor},
    basicstyle=\ttfamily\footnotesize,
    breakatwhitespace=false,
    breaklines=true,
    captionpos=b,
    commentstyle=\color{codegreen},
    keywordstyle=\color{blue},
    numberstyle=\tiny\color{codegray},
    stringstyle=\color{codepurple},
    showstringspaces=false,
    frame=single,
    numbers=left,
    language=Python
}

% Include slide layouts
% SLIDE LAYOUT TEMPLATES FOR NLP COURSE
% Three consistent layouts used throughout the course

% ==============================================================
% LAYOUT 1: CONCEPT SLIDE
% Used for: Theory, definitions, mathematical concepts
% Features: Title, main content area with bullets/equations, optional figure
% ==============================================================

\newcommand{\conceptslide}[3]{ % #1: title, #2: content, #3: optional figure
\begin{frame}[t]{#1}
    \begin{columns}[T]
        \begin{column}{0.6\textwidth}
            #2
        \end{column}
        \begin{column}{0.35\textwidth}
            \centering
            #3
        \end{column}
    \end{columns}
\end{frame}
}

% ==============================================================
% LAYOUT 2: CODE & IMPLEMENTATION SLIDE
% Used for: Code examples, algorithms, implementation details
% Features: Title, code block, explanation text
% ==============================================================

% Note: For code slides, use \begin{frame}[fragile] manually

\newcommand{\codeexplanation}[1]{%
    \vspace{0.5em}
    \small
    #1
}

\newcommand{\codeblock}[2][Python]{%
    \begin{lstlisting}[language=#1, basicstyle=\ttfamily\tiny]
#2
    \end{lstlisting}
}

\newcommand{\codeslide}[3]{%
    \begin{frame}[fragile]{#1}
    \begin{columns}[T]
        \column{0.55\textwidth}
        \codeblock{#2}
        \column{0.43\textwidth}
        \codeexplanation{#3}
    \end{columns}
    \end{frame}
}

% ==============================================================
% LAYOUT 3: RESULTS & VISUALIZATION SLIDE
% Used for: Experimental results, charts, comparisons
% Features: Title, large visualization area, key insights
% ==============================================================

\newcommand{\resultslide}[3]{ % #1: title, #2: visualization, #3: insights
\begin{frame}[t]{#1}
    \centering
    \vspace{-0.5em}
    #2
    \vspace{0.5em}
    
    \begin{block}{Key Insights}
        #3
    \end{block}
\end{frame}
}

% Additional utility commands for consistency

% Highlighted text
\newcommand{\highlight}[1]{\textcolor{blue}{\textbf{#1}}}

% Math notation shortcuts
\newcommand{\prob}[1]{P(#1)}
\newcommand{\given}{\mid}
\newcommand{\argmax}{\operatorname{argmax}}
\newcommand{\softmax}{\operatorname{softmax}}

% Standard figure size
\newcommand{\stdfigsizeinchins}{0.4\textwidth}

% For code blocks, use lstlisting directly:
% \begin{lstlisting}[language=Python]
% ... code ...
% \end{lstlisting}

% Equation highlight box
\newcommand{\eqbox}[1]{
    \begin{center}
    \colorbox{lightblue!20}{
        \parbox{0.8\textwidth}{
            \vspace{0.3em}
            \centering
            #1
            \vspace{0.3em}
        }
    }
    \end{center}
}

\title[Week 6: Pre-trained]{Natural Language Processing Course}
\subtitle{Week 6: Pre-trained Language Models}
\author{Joerg R. Osterrieder \\ \url{www.joergosterrieder.com}}
\date{}

\begin{document}

% Title page
\begin{frame}
    \titlepage
\end{frame}

\section{Week 6: Pre-trained Language Models}

% Title slide
\begin{frame}
    \centering
    \vspace{2cm}
    {\Large \textbf{Week 6}}\\
    \vspace{0.5cm}
    {\huge \textbf{Pre-trained Language Models}}\\
    \vspace{1cm}
    {\large Learning from All of Human Knowledge}
\end{frame}

% Motivation: The waste problem
\begin{frame}[t]{The \$10 Million Problem No One Talked About}
    \textbf{Training a language model from scratch (2018):}
    \vspace{0.5em}
    Company A trains model for sentiment analysis:
    \begin{itemize}
        \item Cost: \$500,000 in compute\footnotemark
        \item Time: 2 weeks on 64 GPUs
        \item Learns: Grammar, syntax, word meanings, sentiment
    \end{itemize}
    
    \vspace{0.5em}
    Company B trains model for question answering:
    \begin{itemize}
        \item Cost: \$500,000 in compute
        \item Time: 2 weeks on 64 GPUs
        \item Learns: Grammar, syntax, word meanings... again!
    \end{itemize}
    
    \vspace{0.5em}
    \begin{center}
    \colorbox{red!20}{
        \parbox{0.8\textwidth}{
            \centering
            Every team re-learning what "the" means from scratch!
        }
    }
    \end{center}
    
    \vspace{0.5em}
    \textbf{The waste:} 90\% of training teaches same basic language understanding
    
    \footnotetext{Based on 2018 cloud GPU pricing and typical training times}
\end{frame}

% The transfer learning revolution
\begin{frame}[t]{The Revolution: Learn Language Once, Use Everywhere}
    \textbf{The breakthrough idea (2018):}\footnotemark
    
    \vspace{0.5em}
    What if we:
    \begin{enumerate}
        \item Train ONE model on massive text (Wikipedia, books, web)
        \item Learn general language understanding
        \item Share this pre-trained model with everyone
        \item Fine-tune for specific tasks with little data
    \end{enumerate}
    
    \vspace{0.5em}
    \textbf{The impact:}
    \begin{itemize}
        \item Training cost: \$10M → \$100\footnotemark
        \item Training time: Weeks → Hours
        \item Data needed: 1M examples → 1K examples
        \item Performance: 70\% → 95\% accuracy
    \end{itemize}
    
    \vspace{0.5em}
    \begin{center}
    \colorbox{green!20}{
        \parbox{0.8\textwidth}{
            \centering
            Transfer Learning: Don't start from scratch, start from knowledge!
        }
    }
    \end{center}
    
    \footnotetext[1]{Howard \& Ruder (2018) ULMFiT; Radford et al. (2018) GPT; Devlin et al. (2019) BERT}
    \footnotetext[2]{Fine-tuning costs vs pre-training from scratch}
\end{frame}

% Real-world impact
\begin{frame}[t]{Pre-trained Models Power Everything (2024)}
    \begin{columns}[T]
        \begin{column}{0.5\textwidth}
            \textbf{Search \& Understanding:}
            \begin{itemize}
                \item Google Search: BERT since 2019\footnotemark
                \item Affects 10\% of all queries
                \item Bing: Multiple BERT variants
                \item Every modern search engine
            \end{itemize}
            
            \vspace{0.5em}
            \textbf{Writing Assistants:}
            \begin{itemize}
                \item Grammarly: BERT-based
                \item Google Docs: Smart compose
                \item Microsoft Editor
                \item All use pre-trained models
            \end{itemize}
        \end{column}
        \begin{column}{0.5\textwidth}
            \textbf{Business Applications:}
            \begin{itemize}
                \item Customer service: 80\% automated\footnotemark
                \item Document analysis
                \item Email classification
                \item Resume screening
            \end{itemize}
            
            \vspace{0.5em}
            \textbf{Key Models:}
            \begin{itemize}
                \item BERT: 110M-340M parameters
                \item GPT-2: 1.5B parameters
                \item RoBERTa: Better BERT training
                \item T5: Unified text-to-text
            \end{itemize}
        \end{column}
    \end{columns}
    
    \vspace{0.5em}
    \begin{center}
    \colorbox{lightblue!30}{
        \parbox{0.8\textwidth}{
            \centering
            2024: You never train from scratch - always start from pre-trained
        }
    }
    \end{center}
    
    \footnotetext[1]{Google blog: "Understanding searches better than ever before"}
    \footnotetext[2]{Gartner report on AI automation}
\end{frame}

% Learning objectives
\begin{frame}[t]{Week 6: What You'll Master}
    \textbf{By the end of this week, you will:}
    \begin{itemize}
        \item \textbf{Understand} why pre-training changes everything
        \item \textbf{Master} BERT's bidirectional approach
        \item \textbf{Implement} masked language modeling
        \item \textbf{Compare} BERT vs GPT architectures
        \item \textbf{Fine-tune} models for your own tasks
    \end{itemize}
    
    \vspace{0.5em}
    \begin{center}
    \colorbox{lightblue!30}{
        \parbox{0.8\textwidth}{
            \centering
            \textbf{Core Insight:} Language understanding is a reusable skill
        }
    }
    \end{center}
\end{frame}

% The pre-training concept
\begin{frame}[t]{Pre-training: Learning from Raw Text}
    \textbf{Traditional supervised learning:}
    \begin{itemize}
        \item Need: Labeled data (expensive!)
        \item Example: 100K sentiment labels
        \item Problem: Starts from random weights
    \end{itemize}
    
    \vspace{0.5em}
    \textbf{Self-supervised pre-training:}
    \begin{itemize}
        \item Need: Just text (free and abundant!)
        \item Example: All of Wikipedia (6B tokens)
        \item Advantage: Learns language before task
    \end{itemize}
    
    \vspace{0.5em}
    \textbf{The clever trick - Create labels from text itself:}
    \begin{enumerate}
        \item Take: "The cat sat on the [?]"
        \item Model predicts: "mat"
        \item No human labeling needed!
        \item Can use internet-scale data
    \end{enumerate}
    
    \vspace{0.5em}
    \begin{center}
    \colorbox{yellow!20}{
        \parbox{0.8\textwidth}{
            \centering
            Pre-training = Teaching models to read before teaching them tasks
        }
    }
    \end{center}
\end{frame}

% BERT's approach
\begin{frame}[t]{BERT: Bidirectional Understanding}
    \textbf{The limitation of GPT (left-to-right):}
    
    "The man went to the [MASK] to buy milk"
    \begin{itemize}
        \item GPT only sees: "The man went to the"
        \item Can't use "to buy milk" as context!
    \end{itemize}
    
    \vspace{0.5em}
    \textbf{BERT's innovation: Look both ways!}\footnotemark
    \begin{itemize}
        \item Sees entire sentence: "The man went to the [MASK] to buy milk"
        \item Can use both left AND right context
        \item Predicts: "store" (using "buy milk" as clue)
    \end{itemize}
    
    \vspace{0.5em}
    \textbf{Masked Language Model (MLM):}
    \begin{enumerate}
        \item Randomly mask 15\% of tokens
        \item Predict masked tokens using all context
        \item Learn deep bidirectional representations
    \end{enumerate}
    
    \vspace{0.5em}
    \begin{center}
    \colorbox{lightblue!30}{
        \parbox{0.8\textwidth}{
            \centering
            BERT = Transformer encoder + Bidirectional context
        }
    }
    \end{center}
    
    \footnotetext{Devlin et al. (2019). "BERT: Pre-training of Deep Bidirectional Transformers"}
\end{frame}

% BERT visualization
\begin{frame}[t]{BERT's Training Process Visualized}
    \centering
    \includegraphics[width=0.7\textwidth]{../figures/bert_training_process.pdf}
    
    \vspace{0.5em}
    \textbf{Key insights:}
    \begin{itemize}
        \item Random masking prevents memorization
        \item Bidirectional context captures full meaning
        \item Also predicts if sentences follow each other (NSP)
        \item Trained on BookCorpus (800M words) + Wikipedia (2.5B words)
    \end{itemize}
\end{frame}

% GPT's approach
\begin{frame}[t]{GPT: Generative Pre-training}
    \textbf{Different philosophy: Predict the future}\footnotemark
    
    \vspace{0.5em}
    Given: "The cat sat on the"
    
    Predict: "mat" (then "in", then "the", then "sun"...)
    
    \vspace{0.5em}
    \textbf{Autoregressive training:}
    \begin{enumerate}
        \item Process text left-to-right
        \item Predict next token at each step
        \item Can generate coherent text!
        \item Same objective as traditional LM
    \end{enumerate}
    
    \vspace{0.5em}
    \textbf{Key differences from BERT:}
    \begin{itemize}
        \item Unidirectional (left-to-right only)
        \item Can generate text naturally
        \item Simpler training (no masking tricks)
        \item Transformer decoder architecture
    \end{itemize}
    
    \vspace{0.5em}
    \begin{center}
    \colorbox{yellow!20}{
        \parbox{0.8\textwidth}{
            \centering
            GPT = Transformer decoder + Autoregressive training
        }
    }
    \end{center}
    
    \footnotetext{Radford et al. (2018). "Improving Language Understanding by Generative Pre-Training"}
\end{frame}

% BERT implementation
\begin{frame}[fragile]{Implementing BERT's Masked Language Model}
    \begin{columns}[T]
\column{0.55\textwidth}
\begin{lstlisting}[language=Python,basicstyle=\ttfamily\tiny]
import torch
import torch.nn as nn
import random

class BERTMasking:
    def __init__(self, vocab_size, mask_token_id, mask_prob=0.15):
        """BERT-style masking for MLM"""
        self.vocab_size = vocab_size
        self.mask_token_id = mask_token_id
        self.mask_prob = mask_prob
        
    def mask_tokens(self, inputs, special_tokens_mask=None):
        """Mask tokens for BERT training"""
        labels = inputs.clone()
        
        # Create probability matrix
        probability_matrix = torch.full(labels.shape, self.mask_prob)
        if special_tokens_mask is not None:
            probability_matrix.masked_fill_(special_tokens_mask, value=0.0)
            
        masked_indices = torch.bernoulli(probability_matrix).bool()
        labels[~masked_indices] = -100  # Only compute loss on masked
        
        # 80% of time, replace with [MASK]
        indices_replaced = torch.bernoulli(
            torch.full(labels.shape, 0.8)).bool() & masked_indices
        inputs[indices_replaced] = self.mask_token_id
        
        # 10% of time, replace with random word
        indices_random = torch.bernoulli(
            torch.full(labels.shape, 0.5)).bool() & masked_indices & ~indices_replaced
        random_words = torch.randint(self.vocab_size, labels.shape, dtype=torch.long)
        inputs[indices_random] = random_words[indices_random]
        
        # 10% of time, keep original (for robustness)
        return inputs, labels
\end{lstlisting}
  \column{0.43\textwidth}
        \codeexplanation{
            \textbf{BERT's 80-10-10 Rule:}
            \begin{itemize}
                \item 80\%: Replace with [MASK]
                \item 10\%: Replace with random token
                \item 10\%: Keep original token
            \end{itemize}
            
            \vspace{0.3em}
            \textbf{Why this complexity?}
            \begin{itemize}
                \item [MASK] never seen in fine-tuning
                \item Random replacement adds noise
                \item Original tokens prevent mismatch
            \end{itemize}
            
            \vspace{0.3em}
            \textbf{Design validates through:}
            \begin{itemize}
                \item Ablation studies in paper
                \item 15\% masking is optimal
                \item Too much masking hurts learning
            \end{itemize}
        }
    \end{columns}
\end{frame}

% Complete BERT model
\begin{frame}[fragile]{BERT Model Architecture}
    \begin{columns}[T]
                \column{0.55\textwidth}
\begin{lstlisting}[language=Python,basicstyle=\ttfamily\tiny]
class BERTModel(nn.Module):
    def __init__(self, vocab_size, hidden_size=768, num_layers=12, 
                 num_heads=12, max_length=512, dropout=0.1):
        """BERT model for masked language modeling"""
        super().__init__()
        
        # Token embeddings + position + segment
        self.token_embeddings = nn.Embedding(vocab_size, hidden_size)
        self.position_embeddings = nn.Embedding(max_length, hidden_size)
        self.segment_embeddings = nn.Embedding(2, hidden_size)
        
        self.layer_norm = nn.LayerNorm(hidden_size)
        self.dropout = nn.Dropout(dropout)
        
        # Transformer encoder layers
        encoder_layer = nn.TransformerEncoderLayer(
            hidden_size, num_heads, dim_feedforward=4*hidden_size,
            dropout=dropout, activation='gelu'
        )
        self.transformer = nn.TransformerEncoder(encoder_layer, num_layers)
        
        # MLM head
        self.mlm_head = nn.Sequential(
            nn.Linear(hidden_size, hidden_size),
            nn.GELU(),
            nn.LayerNorm(hidden_size),
            nn.Linear(hidden_size, vocab_size)
        )
        
    def forward(self, input_ids, segment_ids=None, attention_mask=None):
        """Forward pass for MLM"""
        seq_length = input_ids.size(1)
        position_ids = torch.arange(seq_length, device=input_ids.device)
        
        # Combine embeddings
        token_embeds = self.token_embeddings(input_ids)
        position_embeds = self.position_embeddings(position_ids)
        segment_embeds = self.segment_embeddings(segment_ids) if segment_ids is not None else 0
        
        embeddings = token_embeds + position_embeds + segment_embeds
        embeddings = self.layer_norm(embeddings)
        embeddings = self.dropout(embeddings)
        
        # Transformer encoding
        hidden_states = self.transformer(embeddings.transpose(0, 1))
        
        # MLM predictions
        predictions = self.mlm_head(hidden_states.transpose(0, 1))
        
        return predictions
\end{lstlisting}
                \column{0.43\textwidth}

        \codeexplanation{
            \textbf{BERT Sizes:}
            \begin{itemize}
                \item BASE: 12 layers, 768 hidden, 110M params\footnotemark
                \item LARGE: 24 layers, 1024 hidden, 340M params
            \end{itemize}
            
            \vspace{0.3em}
            \textbf{Key Components:}
            \begin{itemize}
                \item Segment embeddings for sentence pairs
                \item GELU activation (smoother than ReLU)
                \item Layer norm before attention
            \end{itemize}
            
            \footnotetext{Most applications use BERT-BASE}
        }
    \end{columns}
\end{frame}

% Fine-tuning process
\begin{frame}[t]{Fine-tuning: From General to Specific}
    \textbf{The fine-tuning recipe:}
    
    \vspace{0.5em}
    \begin{enumerate}
        \item Start with pre-trained BERT/GPT
        \item Add task-specific head (1-2 layers)
        \item Train on your labeled data
        \item 100x less data needed!
    \end{enumerate}
    
    \vspace{0.5em}
    \centering
    \includegraphics[width=0.8\textwidth]{../figures/fine_tuning_process.pdf}
\end{frame}

% Results comparison
\resultslide{Pre-training Impact: BERT Dominates Every Task}{
    \centering
    \includegraphics[width=0.85\textwidth]{../figures/bert_results_glue.pdf}
}{
    \begin{itemize}
        \item GLUE benchmark: 9 language understanding tasks
        \item BERT improved SOTA on ALL tasks simultaneously
        \item Average score: 82.1 (previous best: 75.1)
        \item Some tasks: 10+ point improvements
        \item Started the "BERT era" of NLP
    \end{itemize}
}

% Transfer learning power
\begin{frame}[t]{The Power of Transfer Learning}
    \centering
    \includegraphics[width=0.55\textwidth]{../figures/transfer_learning_efficiency.pdf}
    
    \vspace{0.5em}
    \textbf{Key insights:}
    \begin{itemize}
        \item With 100 examples: BERT matches 10K from-scratch training
        \item Especially powerful for low-resource languages
        \item Works across very different tasks
        \item The more diverse pre-training, the better transfer
    \end{itemize}
\end{frame}

% Model zoo
\begin{frame}[t]{The Pre-trained Model Zoo (2024)}
    \begin{columns}[T]
        \begin{column}{0.5\textwidth}
            \textbf{General Purpose:}
            \begin{itemize}
                \item BERT: Understanding tasks
                \item GPT-2/3: Generation tasks
                \item T5: Any text-to-text task
                \item ELECTRA: Efficient BERT alternative
            \end{itemize}
            
            \vspace{0.5em}
            \textbf{Domain Specific:}
            \begin{itemize}
                \item BioBERT: Biomedical text
                \item SciBERT: Scientific papers
                \item FinBERT: Financial documents
                \item CodeBERT: Programming code
            \end{itemize}
        \end{column}
        \begin{column}{0.5\textwidth}
            \textbf{Multilingual:}
            \begin{itemize}
                \item mBERT: 104 languages
                \item XLM-R: 100 languages
                \item mT5: Text-to-text multilingual
            \end{itemize}
            
            \vspace{0.5em}
            \textbf{Where to find:}
            \begin{itemize}
                \item Hugging Face: 500K+ models\footnotemark
                \item Google TF Hub
                \item PyTorch Hub
                \item Company model hubs
            \end{itemize}
        \end{column}
    \end{columns}
    
    \vspace{0.5em}
    \begin{center}
    \colorbox{yellow!20}{
        \parbox{0.8\textwidth}{
            \centering
            2024 Reality: Never train from scratch - there's a model for that!
        }
    }
    \end{center}
    
    \footnotetext{Hugging Face hosts majority of open models}
\end{frame}

% Common pitfalls
\begin{frame}[t]{Pre-trained Model Pitfalls}
    \textbf{1. Domain Mismatch}
    \begin{itemize}
        \item Problem: BERT trained on books/wiki, you have tweets
        \item Solution: Domain-adaptive pre-training
        \item Continue pre-training on your domain
    \end{itemize}
    
    \vspace{0.5em}
    \textbf{2. Catastrophic Forgetting}
    \begin{itemize}
        \item Problem: Fine-tuning destroys pre-trained knowledge
        \item Solution: Lower learning rates (2e-5), gradual unfreezing
    \end{itemize}
    
    \vspace{0.5em}
    \textbf{3. Overfitting on Small Data}
    \begin{itemize}
        \item Problem: Large model, small dataset
        \item Solution: Freeze lower layers, only train top layers
    \end{itemize}
    
    \vspace{0.5em}
    \textbf{Real Example - Customer Service Bot:}
    \begin{itemize}
        \item Started with BERT: 72\% accuracy
        \item Continued pre-training on support tickets: 85\%
        \item Fine-tuned with careful LR: 94\%
    \end{itemize}
\end{frame}

% Exercise
\begin{frame}[t]{Week 6 Exercise: Fine-tune Your Own BERT}
    \textbf{Your Mission:} Take pre-trained BERT and make it expert at your task
    
    \vspace{0.5em}
    \textbf{Part 1: Understand Pre-training}
    \begin{itemize}
        \item Load pre-trained BERT from Hugging Face
        \item Examine what it already knows (probe tasks)
        \item Visualize attention patterns
    \end{itemize}
    
    \vspace{0.5em}
    \textbf{Part 2: Fine-tune for Classification}
    \begin{itemize}
        \item Choose: Sentiment, spam, or topic classification
        \item Add classification head to BERT
        \item Compare: 100 vs 1000 vs 10000 examples
        \item Track how quickly it learns
    \end{itemize}
    
    \vspace{0.5em}
    \textbf{Part 3: Advanced Experiments}
    \begin{itemize}
        \item Try different learning rates
        \item Freeze vs unfreeze layers
        \item Compare BERT vs training from scratch
        \item Measure training time savings
    \end{itemize}
    
    \vspace{0.5em}
    \textbf{You'll discover:} Why no one trains from scratch anymore!
\end{frame}

% Summary
\begin{frame}[t]{Key Takeaways: The Pre-training Revolution}
    \textbf{What we learned:}
    \begin{itemize}
        \item Training from scratch wastes resources
        \item Pre-training learns reusable language understanding
        \item BERT: Bidirectional with masking
        \item GPT: Autoregressive generation
        \item Fine-tuning needs 100x less data
    \end{itemize}
    
    \vspace{0.5em}
    \textbf{The paradigm shift:}
    \begin{center}
    \colorbox{lightblue!30}{
        \parbox{0.8\textwidth}{
            \centering
            Task-specific training → Pre-train then fine-tune
        }
    }
    \end{center}
    
    \vspace{0.5em}
    \textbf{Why it matters:}
    \begin{itemize}
        \item Democratized NLP (anyone can use BERT)
        \item Enabled rapid prototyping
        \item Made NLP practical for businesses
    \end{itemize}
    
    \vspace{0.5em}
    \textbf{Next week: Advanced Transformers}
    
    How do we scale to GPT-3's 175B parameters and beyond?
\end{frame}

% References
\begin{frame}[t]{References and Further Reading}
    \textbf{Foundational Papers:}
    \begin{itemize}
        \item Devlin et al. (2019). "BERT: Pre-training of Deep Bidirectional Transformers"
        \item Radford et al. (2018). "Improving Language Understanding by Generative Pre-Training"
        \item Liu et al. (2019). "RoBERTa: A Robustly Optimized BERT"
    \end{itemize}
    
    \textbf{Practical Resources:}
    \begin{itemize}
        \item Hugging Face Transformers library
        \item "The Illustrated BERT" by Jay Alammar
        \item Google's BERT GitHub repository
    \end{itemize}
    
    \textbf{Recent Advances:}
    \begin{itemize}
        \item ELECTRA: More efficient pre-training
        \item DeBERTa: Disentangled attention
        \item Transfer learning survey (Qiu et al., 2020)
    \end{itemize}
\end{frame}

\end{document}
