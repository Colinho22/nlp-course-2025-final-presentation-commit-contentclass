% Act I: The Problem That Started Everything
\begin{frame}{Act I: The Problem That Started Everything}
\begin{center}
{\Large \textbf{1950s: The Mail Sorting Crisis}}
\end{center}
\vspace{5mm}
\begin{columns}
\column{0.48\textwidth}
\textbf{The Challenge:}
\begin{itemize}
\item 150 million letters per day
\item Hand-written addresses
\item Human sorters: slow, expensive, error-prone
\item Traditional programming: useless
\end{itemize}

\column{0.48\textwidth}
\textbf{Why Traditional Code Failed:}
\begin{itemize}
\item Can't write rules for every handwriting style
\item Too many variations of each letter
\item Context matters: "I" vs "l" vs "1"
\item This wasn't computation---it was \highlight{pattern recognition}
\end{itemize}
\end{columns}
\bottomnote{ This problem would take 40 years to solve properly}
\end{frame}

% NEW: Historical Timeline
\begin{frame}{80 Years of Neural Networks: The Complete Journey}
\begin{center}
\includegraphics[width=0.95\textwidth]{../figures/historical_timeline.pdf}
\end{center}
\bottomnote{ From theoretical neurons to ChatGPT: Each breakthrough built on previous failures}
\end{frame}

% Why Rules Don't Work
\begin{frame}[fragile]{Why Can't We Just Write Rules?}
\begin{center}
\textbf{Problem: Recognize the Letter "A"}
\end{center}
\vspace{5mm}
\begin{columns}
\column{0.48\textwidth}
\textbf{Traditional Approach (Failed):}
\begin{lstlisting}[basicstyle=\tiny]
if (has_triangle_top AND
    has_horizontal_bar AND
    two_diagonal_lines) {
  return "A"
}
\end{lstlisting}
\secondary{\small But what about...}
\begin{itemize}
\item Handwritten A's?
\item Different fonts?
\item Rotated A's?
\item Partial A's?
\end{itemize}

\column{0.48\textwidth}
\begin{center}
\includegraphics[width=\textwidth]{../figures/various_a_styles.pdf}
\end{center}
\secondary{\small Just for the letter "A", we'd need thousands of rules!}
\end{columns}
\bottomnote{ The breakthrough: What if machines could learn patterns like children do?}
\end{frame}

% NEW: McCulloch-Pitts Historical Context
\begin{frame}{1943: The First Spark - McCulloch \& Pitts}
\begin{center}
\textbf{The Birth of Computational Neuroscience}
\end{center}
\vspace{5mm}
\begin{columns}
\column{0.48\textwidth}
\textbf{The Revolutionary Paper:}
\begin{itemize}
\item "A Logical Calculus of Ideas Immanent in Nervous Activity"
\item First mathematical model of neurons
\item Proved: Networks can compute ANY logical function
\item Inspired von Neumann's computer architecture
\end{itemize}

\textbf{Key Insight:}
\begin{itemize}
\item Neurons = Logic gates
\item Brain = Computing machine
\item Thinking = Computation
\end{itemize}

\column{0.48\textwidth}
\textbf{The Model:}
\begin{itemize}
\item Binary neurons (0 or 1)
\item Threshold activation
\item Fixed connections
\item No learning yet!
\end{itemize}

\textbf{Historical Impact:}
\begin{itemize}
\item Founded field of neural networks
\item Influenced cybernetics movement
\item Set stage for AI research
\item "The brain is a computer" metaphor
\end{itemize}
\end{columns}
\bottomnote{ 14 years later, Rosenblatt would add the missing piece: learning}
\end{frame}

% 1957: The First Attempt
\begin{frame}{1957: The First Learning Machine - The Perceptron}
\begin{center}
\textbf{Frank Rosenblatt's Radical Idea: Neurons That Learn}
\end{center}
\vspace{5mm}
\begin{columns}
\column{0.48\textwidth}
\textbf{Beyond McCulloch-Pitts:}
\begin{itemize}
\item Adjustable weights (not fixed!)
\item Learning from mistakes
\item Physical machine built (Mark I)
\item Could recognize simple patterns
\end{itemize}

\textbf{The Hardware:}
\begin{itemize}
\item 400 photocells (20$\times$20 ``retina'')
\item 512 motor-driven potentiometers
\item Weights adjusted by electric motors
\item Took 5 minutes to learn patterns
\end{itemize}

\column{0.48\textwidth}
\textbf{Mathematical Model:}
\begin{itemize}
\item Inputs: $x_1, x_2, ..., x_n$
\item Weights: $w_1, w_2, ..., w_n$
\item Sum: $z = \sum_{i=1}^{n} w_i x_i + b$
\item Output: $y = \begin{cases} 1 & \text{if } z > 0 \\ 0 & \text{if } z \leq 0 \end{cases}$
\end{itemize}

\plainmath{Each input gets a vote (weight). We add up all votes plus a bias. If total is positive, output 1; otherwise 0.}

\textbf{Learning Rule:}
If wrong: $w_i = w_i + \eta \cdot error \cdot x_i$
\end{columns}
\bottomnote{ The New York Times, 1958: "The Navy revealed the embryo of an electronic computer that will be able to walk, talk, see, write, reproduce itself and be conscious of its existence."}
\end{frame}

% NEW: Perceptron Hardware Visualization
\begin{frame}{The Mark I Perceptron: A Physical Learning Machine}
\begin{center}
\includegraphics[width=0.85\textwidth]{../figures/perceptron_hardware.pdf}
\end{center}
\bottomnote{ The first neural network wasn't software---it was a room-sized machine with motors and photocells}
\end{frame}