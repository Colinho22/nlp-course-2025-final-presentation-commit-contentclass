% Intermission: From Story to Science
\begin{frame}{Intermission: From Story to Science}
\begin{center}
{\Large \textbf{Let's Understand How This Actually Works}}
\end{center}
\vspace{10mm}

\begin{columns}
\column{0.48\textwidth}
\textbf{We've Seen the History...}
\begin{itemize}
\item McCulloch-Pitts invented the neuron
\item Rosenblatt made it learn
\item The perceptron was born
\end{itemize}

\column{0.48\textwidth}
\textbf{Now Let's See the Science:}
\begin{itemize}
\item How does a neuron calculate?
\item What does learning mean?
\item Why was XOR so hard?
\end{itemize}
\end{columns}

\vspace{10mm}
\begin{center}
\colorbox{checkpointBlue!20}{\parbox{0.7\textwidth}{
\centering
\textbf{Next 5 slides: Hands-on calculations and exercises}\\
\small Get your pencil ready - we're going to work through real examples!
}}
\end{center}

\bottomnote{ Don't worry - we'll return to the story once you understand the basics}
\end{frame}

% NEW: Understanding Checkpoint
\begin{frame}{Understanding Check: Can You Answer These?}
\begin{center}
\textbf{\checkpoint{Let's Make Sure We're Together}}
\end{center}
\vspace{5mm}
\begin{columns}
\column{0.48\textwidth}
\textbf{Quick Questions:}
\begin{enumerate}
\item Why couldn't traditional programming solve mail sorting?
\item What does a weight represent in simple terms?
\item Why do we need the bias term?
\item What was revolutionary about Rosenblatt's perceptron?
\end{enumerate}

\column{0.48\textwidth}
\textbf{Think About It:}
\begin{itemize}
\item A weight is like the importance/trust we give to each input
\item Bias shifts our decision threshold
\item Learning = adjusting these weights
\item The perceptron was the first machine that could learn!
\end{itemize}
\end{columns}
\vspace{5mm}
\tryit{Draw a simple perceptron with 2 inputs. Label the weights, bias, and output. What would the weights be to compute AND logic?}
\bottomnote{ If any of these are unclear, revisit the previous slides before continuing}
\end{frame}

% The Math Behind It (Simple)
\begin{frame}{Making It Concrete: Teaching OR Logic}
\begin{center}
\textbf{Problem: Learn OR function (output 1 if ANY input is 1)}
\end{center}
\vspace{5mm}
\begin{columns}
\column{0.5\textwidth}
\textbf{Training Data:}
\begin{center}
\begin{tabular}{cc|c}
$x_1$ & $x_2$ & Output \\
\hline
0 & 0 & 0 \\
0 & 1 & 1 \\
1 & 0 & 1 \\
1 & 1 & 1 \\
\end{tabular}
\end{center}

\textbf{The Perceptron:}
\begin{align*}
z &= w_1 \cdot x_1 + w_2 \cdot x_2 + b \\
\text{output} &= \begin{cases} 1 & \text{if } z > 0 \\ 0 & \text{if } z \leq 0 \end{cases}
\end{align*}
\plainmath{Multiply first input by first weight, second input by second weight, add bias, check if positive}

\column{0.5\textwidth}
\textbf{Learning Process:}
\begin{enumerate}
\item Start with random weights
\item For each example:
  \begin{itemize}
  \item Calculate output
  \item If wrong: adjust weights
  \item If correct: keep weights
  \end{itemize}
\item Repeat until all correct
\end{enumerate}

\textbf{Final Solution:}
$w_1 = 1$, $w_2 = 1$, $b = -0.5$
\end{columns}
\bottomnote{ Success! But this was just the beginning...}
\end{frame}

% NEW: Let's Calculate Together - Spam Detection
\begin{frame}{Let's Calculate Together: Is This Email Spam?}
\begin{center}
\textbf{\checkpoint{A Real Perceptron Calculation You Can Follow}}
\end{center}
\vspace{5mm}
\begin{columns}
\column{0.48\textwidth}
\textbf{The Email:}
\fbox{\parbox{0.9\textwidth}{\small
"FREE money! Click here NOW for amazing offer!!!"
}}

\textbf{Our Features (Inputs):}
\begin{itemize}
\item $x_1$ = Has "FREE"? = 1
\item $x_2$ = Has "money"? = 1
\item $x_3$ = Many "!"? = 1
\item $x_4$ = From friend? = 0
\end{itemize}

\textbf{Learned Weights:}
\begin{itemize}
\item $w_1$ = +3 (FREE is very spammy)
\item $w_2$ = +2 (money is suspicious)
\item $w_3$ = +2 (!!! is aggressive)
\item $w_4$ = -5 (friends are trusted)
\item $b$ = -2 (threshold)
\end{itemize}

\column{0.48\textwidth}
\textbf{Let's Calculate:}
\begin{align*}
z &= w_1 \cdot x_1 + w_2 \cdot x_2 + w_3 \cdot x_3 + w_4 \cdot x_4 + b \\
  &= 3 \cdot 1 + 2 \cdot 1 + 2 \cdot 1 + (-5) \cdot 0 + (-2) \\
  &= 3 + 2 + 2 + 0 - 2 \\
  &= 5
\end{align*}

\textbf{Decision:}
\begin{itemize}
\item $z = 5 > 0$
\item Output = 1 = SPAM!
\end{itemize}

\tryit{What if this email WAS from a friend ($x_4 = 1$)? Recalculate! Would it still be spam?}

\textbf{Answer:} $z = 5 - 5 = 0$, borderline!
\end{columns}
\bottomnote{ This is exactly how early spam filters worked - and why they failed on clever spam}
\end{frame}

% Notation Explained
\begin{frame}{Understanding the Notation}
\begin{center}
\textbf{Breaking Down the Math Symbols}
\end{center}
\vspace{5mm}
\begin{columns}
\column{0.48\textwidth}
\textbf{Inputs and Weights:}
\begin{itemize}
\item $x_i$ = input value (what we see)
\item $w_i$ = weight (importance/strength)
\item $b$ = bias (threshold adjuster)
\end{itemize}

\textbf{The Computation:}
$$z = \sum_{i=1}^{n} w_i x_i + b$$

This means:
\begin{itemize}
\item Multiply each input by its weight
\item Add them all up
\item Add the bias
\end{itemize}

\column{0.48\textwidth}
\textbf{Real Example:}
\begin{center}
Should I go outside? \\[3mm]
\begin{tabular}{lcc}
Factor & Value & Weight \\
\hline
Sunny? & 1 & +2 \\
Raining? & 0 & -3 \\
Weekend? & 1 & +1 \\
\hline
\end{tabular}
\end{center}
$$z = (1 \times 2) + (0 \times -3) + (1 \times 1) = 3$$
$$\text{Decision: } z > 0 \text{, so YES!}$$
\end{columns}
\bottomnote{ This simple math would evolve into deep learning}
\end{frame}