% FOUNDATIONAL CONCEPT: Neural Networks as Function Approximators
% These three slides establish the core mathematical concept before the historical narrative

% Slide 1: What is Function Approximation?
\begin{frame}{The Core Idea: Neural Networks are Function Approximators}
\begin{center}
\textbf{What does this actually mean?}
\end{center}
\vspace{1mm}

\begin{columns}
\column{0.32\textwidth}
\textbf{The Problem:}
\begin{itemize}
\item We have inputs (x)
\item We want outputs (y)
\item But we don't know the formula!
\item Examples:
  \begin{itemize}
  \footnotesize
  \item Size $\rightarrow$ Price
  \item Image $\rightarrow$ Label
  \item Text $\rightarrow$ Sentiment
  \end{itemize}
\end{itemize}

\column{0.32\textwidth}
\textbf{Traditional Approach:}
\begin{itemize}
\item Guess the formula
\item Write explicit rules
\item Hope it works
\item \warning{Problem:} Real world is too complex!
\end{itemize}

\vspace{3mm}
\textit{Example:}\\
\footnotesize
Price = a $\times$ Size + b\\
(Too simple for real data!)

\column{0.32\textwidth}
\textbf{Neural Network Approach:}
\begin{itemize}
\item Learn from examples
\item Build the formula automatically
\item Adjust until it fits
\item \success{Works for ANY pattern!}
\end{itemize}

\vspace{3mm}
\textit{Magic:}\\
\footnotesize
NN learns: f(x) $\approx$ y\\
No formula needed!
\end{columns}

\vspace{2mm}
\begin{center}
\includegraphics[width=0.75\textwidth]{../figures/function_approx_basics.pdf}
\end{center}

\vspace{2mm}
\plainmath{A function approximator finds patterns in data without being told what the pattern is}
\end{frame}

% Slide 2: How Neural Networks Build Functions
\begin{frame}{How NNs Build Complex Functions from Simple Pieces}
\begin{center}
\textbf{The LEGO Principle: Combine Simple Parts to Build Anything}
\end{center}
\vspace{1mm}

\begin{columns}
\column{0.48\textwidth}
\textbf{The Building Blocks:}
\begin{enumerate}
\item \textbf{Individual Neurons:} Simple decisions
   \begin{itemize}
   \footnotesize
   \item "Is input > threshold?"
   \item Outputs: on/off (smooth version)
   \end{itemize}

\item \textbf{Combine Neurons:} Weight and add
   \begin{itemize}
   \footnotesize
   \item Create complex shapes
   \end{itemize}

\item \textbf{Stack Layers:} Build hierarchy
   \begin{itemize}
   \footnotesize
   \item Each layer adds abstraction
   \end{itemize}
\end{enumerate}

\column{0.48\textwidth}
\textbf{Real-World Analogy:}

\textit{Making a Cake from Ingredients:}
\begin{itemize}
\footnotesize
\item Flour + Sugar + Eggs
\item Mix right amounts
\item $\rightarrow$ Perfect cake!
\end{itemize}

\textbf{In Neural Networks:}
\begin{itemize}
\footnotesize
\item Edges + Curves + Colors
\item Combine with weights
\item $\rightarrow$ Recognize faces!
\end{itemize}
\end{columns}

\begin{center}
\includegraphics[width=0.55\textwidth]{../figures/nn_building_blocks.pdf}
\end{center}
\end{frame}

% Slide 3: Universal Approximation Theorem in Plain English
\begin{frame}{The Universal Approximation Theorem: Why This Always Works}
\begin{center}
\textbf{The Most Important Theorem in Deep Learning (Cybenko, 1989)}
\end{center}
\vspace{1mm}

\begin{columns}
\column{0.48\textwidth}
\textbf{The Theorem (Plain English):}

\colorbox{checkpointBlue!20}{\parbox{0.95\columnwidth}{
\centering
\textit{"A neural network with enough neurons can approximate \textbf{ANY} continuous function to \textbf{ANY} desired accuracy"}
}}

\vspace{2mm}
\textbf{What This Means:}
\begin{itemize}
\item \success{Universal:} Works for any smooth pattern
\item \success{Guaranteed:} Not hoping, but proving
\item \success{Practical:} Just add more neurons!
\end{itemize}

\vspace{1mm}
\textbf{The Catch:}
\begin{itemize}
\item \warning{How many neurons?} Could be millions
\item \warning{How to train?} That's the art
\end{itemize}

\column{0.48\textwidth}
\textbf{Intuitive Proof:}

\textit{Pixel art analogy:}
\begin{itemize}
\footnotesize
\item 4 pixels: Blocky
\item 100 pixels: Recognizable
\item 10,000 pixels: Photo-realistic
\end{itemize}

\textit{Same with neurons:}
\begin{itemize}
\footnotesize
\item Few: Rough
\item More: Better
\item Many: Nearly perfect
\end{itemize}

\vspace{1mm}
\end{columns}

\begin{center}
\includegraphics[width=0.55\textwidth]{../figures/universal_approximation.pdf}
\end{center}

\vspace{2mm}
\confusion{Common mistake: "Universal" doesn't mean "easy" or "fast" - it just means "possible"!}
\end{frame}