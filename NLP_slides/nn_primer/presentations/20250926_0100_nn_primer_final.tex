% Neural Networks Primer: Teaching Machines to See Patterns
% Final BSc-Enhanced Version with Complete Visualizations
% All "Act" references removed, advanced topics moved to appendices

\documentclass[8pt,aspectratio=169]{beamer}
\usetheme{Madrid}
\usecolortheme{seahorse}

% Include preamble with all package imports and command definitions
% Neural Networks Primer: Teaching Machines to See Patterns
% BSc-Enhanced Version with Template Layout from template_beamer_final.tex
% Based on the excellent original with targeted accessibility enhancements

% Additional packages for template compatibility
\usepackage{graphicx}
\usepackage{booktabs}
\usepackage{adjustbox}
\usepackage{multicol}
\usepackage{tikz}
\usepackage{amsmath}
\usepackage{amssymb}
\usepackage{listings}

% ML Theme Color definitions from template
\definecolor{mlblue}{RGB}{0,102,204}
\definecolor{mlpurple}{RGB}{51,51,178}
\definecolor{mllavender}{RGB}{173,173,224}
\definecolor{mllavender2}{RGB}{193,193,232}
\definecolor{mllavender3}{RGB}{204,204,235}
\definecolor{mllavender4}{RGB}{214,214,239}
\definecolor{mlorange}{RGB}{255, 127, 14}
\definecolor{mlgreen}{RGB}{44, 160, 44}
\definecolor{mlred}{RGB}{214, 39, 40}
\definecolor{mlgray}{RGB}{127, 127, 127}

% Additional colors for template compatibility
\definecolor{lightgray}{RGB}{240, 240, 240}
\definecolor{midgray}{RGB}{180, 180, 180}

% Keep checkpointBlue for backward compatibility
\definecolor{checkpointBlue}{RGB}{70,130,180}

% Apply custom colors to Madrid theme
\setbeamercolor{palette primary}{bg=mllavender3,fg=mlpurple}
\setbeamercolor{palette secondary}{bg=mllavender2,fg=mlpurple}
\setbeamercolor{palette tertiary}{bg=mllavender,fg=white}
\setbeamercolor{palette quaternary}{bg=mlpurple,fg=white}

\setbeamercolor{structure}{fg=mlpurple}
\setbeamercolor{section in toc}{fg=mlpurple}
\setbeamercolor{subsection in toc}{fg=mlblue}
\setbeamercolor{title}{fg=mlpurple}
\setbeamercolor{frametitle}{fg=mlpurple,bg=mllavender3}
\setbeamercolor{block title}{bg=mllavender2,fg=mlpurple}
\setbeamercolor{block body}{bg=mllavender4,fg=black}

% Remove navigation symbols
\setbeamertemplate{navigation symbols}{}

% Clean itemize/enumerate
\setbeamertemplate{itemize items}[circle]
\setbeamertemplate{enumerate items}[default]

% Reduce margins for more content space
\setbeamersize{text margin left=5mm,text margin right=5mm}

% Commands with updated colors
\newcommand{\highlight}[1]{{\color{mlpurple}#1}}
\newcommand{\secondary}[1]{{\color{mllavender2}#1}}
\newcommand{\warning}[1]{{\color{mlorange}#1}}
\newcommand{\success}[1]{{\color{mlgreen}#1}}
\newcommand{\checkpoint}[1]{{\color{checkpointBlue}#1}}

% Command for bottom annotation (Madrid-style) from template
\newcommand{\bottomnote}[1]{%
\vfill
\vspace{-2mm}
\textcolor{mllavender2}{\rule{\textwidth}{0.4pt}}
\vspace{1mm}
\footnotesize
\textbf{#1}
}

% Box commands for special content with updated colors
\newcommand{\plainmath}[1]{\fbox{\parbox{0.9\textwidth}{\small\textit{In plain words: #1}}}}
\newcommand{\confusion}[1]{\colorbox{mlorange!20}{\parbox{0.9\textwidth}{\small\warning{Common Confusion:} #1}}}
\newcommand{\whymatters}[1]{\colorbox{mlgreen!10}{\parbox{0.3\textwidth}{\tiny\success{Why it matters:} #1}}}
\newcommand{\tryit}[1]{\colorbox{checkpointBlue!15}{\parbox{0.9\textwidth}{\small\textbf{Try It Yourself:} #1}}}

\title{Teaching Machines to See Patterns}
\subtitle{A Neural Networks Primer: Why We Needed Each Piece of the Puzzle}
\author{NLP Course 2025}
\date{}

\begin{document}

% ============================================================================
% MAIN PRESENTATION
% ============================================================================

% Title slide
% Title Slide
\begin{frame}
\titlepage
\vfill
\begin{center}
\secondary{\small From the 1950s mail sorting crisis to ChatGPT: How humanity taught machines to think}
\end{center}
\end{frame}

% Overview and roadmap
% Journey Roadmap
\begin{frame}{Your Journey Through Neural Networks}
\begin{center}
\textbf{Where We're Going Today}
\end{center}
\vspace{5mm}
\begin{columns}
\column{0.48\textwidth}
\textbf{Part 1: The Problem (1943-1969)}
\begin{itemize}
\item The mail sorting crisis
\item First mathematical neurons
\item The perceptron revolution
\item The XOR catastrophe
\end{itemize}

\vspace{3mm}
\textbf{Intermission: Understanding the Basics}
\begin{itemize}
\item How neurons calculate
\item Why we need layers
\item Following the forward pass
\end{itemize}

\column{0.48\textwidth}
\textbf{Part 2: The Breakthrough (1980s-1990s)}
\begin{itemize}
\item Hidden layers save the day
\item Backpropagation breakthrough
\item Universal approximation proof
\end{itemize}

\vspace{3mm}
\textbf{Part 3: The Revolution (2000s-Present)}
\begin{itemize}
\item Deep learning explosion
\item Modern architectures
\item Real-world impact
\end{itemize}

\vspace{3mm}
\textbf{Your Turn: Building Networks}
\begin{itemize}
\item Build your first network
\item Next steps
\end{itemize}
\end{columns}
\bottomnote{Each part builds on the previous - we'll go step by step!}
\end{frame}

% ============================================================================
% PART 0: FOUNDATIONAL CONCEPTS (NEW!)
% ============================================================================

% NEW: Simple neuron visualizations (3 slides)
% Simple Neurons Introduction - NEW SECTION WITH VISUALS
% Understanding What a Single Neuron Does

\begin{frame}{What is a Neuron? The Building Block}
\begin{center}
\textbf{Before we tell the story, let's understand the fundamental building block}
\end{center}
\vspace{5mm}

\includegraphics[width=0.95\textwidth]{../figures/single_neuron_explained.pdf}

\bottomnote{ A neuron is just a mathematical function that takes inputs and produces an output}
\end{frame}

\begin{frame}{A Neuron as a Decision Maker}
\begin{center}
\textbf{How Weights and Bias Control Decisions}
\end{center}
\vspace{5mm}

\includegraphics[width=0.95\textwidth]{../figures/neuron_as_classifier.pdf}

\bottomnote{ Weights control the angle, bias shifts the position - together they define the decision boundary}
\end{frame}

\begin{frame}{Watching a Neuron Learn}
\begin{center}
\textbf{Learning = Gradually Adjusting Weights to Fit the Data}
\end{center}
\vspace{5mm}

\includegraphics[width=0.95\textwidth]{../figures/neuron_learning_progress.pdf}

\tryit{Notice how the decision boundary moves closer to the correct position with each epoch!}
\bottomnote{ This simple process - repeated millions of times - is how all neural networks learn}
\end{frame}

% Foundational concept: Function approximation (3 slides)
% FOUNDATIONAL CONCEPT: Neural Networks as Function Approximators
% These three slides establish the core mathematical concept before the historical narrative

% Slide 1: What is Function Approximation?
\begin{frame}{The Core Idea: Neural Networks are Function Approximators}
\begin{center}
\textbf{What does this actually mean?}
\end{center}
\vspace{1mm}

\begin{columns}
\column{0.32\textwidth}
\textbf{The Problem:}
\begin{itemize}
\item We have inputs (x)
\item We want outputs (y)
\item But we don't know the formula!
\item Examples:
  \begin{itemize}
  \footnotesize
  \item Size $\rightarrow$ Price
  \item Image $\rightarrow$ Label
  \item Text $\rightarrow$ Sentiment
  \end{itemize}
\end{itemize}

\column{0.32\textwidth}
\textbf{Traditional Approach:}
\begin{itemize}
\item Guess the formula
\item Write explicit rules
\item Hope it works
\item \warning{Problem:} Real world is too complex!
\end{itemize}

\vspace{3mm}
\textit{Example:}\\
\footnotesize
Price = a $\times$ Size + b\\
(Too simple for real data!)

\column{0.32\textwidth}
\textbf{Neural Network Approach:}
\begin{itemize}
\item Learn from examples
\item Build the formula automatically
\item Adjust until it fits
\item \success{Works for ANY pattern!}
\end{itemize}

\vspace{3mm}
\textit{Magic:}\\
\footnotesize
NN learns: f(x) $\approx$ y\\
No formula needed!
\end{columns}

\vspace{2mm}
\begin{center}
\includegraphics[width=0.75\textwidth]{../figures/function_approx_basics.pdf}
\end{center}

\vspace{2mm}
\plainmath{A function approximator finds patterns in data without being told what the pattern is}
\end{frame}

% Slide 2: How Neural Networks Build Functions
\begin{frame}{How NNs Build Complex Functions from Simple Pieces}
\begin{center}
\textbf{The LEGO Principle: Combine Simple Parts to Build Anything}
\end{center}
\vspace{1mm}

\begin{columns}
\column{0.48\textwidth}
\textbf{The Building Blocks:}
\begin{enumerate}
\item \textbf{Individual Neurons:} Simple decisions
   \begin{itemize}
   \footnotesize
   \item "Is input > threshold?"
   \item Outputs: on/off (smooth version)
   \end{itemize}

\item \textbf{Combine Neurons:} Weight and add
   \begin{itemize}
   \footnotesize
   \item Create complex shapes
   \end{itemize}

\item \textbf{Stack Layers:} Build hierarchy
   \begin{itemize}
   \footnotesize
   \item Each layer adds abstraction
   \end{itemize}
\end{enumerate}

\column{0.48\textwidth}
\textbf{Real-World Analogy:}

\textit{Making a Cake from Ingredients:}
\begin{itemize}
\footnotesize
\item Flour + Sugar + Eggs
\item Mix right amounts
\item $\rightarrow$ Perfect cake!
\end{itemize}

\textbf{In Neural Networks:}
\begin{itemize}
\footnotesize
\item Edges + Curves + Colors
\item Combine with weights
\item $\rightarrow$ Recognize faces!
\end{itemize}
\end{columns}

\begin{center}
\includegraphics[width=0.55\textwidth]{../figures/nn_building_blocks.pdf}
\end{center}
\end{frame}

% Slide 3: Universal Approximation Theorem in Plain English
\begin{frame}{The Universal Approximation Theorem: Why This Always Works}
\begin{center}
\textbf{The Most Important Theorem in Deep Learning (Cybenko, 1989)}
\end{center}
\vspace{1mm}

\begin{columns}
\column{0.48\textwidth}
\textbf{The Theorem (Plain English):}

\colorbox{checkpointBlue!20}{\parbox{0.95\columnwidth}{
\centering
\textit{"A neural network with enough neurons can approximate \textbf{ANY} continuous function to \textbf{ANY} desired accuracy"}
}}

\vspace{2mm}
\textbf{What This Means:}
\begin{itemize}
\item \success{Universal:} Works for any smooth pattern
\item \success{Guaranteed:} Not hoping, but proving
\item \success{Practical:} Just add more neurons!
\end{itemize}

\vspace{1mm}
\textbf{The Catch:}
\begin{itemize}
\item \warning{How many neurons?} Could be millions
\item \warning{How to train?} That's the art
\end{itemize}

\column{0.48\textwidth}
\textbf{Intuitive Proof:}

\textit{Pixel art analogy:}
\begin{itemize}
\footnotesize
\item 4 pixels: Blocky
\item 100 pixels: Recognizable
\item 10,000 pixels: Photo-realistic
\end{itemize}

\textit{Same with neurons:}
\begin{itemize}
\footnotesize
\item Few: Rough
\item More: Better
\item Many: Nearly perfect
\end{itemize}

\vspace{1mm}
\end{columns}

\begin{center}
\includegraphics[width=0.55\textwidth]{../figures/universal_approximation.pdf}
\end{center}

\vspace{2mm}
\confusion{Common mistake: "Universal" doesn't mean "easy" or "fast" - it just means "possible"!}
\end{frame}

% ============================================================================
% PART 1: THE PROBLEM THAT STARTED EVERYTHING
% ============================================================================

% Historical context and early attempts
% NOTE: Removed "Mark I Perceptron" slide (moved to Appendix B)
\begin{frame}{Part 1: The Problem That Started Everything}
\begin{center}
{\Large \textbf{1950s: The Mail Sorting Crisis}}
\end{center}
\vspace{5mm}
\begin{columns}
\column{0.48\textwidth}
\textbf{The Challenge:}
\begin{itemize}
\item 150 million letters per day
\item Hand-written addresses
\item Human sorters: slow, expensive, error-prone
\item Traditional programming: useless
\end{itemize}

\column{0.48\textwidth}
\textbf{Why Traditional Code Failed:}
\begin{itemize}
\item Can't write rules for every handwriting style
\item Too many variations of each letter
\item Context matters: "I" vs "l" vs "1"
\item This wasn't computation---it was \highlight{pattern recognition}
\end{itemize}
\end{columns}
\bottomnote{ This problem would take 40 years to solve properly}
\end{frame}

% Historical Timeline
\begin{frame}{80 Years of Neural Networks: The Complete Journey}
\begin{center}
\includegraphics[width=0.95\textwidth]{../figures/historical_timeline.pdf}
\end{center}
\bottomnote{ From theoretical neurons to ChatGPT: Each breakthrough built on previous failures}
\end{frame}

% Why Rules Don't Work
\begin{frame}[fragile]{Why Can't We Just Write Rules?}
\begin{center}
\textbf{Problem: Recognize the Letter "A"}
\end{center}
\vspace{5mm}
\begin{columns}
\column{0.48\textwidth}
\textbf{Traditional Approach (Failed):}
\begin{lstlisting}[basicstyle=\tiny]
if (has_triangle_top AND
    has_horizontal_bar AND
    two_diagonal_lines) {
  return "A"
}
\end{lstlisting}
\secondary{\small But what about...}
\begin{itemize}
\item Handwritten A's?
\item Different fonts?
\item Rotated A's?
\item Partial A's?
\end{itemize}

\column{0.48\textwidth}
\begin{center}
\includegraphics[width=\textwidth]{../figures/various_a_styles.pdf}
\end{center}
\secondary{\small Just for the letter "A", we'd need thousands of rules!}
\end{columns}
\bottomnote{ The breakthrough: What if machines could learn patterns like children do?}
\end{frame}

% McCulloch-Pitts
\begin{frame}{1943: The First Spark - McCulloch \& Pitts}
\begin{center}
\textbf{The Birth of Computational Neuroscience}
\end{center}
\vspace{5mm}
\begin{columns}
\column{0.48\textwidth}
\textbf{The Revolutionary Paper:}
\begin{itemize}
\item "A Logical Calculus of Ideas Immanent in Nervous Activity"
\item First mathematical model of neurons
\item Proved: Networks can compute ANY logical function
\item Inspired von Neumann's computer architecture
\end{itemize}

\textbf{Key Insight:}
\begin{itemize}
\item Neurons = Logic gates
\item Brain = Computing machine
\item Thinking = Computation
\end{itemize}

\column{0.48\textwidth}
\textbf{The Model:}
\begin{itemize}
\item Binary neurons (0 or 1)
\item Threshold activation
\item Fixed connections
\item No learning yet!
\end{itemize}

\textbf{Historical Impact:}
\begin{itemize}
\item Founded field of neural networks
\item Influenced cybernetics movement
\item Set stage for AI research
\item "The brain is a computer" metaphor
\end{itemize}
\end{columns}
\bottomnote{ 14 years later, Rosenblatt would add the missing piece: learning}
\end{frame}

% Perceptron
\begin{frame}{1957: The First Learning Machine - The Perceptron}
\begin{center}
\textbf{Frank Rosenblatt's Radical Idea: Neurons That Learn}
\end{center}
\vspace{5mm}
\begin{columns}
\column{0.48\textwidth}
\textbf{Beyond McCulloch-Pitts:}
\begin{itemize}
\item Adjustable weights (not fixed!)
\item Learning from mistakes
\item Physical machine built (Mark I)
\item Could recognize simple patterns
\end{itemize}

\textbf{The Hardware:}
\begin{itemize}
\item 400 photocells (20$\times$20 ``retina'')
\item 512 motor-driven potentiometers
\item Weights adjusted by electric motors
\item Took 5 minutes to learn patterns
\end{itemize}

\column{0.48\textwidth}
\textbf{Mathematical Model:}
\begin{itemize}
\item Inputs: $x_1, x_2, ..., x_n$
\item Weights: $w_1, w_2, ..., w_n$
\item Sum: $z = \sum_{i=1}^{n} w_i x_i + b$
\item Output: $y = \begin{cases} 1 & \text{if } z > 0 \\ 0 & \text{if } z \leq 0 \end{cases}$
\end{itemize}

\plainmath{Each input gets a vote (weight). We add up all votes plus a bias. If total is positive, output 1; otherwise 0.}

\textbf{Learning Rule:}
If wrong: $w_i = w_i + \eta \cdot error \cdot x_i$
\end{columns}
\bottomnote{ The New York Times, 1958: "The Navy revealed the embryo of an electronic computer that will be able to walk, talk, see, write, reproduce itself and be conscious of its existence."}
\end{frame}

% ============================================================================
% INTERMISSION: FROM STORY TO SCIENCE
% ============================================================================

% Intermission: From Story to Science
\begin{frame}{Intermission: From Story to Science}
\begin{center}
{\Large \textbf{Let's Understand How This Actually Works}}
\end{center}
\vspace{10mm}

\begin{columns}
\column{0.48\textwidth}
\textbf{We've Seen the History...}
\begin{itemize}
\item McCulloch-Pitts invented the neuron
\item Rosenblatt made it learn
\item The perceptron was born
\end{itemize}

\column{0.48\textwidth}
\textbf{Now Let's See the Science:}
\begin{itemize}
\item How does a neuron calculate?
\item What does learning mean?
\item Why was XOR so hard?
\end{itemize}
\end{columns}

\vspace{10mm}
\begin{center}
\colorbox{checkpointBlue!20}{\parbox{0.7\textwidth}{
\centering
\textbf{Next 5 slides: Hands-on calculations and exercises}\\
\small Get your pencil ready - we're going to work through real examples!
}}
\end{center}

\bottomnote{ Don't worry - we'll return to the story once you understand the basics}
\end{frame}

% NEW: Understanding Checkpoint
\begin{frame}{Understanding Check: Can You Answer These?}
\begin{center}
\textbf{\checkpoint{Let's Make Sure We're Together}}
\end{center}
\vspace{5mm}
\begin{columns}
\column{0.48\textwidth}
\textbf{Quick Questions:}
\begin{enumerate}
\item Why couldn't traditional programming solve mail sorting?
\item What does a weight represent in simple terms?
\item Why do we need the bias term?
\item What was revolutionary about Rosenblatt's perceptron?
\end{enumerate}

\column{0.48\textwidth}
\textbf{Think About It:}
\begin{itemize}
\item A weight is like the importance/trust we give to each input
\item Bias shifts our decision threshold
\item Learning = adjusting these weights
\item The perceptron was the first machine that could learn!
\end{itemize}
\end{columns}
\vspace{5mm}
\tryit{Draw a simple perceptron with 2 inputs. Label the weights, bias, and output. What would the weights be to compute AND logic?}
\bottomnote{ If any of these are unclear, revisit the previous slides before continuing}
\end{frame}

% The Math Behind It (Simple)
\begin{frame}{Making It Concrete: Teaching OR Logic}
\begin{center}
\textbf{Problem: Learn OR function (output 1 if ANY input is 1)}
\end{center}
\vspace{5mm}
\begin{columns}
\column{0.5\textwidth}
\textbf{Training Data:}
\begin{center}
\begin{tabular}{cc|c}
$x_1$ & $x_2$ & Output \\
\hline
0 & 0 & 0 \\
0 & 1 & 1 \\
1 & 0 & 1 \\
1 & 1 & 1 \\
\end{tabular}
\end{center}

\textbf{The Perceptron:}
\begin{align*}
z &= w_1 \cdot x_1 + w_2 \cdot x_2 + b \\
\text{output} &= \begin{cases} 1 & \text{if } z > 0 \\ 0 & \text{if } z \leq 0 \end{cases}
\end{align*}
\plainmath{Multiply first input by first weight, second input by second weight, add bias, check if positive}

\column{0.5\textwidth}
\textbf{Learning Process:}
\begin{enumerate}
\item Start with random weights
\item For each example:
  \begin{itemize}
  \item Calculate output
  \item If wrong: adjust weights
  \item If correct: keep weights
  \end{itemize}
\item Repeat until all correct
\end{enumerate}

\textbf{Final Solution:}
$w_1 = 1$, $w_2 = 1$, $b = -0.5$
\end{columns}
\bottomnote{ Success! But this was just the beginning...}
\end{frame}

% NEW: Let's Calculate Together - Spam Detection
\begin{frame}{Let's Calculate Together: Is This Email Spam?}
\begin{center}
\textbf{\checkpoint{A Real Perceptron Calculation You Can Follow}}
\end{center}
\vspace{5mm}
\begin{columns}
\column{0.48\textwidth}
\textbf{The Email:}
\fbox{\parbox{0.9\textwidth}{\small
"FREE money! Click here NOW for amazing offer!!!"
}}

\textbf{Our Features (Inputs):}
\begin{itemize}
\item $x_1$ = Has "FREE"? = 1
\item $x_2$ = Has "money"? = 1
\item $x_3$ = Many "!"? = 1
\item $x_4$ = From friend? = 0
\end{itemize}

\textbf{Learned Weights:}
\begin{itemize}
\item $w_1$ = +3 (FREE is very spammy)
\item $w_2$ = +2 (money is suspicious)
\item $w_3$ = +2 (!!! is aggressive)
\item $w_4$ = -5 (friends are trusted)
\item $b$ = -2 (threshold)
\end{itemize}

\column{0.48\textwidth}
\textbf{Let's Calculate:}
\begin{align*}
z &= w_1 \cdot x_1 + w_2 \cdot x_2 + w_3 \cdot x_3 + w_4 \cdot x_4 + b \\
  &= 3 \cdot 1 + 2 \cdot 1 + 2 \cdot 1 + (-5) \cdot 0 + (-2) \\
  &= 3 + 2 + 2 + 0 - 2 \\
  &= 5
\end{align*}

\textbf{Decision:}
\begin{itemize}
\item $z = 5 > 0$
\item Output = 1 = SPAM!
\end{itemize}

\tryit{What if this email WAS from a friend ($x_4 = 1$)? Recalculate! Would it still be spam?}

\textbf{Answer:} $z = 5 - 5 = 0$, borderline!
\end{columns}
\bottomnote{ This is exactly how early spam filters worked - and why they failed on clever spam}
\end{frame}

% Notation Explained
\begin{frame}{Understanding the Notation}
\begin{center}
\textbf{Breaking Down the Math Symbols}
\end{center}
\vspace{5mm}
\begin{columns}
\column{0.48\textwidth}
\textbf{Inputs and Weights:}
\begin{itemize}
\item $x_i$ = input value (what we see)
\item $w_i$ = weight (importance/strength)
\item $b$ = bias (threshold adjuster)
\end{itemize}

\textbf{The Computation:}
$$z = \sum_{i=1}^{n} w_i x_i + b$$

This means:
\begin{itemize}
\item Multiply each input by its weight
\item Add them all up
\item Add the bias
\end{itemize}

\column{0.48\textwidth}
\textbf{Real Example:}
\begin{center}
Should I go outside? \\[3mm]
\begin{tabular}{lcc}
Factor & Value & Weight \\
\hline
Sunny? & 1 & +2 \\
Raining? & 0 & -3 \\
Weekend? & 1 & +1 \\
\hline
\end{tabular}
\end{center}
$$z = (1 \times 2) + (0 \times -3) + (1 \times 1) = 3$$
$$\text{Decision: } z > 0 \text{, so YES!}$$
\end{columns}
\bottomnote{ This simple math would evolve into deep learning}
\end{frame}

% ============================================================================
% NEW: ADVANCED VISUALIZATIONS (Key visualizations integrated here)
% ============================================================================

% NEW: 3D neuron visualization
\begin{frame}{The Neuron as a 3D Function}
\begin{center}
\textbf{Visualizing How Activation Functions Transform the Output Space}
\end{center}
\vspace{5mm}

\includegraphics[width=0.95\textwidth]{../figures/neuron_3d_visualization.pdf}

\bottomnote{ Left: Linear (just a plane), Right: With activation (curved surface) - this non-linearity is what makes learning possible!}
\end{frame}

% NEW: Network complexity visualization
\begin{frame}{From Simple to Complex: Network Depth Creates Complexity}
\begin{center}
\textbf{How More Neurons Enable More Complex Decision Boundaries}
\end{center}
\vspace{5mm}

\includegraphics[width=0.95\textwidth]{../figures/network_complexity_visualization.pdf}

\bottomnote{ Each neuron adds a new "dimension" to what the network can learn}
\end{frame}

% ============================================================================
% PART 2: THE STRUGGLES AND SOLUTIONS
% ============================================================================

% XOR Crisis through Backpropagation
% NOTE: Removed "NetTalk" slide (moved to Appendix B)
\begin{frame}{Part 2: The Struggles (1969-1989)}
\begin{center}
{\Large \textbf{1969: The Crisis - XOR Problem}}
\end{center}
\vspace{5mm}
\begin{columns}
\column{0.48\textwidth}
\textbf{XOR (Exclusive OR):}
\begin{center}
\begin{tabular}{cc|c}
$x_1$ & $x_2$ & Output \\
\hline
0 & 0 & 0 \\
0 & 1 & 1 \\
1 & 0 & 1 \\
1 & 1 & 0 \\
\end{tabular}
\end{center}

\textbf{The Problem:}
\begin{itemize}
\item Can't draw a single line to separate
\item Perceptron only learns linear boundaries
\item Real-world problems are non-linear!
\end{itemize}

\column{0.48\textwidth}
\begin{center}
\includegraphics[width=\textwidth]{../figures/xor_visualization.pdf}
\end{center}
\textbf{Impact:}
\begin{itemize}
\item Funding dried up
\item "AI Winter" begins
\item Neural networks abandoned
\end{itemize}
\end{columns}
\bottomnote{ The field would be dormant for over a decade...}
\end{frame}

% XOR Solution with NEW visualization
\begin{frame}{XOR Problem Solved Visually}
\begin{center}
\textbf{Why We Need Hidden Layers: The XOR Solution}
\end{center}
\vspace{5mm}

\includegraphics[width=0.95\textwidth]{../figures/xor_solution_visualization.pdf}

\bottomnote{ Two hidden neurons working together can solve what one neuron cannot}
\end{frame}

% Continue with rest of Part 2 (hidden layers, backpropagation, etc.)
% Using content from act2_struggles.tex but without NetTalk

% Include remaining key slides from Part 2
\begin{frame}{When One Line Isn't Enough: Real Problems Need More}
\begin{center}
\textbf{\checkpoint{Let's See Why We Need Hidden Layers}}
\end{center}
\vspace{5mm}
\begin{columns}
\column{0.48\textwidth}
\textbf{Problem 1: Spam Detection (Easy)}
\begin{itemize}
\item Has many spam words? $\rightarrow$ SPAM
\item Has few spam words? $\rightarrow$ NOT SPAM
\item \success{One line (threshold) works!}
\end{itemize}

\vspace{5mm}
\textbf{Problem 2: Cat or Dog Photo (Hard)}
\begin{itemize}
\item Small + fluffy? Could be either!
\item Large + smooth? Could be either!
\item Pointy ears + whiskers? $\rightarrow$ Cat
\item Floppy ears + wet nose? $\rightarrow$ Dog
\item \warning{Need multiple feature detectors!}
\end{itemize}

\column{0.48\textwidth}
\includegraphics[width=\textwidth]{../figures/linear_to_nonlinear_bridge.pdf}

\textbf{The Solution:}
\begin{enumerate}
\item First layer: Multiple detectors
  \begin{itemize}
  \item Detector 1: "Has cat features?"
  \item Detector 2: "Has dog features?"
  \end{itemize}
\item Second layer: Combine detections
  \begin{itemize}
  \item If cat features > dog features $\rightarrow$ Cat
  \end{itemize}
\end{enumerate}
\end{columns}
\bottomnote{ This is why deep learning works: each layer builds more complex detectors from simpler ones}
\end{frame}

% Hidden layers solution
\begin{frame}{Part 2: The Journey Back}
\begin{center}
{\Large \textbf{1980s: The Hidden Layer Revolution}}
\end{center}
\vspace{5mm}
\begin{columns}
\column{0.48\textwidth}
\textbf{The Insight:}
\begin{itemize}
\item Stack multiple layers!
\item First layer: detect simple features
\item Hidden layer: combine features
\item Output layer: final decision
\end{itemize}

\textbf{Solving XOR:}
\begin{itemize}
\item Hidden neuron 1: Is it (0,1)?
\item Hidden neuron 2: Is it (1,0)?
\item Output: OR of hidden neurons
\end{itemize}

\column{0.48\textwidth}
\begin{center}
\includegraphics[width=\textwidth]{../figures/multilayer_network.pdf}
\end{center}
\textbf{New Architecture:}
\begin{itemize}
\item Input layer: raw data
\item Hidden layer(s): feature extraction
\item Output layer: final classification
\end{itemize}
\end{columns}
\bottomnote{ But how do we train multiple layers?}
\end{frame}

% NEW: Forward pass visualization
\begin{frame}{Forward Pass: Signal Propagation Step-by-Step}
\begin{center}
\textbf{Following Data as it Flows Through the Network}
\end{center}
\vspace{5mm}

\includegraphics[width=0.95\textwidth]{../figures/forward_pass_frames.pdf}

\bottomnote{ Each frame shows one step: computing weighted sums, applying activations, passing to next layer}
\end{frame}

% Backpropagation
\begin{frame}{1986: Backpropagation - Teaching Networks to Learn}
\begin{center}
\textbf{Rumelhart, Hinton, Williams: The Learning Algorithm}
\end{center}
\vspace{5mm}
\begin{columns}
\column{0.48\textwidth}
\textbf{The Problem:}
\begin{itemize}
\item Perceptron learning only works for 1 layer
\item How to adjust hidden layer weights?
\item No direct error signal for hidden neurons
\end{itemize}

\textbf{The Solution:}
\begin{itemize}
\item Propagate error backwards!
\item Each layer gets blame for output error
\item Use calculus (chain rule) to distribute blame
\end{itemize}

\column{0.48\textwidth}
\textbf{The Algorithm:}
\begin{enumerate}
\item Forward: Calculate output
\item Compare: Find error
\item Backward: Distribute blame
\item Update: Adjust all weights
\end{enumerate}

\plainmath{Like a teacher marking an essay: finds the final error, then traces back to see which paragraphs, sentences, and words caused it}

\textbf{Impact:}
\begin{itemize}
\item Finally could train deep networks!
\item Neural networks reborn
\item Foundation of all modern AI
\end{itemize}
\end{columns}
\bottomnote{ This algorithm runs billions of times to train ChatGPT}
\end{frame}

% Universal Approximation
\begin{frame}{1989: Universal Approximation - The Mathematical Foundation}
\begin{center}
\textbf{Cybenko, Hornik: The Ultimate Proof}
\end{center}
\vspace{5mm}
\begin{columns}
\column{0.48\textwidth}
\textbf{The Theorem:}
\colorbox{checkpointBlue!20}{\parbox{0.95\columnwidth}{
\small
A feedforward network with:
\begin{itemize}
\item One hidden layer
\item Enough neurons
\item Non-linear activation
\end{itemize}
Can approximate ANY continuous function to arbitrary accuracy!
}}

\textbf{What This Means:}
\begin{itemize}
\item Neural networks are universal computers
\item No function is too complex
\item Just need enough neurons and data
\end{itemize}

\column{0.48\textwidth}
\textbf{Real-World Analogy:}

\textit{LEGO blocks can build anything:}
\begin{itemize}
\item Few blocks = rough shape
\item Many blocks = detailed model
\item Infinite blocks = perfect replica
\end{itemize}

\textit{Same with neurons:}
\begin{itemize}
\item Few neurons = rough approximation
\item Many neurons = good function
\item Infinite neurons = exact function
\end{itemize}

\tryit{Think of any pattern or function. This theorem guarantees a neural network can learn it!}
\end{columns}
\bottomnote{ This gave theoretical backing to the neural network revolution}
\end{frame}

% Activation functions
% Activation Functions and Visualization
\begin{frame}{Why Linear Doesn't Work: Activation Functions}
\begin{center}
\textbf{The Need for Non-Linearity}
\end{center}
\vspace{5mm}
\begin{columns}
\column{0.48\textwidth}
\textbf{Problem with Linear:}
\begin{itemize}
\item Stack of linear layers = still linear!
\item $f(g(x)) = (wx + b_1)w' + b_2 = w'wx + ...$
\item Can't learn complex patterns
\end{itemize}

\textbf{Solution: Activation Functions}
\begin{itemize}
\item Add non-linearity after each layer
\item Allows learning complex boundaries
\item Different functions for different needs
\end{itemize}

\column{0.48\textwidth}
\textbf{Common Activation Functions:}
\begin{itemize}
\item \textbf{Sigmoid:} $\sigma(x) = \frac{1}{1 + e^{-x}}$
  \begin{itemize}
  \item Smooth, outputs 0-1
  \item Good for probabilities
  \end{itemize}
  \plainmath{Squashes any input to range 0-1. Large positive becomes 1, large negative becomes 0}
\item \textbf{ReLU:} $f(x) = \max(0, x)$
  \begin{itemize}
  \item Simple, fast
  \item Solves vanishing gradient
  \end{itemize}
\item \textbf{Tanh:} $\tanh(x) = \frac{e^x - e^{-x}}{e^x + e^{-x}}$
  \begin{itemize}
  \item Outputs -1 to 1
  \item Zero-centered
  \end{itemize}
\end{itemize}
\end{columns}
\bottomnote{ ReLU's simplicity revolutionized deep learning in 2011}
\end{frame}

% Simple 2D Example
\begin{frame}{Visualizing Learning: 2D Classification}
\begin{center}
\textbf{Teaching a Network to Separate Red from Blue Points}
\end{center}
\vspace{5mm}
\begin{columns}
\column{0.48\textwidth}
\textbf{The Setup:}
\begin{itemize}
\item Input: (x, y) coordinates
\item Output: Red or Blue class
\item Network: 2 $\rightarrow$ 4 $\rightarrow$ 2 neurons
\end{itemize}

\textbf{Training Process:}
\begin{enumerate}
\item Epoch 1: Random boundary
\item Epoch 10: Rough separation
\item Epoch 50: Good boundary
\item Epoch 100: Perfect fit
\end{enumerate}

\column{0.48\textwidth}
\begin{center}
\includegraphics[width=\textwidth]{../figures/2d_classification_evolution.pdf}
\end{center}
\textbf{What Each Layer Learns:}
\begin{itemize}
\item Layer 1: Simple boundaries
\item Hidden: Combine boundaries
\item Output: Final decision
\end{itemize}
\end{columns}
\bottomnote{ This same principle scales to millions of parameters}
\end{frame}

% NEW: Gradient landscape visualization
\begin{frame}{The Optimization Landscape}
\begin{center}
\textbf{Gradient Descent: Finding the Valley in 3D Space}
\end{center}
\vspace{5mm}

\includegraphics[width=0.95\textwidth]{../figures/gradient_landscape_3d.pdf}

\bottomnote{ Training a neural network = rolling a ball down this landscape to find the lowest point}
\end{frame}

% ============================================================================
% PART 3: THE BREAKTHROUGH YEARS
% ============================================================================

% Main slides from act3_breakthrough.tex (without excessive detail)
\begin{frame}{Part 3: The Breakthrough Years}
\begin{center}
{\Large \textbf{1998-2012: From Digits to ImageNet}}
\end{center}
\vspace{5mm}
\begin{columns}
\column{0.48\textwidth}
\textbf{1998 - LeNet: First Success}
\begin{itemize}
\item Yann LeCun's CNN for digits
\item 32$\times$32 pixels $\rightarrow$ 10 classes
\item 60,000 parameters
\item Banks adopt for check reading
\end{itemize}

\textbf{Key Innovation: Convolutions}
\begin{itemize}
\item Share weights across image
\item Detect features anywhere
\item Build complexity layer by layer
\end{itemize}

\column{0.48\textwidth}
\textbf{2012 - AlexNet: The Revolution}
\begin{itemize}
\item 1000 ImageNet classes
\item 60 million parameters
\item GPUs enable training
\item Error rate: 26\% $\rightarrow$ 16\%
\end{itemize}

\textbf{What Changed:}
\begin{itemize}
\item Big Data (millions of images)
\item GPU computing (100x faster)
\item ReLU activation
\item Dropout regularization
\end{itemize}
\end{columns}
\bottomnote{ This victory ended the second AI winter permanently}
\end{frame}

% Continue with key slides from Part 3...
% Include CNN explanation, gradient descent, learning types, overfitting
% Skip some of the very detailed slides to keep it focused

% NEW: Learning process visualization
\begin{frame}{The Learning Process: Frame by Frame}
\begin{center}
\textbf{Watching Decision Boundaries Evolve During Training}
\end{center}
\vspace{5mm}

\includegraphics[width=0.95\textwidth]{../figures/decision_boundary_evolution.pdf}

\tryit{Notice how the boundary starts random and gradually fits the data pattern!}
\bottomnote{ This is what "learning" looks like - not magic, just systematic improvement}
\end{frame}

% ============================================================================
% PART 4: THE REVOLUTION
% ============================================================================

% Key slides from act4_revolution.tex
% NOTE: Removed these advanced slides (moved to Appendix A):
% - Lottery Ticket Hypothesis
% - Inductive Biases
% - Scaling Laws
% - Deep vs Wide
% - Emergent Abilities
% - Optimization Algorithms (detailed)

\begin{frame}{Part 3 (continued): The Deep Learning Revolution}
\begin{center}
{\Large \textbf{2014-Present: Networks That Changed the World}}
\end{center}
\vspace{5mm}
\begin{columns}
\column{0.48\textwidth}
\textbf{The Depth Revolution:}
\begin{itemize}
\item 2014 - VGGNet: 19 layers
\item 2015 - ResNet: 152 layers
\item 2017 - Transformers: Attention
\item 2020 - GPT-3: 175B parameters
\end{itemize}

\textbf{Why Depth Matters:}
\begin{itemize}
\item Each layer = abstraction level
\item Deep = complex reasoning
\item Hierarchical feature learning
\end{itemize}

\column{0.48\textwidth}
\textbf{Real-World Impact:}
\begin{itemize}
\item \textbf{Vision:} Self-driving cars
\item \textbf{Language:} Google Translate
\item \textbf{Speech:} Siri, Alexa
\item \textbf{Medicine:} Disease diagnosis
\item \textbf{Science:} Protein folding
\end{itemize}

\textbf{The Scale:}
\begin{itemize}
\item Billions of parameters
\item Trained on internet-scale data
\item Months of GPU time
\item Emergent abilities appear
\end{itemize}
\end{columns}
\bottomnote{ We went from recognizing digits to passing the bar exam in 25 years}
\end{frame}

% ResNet
\begin{frame}{2015: ResNet - The Skip Connection Revolution}
\begin{center}
\textbf{Problem: Networks Couldn't Get Deeper}
\end{center}
\vspace{5mm}
\begin{columns}
\column{0.48\textwidth}
\textbf{The Vanishing Gradient:}
\begin{itemize}
\item Gradients multiply through layers
\item Become exponentially small
\item Deep layers stop learning
\item 20 layers was the limit
\end{itemize}

\textbf{The Breakthrough: Skip Connections}
\begin{itemize}
\item Add input directly to output
\item $F(x) + x$ instead of just $F(x)$
\item Gradients flow directly backward
\item Can train 1000+ layers!
\end{itemize}

\column{0.48\textwidth}
\begin{center}
\includegraphics[width=\textwidth]{../figures/resnet_skip_connection.pdf}
\end{center}
\textbf{Why It Works:}
\begin{itemize}
\item Learn residual (difference) only
\item Identity mapping is easy default
\item Gradients have direct path
\item Each layer refines previous result
\end{itemize}
\end{columns}
\bottomnote{ This simple trick enabled the deep learning revolution}
\end{frame}

% NEW: Architecture comparison visualization
\begin{frame}{Architecture Evolution Over Time}
\begin{center}
\textbf{From 20 Parameters to 1.8 Trillion: The Growth of Neural Networks}
\end{center}
\vspace{5mm}

\includegraphics[width=0.95\textwidth]{../figures/architecture_size_comparison.pdf}

\bottomnote{ Each 10x increase in size unlocked new capabilities}
\end{frame}

% NEW: Architecture types
\begin{frame}{Architecture Types Comparison}
\begin{center}
\textbf{Different Architectures for Different Problems}
\end{center}
\vspace{5mm}

\includegraphics[width=0.95\textwidth]{../figures/architecture_type_comparison.pdf}

\bottomnote{ Each architecture encodes different assumptions about the data structure}
\end{frame}

% ============================================================================
% MODERN TECHNIQUES AND PRACTICAL ADVICE
% ============================================================================

% From modern_architectures.tex (keeping practical content, removing advanced theory)

% Architectures overview
\begin{frame}{Neural Network Architectures: Right Tool for Right Job}
\begin{columns}
\column{0.48\textwidth}
\textbf{Feedforward Networks:}
\begin{itemize}
\item Information flows forward only
\item Fixed-size input and output
\item Good for: Classification, regression
\end{itemize}

\textbf{Convolutional (CNN):}
\begin{itemize}
\item Spatial feature detection
\item Translation invariance
\item Good for: Images, video
\end{itemize}

\column{0.48\textwidth}
\textbf{Recurrent (RNN):}
\begin{itemize}
\item Process sequences
\item Maintain memory/state
\item Good for: Text, time-series
\end{itemize}

\textbf{Transformer:}
\begin{itemize}
\item Attention mechanism
\item Parallel processing
\item Good for: Language, everything else
\end{itemize}
\end{columns}
\bottomnote{ Each architecture encodes different assumptions about the data}
\end{frame}

% Modern training
\begin{frame}{Modern Training: Standing on Shoulders of Giants}
\begin{columns}
\column{0.48\textwidth}
\textbf{Transfer Learning:}
\begin{itemize}
\item Start with pre-trained network
\item Fine-tune on your task
\item 100x less data needed
\item Days $\rightarrow$ Hours training
\end{itemize}

\textbf{Data Augmentation:}
\begin{itemize}
\item Create variations of training data
\item Rotations, crops, color shifts
\item Prevents overfitting
\item Free performance boost
\end{itemize}

\column{0.48\textwidth}
\textbf{Simple Optimizers to Start:}
\begin{itemize}
\item \textbf{SGD:} Basic gradient descent
\item \textbf{Adam:} Adaptive learning rates (use this!)
\end{itemize}

\textbf{Mixed Precision:}
\begin{itemize}
\item Use 16-bit floats where possible
\item Keep 32-bit for critical ops
\item 2-3x speedup
\item Same accuracy
\end{itemize}
\end{columns}
\bottomnote{ These techniques make deep learning practical for everyone}
\end{frame}

% Common misconceptions
\begin{frame}{Common Mental Models That Are WRONG}
\begin{center}
\textbf{\warning{Misconceptions That Will Confuse You}}
\end{center}
\vspace{5mm}
\begin{columns}
\column{0.48\textwidth}
\textbf{WRONG: "Neurons are like brain neurons"}
\begin{itemize}
\item \warning{Brain neurons:} Complex, chemical, adaptive
\item \success{Artificial neurons:} Simple math functions
\item Just multiply and add!
\item No biology involved
\end{itemize}

\vspace{3mm}
\textbf{WRONG: "Networks understand concepts"}
\begin{itemize}
\item \warning{What you think:} "It knows what a cat is"
\item \success{Reality:} It found statistical patterns
\item No understanding, just correlation
\item Can be fooled by tiny changes
\end{itemize}

\column{0.48\textwidth}
\textbf{WRONG: "More layers = always better"}
\begin{itemize}
\item \warning{Too deep:} Vanishing gradients
\item \warning{Too deep:} Overfitting
\item \success{Right depth:} Depends on problem complexity
\item Simple problems need shallow networks
\end{itemize}

\vspace{3mm}
\textbf{WRONG: "It learns like humans"}
\begin{itemize}
\item \warning{Humans:} Learn from few examples
\item \warning{Humans:} Transfer knowledge easily
\item \success{Networks:} Need thousands of examples
\item \success{Networks:} Struggle with new situations
\end{itemize}
\end{columns}

\vspace{5mm}
\begin{center}
\colorbox{mlorange!20}{\parbox{0.8\textwidth}{
\centering
\textbf{Remember:} Neural networks are just fancy pattern matchers.\\
They don't think, understand, or reason - they find correlations in data.
}}
\end{center}
\bottomnote{ Understanding these limits helps you use neural networks effectively}
\end{frame}

% Why now
\begin{frame}{Why Deep Learning Exploded Now: The Perfect Storm}
\begin{columns}
\column{0.48\textwidth}
\textbf{1. Data Explosion:}
\begin{itemize}
\item Internet = infinite training data
\item ImageNet: 14M labeled images
\item Common Crawl: 300TB of text
\item YouTube: 500 hours/minute
\end{itemize}

\textbf{2. Hardware Revolution:}
\begin{itemize}
\item GPUs: 100x faster than CPUs
\item TPUs: Built for neural nets
\item Cloud computing: Rent supercomputers
\item Mobile chips with NPUs
\end{itemize}

\column{0.48\textwidth}
\textbf{3. Algorithm Breakthroughs:}
\begin{itemize}
\item ReLU activation (2011)
\item Batch normalization (2015)
\item Skip connections (2015)
\item Attention mechanism (2017)
\end{itemize}

\textbf{4. Open Source Culture:}
\begin{itemize}
\item TensorFlow, PyTorch free
\item Pre-trained models shared
\item Papers with code
\item Collaborative research
\end{itemize}
\end{columns}
\bottomnote{ The same ideas from 1980s finally had the resources to work}
\end{frame}

% Practical implementation
\begin{frame}[fragile]{From Theory to Practice: Your First Network}
\begin{center}
\textbf{Building a Digit Classifier in 10 Lines}
\end{center}
\vspace{5mm}
\begin{columns}
\column{0.58\textwidth}
\textbf{PyTorch Implementation:}
\begin{lstlisting}[language=Python,basicstyle=\tiny]
import torch
import torch.nn as nn

class SimpleNet(nn.Module):
    def __init__(self):
        super().__init__()
        self.fc1 = nn.Linear(784, 128)
        self.fc2 = nn.Linear(128, 10)

    def forward(self, x):
        x = torch.relu(self.fc1(x))
        return self.fc2(x)

# Train
model = SimpleNet()
optimizer = torch.optim.Adam(model.parameters())
criterion = nn.CrossEntropyLoss()
\end{lstlisting}

\column{0.38\textwidth}
\textbf{What This Does:}
\begin{itemize}
\item Input: 28$\times$28 pixel image
\item Hidden: 128 neurons
\item Output: 10 digit classes
\item Activation: ReLU
\item Training: Adam optimizer
\end{itemize}

\textbf{Training Loop:}
\begin{itemize}
\item Forward pass
\item Calculate loss
\item Backward pass
\item Update weights
\item Repeat
\end{itemize}
\end{columns}
\bottomnote{ This simple network achieves 97\% accuracy on MNIST}
\end{frame}

% Debugging
\begin{frame}{Your Debugging Checklist: When Things Go Wrong}
\begin{center}
\textbf{\checkpoint{Systematic Debugging Saves Hours}}
\end{center}
\vspace{5mm}
\tryit{Save this checklist - you'll need it for every project!}
\vspace{3mm}
\begin{columns}
\column{0.48\textwidth}
\textbf{Step 1: Sanity Checks}
\begin{itemize}
\item [$\square$] Can you overfit a single batch?
\item [$\square$] Are inputs normalized?
\item [$\square$] Is output layer correct?
\item [$\square$] Loss function matches task?
\end{itemize}

\textbf{Step 2: Data Checks}
\begin{itemize}
\item [$\square$] Plot sample inputs
\item [$\square$] Check label distribution
\item [$\square$] Verify train/val split
\item [$\square$] Look for data leakage
\end{itemize}

\column{0.48\textwidth}
\textbf{Step 3: Training Checks}
\begin{itemize}
\item [$\square$] Plot loss curves
\item [$\square$] Check gradient norms
\item [$\square$] Monitor weight updates
\item [$\square$] Try different learning rates
\end{itemize}

\textbf{Step 4: Architecture}
\begin{itemize}
\item [$\square$] Start with known working model
\item [$\square$] Add complexity gradually
\item [$\square$] Check activation distributions
\item [$\square$] Verify dimensions match
\end{itemize}

\confusion{90\% of bugs are in data preprocessing, not the model!}
\end{columns}
\bottomnote{ Print this slide and keep it handy}
\end{frame}

% ============================================================================
% EPILOGUE: YOUR TURN
% ============================================================================

% Epilogue: Key Takeaways and Practical Implementation
% Key Takeaways
\begin{frame}{Key Takeaways: From Mail Sorting to ChatGPT}
\begin{center}
{\Large \textbf{The Journey So Far}}
\end{center}
\vspace{5mm}
\begin{columns}
\column{0.48\textwidth}
\textbf{Core Concepts:}
\begin{enumerate}
\item \textbf{Neurons:} $y = f(\sum w_i x_i + b)$
\item \textbf{Learning:} Adjust weights to minimize error
\item \textbf{Depth:} Each layer adds abstraction
\item \textbf{Backpropagation:} Distribute error backwards
\item \textbf{Non-linearity:} Enables complex functions
\end{enumerate}

\column{0.48\textwidth}
\textbf{Historical Lessons:}
\begin{enumerate}
\item Every limitation spawned innovation
\item Simple ideas + scale = revolution
\item Biology inspires but doesn't limit
\item Persistence through AI winters
\item Theory + engineering = breakthroughs
\end{enumerate}
\end{columns}
\bottomnote{ You now understand the fundamentals that power all modern AI}
\end{frame}

% NEW: Your First Neural Network
\begin{frame}[fragile]{Epilogue: Your First Neural Network in 5 Minutes}
\begin{center}
\textbf{\checkpoint{Let's Build Something Real!}}
\end{center}
\vspace{1mm}
\begin{columns}
\column{0.55\textwidth}
\textbf{Complete MNIST Classifier:}
\begin{lstlisting}[language=Python,basicstyle=\tiny]
import torch
import torch.nn as nn
from torchvision import datasets, transforms
from torch.utils.data import DataLoader

# 1. Define Network
class Net(nn.Module):
    def __init__(self):
        super(Net, self).__init__()
        self.fc1 = nn.Linear(784, 128)
        self.fc2 = nn.Linear(128, 64)
        self.fc3 = nn.Linear(64, 10)

    def forward(self, x):
        x = x.view(-1, 784)  # Flatten
        x = torch.relu(self.fc1(x))
        x = torch.relu(self.fc2(x))
        return self.fc3(x)

# 2. Load Data
transform = transforms.ToTensor()
train_data = datasets.MNIST('.', train=True,
                           download=True,
                           transform=transform)
train_loader = DataLoader(train_data,
                         batch_size=64,
                         shuffle=True)

# 3. Setup Training
model = Net()
optimizer = torch.optim.Adam(model.parameters())
criterion = nn.CrossEntropyLoss()
\end{lstlisting}

\column{0.42\textwidth}
\begin{lstlisting}[language=Python,basicstyle=\tiny]
# 4. Training Loop
for epoch in range(3):
    for batch_idx, (data, target) in
                enumerate(train_loader):
        # Forward pass
        output = model(data)
        loss = criterion(output, target)

        # Backward pass
        optimizer.zero_grad()
        loss.backward()
        optimizer.step()

        # Print progress
        if batch_idx % 100 == 0:
            print(f'Epoch {epoch}: {loss:.4f}')

# 5. Test One Example
model.eval()
test_image = train_data[0][0]
prediction = model(test_image.unsqueeze(0))
print(f"Predicted: {prediction.argmax()}")
\end{lstlisting}

\vspace{1mm}
\textbf{What You Just Built:}
\begin{itemize}
\item 3-layer neural network
\item 60K training images
\item 97\% accuracy in 3 epochs
\item Runs in 2 minutes on laptop
\end{itemize}

\tryit{Copy this code and run it now!}
\end{columns}
\bottomnote{ Congratulations - you've just trained your first neural network!}
\end{frame}

% Where to Go Next
\begin{frame}{Where to Go Next: Your Learning Path}
\begin{center}
\textbf{Continue Your Neural Network Journey}
\end{center}
\vspace{2mm}
\begin{columns}
\column{0.48\textwidth}
\textbf{Next Topics to Learn:}
\begin{enumerate}
\item \textbf{CNNs:} Computer vision
\item \textbf{RNNs/LSTMs:} Sequences
\item \textbf{Transformers:} Modern NLP
\item \textbf{GANs:} Generation
\item \textbf{RL:} Decision making
\end{enumerate}

\textbf{Practical Projects:}
\begin{itemize}
\item Image classifier for your photos
\item Sentiment analysis of tweets
\item Chatbot for customer service
\item Style transfer for art
\item Game-playing agent
\end{itemize}

\column{0.48\textwidth}
\textbf{Resources:}
\begin{itemize}
\item \textbf{Fast.ai:} Practical deep learning
\item \textbf{PyTorch Tutorials:} Official guides
\item \textbf{Papers with Code:} Latest research
\item \textbf{Kaggle:} Competitions and datasets
\item \textbf{3Blue1Brown:} Visual explanations
\end{itemize}

\textbf{Remember:}
\begin{itemize}
\item Start simple, build up
\item Theory + practice together
\item Join communities
\item Build projects you care about
\item Share what you learn
\end{itemize}
\end{columns}

\vspace{2mm}
\begin{center}
\colorbox{mlgreen!10}{\parbox{0.8\textwidth}{
\centering
\textbf{You've learned how humanity taught machines to think.\\
Now it's your turn to push the boundaries!}
}}
\end{center}
\bottomnote{ The future of AI is being written now - be part of it!}
\end{frame}

% ============================================================================
% APPENDICES (Advanced topics not essential for BSc)
% ============================================================================

\input{sections/appendix_a_advanced_topics}
% Appendix B: Extended History
\begin{frame}{Appendix B: Extended Historical Details}
\begin{center}
{\Large \textbf{The Full Story: Historical Deep Dives}}
\end{center}
\vspace{10mm}

\textbf{This section contains fascinating historical details that enrich the narrative but aren't essential for understanding the technical concepts.}

\vspace{5mm}
Topics covered:
\begin{itemize}
\item The Mark I Perceptron: Physical Hardware
\item NetTalk: Networks Learn to Speak (1987)
\item Batch Normalization: Keeping Networks Stable
\end{itemize}

\vspace{5mm}
\textit{These stories show how each breakthrough built on previous work and overcame specific limitations.}
\end{frame}

% MARK I PERCEPTRON
\begin{frame}{The Mark I Perceptron: A Physical Learning Machine}
\begin{center}
\includegraphics[width=0.85\textwidth]{../figures/perceptron_hardware.pdf}
\end{center}
\bottomnote{ The first neural network wasn't software---it was a room-sized machine with motors and photocells}
\end{frame}

% NETTALK
\begin{frame}{1987: NetTalk - Networks Learn to Speak}
\begin{center}
\textbf{Sejnowski \& Rosenberg: The First Viral NN Demo}
\end{center}
\vspace{5mm}
\begin{columns}
\column{0.48\textwidth}
\textbf{The Challenge:}
\begin{itemize}
\item Convert written text to speech
\item English is irregular (tough, though, through)
\item Rule-based systems had 1000s of exceptions
\end{itemize}

\textbf{The Network:}
\begin{itemize}
\item 7$\times$29 input (7-letter window)
\item 80 hidden neurons
\item 26 output phonemes
\item Trained overnight on DEC workstation
\end{itemize}

\column{0.48\textwidth}
\textbf{The Magic:}
\begin{itemize}
\item Started: Random babbling
\item Hour 1: Vowel-consonant patterns
\item Hour 5: Recognizable words
\item Hour 10: 95\% accuracy!
\end{itemize}

\textbf{Hidden Neurons Learned:}
\begin{itemize}
\item Vowel detectors
\item Consonant clusters
\item Word boundaries
\item \highlight{Nobody programmed these!}
\end{itemize}

\confusion{The network discovered linguistic concepts on its own - features linguists took centuries to identify!}
\end{columns}
\bottomnote{ Media sensation: "Computer teaches itself to read aloud overnight"}
\end{frame}

% BATCH NORMALIZATION
\begin{frame}{Batch Normalization: Keeping Networks Stable}
\begin{center}
\textbf{The Internal Covariate Shift Problem}
\end{center}
\vspace{2mm}
\begin{columns}
\column{0.48\textwidth}
\textbf{The Issue:}
\begin{itemize}
\item Each layer's input distribution changes
\item As previous layers update
\item Makes learning unstable
\item Requires tiny learning rates
\end{itemize}

\textbf{The Solution:}
\begin{itemize}
\item Normalize inputs to each layer
\item Mean = 0, Variance = 1
\item Learn scale and shift parameters
\item Apply during training and testing
\end{itemize}

\column{0.48\textwidth}
\textbf{BatchNorm Algorithm:}
\begin{align*}
\mu_B &= \frac{1}{m} \sum_{i=1}^{m} x_i \\
\sigma_B^2 &= \frac{1}{m} \sum_{i=1}^{m} (x_i - \mu_B)^2 \\
\hat{x}_i &= \frac{x_i - \mu_B}{\sqrt{\sigma_B^2 + \epsilon}} \\
y_i &= \gamma \hat{x}_i + \beta
\end{align*}
\plainmath{1) Find average, 2) Find spread, 3) Normalize to standard range, 4) Scale and shift as needed}

\textbf{Benefits:}
\begin{itemize}
\item 10x faster training
\item Higher learning rates OK
\item Less sensitive to initialization
\item Acts as regularization
\end{itemize}
\end{columns}
\bottomnote{ Now standard in every deep network}
\end{frame}

\end{document}