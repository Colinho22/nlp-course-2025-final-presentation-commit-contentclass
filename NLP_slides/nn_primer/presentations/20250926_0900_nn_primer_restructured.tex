% Neural Networks Primer: Teaching Machines to See Patterns
% RESTRUCTURED VERSION - Optimal Logical Flow
% Changes from original:
% - Moved function approximation AFTER Universal Approximation Theorem
% - Moved activation functions BEFORE XOR problem
% - Removed redundant neuron explanations from Part 0
% - Added geometric XOR bridge slide
% - Updated overview to reflect new structure

\documentclass[8pt,aspectratio=169]{beamer}
\usetheme{Madrid}
\usecolortheme{seahorse}

% Include preamble with all package imports and command definitions
% Neural Networks Primer: Teaching Machines to See Patterns
% BSc-Enhanced Version with Template Layout from template_beamer_final.tex
% Based on the excellent original with targeted accessibility enhancements

% Additional packages for template compatibility
\usepackage{graphicx}
\usepackage{booktabs}
\usepackage{adjustbox}
\usepackage{multicol}
\usepackage{tikz}
\usepackage{amsmath}
\usepackage{amssymb}
\usepackage{listings}

% ML Theme Color definitions from template
\definecolor{mlblue}{RGB}{0,102,204}
\definecolor{mlpurple}{RGB}{51,51,178}
\definecolor{mllavender}{RGB}{173,173,224}
\definecolor{mllavender2}{RGB}{193,193,232}
\definecolor{mllavender3}{RGB}{204,204,235}
\definecolor{mllavender4}{RGB}{214,214,239}
\definecolor{mlorange}{RGB}{255, 127, 14}
\definecolor{mlgreen}{RGB}{44, 160, 44}
\definecolor{mlred}{RGB}{214, 39, 40}
\definecolor{mlgray}{RGB}{127, 127, 127}

% Additional colors for template compatibility
\definecolor{lightgray}{RGB}{240, 240, 240}
\definecolor{midgray}{RGB}{180, 180, 180}

% Keep checkpointBlue for backward compatibility
\definecolor{checkpointBlue}{RGB}{70,130,180}

% Apply custom colors to Madrid theme
\setbeamercolor{palette primary}{bg=mllavender3,fg=mlpurple}
\setbeamercolor{palette secondary}{bg=mllavender2,fg=mlpurple}
\setbeamercolor{palette tertiary}{bg=mllavender,fg=white}
\setbeamercolor{palette quaternary}{bg=mlpurple,fg=white}

\setbeamercolor{structure}{fg=mlpurple}
\setbeamercolor{section in toc}{fg=mlpurple}
\setbeamercolor{subsection in toc}{fg=mlblue}
\setbeamercolor{title}{fg=mlpurple}
\setbeamercolor{frametitle}{fg=mlpurple,bg=mllavender3}
\setbeamercolor{block title}{bg=mllavender2,fg=mlpurple}
\setbeamercolor{block body}{bg=mllavender4,fg=black}

% Remove navigation symbols
\setbeamertemplate{navigation symbols}{}

% Clean itemize/enumerate
\setbeamertemplate{itemize items}[circle]
\setbeamertemplate{enumerate items}[default]

% Reduce margins for more content space
\setbeamersize{text margin left=5mm,text margin right=5mm}

% Commands with updated colors
\newcommand{\highlight}[1]{{\color{mlpurple}#1}}
\newcommand{\secondary}[1]{{\color{mllavender2}#1}}
\newcommand{\warning}[1]{{\color{mlorange}#1}}
\newcommand{\success}[1]{{\color{mlgreen}#1}}
\newcommand{\checkpoint}[1]{{\color{checkpointBlue}#1}}

% Command for bottom annotation (Madrid-style) from template
\newcommand{\bottomnote}[1]{%
\vfill
\vspace{-2mm}
\textcolor{mllavender2}{\rule{\textwidth}{0.4pt}}
\vspace{1mm}
\footnotesize
\textbf{#1}
}

% Box commands for special content with updated colors
\newcommand{\plainmath}[1]{\fbox{\parbox{0.9\textwidth}{\small\textit{In plain words: #1}}}}
\newcommand{\confusion}[1]{\colorbox{mlorange!20}{\parbox{0.9\textwidth}{\small\warning{Common Confusion:} #1}}}
\newcommand{\whymatters}[1]{\colorbox{mlgreen!10}{\parbox{0.3\textwidth}{\tiny\success{Why it matters:} #1}}}
\newcommand{\tryit}[1]{\colorbox{checkpointBlue!15}{\parbox{0.9\textwidth}{\small\textbf{Try It Yourself:} #1}}}

\title{Teaching Machines to See Patterns}
\subtitle{A Neural Networks Primer: Why We Needed Each Piece of the Puzzle}
\author{NLP Course 2025}
\date{}

\begin{document}

% ============================================================================
% PHASE 1: THE MOTIVATION (Slides 1-10)
% ============================================================================

% Title slide
% Title Slide
\begin{frame}
\titlepage
\vfill
\begin{center}
\secondary{\small From the 1950s mail sorting crisis to ChatGPT: How humanity taught machines to think}
\end{center}
\end{frame}

% Updated overview
% Journey Roadmap - RESTRUCTURED VERSION
\begin{frame}{Your Journey Through Neural Networks}
\begin{center}
\textbf{Where We're Going Today - Optimized Flow}
\end{center}
\vspace{5mm}
\begin{columns}
\column{0.48\textwidth}
\textbf{Phase 1: The Motivation}
\begin{itemize}
\item The mail sorting crisis
\item Why rules don't work
\item First mathematical neurons
\item Rosenblatt's learning breakthrough
\end{itemize}

\vspace{3mm}
\textbf{Phase 2: Understanding Basics}
\begin{itemize}
\item How neurons calculate
\item \highlight{Activation functions (NEW placement!)}
\item Hand calculations and examples
\end{itemize}

\vspace{3mm}
\textbf{Phase 3: Crisis \& Solution}
\begin{itemize}
\item The XOR problem
\item \highlight{Geometric intuition (NEW!)}
\item Hidden layers solution
\item Backpropagation breakthrough
\item Universal Approximation Theorem
\item \highlight{Function approximation (moved here!)}
\end{itemize}

\column{0.48\textwidth}
\textbf{Phase 4: Theory to Practice}
\begin{itemize}
\item Gradient landscape visualization
\item LeNet (1998): First success
\item AlexNet (2012): Deep learning explosion
\item Modern architectures
\item Real-world applications
\end{itemize}

\vspace{3mm}
\textbf{Phase 5: Your Turn}
\begin{itemize}
\item Building your first network
\item Debugging tips
\item Next steps \& resources
\end{itemize}

\end{columns}
\bottomnote{Restructured for optimal learning flow - each concept builds naturally on previous ones!}
\end{frame}

% ============================================================================
% PART 1: THE PROBLEM THAT STARTED EVERYTHING
% ============================================================================

% Historical context and early attempts
\begin{frame}{Part 1: The Problem That Started Everything}
\begin{center}
{\Large \textbf{1950s: The Mail Sorting Crisis}}
\end{center}
\vspace{5mm}
\begin{columns}
\column{0.48\textwidth}
\textbf{The Challenge:}
\begin{itemize}
\item 150 million letters per day
\item Hand-written addresses
\item Human sorters: slow, expensive, error-prone
\item Traditional programming: useless
\end{itemize}

\column{0.48\textwidth}
\textbf{Why Traditional Code Failed:}
\begin{itemize}
\item Can't write rules for every handwriting style
\item Too many variations of each letter
\item Context matters: ``I'' vs ``l'' vs ``1''
\item This wasn't computation---it was \highlight{pattern recognition}
\end{itemize}
\end{columns}
\bottomnote{ This problem would take 40 years to solve properly}
\end{frame}

% Historical Timeline
\begin{frame}{80 Years of Neural Networks: The Complete Journey}
\begin{center}
\includegraphics[width=0.95\textwidth]{../figures/historical_timeline.pdf}
\end{center}
\bottomnote{ From theoretical neurons to ChatGPT: Each breakthrough built on previous failures}
\end{frame}

% Why Rules Don't Work
\begin{frame}[fragile]{Why Can't We Just Write Rules?}
\begin{center}
\textbf{Problem: Recognize the Letter ``A''}
\end{center}
\vspace{5mm}
\begin{columns}
\column{0.48\textwidth}
\textbf{Traditional Approach (Failed):}
\begin{lstlisting}[basicstyle=\tiny]
if (has_triangle_top AND
    has_horizontal_bar AND
    two_diagonal_lines) {
  return "A"
}
\end{lstlisting}
\secondary{\small But what about...}
\begin{itemize}
\item Handwritten A's?
\item Different fonts?
\item Rotated A's?
\item Partial A's?
\end{itemize}

\column{0.48\textwidth}
\begin{center}
\includegraphics[width=0.7\textwidth]{../figures/various_a_styles.pdf}
\end{center}
\textbf{The Insight:}
\begin{itemize}
\item We need \highlight{pattern recognition}, not rules
\item System must \highlight{learn from examples}
\item Similar to how humans learn
\end{itemize}
\end{columns}
\bottomnote{ This realization launched the field of machine learning}
\end{frame}

% McCulloch-Pitts
\begin{frame}{1943: The First Mathematical Neuron}
\begin{columns}
\column{0.48\textwidth}
\textbf{McCulloch \& Pitts:}
\begin{itemize}
\item Neurophysiologists studying brain
\item Asked: Can neurons be modeled mathematically?
\item Created first artificial neuron
\end{itemize}

\vspace{5mm}
\textbf{The Model:}
\begin{itemize}
\item Multiple inputs (dendrites)
\item Weighted sum (cell body)
\item Threshold activation (axon)
\item Binary output (fire or not)
\end{itemize}

\column{0.48\textwidth}
\begin{center}
\includegraphics[width=0.9\textwidth]{../figures/nn_building_blocks.pdf}
\end{center}

\textbf{The Limitation:}
\begin{itemize}
\item Weights were \highlight{fixed}
\item No learning mechanism
\item Programmer had to set weights manually
\end{itemize}
\end{columns}
\bottomnote{ Revolutionary idea, but missing the key ingredient: learning}
\end{frame}

% Rosenblatt's Perceptron
\begin{frame}{1958: Rosenblatt's Learning Breakthrough}
\begin{columns}
\column{0.48\textwidth}
\textbf{Frank Rosenblatt's Insight:}
\begin{quote}
\textit{``What if the machine could adjust its own weights based on mistakes?''}
\end{quote}

\textbf{The Perceptron Learning Rule:}
\begin{enumerate}
\item Make a prediction
\item Check if wrong
\item If wrong: adjust weights
\item Repeat until correct
\end{enumerate}

\vspace{3mm}
\textbf{Historic Demo (1958):}
\begin{itemize}
\item Mark I Perceptron machine
\item Learned to recognize simple shapes
\item Press coverage: ``Thinking machine!''
\end{itemize}

\column{0.48\textwidth}
\begin{center}
\includegraphics[width=0.9\textwidth]{../figures/perceptron_hardware.pdf}
\end{center}

\textbf{Why Revolutionary:}
\begin{itemize}
\item First machine that could \highlight{learn}
\item Weights adjusted automatically
\item Learned from examples, not rules
\item Mathematically proven to converge
\end{itemize}
\end{columns}
\bottomnote{ This was the birth of machine learning}
\end{frame}

% ============================================================================
% PHASE 2: UNDERSTANDING THE BASICS (Slides 11-20)
% ============================================================================

% Intermission transition
% Intermission: From Story to Science
\begin{frame}{Intermission: From Story to Science}
\begin{center}
{\Large \textbf{Let's Understand How This Actually Works}}
\end{center}
\vspace{10mm}

\begin{columns}
\column{0.48\textwidth}
\textbf{We've Seen the History...}
\begin{itemize}
\item McCulloch-Pitts invented the neuron
\item Rosenblatt made it learn
\item The perceptron was born
\end{itemize}

\column{0.48\textwidth}
\textbf{Now Let's See the Science:}
\begin{itemize}
\item How does a neuron calculate?
\item What does learning mean?
\item Why was XOR so hard?
\end{itemize}
\end{columns}

\vspace{10mm}
\begin{center}
\colorbox{checkpointBlue!20}{\parbox{0.7\textwidth}{
\centering
\textbf{Next 5 slides: Hands-on calculations and exercises}\\
\small Get your pencil ready - we're going to work through real examples!
}}
\end{center}

\bottomnote{ Don't worry - we'll return to the story once you understand the basics}
\end{frame}

% NEW: Understanding Checkpoint
\begin{frame}{Understanding Check: Can You Answer These?}
\begin{center}
\textbf{\checkpoint{Let's Make Sure We're Together}}
\end{center}
\vspace{5mm}
\begin{columns}
\column{0.48\textwidth}
\textbf{Quick Questions:}
\begin{enumerate}
\item Why couldn't traditional programming solve mail sorting?
\item What does a weight represent in simple terms?
\item Why do we need the bias term?
\item What was revolutionary about Rosenblatt's perceptron?
\end{enumerate}

\column{0.48\textwidth}
\textbf{Think About It:}
\begin{itemize}
\item A weight is like the importance/trust we give to each input
\item Bias shifts our decision threshold
\item Learning = adjusting these weights
\item The perceptron was the first machine that could learn!
\end{itemize}
\end{columns}
\vspace{5mm}
\tryit{Draw a simple perceptron with 2 inputs. Label the weights, bias, and output. What would the weights be to compute AND logic?}
\bottomnote{ If any of these are unclear, revisit the previous slides before continuing}
\end{frame}

% The Math Behind It (Simple)
\begin{frame}{Making It Concrete: Teaching OR Logic}
\begin{center}
\textbf{Problem: Learn OR function (output 1 if ANY input is 1)}
\end{center}
\vspace{5mm}
\begin{columns}
\column{0.5\textwidth}
\textbf{Training Data:}
\begin{center}
\begin{tabular}{cc|c}
$x_1$ & $x_2$ & Output \\
\hline
0 & 0 & 0 \\
0 & 1 & 1 \\
1 & 0 & 1 \\
1 & 1 & 1 \\
\end{tabular}
\end{center}

\textbf{The Perceptron:}
\begin{align*}
z &= w_1 \cdot x_1 + w_2 \cdot x_2 + b \\
\text{output} &= \begin{cases} 1 & \text{if } z > 0 \\ 0 & \text{if } z \leq 0 \end{cases}
\end{align*}
\plainmath{Multiply first input by first weight, second input by second weight, add bias, check if positive}

\column{0.5\textwidth}
\textbf{Learning Process:}
\begin{enumerate}
\item Start with random weights
\item For each example:
  \begin{itemize}
  \item Calculate output
  \item If wrong: adjust weights
  \item If correct: keep weights
  \end{itemize}
\item Repeat until all correct
\end{enumerate}

\textbf{Final Solution:}
$w_1 = 1$, $w_2 = 1$, $b = -0.5$
\end{columns}
\bottomnote{ Success! But this was just the beginning...}
\end{frame}

% NEW: Let's Calculate Together - Spam Detection
\begin{frame}{Let's Calculate Together: Is This Email Spam?}
\begin{center}
\textbf{\checkpoint{A Real Perceptron Calculation You Can Follow}}
\end{center}
\vspace{5mm}
\begin{columns}
\column{0.48\textwidth}
\textbf{The Email:}
\fbox{\parbox{0.9\textwidth}{\small
"FREE money! Click here NOW for amazing offer!!!"
}}

\textbf{Our Features (Inputs):}
\begin{itemize}
\item $x_1$ = Has "FREE"? = 1
\item $x_2$ = Has "money"? = 1
\item $x_3$ = Many "!"? = 1
\item $x_4$ = From friend? = 0
\end{itemize}

\textbf{Learned Weights:}
\begin{itemize}
\item $w_1$ = +3 (FREE is very spammy)
\item $w_2$ = +2 (money is suspicious)
\item $w_3$ = +2 (!!! is aggressive)
\item $w_4$ = -5 (friends are trusted)
\item $b$ = -2 (threshold)
\end{itemize}

\column{0.48\textwidth}
\textbf{Let's Calculate:}
\begin{align*}
z &= w_1 \cdot x_1 + w_2 \cdot x_2 + w_3 \cdot x_3 + w_4 \cdot x_4 + b \\
  &= 3 \cdot 1 + 2 \cdot 1 + 2 \cdot 1 + (-5) \cdot 0 + (-2) \\
  &= 3 + 2 + 2 + 0 - 2 \\
  &= 5
\end{align*}

\textbf{Decision:}
\begin{itemize}
\item $z = 5 > 0$
\item Output = 1 = SPAM!
\end{itemize}

\tryit{What if this email WAS from a friend ($x_4 = 1$)? Recalculate! Would it still be spam?}

\textbf{Answer:} $z = 5 - 5 = 0$, borderline!
\end{columns}
\bottomnote{ This is exactly how early spam filters worked - and why they failed on clever spam}
\end{frame}

% Notation Explained
\begin{frame}{Understanding the Notation}
\begin{center}
\textbf{Breaking Down the Math Symbols}
\end{center}
\vspace{5mm}
\begin{columns}
\column{0.48\textwidth}
\textbf{Inputs and Weights:}
\begin{itemize}
\item $x_i$ = input value (what we see)
\item $w_i$ = weight (importance/strength)
\item $b$ = bias (threshold adjuster)
\end{itemize}

\textbf{The Computation:}
$$z = \sum_{i=1}^{n} w_i x_i + b$$

This means:
\begin{itemize}
\item Multiply each input by its weight
\item Add them all up
\item Add the bias
\end{itemize}

\column{0.48\textwidth}
\textbf{Real Example:}
\begin{center}
Should I go outside? \\[3mm]
\begin{tabular}{lcc}
Factor & Value & Weight \\
\hline
Sunny? & 1 & +2 \\
Raining? & 0 & -3 \\
Weekend? & 1 & +1 \\
\hline
\end{tabular}
\end{center}
$$z = (1 \times 2) + (0 \times -3) + (1 \times 1) = 3$$
$$\text{Decision: } z > 0 \text{, so YES!}$$
\end{columns}
\bottomnote{ This simple math would evolve into deep learning}
\end{frame}

% NEW: Activation functions introduced BEFORE XOR problem
% Activation Functions and Visualization
\begin{frame}{Why Linear Doesn't Work: Activation Functions}
\begin{center}
\textbf{The Need for Non-Linearity}
\end{center}
\vspace{5mm}
\begin{columns}
\column{0.48\textwidth}
\textbf{Problem with Linear:}
\begin{itemize}
\item Stack of linear layers = still linear!
\item $f(g(x)) = (wx + b_1)w' + b_2 = w'wx + ...$
\item Can't learn complex patterns
\end{itemize}

\textbf{Solution: Activation Functions}
\begin{itemize}
\item Add non-linearity after each layer
\item Allows learning complex boundaries
\item Different functions for different needs
\end{itemize}

\column{0.48\textwidth}
\textbf{Common Activation Functions:}
\begin{itemize}
\item \textbf{Sigmoid:} $\sigma(x) = \frac{1}{1 + e^{-x}}$
  \begin{itemize}
  \item Smooth, outputs 0-1
  \item Good for probabilities
  \end{itemize}
  \plainmath{Squashes any input to range 0-1. Large positive becomes 1, large negative becomes 0}
\item \textbf{ReLU:} $f(x) = \max(0, x)$
  \begin{itemize}
  \item Simple, fast
  \item Solves vanishing gradient
  \end{itemize}
\item \textbf{Tanh:} $\tanh(x) = \frac{e^x - e^{-x}}{e^x + e^{-x}}$
  \begin{itemize}
  \item Outputs -1 to 1
  \item Zero-centered
  \end{itemize}
\end{itemize}
\end{columns}
\bottomnote{ ReLU's simplicity revolutionized deep learning in 2011}
\end{frame}

% Simple 2D Example
\begin{frame}{Visualizing Learning: 2D Classification}
\begin{center}
\textbf{Teaching a Network to Separate Red from Blue Points}
\end{center}
\vspace{5mm}
\begin{columns}
\column{0.48\textwidth}
\textbf{The Setup:}
\begin{itemize}
\item Input: (x, y) coordinates
\item Output: Red or Blue class
\item Network: 2 $\rightarrow$ 4 $\rightarrow$ 2 neurons
\end{itemize}

\textbf{Training Process:}
\begin{enumerate}
\item Epoch 1: Random boundary
\item Epoch 10: Rough separation
\item Epoch 50: Good boundary
\item Epoch 100: Perfect fit
\end{enumerate}

\column{0.48\textwidth}
\begin{center}
\includegraphics[width=\textwidth]{../figures/2d_classification_evolution.pdf}
\end{center}
\textbf{What Each Layer Learns:}
\begin{itemize}
\item Layer 1: Simple boundaries
\item Hidden: Combine boundaries
\item Output: Final decision
\end{itemize}
\end{columns}
\bottomnote{ This same principle scales to millions of parameters}
\end{frame}

% ============================================================================
% PHASE 3: THE CRISIS & SOLUTION (Slides 21-32)
% ============================================================================

\begin{frame}{Part 2: The XOR Crisis}
\begin{center}
{\Large \textbf{1969: The Problem That Killed AI}}
\end{center}
\end{frame}

% XOR Problem Introduction
\begin{frame}{The XOR Problem: Why Single Neurons Fail}
\begin{columns}
\column{0.48\textwidth}
\textbf{XOR (Exclusive OR):}
\begin{itemize}
\item Output 1 if inputs are different
\item Output 0 if inputs are same
\end{itemize}

\vspace{3mm}
\begin{center}
\textbf{Truth Table:}\\[3mm]
\begin{tabular}{cc|c}
$x_1$ & $x_2$ & Output \\
\hline
0 & 0 & 0 \\
0 & 1 & 1 \\
1 & 0 & 1 \\
1 & 1 & 0
\end{tabular}
\end{center}

\vspace{3mm}
\textbf{The Challenge:}
\begin{itemize}
\item Try drawing ONE straight line
\item That separates 1's from 0's
\item Impossible!
\end{itemize}

\column{0.48\textwidth}
\begin{center}
\includegraphics[width=0.9\textwidth]{../figures/xor_impossible_line.pdf}
\end{center}

\textbf{Why It Matters:}
\begin{itemize}
\item Perceptrons can only draw straight lines
\item XOR requires curved boundary
\item This is the simplest non-linear problem
\item Minsky \& Papert proved this mathematically
\end{itemize}
\end{columns}
\bottomnote{ This proof triggered the first AI Winter (1970-1980)}
\end{frame}

% NEW: Geometric intuition bridge slide
\begin{frame}{NEW: Geometric Intuition - Why We Need Two Boundaries}
\begin{center}
\textbf{The Key Insight: XOR Needs TWO Lines, Not One}
\end{center}
\vspace{2mm}
\begin{columns}
\column{0.48\textwidth}
\textbf{Visualization:}
\begin{center}
\includegraphics[width=0.85\textwidth]{../figures/xor_solution_steps.pdf}
\end{center}

\textbf{What We See:}
\begin{itemize}
\item Red line: Separates (0,0) from others
\item Blue line: Separates (1,1) from others
\item Green region: Intersection of both
\item Points (0,1) and (1,0) in green = Output 1!
\end{itemize}

\column{0.48\textwidth}
\textbf{The Breakthrough Idea:}
\begin{enumerate}
\item Use TWO neurons (not one)
\item Each neuron creates one boundary
\item Combine their outputs
\item Intersection solves XOR!
\end{enumerate}

\vspace{1mm}
\textbf{This Requires:}
\begin{itemize}
\item Hidden layer with 2 neurons
\item Output layer combines results
\item This is a 2-layer network
\item First layer: create boundaries
\item Second layer: find intersection
\end{itemize}

\vspace{3mm}
\colorbox{green!20}{\parbox{0.95\textwidth}{\centering
\textbf{Key Insight:} Hidden layers let us combine simple boundaries into complex decision regions!
}}
\end{columns}
\bottomnote{ This geometric intuition explains WHY hidden layers work}
\end{frame}

% Hidden Layers Solution
\begin{frame}{The Solution: Hidden Layers}
\begin{columns}
\column{0.48\textwidth}
\textbf{Architecture with Hidden Layer:}
\begin{center}
\includegraphics[width=0.9\textwidth]{../figures/multilayer_network.pdf}
\end{center}

\textbf{How It Works:}
\begin{itemize}
\item Input layer: 2 neurons ($x_1$, $x_2$)
\item Hidden layer: 2 neurons (two boundaries)
\item Output layer: 1 neuron (combines)
\end{itemize}

\column{0.48\textwidth}
\textbf{Forward Pass for XOR:}

Given weights:
\begin{itemize}
\item Hidden 1: $w=[1,1], b=-0.5$
\item Hidden 2: $w=[1,1], b=-1.5$
\item Output: $w=[1,-1], b=0$
\end{itemize}

\vspace{3mm}
For input $(1,0)$:
\begin{align*}
h_1 &= \sigma(1 \cdot 1 + 1 \cdot 0 - 0.5) \\
&= \sigma(0.5) \approx 0.62 \\
h_2 &= \sigma(1 \cdot 1 + 1 \cdot 0 - 1.5) \\
&= \sigma(-0.5) \approx 0.38 \\
y &= \sigma(1 \cdot 0.62 - 1 \cdot 0.38) \\
&= \sigma(0.24) \approx 0.56 \text{ (close to 1!)}
\end{align*}
\end{columns}
\bottomnote{ Hidden layers unlock non-linear patterns}
\end{frame}

% The Training Problem
\begin{frame}{The New Challenge: How to Train Hidden Layers?}
\begin{columns}
\column{0.48\textwidth}
\textbf{The Credit Assignment Problem:}
\begin{itemize}
\item Output is wrong - we know the error
\item But which hidden neuron caused it?
\item How much should each weight change?
\item Perceptron rule only works for output layer
\end{itemize}

\vspace{3mm}
\textbf{Why It's Hard:}
\begin{center}
\includegraphics[width=0.7\textwidth]{../figures/blame_distribution.pdf}
\end{center}

\column{0.48\textwidth}
\textbf{Early Failed Attempts (1970s):}
\begin{itemize}
\item Random weight adjustment
\item Genetic algorithms
\item Simulated annealing
\item All too slow or unreliable
\end{itemize}

\vspace{3mm}
\textbf{What We Needed:}
\begin{itemize}
\item Systematic way to assign blame
\item Efficient computation
\item Guaranteed to improve
\item Works for many layers
\end{itemize}

\vspace{3mm}
\colorbox{orange!20}{\parbox{0.95\textwidth}{\centering
\textbf{The solution would come from calculus...}
}}
\end{columns}
\bottomnote{ This problem was solved in 1986 with backpropagation}
\end{frame}

% NEW: Before Backpropagation - Explain Loss
\begin{frame}{Before We Can Learn: Measuring Error}
\begin{center}
\textbf{How Do We Know If Our Network Is Wrong?}
\end{center}
\vspace{2mm}
\begin{columns}
\column{0.48\textwidth}
\textbf{A Concrete Example:}

\textit{Predicting house price:}
\begin{itemize}
\item Network predicts: \$300,000
\item Actual price: \$400,000
\item Error = \$400k - \$300k = \$100k
\end{itemize}

\vspace{2mm}
\textbf{The Problem:}
\begin{itemize}
\item Positive errors (+\$100k) and negative errors (-\$100k) cancel out
\item We care about magnitude, not direction
\item Solution: Square the error!
\end{itemize}

\vspace{2mm}
\textbf{Why Square It?}
\begin{itemize}
\item Always positive: (\$100k)$^2$ = 10,000M$^2$
\item Big mistakes hurt more: (\$200k)$^2$ = 40,000M$^2$
\item Math works nicely for optimization
\end{itemize}

\column{0.48\textwidth}
\textbf{The Loss Function:}

For one example:
$$\text{Error} = (\text{predicted} - \text{actual})^2$$

For all training examples:
$$\text{Loss} = \frac{1}{n} \sum_{i=1}^{n} (\text{predicted}_i - \text{actual}_i)^2$$

\plainmath{Average of squared errors across all examples}

\vspace{2mm}
\textbf{Real Numbers:}
\begin{center}
\small
\begin{tabular}{ccc}
Predicted & Actual & Error$^2$ \\
\hline
0.3 & 1.0 & 0.49 \\
0.8 & 1.0 & 0.04 \\
0.1 & 0.0 & 0.01 \\
\hline
\multicolumn{2}{r}{Average:} & 0.18 \\
\end{tabular}
\end{center}

\vspace{2mm}
\colorbox{checkpointBlue!20}{\parbox{0.95\columnwidth}{
\centering\small
\textbf{Goal of training:} Make this loss as small as possible!
}}
\end{columns}
\bottomnote{ Now we can talk about backpropagation - the algorithm that minimizes this loss}
\end{frame}

% Backpropagation
\begin{frame}{1986: Backpropagation - The Breakthrough Algorithm}
\begin{columns}
\column{0.48\textwidth}
\textbf{The Algorithm (Rumelhart et al.):}
\begin{enumerate}
\item \textbf{Forward pass}: Compute output
\item \textbf{Compute error}: Compare to target
\item \textbf{Backward pass}: Use chain rule to compute gradients
\item \textbf{Update weights}: Gradient descent
\end{enumerate}

\vspace{3mm}
\textbf{The Key Insight:}
\begin{itemize}
\item Use calculus (chain rule)
\item Error flows backward through network
\item Each layer gets its share of blame
\item Weights adjusted proportionally
\end{itemize}

\column{0.48\textwidth}
\begin{center}
\includegraphics[width=0.9\textwidth]{../figures/gradient_flow_visualization.pdf}
\end{center}

\textbf{Mathematical Foundation:}
\begin{align*}
\frac{\partial L}{\partial w_{ij}} &= \frac{\partial L}{\partial a_j} \cdot \frac{\partial a_j}{\partial z_j} \cdot \frac{\partial z_j}{\partial w_{ij}} \\
&= \delta_j \cdot a_i
\end{align*}

\textbf{Why Revolutionary:}
\begin{itemize}
\item Efficient: One backward pass
\item General: Works for any architecture
\item Automatic: No manual tuning
\end{itemize}
\end{columns}
\bottomnote{ This paper revived neural networks and enabled modern deep learning}
\end{frame}

% Gradient visualization
\begin{frame}{Visualizing Learning: The Gradient Landscape}
\begin{center}
\includegraphics[width=0.75\textwidth]{../figures/gradient_landscape_3d.pdf}
\end{center}
\bottomnote{ Gradient descent finds the valley where error is minimized}
\end{frame}

% Universal Approximation Theorem
\begin{frame}{1989: The Theoretical Guarantee}
\begin{columns}
\column{0.48\textwidth}
\textbf{Cybenko's Universal Approximation Theorem:}

\begin{quote}
\textit{A neural network with one hidden layer and finite neurons can approximate ANY continuous function to ANY desired accuracy}
\end{quote}

\vspace{3mm}
\textbf{What This Means:}
\begin{itemize}
\item Mathematical proof
\item Not just XOR - ANY pattern!
\item Theoretical justification
\item Explains why NNs are so powerful
\end{itemize}

\column{0.48\textwidth}
\begin{center}
\includegraphics[width=0.9\textwidth]{../figures/universal_approximation.pdf}
\end{center}

\textbf{Caveats:}
\begin{itemize}
\item Guarantees existence, not learning
\item May need exponential neurons
\item Deep networks often more efficient
\item Still need good training algorithm
\end{itemize}
\end{columns}
\bottomnote{ Theory meets practice: NNs CAN learn any pattern, backprop shows us HOW}
\end{frame}

% NOW: Function approximation (moved from Part 0)
% FOUNDATIONAL CONCEPT: Neural Networks as Function Approximators
% These three slides establish the core mathematical concept before the historical narrative

% Slide 1: What is Function Approximation?
\begin{frame}{The Core Idea: Neural Networks are Function Approximators}
\begin{center}
\textbf{What does this actually mean?}
\end{center}
\vspace{1mm}

\begin{columns}
\column{0.32\textwidth}
\textbf{The Problem:}
\begin{itemize}
\item We have inputs (x)
\item We want outputs (y)
\item But we don't know the formula!
\item Examples:
  \begin{itemize}
  \footnotesize
  \item Size $\rightarrow$ Price
  \item Image $\rightarrow$ Label
  \item Text $\rightarrow$ Sentiment
  \end{itemize}
\end{itemize}

\column{0.32\textwidth}
\textbf{Traditional Approach:}
\begin{itemize}
\item Guess the formula
\item Write explicit rules
\item Hope it works
\item \warning{Problem:} Real world is too complex!
\end{itemize}

\vspace{3mm}
\textit{Example:}\\
\footnotesize
Price = a $\times$ Size + b\\
(Too simple for real data!)

\column{0.32\textwidth}
\textbf{Neural Network Approach:}
\begin{itemize}
\item Learn from examples
\item Build the formula automatically
\item Adjust until it fits
\item \success{Works for ANY pattern!}
\end{itemize}

\vspace{3mm}
\textit{Magic:}\\
\footnotesize
NN learns: f(x) $\approx$ y\\
No formula needed!
\end{columns}

\vspace{2mm}
\begin{center}
\includegraphics[width=0.75\textwidth]{../figures/function_approx_basics.pdf}
\end{center}

\vspace{2mm}
\plainmath{A function approximator finds patterns in data without being told what the pattern is}
\end{frame}

% Slide 2: How Neural Networks Build Functions
\begin{frame}{How NNs Build Complex Functions from Simple Pieces}
\begin{center}
\textbf{The LEGO Principle: Combine Simple Parts to Build Anything}
\end{center}
\vspace{1mm}

\begin{columns}
\column{0.48\textwidth}
\textbf{The Building Blocks:}
\begin{enumerate}
\item \textbf{Individual Neurons:} Simple decisions
   \begin{itemize}
   \footnotesize
   \item "Is input > threshold?"
   \item Outputs: on/off (smooth version)
   \end{itemize}

\item \textbf{Combine Neurons:} Weight and add
   \begin{itemize}
   \footnotesize
   \item Create complex shapes
   \end{itemize}

\item \textbf{Stack Layers:} Build hierarchy
   \begin{itemize}
   \footnotesize
   \item Each layer adds abstraction
   \end{itemize}
\end{enumerate}

\column{0.48\textwidth}
\textbf{Real-World Analogy:}

\textit{Making a Cake from Ingredients:}
\begin{itemize}
\footnotesize
\item Flour + Sugar + Eggs
\item Mix right amounts
\item $\rightarrow$ Perfect cake!
\end{itemize}

\textbf{In Neural Networks:}
\begin{itemize}
\footnotesize
\item Edges + Curves + Colors
\item Combine with weights
\item $\rightarrow$ Recognize faces!
\end{itemize}
\end{columns}

\begin{center}
\includegraphics[width=0.55\textwidth]{../figures/nn_building_blocks.pdf}
\end{center}
\end{frame}

% Slide 3: Universal Approximation Theorem in Plain English
\begin{frame}{The Universal Approximation Theorem: Why This Always Works}
\begin{center}
\textbf{The Most Important Theorem in Deep Learning (Cybenko, 1989)}
\end{center}
\vspace{1mm}

\begin{columns}
\column{0.48\textwidth}
\textbf{The Theorem (Plain English):}

\colorbox{checkpointBlue!20}{\parbox{0.95\columnwidth}{
\centering
\textit{"A neural network with enough neurons can approximate \textbf{ANY} continuous function to \textbf{ANY} desired accuracy"}
}}

\vspace{2mm}
\textbf{What This Means:}
\begin{itemize}
\item \success{Universal:} Works for any smooth pattern
\item \success{Guaranteed:} Not hoping, but proving
\item \success{Practical:} Just add more neurons!
\end{itemize}

\vspace{1mm}
\textbf{The Catch:}
\begin{itemize}
\item \warning{How many neurons?} Could be millions
\item \warning{How to train?} That's the art
\end{itemize}

\column{0.48\textwidth}
\textbf{Intuitive Proof:}

\textit{Pixel art analogy:}
\begin{itemize}
\footnotesize
\item 4 pixels: Blocky
\item 100 pixels: Recognizable
\item 10,000 pixels: Photo-realistic
\end{itemize}

\textit{Same with neurons:}
\begin{itemize}
\footnotesize
\item Few: Rough
\item More: Better
\item Many: Nearly perfect
\end{itemize}

\vspace{1mm}
\end{columns}

\begin{center}
\includegraphics[width=0.55\textwidth]{../figures/universal_approximation.pdf}
\end{center}

\vspace{2mm}
\confusion{Common mistake: "Universal" doesn't mean "easy" or "fast" - it just means "possible"!}
\end{frame}

% ============================================================================
% PHASE 4: FROM THEORY TO PRACTICE (Slides 33-44)
% ============================================================================

% Advanced visualizations
% Advanced Visualizations - NEW SLIDES

\begin{frame}{The Neuron as a 3D Function}
\begin{center}
\textbf{Visualizing How Activation Functions Transform the Output Space}
\end{center}
\vspace{2mm}

\includegraphics[width=0.8\textwidth]{../figures/neuron_3d_visualization.pdf}

\bottomnote{ Left: Linear (just a plane), Right: With activation (curved surface) - this non-linearity is what makes learning possible!}
\end{frame}

\begin{frame}{From Simple to Complex: Network Depth Creates Complexity}
\begin{center}
\textbf{How More Neurons Enable More Complex Decision Boundaries}
\end{center}
\vspace{2mm}

\includegraphics[width=0.8\textwidth]{../figures/network_complexity_visualization.pdf}

\bottomnote{ Each neuron adds a new "dimension" to what the network can learn}
\end{frame}

\begin{frame}{The Learning Process: Frame by Frame}
\begin{center}
\textbf{Watching Decision Boundaries Evolve During Training}
\end{center}
\vspace{2mm}

\includegraphics[width=0.8\textwidth]{../figures/decision_boundary_evolution.pdf}

\tryit{Notice how the boundary starts random and gradually fits the data pattern!}
\bottomnote{ This is what "learning" looks like - not magic, just systematic improvement}
\end{frame}

\begin{frame}{XOR Problem Solved Visually}
\begin{center}
\textbf{Why We Need Hidden Layers: The XOR Solution}
\end{center}
\vspace{2mm}

\includegraphics[width=0.8\textwidth]{../figures/xor_solution_visualization.pdf}

\bottomnote{ Two hidden neurons working together can solve what one neuron cannot}
\end{frame}

\begin{frame}{Forward Pass: Signal Propagation Step-by-Step}
\begin{center}
\textbf{Following Data as it Flows Through the Network}
\end{center}
\vspace{2mm}

\includegraphics[width=0.8\textwidth]{../figures/forward_pass_frames.pdf}

\bottomnote{ Each frame shows one step: computing weighted sums, applying activations, passing to next layer}
\end{frame}

\begin{frame}{Detailed Computation: Inside One Neuron}
\begin{center}
\textbf{The Math Behind a Single Neuron's Calculation}
\end{center}
\vspace{2mm}

\includegraphics[width=0.8\textwidth]{../figures/neuron_computation_detail.pdf}

\tryit{Follow along: multiply each input by its weight, add them up, add bias, apply activation!}
\bottomnote{ This calculation happens millions of times per second in modern networks}
\end{frame}

\begin{frame}{The Optimization Landscape}
\begin{center}
\textbf{Gradient Descent: Finding the Valley in 3D Space}
\end{center}
\vspace{2mm}

\includegraphics[width=0.8\textwidth]{../figures/gradient_landscape_3d.pdf}

\bottomnote{ Training a neural network = rolling a ball down this landscape to find the lowest point}
\end{frame}

\begin{frame}{Comparing Optimization Algorithms}
\begin{center}
\textbf{Why Adam Outperforms Simple Gradient Descent}
\end{center}
\includegraphics[width=0.68\textwidth]{../figures/optimizer_paths_comparison.pdf}

\bottomnote{ Different optimizers take different paths - some are much more efficient}
\end{frame}

\begin{frame}{Architecture Evolution Over Time}
\begin{center}
\textbf{From 20 Parameters to 1.8 Trillion: The Growth of Neural Networks}
\end{center}
\vspace{2mm}

\includegraphics[width=0.8\textwidth]{../figures/architecture_size_comparison.pdf}

\bottomnote{ Each 10x increase in size unlocked new capabilities}
\end{frame}

\begin{frame}{Architecture Types Comparison}
\begin{center}
\textbf{Different Architectures for Different Problems}
\end{center}
\vspace{2mm}

\includegraphics[width=0.8\textwidth]{../figures/architecture_type_comparison.pdf}

\bottomnote{ Each architecture encodes different assumptions about the data structure}
\end{frame}

\begin{frame}{Gradient Flow: Healthy vs Vanishing}
\begin{center}
\textbf{Why Deep Networks Were Hard Before ReLU}
\end{center}
\vspace{2mm}

\includegraphics[width=0.8\textwidth]{../figures/gradient_flow_visualization.pdf}

\bottomnote{ Left: Healthy gradient flow, Right: Vanishing gradients - this problem limited networks to 2-3 layers for decades}
\end{frame}

% Part 3: Breakthrough years
\begin{frame}{Part 3: The Breakthrough Years (1998-2012)}
\begin{center}
{\Large \textbf{From Theory to Working Systems}}
\end{center}
\end{frame}

% Act III: The Breakthrough Years
\begin{frame}{Act III: The Breakthrough Years}
\begin{center}
{\Large \textbf{1998-2012: From Digits to ImageNet}}
\end{center}
\vspace{5mm}
\begin{columns}
\column{0.48\textwidth}
\textbf{1998 - LeNet: First Success}
\begin{itemize}
\item Yann LeCun's CNN for digits
\item 32$\times$32 pixels $\rightarrow$ 10 classes
\item 60,000 parameters
\item Banks adopt for check reading
\end{itemize}

\textbf{Key Innovation: Convolutions}
\begin{itemize}
\item Share weights across image
\item Detect features anywhere
\item Build complexity layer by layer
\end{itemize}

\column{0.48\textwidth}
\textbf{2012 - AlexNet: The Revolution}
\begin{itemize}
\item 1000 ImageNet classes
\item 60 million parameters
\item GPUs enable training
\item Error rate: 26\% $\rightarrow$ 16\%
\end{itemize}

\textbf{What Changed:}
\begin{itemize}
\item Big Data (millions of images)
\item GPU computing (100x faster)
\item ReLU activation
\item Dropout regularization
\end{itemize}
\end{columns}
\bottomnote{ This victory ended the second AI winter permanently}
\end{frame}

% Understanding Convolutions
\begin{frame}{The Convolution Innovation: See Like Humans Do}
\begin{center}
\textbf{How We Actually Recognize Objects}
\end{center}
\vspace{5mm}
\begin{columns}
\column{0.48\textwidth}
\textbf{Human Vision Process:}
\begin{enumerate}
\item Detect edges
\item Find shapes
\item Identify parts
\item Recognize object
\end{enumerate}

\textbf{CNN Mimics This:}
\begin{itemize}
\item Layer 1: Edge detectors
\item Layer 2: Corner/curve detectors
\item Layer 3: Part detectors
\item Layer 4: Object detectors
\end{itemize}

\column{0.48\textwidth}
\begin{center}
\includegraphics[width=\textwidth]{../figures/cnn_feature_hierarchy.pdf}
\end{center}
\textbf{Key Insight:}
\begin{itemize}
\item A "wheel detector" works anywhere in image
\item Share the same detector across positions
\item Reduces parameters dramatically
\item Makes network translation-invariant
\end{itemize}
\end{columns}
\bottomnote{ This is why CNNs dominate computer vision}
\end{frame}

% The Mathematics of Learning
\begin{frame}{The Mathematics of Learning: Gradient Descent}
\begin{center}
\textbf{Finding the Best Weights: Like Hiking Down a Mountain}
\end{center}
\vspace{2mm}
\begin{columns}
\column{0.48\textwidth}
\textbf{The Optimization Problem:}
\begin{itemize}
\item Millions of weights to adjust
\item Each affects the error
\item Need to find best combination
\end{itemize}

\textbf{Gradient Descent:}
\begin{enumerate}
\item Calculate error (loss)
\item Find slope (gradient) for each weight
\item Step downhill: $w = w - \alpha \cdot \nabla L$
  \plainmath{New weight = old weight - (step size times slope)}
\item Repeat until bottom
\end{enumerate}

\column{0.48\textwidth}
\begin{center}
\includegraphics[width=0.85\textwidth]{../figures/gradient_descent.pdf}
\end{center}
\textbf{Learning Rate ($\alpha$):}
\begin{itemize}
\item Too small: takes forever
\item Too large: overshoot minimum
\item Just right: smooth convergence
\end{itemize}
\end{columns}
\bottomnote{ Modern optimizers like Adam adapt the learning rate automatically}
\end{frame}

% Types of Learning
\begin{frame}{Types of Learning: Different Problems, Different Approaches}
\begin{columns}
\column{0.48\textwidth}
\textbf{Supervised Learning:}
\begin{itemize}
\item Have input-output pairs
\item Learn mapping function
\item Examples: Classification, Regression
\end{itemize}

\textbf{Unsupervised Learning:}
\begin{itemize}
\item Only have inputs
\item Find patterns/structure
\item Examples: Clustering, Compression
\end{itemize}

\column{0.48\textwidth}
\textbf{Reinforcement Learning:}
\begin{itemize}
\item Learn through trial/error
\item Maximize reward signal
\item Examples: Games, Robotics
\end{itemize}

\textbf{Self-Supervised (Modern):}
\begin{itemize}
\item Create labels from data itself
\item Predict next word, masked words
\item Examples: GPT, BERT
\end{itemize}
\end{columns}
\bottomnote{ Self-supervised learning powers all modern language models}
\end{frame}

% NEW: Understanding Checkpoint - Types of Learning
\begin{frame}{Check Your Understanding: Learning Types}
\begin{center}
\textbf{\checkpoint{Can You Match These Examples?}}
\end{center}
\vspace{5mm}
\tryit{Match each scenario to a learning type: Supervised, Unsupervised, Reinforcement, Self-Supervised}
\vspace{5mm}
\begin{columns}
\column{0.48\textwidth}
\textbf{Scenarios:}
\begin{enumerate}
\item Teaching a robot to walk by giving rewards for standing
\item Showing 1000 cat photos labeled "cat"
\item Giving GPT text with words masked out
\item Finding groups in customer data
\end{enumerate}

\column{0.48\textwidth}
\textbf{Answers:}
\begin{enumerate}
\item Reinforcement (trial and error)
\item Supervised (labeled examples)
\item Self-supervised (creates own labels)
\item Unsupervised (finds patterns)
\end{enumerate}

\confusion{Self-supervised IS supervised learning - we just create the labels automatically from the data itself!}
\end{columns}
\bottomnote{ Understanding these differences helps you choose the right approach}
\end{frame}

% Overfitting Problem
\begin{frame}{The Overfitting Problem: When Learning Goes Too Far}
\begin{center}
\textbf{Memorization vs. Understanding}
\end{center}
\vspace{5mm}
\begin{columns}
\column{0.48\textwidth}
\textbf{The Problem:}
\begin{itemize}
\item Network memorizes training data
\item Fails on new, unseen data
\item Like student memorizing answers
\end{itemize}

\textbf{Signs of Overfitting:}
\begin{itemize}
\item Training accuracy: 99\%
\item Test accuracy: 60\%
\item Complex decision boundaries
\item High variance
\end{itemize}

\column{0.48\textwidth}
\begin{center}
\includegraphics[width=\textwidth]{../figures/overfitting_visualization.pdf}
\end{center}
\textbf{Solutions:}
\begin{itemize}
\item \textbf{More data:} Can't memorize everything
\item \textbf{Dropout:} Randomly disable neurons
\item \textbf{Regularization:} Penalize complexity
\item \textbf{Early stopping:} Stop before overfitting
\end{itemize}
\end{columns}
\bottomnote{ "With four parameters I can fit an elephant, with five I can make him wiggle his trunk" - von Neumann}
\end{frame}

% NEW: Feature Hierarchy Progression
\begin{frame}{How Deep Networks See: Building Features Layer by Layer}
\begin{center}
\textbf{From Pixels to Concepts: The Hierarchy of Understanding}
\end{center}
\vspace{5mm}
\begin{columns}
\column{0.55\textwidth}
\includegraphics[width=\textwidth]{../figures/feature_hierarchy_progression.pdf}

\textbf{What Each Layer Learns:}
\begin{itemize}
\item \textbf{Layer 1:} Edges, colors, gradients
\item \textbf{Layer 2:} Corners, textures, curves
\item \textbf{Layer 3:} Parts (eyes, wheels, patterns)
\item \textbf{Layer 4:} Objects (faces, cars, scenes)
\item \textbf{Layer 5:} Concepts (identity, style, context)
\end{itemize}

\column{0.42\textwidth}
\textbf{Why Hierarchy Matters:}
\begin{itemize}
\item Reusable features
\item Efficient representation
\item Transfer learning works
\item Mimics visual cortex
\end{itemize}

\textbf{Discovered Automatically:}
\begin{itemize}
\item No manual feature engineering
\item Emerges from data
\item Different tasks, same hierarchy
\item Universal pattern
\end{itemize}
\end{columns}
\bottomnote{ Each layer combines features from the previous layer into more abstract concepts}
\end{frame}

% NEW: Training Dynamics Dashboard
\begin{frame}{Training Dynamics: Watching Networks Learn}
\begin{center}
\textbf{Real-Time Monitoring: The Training Dashboard}
\end{center}
\vspace{1mm}
\begin{columns}
\column{0.55\textwidth}
\includegraphics[width=0.9\textwidth]{../figures/training_dynamics_dashboard.pdf}

\textbf{Key Metrics to Track:}
\begin{itemize}
\item \textbf{Loss Curves:} Training vs validation
\item \textbf{Accuracy:} How often we're right
\item \textbf{Learning Rate:} Speed of updates
\item \textbf{Gradient Norm:} Update magnitude
\end{itemize}

\column{0.42\textwidth}
\textbf{Warning Signs:}
\begin{itemize}
\item Gap = Overfitting
\item Flat = Learning stopped
\item Spikes = Instability
\item NaN = Numerical issues
\end{itemize}

\textbf{Healthy Training:}
\begin{itemize}
\item Smooth decrease
\item Val follows train
\item Gradients stable
\item LR decays properly
\end{itemize}

\textbf{When to Stop:}
\begin{itemize}
\item Validation plateaus
\item Gap increasing
\item Diminishing returns
\end{itemize}
\end{columns}
\bottomnote{ Modern training requires constant monitoring - it's more art than science}
\end{frame}

% Part 4: Modern revolution
\begin{frame}{Part 4: The Deep Learning Revolution (2012-Present)}
\begin{center}
{\Large \textbf{The Explosion of Modern AI}}
\end{center}
\end{frame}

% Act IV: The Deep Learning Revolution
\begin{frame}{Act IV: The Deep Learning Revolution}
\begin{center}
{\Large \textbf{2014-Present: Networks That Changed the World}}
\end{center}
\vspace{5mm}
\begin{columns}
\column{0.48\textwidth}
\textbf{The Depth Revolution:}
\begin{itemize}
\item 2014 - VGGNet: 19 layers
\item 2015 - ResNet: 152 layers
\item 2017 - Transformers: Attention
\item 2020 - GPT-3: 175B parameters
\end{itemize}

\textbf{Why Depth Matters:}
\begin{itemize}
\item Each layer = abstraction level
\item Deep = complex reasoning
\item Hierarchical feature learning
\end{itemize}

\column{0.48\textwidth}
\textbf{Real-World Impact:}
\begin{itemize}
\item \textbf{Vision:} Self-driving cars
\item \textbf{Language:} Google Translate
\item \textbf{Speech:} Siri, Alexa
\item \textbf{Medicine:} Disease diagnosis
\item \textbf{Science:} Protein folding
\end{itemize}

\textbf{The Scale:}
\begin{itemize}
\item Billions of parameters
\item Trained on internet-scale data
\item Months of GPU time
\item Emergent abilities appear
\end{itemize}
\end{columns}
\bottomnote{ We went from recognizing digits to passing the bar exam in 25 years}
\end{frame}

% ResNet Innovation
\begin{frame}{2015: ResNet - The Skip Connection Revolution}
\begin{center}
\textbf{Problem: Networks Couldn't Get Deeper}
\end{center}
\vspace{5mm}
\begin{columns}
\column{0.48\textwidth}
\textbf{The Vanishing Gradient:}
\begin{itemize}
\item Gradients multiply through layers
\item Become exponentially small
\item Deep layers stop learning
\item 20 layers was the limit
\end{itemize}

\textbf{The Breakthrough: Skip Connections}
\begin{itemize}
\item Add input directly to output
\item $F(x) + x$ instead of just $F(x)$
\item Gradients flow directly backward
\item Can train 1000+ layers!
\end{itemize}

\column{0.48\textwidth}
\begin{center}
\includegraphics[width=\textwidth]{../figures/resnet_skip_connection.pdf}
\end{center}
\textbf{Why It Works:}
\begin{itemize}
\item Learn residual (difference) only
\item Identity mapping is easy default
\item Gradients have direct path
\item Each layer refines previous result
\end{itemize}
\end{columns}
\bottomnote{ This simple trick enabled the deep learning revolution}
\end{frame}

% Batch Normalization
\begin{frame}{Batch Normalization: Keeping Networks Stable}
\begin{center}
\textbf{The Internal Covariate Shift Problem}
\end{center}
\vspace{2mm}
\begin{columns}
\column{0.48\textwidth}
\textbf{The Issue:}
\begin{itemize}
\item Each layer's input distribution changes
\item As previous layers update
\item Makes learning unstable
\item Requires tiny learning rates
\end{itemize}

\textbf{The Solution:}
\begin{itemize}
\item Normalize inputs to each layer
\item Mean = 0, Variance = 1
\item Learn scale and shift parameters
\item Apply during training and testing
\end{itemize}

\column{0.48\textwidth}
\textbf{BatchNorm Algorithm:}
\begin{align*}
\mu_B &= \frac{1}{m} \sum_{i=1}^{m} x_i \\
\sigma_B^2 &= \frac{1}{m} \sum_{i=1}^{m} (x_i - \mu_B)^2 \\
\hat{x}_i &= \frac{x_i - \mu_B}{\sqrt{\sigma_B^2 + \epsilon}} \\
y_i &= \gamma \hat{x}_i + \beta
\end{align*}
\plainmath{1) Find average, 2) Find spread, 3) Normalize to standard range, 4) Scale and shift as needed}

\textbf{Benefits:}
\begin{itemize}
\item 10x faster training
\item Higher learning rates OK
\item Less sensitive to initialization
\item Acts as regularization
\end{itemize}
\end{columns}
\bottomnote{ Now standard in every deep network}
\end{frame}

% NEW: The Lottery Ticket Hypothesis
\begin{frame}{2019: The Lottery Ticket Hypothesis}
\begin{center}
\textbf{Most Network Weights Don't Matter!}
\end{center}
\vspace{5mm}
\begin{columns}
\column{0.48\textwidth}
\textbf{The Discovery:}
\begin{itemize}
\item Networks contain "winning tickets"
\item Subnetworks that train well alone
\item 90-95\% of weights can be removed
\item Performance stays the same!
\end{itemize}

\textbf{The Hypothesis:}
"Dense networks succeed because they contain sparse subnetworks that are capable of training effectively"

\column{0.48\textwidth}
\textbf{Implications:}
\begin{itemize}
\item We massively overparameterize
\item Training finds the needle in haystack
\item Future: Train small from start?
\item Mobile deployment possible
\end{itemize}

\textbf{Why It Matters:}
\begin{itemize}
\item Explains why big networks train better
\item Pruning after training works
\item Efficiency revolution starting
\item Changes how we think about learning
\end{itemize}
\end{columns}
\bottomnote{ A 1 billion parameter model might only need 50 million}
\end{frame}

% NEW: Inductive Biases
\begin{frame}{Inductive Biases: Building in Assumptions}
\begin{center}
\textbf{The Right Architecture for the Right Problem}
\end{center}
\vspace{5mm}
\begin{columns}
\column{0.48\textwidth}
\textbf{What Are Inductive Biases?}
\begin{itemize}
\item Assumptions built into architecture
\item Guide learning toward solutions
\item Trade flexibility for efficiency
\item "Priors" about the problem
\end{itemize}

\textbf{Examples:}
\begin{itemize}
\item \textbf{CNN:} Spatial locality matters
\item \textbf{RNN:} Order/time matters
\item \textbf{GNN:} Graph structure matters
\item \textbf{Transformer:} All positions can interact
\end{itemize}

\column{0.48\textwidth}
\textbf{Why They Matter:}
\begin{itemize}
\item Reduce search space
\item Faster convergence
\item Better generalization
\item Less data needed
\end{itemize}

\textbf{The Tradeoff:}
\begin{itemize}
\item Right bias = 10x better
\item Wrong bias = 10x worse
\item General architectures = safe but slow
\item Specialized = fast but limited
\end{itemize}
\end{columns}
\bottomnote{ Choosing the right inductive bias is still an art}
\end{frame}

% NEW: Emergent Abilities
\begin{frame}{Emergent Abilities: When Scale Creates Intelligence}
\begin{center}
\textbf{Capabilities That Appear Suddenly with Scale}
\end{center}
\vspace{5mm}
\begin{columns}
\column{0.48\textwidth}
\textbf{The Phenomenon:}
\begin{itemize}
\item Small models: Can't do task at all
\item Medium models: Still can't
\item Large models: Suddenly can!
\item No gradual improvement
\end{itemize}

\textbf{Examples:}
\begin{itemize}
\item 3-digit arithmetic (>10B params)
\item Chain-of-thought reasoning (>50B)
\item Code generation (>20B)
\item Multilingual translation (>100B)
\end{itemize}

\column{0.48\textwidth}
\textbf{Why It Happens:}
\begin{itemize}
\item Complex patterns need capacity
\item Phase transitions in learning
\item Composition of simpler abilities
\item "Grokking" - sudden understanding
\end{itemize}

\textbf{Implications:}
\begin{itemize}
\item We can't predict what's next
\item Scaling might unlock AGI
\item Or hit fundamental limits
\item Active area of research
\end{itemize}
\end{columns}
\bottomnote{ GPT-3 showed abilities nobody expected or programmed}
\end{frame}

% NEW: Depth vs Width Comparison
\begin{frame}{Architecture Choices: Deep vs Wide Networks}
\begin{center}
\textbf{The Fundamental Tradeoff in Neural Architecture}
\end{center}
\vspace{2mm}
\begin{columns}
\column{0.55\textwidth}
\includegraphics[width=0.9\textwidth]{../figures/depth_width_comparison.pdf}

\textbf{Deep Networks (Many Layers):}
\begin{itemize}
\item Complex hierarchical features
\item Exponential expressiveness growth
\item Harder to train (vanishing gradients)
\item Better for vision, NLP
\end{itemize}

\textbf{Wide Networks (Many Neurons):}
\begin{itemize}
\item More parallel processing
\item Easier optimization landscape
\item Linear expressiveness growth
\item Better for tabular data
\end{itemize}

\column{0.42\textwidth}
\textbf{The Sweet Spot:}
\begin{itemize}
\item Vision: Deep (100+ layers)
\item Language: Very deep (24-96 layers)
\item Tabular: Wide and shallow (2-4 layers)
\item Time series: Moderate (5-10 layers)
\end{itemize}

\textbf{Modern Insights:}
\begin{itemize}
\item Depth beats width for same parameters
\item Skip connections enable extreme depth
\item Width helps with memorization
\item Depth helps with generalization
\end{itemize}

\textbf{Scaling Laws:}
\begin{itemize}
\item Performance $\propto$ depth$^{0.8}$
\item Performance $\propto$ width$^{0.5}$
\end{itemize}
\end{columns}
\bottomnote{ The depth revolution (2012-2016) showed that deep beats wide for most problems}
\end{frame}

% NEW: Data Scaling Laws
\begin{frame}{Scaling Laws: How Performance Grows with Data}
\begin{center}
\textbf{The Predictable Relationship Between Data, Model Size, and Performance}
\end{center}
\vspace{2mm}
\begin{columns}
\column{0.55\textwidth}
\includegraphics[width=0.82\textwidth]{../figures/data_scaling_laws.pdf}

\textbf{The Chinchilla Law (2022):}
\begin{itemize}
\item Optimal ratio: 20 tokens per parameter
\item 10B model needs 200B tokens
\item Most models are undertrained
\item Data quality matters more than quantity
\end{itemize}

\textbf{Power Law Scaling:}
$$\text{Loss} = A \cdot N^{-\alpha} + B \cdot D^{-\beta} + C$$
\begin{itemize}
\item $N$ = model parameters
\item $D$ = dataset size
\item $\alpha \approx 0.07$, $\beta \approx 0.10$
\end{itemize}

\column{0.42\textwidth}
\textbf{Practical Implications:}
\begin{itemize}
\item 10x data $\rightarrow$ 2x performance
\item 10x parameters $\rightarrow$ 1.7x performance
\item 10x compute $\rightarrow$ 3x performance
\item Diminishing returns always
\end{itemize}

\textbf{Data Efficiency Tricks:}
\begin{itemize}
\item Data augmentation
\item Synthetic data generation
\item Active learning
\item Curriculum learning
\item Multi-task training
\end{itemize}

\whymatters{These laws predict costs before training}

\textbf{Current Limits:}
\begin{itemize}
\item Internet has ~10T tokens
\item Already using most of it
\item Quality > Quantity now
\end{itemize}
\end{columns}
\bottomnote{ These laws let us predict performance before training - saving millions in compute}
\end{frame}

% NEW: Optimizer Comparison
\begin{frame}{Optimization Algorithms: How Networks Learn}
\begin{center}
\textbf{The Evolution of Gradient Descent}
\end{center}
\vspace{2mm}
\begin{columns}
\column{0.55\textwidth}
\includegraphics[width=0.82\textwidth]{../figures/optimizer_comparison.pdf}

\textbf{SGD (1951):}
\begin{itemize}
\item Basic gradient descent
\item Learning rate: Fixed
\item Slow but reliable
\item Still used for fine-tuning
\end{itemize}

\textbf{Momentum (1964):}
\begin{itemize}
\item Remember past gradients
\item Accelerate in consistent directions
\item Escape shallow local minima
\item $v_t = \beta v_{t-1} + (1-\beta) g_t$
\end{itemize}

\column{0.42\textwidth}
\textbf{Adam (2014):}
\begin{itemize}
\item Adaptive learning rates per parameter
\item Combines momentum + RMSprop
\item De facto standard
\item Works out-of-the-box
\end{itemize}

\textbf{Modern Variants:}
\begin{itemize}
\item \textbf{AdamW:} Decoupled weight decay
\item \textbf{RAdam:} Rectified Adam
\item \textbf{LAMB:} Large batch training
\item \textbf{Sophia:} 2nd-order approximation
\end{itemize}

\textbf{Choosing an Optimizer:}
\begin{itemize}
\item Start with Adam (lr=3e-4)
\item Large batch: LAMB
\item Fine-tuning: SGD with momentum
\item Transformers: AdamW
\end{itemize}
\end{columns}
\bottomnote{ Adam works 90\% of the time - but that last 10\% matters for SOTA results}
\end{frame}

% NEW: Architecture Decision Tree
\begin{frame}{Quick Guide: Choosing Your Architecture}
\begin{center}
\textbf{\checkpoint{Which Network Should You Use?}}
\end{center}
\vspace{2mm}
\begin{columns}
\column{0.55\textwidth}
\includegraphics[width=0.9\textwidth]{../figures/architecture_decision_tree.pdf}

\tryit{You have 10,000 customer reviews to classify as positive/negative. Which architecture? Why?}

\column{0.42\textwidth}
\textbf{Decision Questions:}
\begin{enumerate}
\item Is your data sequential?
\item Does position matter?
\item Is it images/spatial?
\item Fixed or variable size?
\end{enumerate}

\textbf{Quick Rules:}
\begin{itemize}
\item Images $\rightarrow$ CNN
\item Text $\rightarrow$ Transformer/RNN
\item Tabular $\rightarrow$ Feedforward
\item Audio $\rightarrow$ CNN or RNN
\item Video $\rightarrow$ CNN + RNN
\end{itemize}

\confusion{Transformers now dominate most tasks, but specialized architectures still win for specific problems!}
\end{columns}
\bottomnote{ Answer: Transformer or RNN - text is sequential and context matters}
\end{frame}

% ============================================================================
% PHASE 5: YOUR TURN (Slides 45-48)
% ============================================================================

% Epilogue: Key Takeaways and Practical Implementation
% Key Takeaways
\begin{frame}{Key Takeaways: From Mail Sorting to ChatGPT}
\begin{center}
{\Large \textbf{The Journey So Far}}
\end{center}
\vspace{5mm}
\begin{columns}
\column{0.48\textwidth}
\textbf{Core Concepts:}
\begin{enumerate}
\item \textbf{Neurons:} $y = f(\sum w_i x_i + b)$
\item \textbf{Learning:} Adjust weights to minimize error
\item \textbf{Depth:} Each layer adds abstraction
\item \textbf{Backpropagation:} Distribute error backwards
\item \textbf{Non-linearity:} Enables complex functions
\end{enumerate}

\column{0.48\textwidth}
\textbf{Historical Lessons:}
\begin{enumerate}
\item Every limitation spawned innovation
\item Simple ideas + scale = revolution
\item Biology inspires but doesn't limit
\item Persistence through AI winters
\item Theory + engineering = breakthroughs
\end{enumerate}
\end{columns}
\bottomnote{ You now understand the fundamentals that power all modern AI}
\end{frame}

% NEW: Your First Neural Network
\begin{frame}[fragile]{Epilogue: Your First Neural Network in 5 Minutes}
\begin{center}
\textbf{\checkpoint{Let's Build Something Real!}}
\end{center}
\vspace{1mm}
\begin{columns}
\column{0.55\textwidth}
\textbf{Complete MNIST Classifier:}
\begin{lstlisting}[language=Python,basicstyle=\tiny]
import torch
import torch.nn as nn
from torchvision import datasets, transforms
from torch.utils.data import DataLoader

# 1. Define Network
class Net(nn.Module):
    def __init__(self):
        super(Net, self).__init__()
        self.fc1 = nn.Linear(784, 128)
        self.fc2 = nn.Linear(128, 64)
        self.fc3 = nn.Linear(64, 10)

    def forward(self, x):
        x = x.view(-1, 784)  # Flatten
        x = torch.relu(self.fc1(x))
        x = torch.relu(self.fc2(x))
        return self.fc3(x)

# 2. Load Data
transform = transforms.ToTensor()
train_data = datasets.MNIST('.', train=True,
                           download=True,
                           transform=transform)
train_loader = DataLoader(train_data,
                         batch_size=64,
                         shuffle=True)

# 3. Setup Training
model = Net()
optimizer = torch.optim.Adam(model.parameters())
criterion = nn.CrossEntropyLoss()
\end{lstlisting}

\column{0.42\textwidth}
\begin{lstlisting}[language=Python,basicstyle=\tiny]
# 4. Training Loop
for epoch in range(3):
    for batch_idx, (data, target) in
                enumerate(train_loader):
        # Forward pass
        output = model(data)
        loss = criterion(output, target)

        # Backward pass
        optimizer.zero_grad()
        loss.backward()
        optimizer.step()

        # Print progress
        if batch_idx % 100 == 0:
            print(f'Epoch {epoch}: {loss:.4f}')

# 5. Test One Example
model.eval()
test_image = train_data[0][0]
prediction = model(test_image.unsqueeze(0))
print(f"Predicted: {prediction.argmax()}")
\end{lstlisting}

\vspace{1mm}
\textbf{What You Just Built:}
\begin{itemize}
\item 3-layer neural network
\item 60K training images
\item 97\% accuracy in 3 epochs
\item Runs in 2 minutes on laptop
\end{itemize}

\tryit{Copy this code and run it now!}
\end{columns}
\bottomnote{ Congratulations - you've just trained your first neural network!}
\end{frame}

% Where to Go Next
\begin{frame}{Where to Go Next: Your Learning Path}
\begin{center}
\textbf{Continue Your Neural Network Journey}
\end{center}
\vspace{2mm}
\begin{columns}
\column{0.48\textwidth}
\textbf{Next Topics to Learn:}
\begin{enumerate}
\item \textbf{CNNs:} Computer vision
\item \textbf{RNNs/LSTMs:} Sequences
\item \textbf{Transformers:} Modern NLP
\item \textbf{GANs:} Generation
\item \textbf{RL:} Decision making
\end{enumerate}

\textbf{Practical Projects:}
\begin{itemize}
\item Image classifier for your photos
\item Sentiment analysis of tweets
\item Chatbot for customer service
\item Style transfer for art
\item Game-playing agent
\end{itemize}

\column{0.48\textwidth}
\textbf{Resources:}
\begin{itemize}
\item \textbf{Fast.ai:} Practical deep learning
\item \textbf{PyTorch Tutorials:} Official guides
\item \textbf{Papers with Code:} Latest research
\item \textbf{Kaggle:} Competitions and datasets
\item \textbf{3Blue1Brown:} Visual explanations
\end{itemize}

\textbf{Remember:}
\begin{itemize}
\item Start simple, build up
\item Theory + practice together
\item Join communities
\item Build projects you care about
\item Share what you learn
\end{itemize}
\end{columns}

\vspace{2mm}
\begin{center}
\colorbox{mlgreen!10}{\parbox{0.8\textwidth}{
\centering
\textbf{You've learned how humanity taught machines to think.\\
Now it's your turn to push the boundaries!}
}}
\end{center}
\bottomnote{ The future of AI is being written now - be part of it!}
\end{frame}

% ============================================================================
% APPENDICES
% ============================================================================

\begin{frame}{Appendices}
\begin{center}
{\Large \textbf{Additional Material for Deep Dive}}
\end{center}
\vspace{10mm}
\begin{itemize}
\item Appendix A: Advanced Topics
\item Appendix B: Extended History
\end{itemize}
\end{frame}

\input{sections/appendix_a_advanced_topics}
% Appendix B: Extended History
\begin{frame}{Appendix B: Extended Historical Details}
\begin{center}
{\Large \textbf{The Full Story: Historical Deep Dives}}
\end{center}
\vspace{10mm}

\textbf{This section contains fascinating historical details that enrich the narrative but aren't essential for understanding the technical concepts.}

\vspace{5mm}
Topics covered:
\begin{itemize}
\item The Mark I Perceptron: Physical Hardware
\item NetTalk: Networks Learn to Speak (1987)
\item Batch Normalization: Keeping Networks Stable
\end{itemize}

\vspace{5mm}
\textit{These stories show how each breakthrough built on previous work and overcame specific limitations.}
\end{frame}

% MARK I PERCEPTRON
\begin{frame}{The Mark I Perceptron: A Physical Learning Machine}
\begin{center}
\includegraphics[width=0.85\textwidth]{../figures/perceptron_hardware.pdf}
\end{center}
\bottomnote{ The first neural network wasn't software---it was a room-sized machine with motors and photocells}
\end{frame}

% NETTALK
\begin{frame}{1987: NetTalk - Networks Learn to Speak}
\begin{center}
\textbf{Sejnowski \& Rosenberg: The First Viral NN Demo}
\end{center}
\vspace{5mm}
\begin{columns}
\column{0.48\textwidth}
\textbf{The Challenge:}
\begin{itemize}
\item Convert written text to speech
\item English is irregular (tough, though, through)
\item Rule-based systems had 1000s of exceptions
\end{itemize}

\textbf{The Network:}
\begin{itemize}
\item 7$\times$29 input (7-letter window)
\item 80 hidden neurons
\item 26 output phonemes
\item Trained overnight on DEC workstation
\end{itemize}

\column{0.48\textwidth}
\textbf{The Magic:}
\begin{itemize}
\item Started: Random babbling
\item Hour 1: Vowel-consonant patterns
\item Hour 5: Recognizable words
\item Hour 10: 95\% accuracy!
\end{itemize}

\textbf{Hidden Neurons Learned:}
\begin{itemize}
\item Vowel detectors
\item Consonant clusters
\item Word boundaries
\item \highlight{Nobody programmed these!}
\end{itemize}

\confusion{The network discovered linguistic concepts on its own - features linguists took centuries to identify!}
\end{columns}
\bottomnote{ Media sensation: "Computer teaches itself to read aloud overnight"}
\end{frame}

% BATCH NORMALIZATION
\begin{frame}{Batch Normalization: Keeping Networks Stable}
\begin{center}
\textbf{The Internal Covariate Shift Problem}
\end{center}
\vspace{2mm}
\begin{columns}
\column{0.48\textwidth}
\textbf{The Issue:}
\begin{itemize}
\item Each layer's input distribution changes
\item As previous layers update
\item Makes learning unstable
\item Requires tiny learning rates
\end{itemize}

\textbf{The Solution:}
\begin{itemize}
\item Normalize inputs to each layer
\item Mean = 0, Variance = 1
\item Learn scale and shift parameters
\item Apply during training and testing
\end{itemize}

\column{0.48\textwidth}
\textbf{BatchNorm Algorithm:}
\begin{align*}
\mu_B &= \frac{1}{m} \sum_{i=1}^{m} x_i \\
\sigma_B^2 &= \frac{1}{m} \sum_{i=1}^{m} (x_i - \mu_B)^2 \\
\hat{x}_i &= \frac{x_i - \mu_B}{\sqrt{\sigma_B^2 + \epsilon}} \\
y_i &= \gamma \hat{x}_i + \beta
\end{align*}
\plainmath{1) Find average, 2) Find spread, 3) Normalize to standard range, 4) Scale and shift as needed}

\textbf{Benefits:}
\begin{itemize}
\item 10x faster training
\item Higher learning rates OK
\item Less sensitive to initialization
\item Acts as regularization
\end{itemize}
\end{columns}
\bottomnote{ Now standard in every deep network}
\end{frame}

\end{document}